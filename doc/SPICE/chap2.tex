\chapter{Getting Started\label{chapter:chap2}}

In this chapter, we simulate a small circuit in order to
introduce you to \spice.  We describe the input file, or ``circuit''
file, showing you the generic structure of the file, and
giving a number of examples.   Though in each example we describe
what is shown, we do not list all the options and variations
for each item described.  The reader is referred to the reference
sections
(chapters \ref{chapter:statement} and \ref{chapter:element})
for that.  We then show
you how to run \spice\ and discuss the different features of the
output file.

\section{The Input File}

A typical input file, and a schematic of the circuit and input
waveform it is simulating, is shown in Figure~\ref{input}.
%
\begin{figure}
\centerline{\pfig{simpleex.ps}}
\par\noindent
\begin{center}
\begin{tabular}{ll}
{\tt Simple RC Network} & {\em Title Line} \\
{\tt * } & \\
{\tt * This is a comment} & {\em Comments Lines} \\
{\tt * } & \\
{\tt R1 1 2 500} & {\em Element Lines} \\
{\tt C1 2 0 5p } & \\
{\tt vin 1 0 pulse (0 1.5 4ns 3ns 5ns 2ns 17ns)} & \\
{\tt * } & \\
{\tt .tran 500ps 34ns} & {\em Control Statement Lines} \\
{\tt .print tran v(1) v(2) i(vin)} & \\
{\tt * } & \\
{\tt .end} & {\em End Line} 
\end{tabular}
\end{center}
\caption{Example circuit and corresponding input file.}
\label{input}
\end{figure}
%
The input file is created with a text editor  and is typically
named something like `test.cir'.  The file is made up of five
types of lines:
\begin{itemize}
  \item A {\em title line}, up to 80 characters long, placed
	at the start of the file.
  \item An {\tt .end} statement at the end of the file.
	\notforsspice{This statement can
	be safely omitted in many simulators but its usage is recommended for
	compatibility purposes.}
  \item Any number of {\em comment lines}, each starting with {\em `*'}, can
	be placed anywhere after the title and before the end.
  \item Any number of {\em element lines} that describe the circuit
	to be simulated.  The basic syntax of the element line is
\begin{center}
{\it name node node ... value}
%  $<$name$>$ $<$node$>$ $<$node$>$ ... $<$value$>$
\end{center}
where  {\it name} is the name that you assign to the element.  The first
character in the name identifies the type of circuit element being
described, e.g.
`{\tt R}' for a resistor.  From one to seven characters must then be added to
the name to identify it uniquely, e.g. `{\tt R1}', or `{\tt Rpulldn}'.  Numbers are
usually used.  The  {\it node}'s identify the nodes in the circuit
to which the
terminals or `leads' of the circuit element are connected.   For example,
one terminal of the capacitor `{\tt C1}' is connected to the node numbered
`{\tt 0}',
which must be used for the ground (or common reference) node, and the
other terminal is
connected to node number `{\tt 2}'.  In \spice\ each circuit node is identified
by a unique number.  {\it value} describes the value(s) needed to
describe the element.

  \item Any number of {\em control statement lines} that specify 
	what type of circuit analysis is to be performed and how the
	results are to be reported.
\end{itemize}
The elements and control statement lines can be written in any order,
even intermixed.

The first two element lines describe a 1~k$\Omega$ resistor and a 
5~fF capacitor.  Almost-standard metric prefixes are used in
\spice, the prefix abbreviation, the full metric name, and the represented
scale factors being as follows:
\begin{center}
  \begin{tabular}{lll}
{\bf Spice Prefix} & {\bf Metric equivalent} & {\bf Scale} \\
  \ \ \ \ \ F & femto & $10^{-15}$ \\
  \ \ \ \ \ P & pico  & $10^{-12}$ \\
  \ \ \ \ \ N & nano  & $10^{-9}$ \\
  \ \ \ \ \ U & micro & $10^{-6}$ \\
  \ \ \ \ \ M & milli & $10^{-3}$ \\
  \ \ \ \ \ K & kilo  & $10^{+3}$ \\
  \ \ \ \ \ MEG & mega  & $10^{+6}$ \\
  \ \ \ \ \ G & giga  & $10^{+9}$ \\
  \ \ \ \ \ T & tera  & $10^{+12}$ \\
  \end{tabular}
\end{center}
As \spice\ does not differentiate between upper and lower case, `{\tt MEG}'
(or `{\tt meg}')
is used for `mega' instead of the standard metric upper case `M'.

The value of an element is specified in terms of the conventionally
accepted units,
e.g. resistance in Ohms, capacitance in Farads, and inductance in Henries.
If you wish you can spell it out more fully, e.g.
\begin{center}
\begin{tabular}{ll}
{\tt  C1  0  2  5f} & or \\
{\tt  C1  0  2  5fF} & or \\
{\tt C1  0  2  5fFarad}  &or even \\
{\tt C1 0 2 5fthingies} &
\end{tabular}
\end{center}
The last alternative is allowed as
\spice\ actually ignores whatever follows the `{\tt f}' and assumes Farads.

The element named `{\tt Vin}' is an example of an independent voltage source.
In this case the voltage source produces a repeating pulse, as shown below:
\par\noindent
%\vspace{5mm}
\par\noindent
\centerline{\pfig{pulseex.ps}}
\par\noindent
{\tt vin 1 0 pulse (0 1 5 4ns 3ns 2ns 5ns 17ns)}
\vspace{5mm}
\par\noindent
The general format for a pulse independent source is:
\vspace{5mm}
\par\noindent
{\it {\tt V}name Node1 Node2 pulse (Initial-Value Pulsed-Value Delay-Time
Rise-Time  Fall-Time Pulse-Width Period)}
\vspace{5mm}
\par\noindent
The first control statement, {\tt .tran 200ps 34ns}, specifies that a 
transient response simulation is to be run, i.e. we wish to know how
the circuit behaves as a function of time.   The total length of
the simulation is to be 34~ns and the outputs are to be obtained
every 500ps.  The second control statement, {\tt .print tran v(1) v(2) i(vin)},
specifies that the output is to be in the format of a printed table
({\tt .print});
that transient waveforms ({\tt tran}) are to be tabulated
as a function of time; and that we wish to know the values for the voltage
at the input (node 1 {\tt v(1)}, the output (node 2 {\tt v(2)}) and the
current through the voltage source {\tt Vin}
at the time steps specified in the {\tt .tran} statement.
Note that all control statements start with a period `{\tt .}'.
We are now ready to run the simulator.

\section{Running Spice and Viewing the Output}

\notforsspice{The details of how to run \spice\ vary from system to system.  On a 
computer running the Unix operating system,}
\spice\  can be run with a command line like the following:
\begin{center}
{\tt spice test.cir test.out}
\end{center}
\par\noindent
This will cause \spice\ to read in the file and produce the output listing
file `test.out'.  Parts of the test.out are shown in Figure~\ref{out}.
%
\begin{figure}
\centerline{\epsfysize=15cm\pfig{simpout.ps}}
\caption{Portions of the \spice\ output file.}
\label{out}
\end{figure}
%
The first part of the file is a header.  Then the input file
is listed.  Just like a human circuit analyzer, \spice\ has to first 
calculate the initial DC conditions before running the transient analysis.
The results of this calculation are shown next.  Finally, the transient
response is tabulated.

In this output, `{\tt D}' and `{\tt E}' both indicate scientific
notation, e.g. {\tt 4.500D-09}  means $4.5 \times 10^{-9}$.

There are other ways to view the output.  For example if a {\tt .plot}
control statement is used instead of a {\tt .print} statement, the
voltages and currents are plotted using character `graphics', an
example of which is given in Figure~\ref{cg}.
%
\begin{figure}
{\tt
\begin{verbatim}
0LEGEND:

 *: V(1)
 +: V(2)
 =: I(VIN)
X
     TIME      V(1)

 ...

(=)-----------------  -4.000D-06               -2.000D-06
                           - - - - - - - - - - - - - - - - - - - - - - -
 0.000D+00    0.000D+00   *                        +
 5.000D-10    0.000D+00   *                        +
 1.000D-09    0.000D+00   *                        +
 1.500D-09    0.000D+00   *                        +

 ....

  4.500D-09    2.500D-01   .            *      =    .     +
\end{verbatim}
}
(Only part of the file is shown here:  the legend, the `y-axis' scale for
{\tt I(VIN)} (the `y-axis' is left to right across the page) and part of
the plot with the `time' axis going down the page.)
\caption{Example of output produced by the {\tt .plot} control statement.}
\label{cg}
\end{figure}

\notforsspice{Using appropriate graphical packages, the output can also be plotted 
as smooth curves.  For example, gnuplot was used to create the
graph shown in Figure~\ref{op}.
%
\begin{figure}
\centerline{\epsfxsize=6in\pfig{simple.ps}}
\caption{Results plotted graphically.}
\label{op}
\end{figure}
%
%This will be discussed later.
}

\section{Error Messages}
\label{err1}

There are many ways to produce errors in \spice.  The most common error
produced by novice users is to `connect the circuit' up incorrectly. 
For example, in the \spice\ input file discussed above,
if we connect one of the nodes of the capacitor to node 1 instead of
node 2, viz.
{\tt
\begin{verbatim}
r1 1 2 1k
c1 0 1 5f
vin 1 0 pulse (0 1.5 4ns 3ns 5ns 2ns 17ns)
\end{verbatim}}
\noindent then we have not specified our circuit as drawn.  In this case, we also
leave one terminal of the resistor unconnected to anything else
and \spice\ detects the error and reports it in the output file:
{\tt
\begin{verbatim}
0*ERROR*:  LESS THAN 2 CONNECTIONS AT NODE      2
\end{verbatim} } 
However, life  is rarely so simple.  In a complex circuit it is always easy
to get one node number wrong on one element but leave all of the nodes connected
to two or more elements.  In this case \spice\ might detect no errors.  If the
output looks `wrong' for any reason, the first thing to do is 
to draw your circuit by looking at the \spice\ file as written and check
that against your intended circuit.

Another important thing to remember about error messages is that \justspice\
is not very good at drawing attention to them.
\justspice\ output files 
tend to be long and are cryptic looking.  Error and Warning messages
can be found almost anywhere within them.
\notforsspice{Read the entire file.
Errors are discussed further in Chapter \ref{chapter:chap3} and Appendix
\ref{chapter:error}.}
