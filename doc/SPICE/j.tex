\element{J}{Junction Field-Effect Transistor}
\begin{figure}[h]
\centering
\ \pfig{j_spice.ps}
\caption[J --- Junction field effect transistor element]{J --- Junction field effect
transistor element: (a) {\it n} channel JFET; (b) {\it p} channel JFET.\label{j.ps}}
\end{figure}

\form{{\tt J}name NDrain NGate NSource ModelName
 \B Area\E  \B {\tt OFF}\E  \B {\tt IC}=$V_{DS},V_{GS}$\E }

\pspiceform{{\tt J}name NDrain NGate NSource ModelName \B Area\E }
\example{J1 7 2 3 JM1 OFF}

\begin{widelist}
\item[{\it NDrain}] is the drain node
\item[{\it NGate}] is the gate node
\item[{\it NSource}] is the source node
\item[{\tt ModelName}]  is  the  model name
\item[{\it Area}] is the area factor.
               (Units: none; Optional; Default: 1; Symbol: $Area$)
\item[{\tt OFF}] indicates an (optional) initial condition on the device for
\dc\ operating point analysis.
If specified the \dc\ operating point is calculated with the terminal voltages
set to zero.  Once convergence is obtained, the
program continues to iterate to obtain the exact  value of
the  terminal  voltages.  The OFF option is used to enforce the solution
to  correspond  to  a  desired  state if the circuit has more than one stable
state.
\item[{\tt IC}] is the optional  initial condition specification.
Using {\tt IC=}$V_{DS},V_{GS}$
is intended for use with the {\tt UIC} option
on  the  {\tt .TRAN}  line,  when  a transient analysis is desired
starting from other than the quiescent operating point.
Specification of the transient initial conditions using the {\tt .IC}
statement (see page \pageref{.ICstatement}) is preferred and is more
convenient.
\end{widelist}

\model{NJF}{N-Channel JFET Model}
\model{PJF}{P-Channel JFET Model}
\begin{figure}[h]
\centering
\ \epsfxsize=2.75in\pfig{jfet.ps}
\caption[Schematic of the JFET model]{Schematic of the JFET model. \label{jfet}
$V_{GS}$, $V_{DS}$, and $V_{GD}$ are intrinsic gate-source,
drain-source and gate-drain voltages
between the internal gate, drain, and source terminals designated
$G$, $D$, and $C$ respectively.  }
\end{figure}

The parameters of the {\it n}-channel (NJF) and of
the {\it p}-channel (PJF) models are the same and are given in table
\ref{jtable}.

The parameters of the JFET can be completely specified in the model
{\it ModelName}. This facilitates the use of standard transistors by using
absolute quantities in the model.  Alternatively scalable process parameters can
be specified in the model {\it ModelName} and these scaled by the $Area$
parameter on the JFET element line.   The parameters that can be scaled
by $Area$ are
{\tt BETA}, {\tt CGS}, {\tt CGD}, {\tt IS}, {\tt RD} and {\tt RS}.

\begin{table}[h]
\caption[NJF and PJF model keywords for the junction field effect transistor]{NJF
and PJF model keywords for the junction field effect transistor.
The Area column indicates parameters that are scaled by $Area$.
\label{jtable}}
\keywordtwotable{Area}{
{\tt AF} & flicker noise exponent\sym{A_F}& -      & 1 &  \X
{\tt BETA} & transconductance parameter
      \sym{\beta}& $\mbox{A/V}^2$& 1.0E-4   &  \STAR\X
{\tt CGS} & zero-bias G-S junction capacitance per unit area\sym{C'_{GS}}
     & F      & 0        &  \STAR\X
{\tt CGD} & zero-bias G-D junction capacitance per unit area\sym{C'_{GD}}
     & F      & 0        &  \STAR\X
{\tt FC} & coefficient for forward-bias depletion\newline capacitance formula\sym{F_C}
       & -    &   0.5 &\X
{\tt IS}&gate junction saturation current\sym{I_S}&A& 1.0E-14  &  \STAR\X
{\tt KF} & flicker noise coefficient\sym{K_F}& -      & 0 &  \X
{\tt LAMBDA}
 & channel length modulation parameter\sym{\lambda}& 1/V &0  & \X
{\tt PB} & gate junction potential\sym{\phi_J}& V      & 1        & \X
{\tt RD} & drain ohmic sheet resistance\sym{R_D}&$\Omega$& 0&  \STAR\X
{\tt RS} & source ohmic sheet resistance\sym{R_S}&$\Omega$& 0&  \STAR\X
{\tt VTO} & threshold voltage (VT-oh)\sym{V_{T0}}\newline
           {\tt VTO} $<$ 0 indicates a depletion mode JFET\newline
           {\tt VTO} $\ge$ 0 indicates an enhancement mode JFET
	  & V& -2.0     & \X
{\tt BETATC}&temperature coefficient of the transconductance parameter {\tt BETA}
                   \sym{T_{C,\beta}} & \%/$^\circ$C& 0\X
{\tt M}    & gate {\it p-n} junction grading  coefficient
                   \sym{M} & -   &   0.5 \X
{\tt VTOTC}& temperature coefficient of threshold voltage {\tt VTO}
                   \sym{T_{C,VT0}} & V/$^\circ$C &   0 \X
}
\end{table}

The physical constants used in the model evaluation are
\begin{center}
\begin{tabular}{|l|l|l|}
\hline
$k$ & Boltzman's constant           &  $1.3806226\,10^{-23}$~J/K\\
$q$ & electronic charge             & $1.6021918\,10^{-19}$~C\\
\hline
\end{tabular}
\end{center}
\noindent\underline{\sl \large Standard Calculations}\\[0.1in]
Absolute temperatures (in kelvins, K) are used.
The thermal voltage
\begin{equation}
V_{\ms{TH}} = {{kT_{\ms{NOM}}} \over q} .
\end{equation}
\noindent The silicon bandgap energy
\begin{equation}
E_G = 1.16 - 0.000702{{4T_{\ms{NOM}}^2} \over {T_{\ms{NOM}}+1108}} .
\end{equation}
\noindent\underline{\sl \large Temperature Dependence}
\index{JFET, Temperature Dependence}
\index{Temperature Dependence, see JFET}
\\[0.1in]
Temperature effects are incorporated as follows where $T$ and $T_{\ms{NOM}}$
are absolute temperatures in Kelvins (K).
\begin{eqnarray}
V_{\ms{TH}} & = & {{kT} \over q}\\
I_S (T) & = & I_S e^{\left( \textstyle E_g(T) {T \over {T_{\ms{NOM}}}}
 - E_G(T) \right) /(nV_{\ms{TH}})}\\
V_{BI} (T) & = & V_{BI} {T \over {T_{\ms{NOM}}}}
 - 3V_{\ms{TH}} \ln{ {T \over {T_{\ms{NOM}}}} }
              + E_G (T_{\ms{NOM}}) {T \over {T_{\ms{NOM}}}} -E_G(T)\\
C'_{GS} (T)&=&C'_{GS}\{1 + M [0.0004(T-T_{\ms{NOM}})+(1-V_{BI} (T)/V_{BI})]\} \\
C'_{GD} (T)&=&C'_{GD}\{1 + M [0.0004(T-T_{\ms{NOM}})+(1-V_{BI} (T)/V_{BI})]\} \\
\beta(T) &=& \beta 1.01^{T_{C,\beta}(T-T_{\ms{NOM}})}\\
V_{T0}(T) &=& V_{T0} + T_{C,VT0}\left(T-T_{\ms{NOM}}\right)
\end{eqnarray}
\noindent\underline{\sl \large Parasitic Resistances}\\[0.1in]
\index{Parasitic Resistances, see JFET}
\index{JFET, \pspice\ Parasitic Resistance}
\index{JFET, \pspice\ $R_S$}
\index{JFET, \pspice\ $R_G$}
\index{JFET, \pspice\ $R_D$}
\index{Junction Field Effect Transistor, see JFET}
The parasitic resistances
are calculated from the sheet resistivities
{\tt RS}, {\tt RG}, {\tt RD}, and the
$Area$ specified on the element line.
\begin{eqnarray}
R_S & = & R'_S Area\\
R_G & = & R'_G Area\\
R_D & = & R'_D Area
\end{eqnarray}

The parasitic resistance parameter dependencies are summarized in
figure \ref{jpara}.\\[0.2in]
\begin{figure}[hbp]
\parbox[t]{1.3in}{
\begin{tabular}[t]{|p{1in}|}
\hline
\multicolumn{1}{|c|}{PROCESS} \\
\multicolumn{1}{|c|}{PARAMETERS} \\
\hline
{\tt RSH} \hfill $\RSH$\\
\hline
\end{tabular}
}
\hfill
\parbox{0.1in}{\ \vspace*{0.2in}\newline +}
\hfill
\begin{tabular}[t]{|p{1in}|}
\hline
\multicolumn{1}{|c|}{GEOMETRY} \\
\multicolumn{1}{|c|}{PARAMETERS} \\
\hline
{\tt NRS} \hfill $\NRS$\\
{\tt NRD} \hfill $\NRD$\\
{\tt NRG} \hfill $\NRG$\\
{\tt NRB} \hfill $\NRB$\\
\hline
\end{tabular}
\hfill
\parbox{0.1in}{\ \vspace*{0.2in}\newline $\rightarrow$}
\hfill
\begin{tabular}[t]{|p{1.8in}|}
\hline
\multicolumn{1}{|c|}{DEVICE} \\
\multicolumn{1}{|c|}{PARAMETERS} \\
\hline
{\tt RD} \hfill $\RD = f(\RSH, \NRD)$ \\
{\tt RS} \hfill $\RS = f(\RSH, \NRS)$ \\
{\tt RG} \hfill $\RG = f(\RSH, \NRG)$ \\
{\tt RB} \hfill $\RB = f(\RSH, \NRB)$ \\
\hline
\end{tabular}
\caption{JFET parasitic resistance parameter
relationships. \label{jpara}}
\end{figure}

\noindent\underline{\sl \large Leakage Currents}\\[0.1in]
\index{Leakage Currents, see JFET}
\index{JFET, I/V Characteristics, Leakage Currents}
\index{JFET, Leakage Currents}
Current flows across the normally reverse biased source-bulk and drain-bulk
junctions.
The gate-source leakage current
\begin{equation}
I_{GS} = Area\,I_{S}e^{(\textstyle V_{GS}/V_{\ms{TH}} -1)}
\end{equation}
and the gate-source leakage current
\begin{equation}
I_{GD} = Area\,I_{S}e^{(\textstyle V_{GD}/V_{\ms{TH}} -1)}
\end{equation}
The dependencies of the parameters describing the leakage current
in the JFET model are summarized in
figure \ref{jleakage}.\\[0.2in]
\begin{figure}[hbp]
\begin{tabular}[t]{|p{1in}|}
\hline
\multicolumn{1}{|c|}{PROCESS} \\
\multicolumn{1}{|c|}{PARAMETERS} \\
\hline
{\tt IS} \hfill $I_S$\\
\hline
\end{tabular}
\hfill
\parbox{0.1in}{\ \vspace*{0.2in}\newline +}
\hfill
\begin{tabular}[t]{|p{1in}|}
\hline
\multicolumn{1}{|c|}{GEOMETRY} \\
\multicolumn{1}{|c|}{PARAMETERS} \\
\hline
$Area$\\
\hline
\end{tabular}
\hfill
\parbox{0.1in}{\ \vspace*{0.2in}\newline $\rightarrow$}
\hfill
\begin{tabular}[t]{|p{1.8in}|}
\hline
\multicolumn{1}{|c|}{DEVICE} \\
\multicolumn{1}{|c|}{PARAMETERS} \\
\hline
\hspace*{\fill}$I_{GS} = f(I_S, Area)$\\
\hspace*{\fill}$I_{GD} = f(I_S, Area)$\\
\hline
\end{tabular}
\caption{JFET leakage current parameter dependecies. \label{jleakage}}
\end{figure}

\noindent\underline{\sl\large I/V Characteristics}\\[0.1in]
The current/voltage characteristics are evaluated after first
determining the mode (normal: $V_{DS} \ge 0$ or inverted:
$V_{DS} < 0$) and the region (cutoff,
linear or saturation) of the current
$(V_{DS}, V_{GS})$ operating point.\\[0.1in]

\noindent{\sl Normal Mode: ($V_{DS} \ge 0$)}\\[0.2in]
The regions are as follows:\\[0.1in]
\hspace*{\fill}\offsetparbox{
\begin{tabular}{ll}
cutoff region:&$V_{GS} < V_{T0}$\\
linear region:&$V_{GS} \ge V_{T0}$
and $V_{GS} > V_{DS}+V_{T0}$\\
saturation region:&$V_{GS} \ge V_{T0}$
and $V_{GS} \le V_{DS}+V_{T0}$\\
\end{tabular}}\\[0.1in]
Then
\begin{equation}
I_{D} = \left\{ \begin{array}{ll}
      0  & \mbox{cutoff region} \\ \\
      Area\,\beta \left(1 + \lambda V_{DS}\right)V_{DS}
      \left[2\left(V_{GS}-V_{T0}\right)-V_{DS}\right]
         &\mbox{linear region}\\
      Area\,\beta \left(1 + \lambda V_{DS}\right)
      \left(V_{GS}-V_{T0}\right)^2
         &\mbox{saturation region} \end{array} \right. %}
      \label{jid}
\end{equation}

\noindent{\sl Inverted Mode: ($V_{DS} < 0)$}\\[0.2in]
In the inverted mode the JFET I/V characteristics are evaluated as in the
normal mode (\ref{jid}) but with the drain and source subscripts
exchanged.
The relationships of the parameters describing the I/V
characteristics are summarized in figure
\ref{ji/v}.\\[0.1in]
\begin{figure}[hbp]
\begin{tabular}[t]{|p{1in}|}
\hline
\multicolumn{1}{|c|}{PROCESS} \\
\multicolumn{1}{|c|}{PARAMETERS} \\
\hline
{\tt ALPHA} \hfill $\alpha$\\
{\tt BETA} \hfill $\beta$\\
{\tt LAMBDA} \hfill $\lambda$\\
{\tt VTO} \hfill $V_{T0}$\\
\hline
\end{tabular}
\hfill
\parbox{0.1in}{\ \vspace*{0.2in}\newline +}
\hfill
\begin{tabular}[t]{|p{1in}|}
\hline
\multicolumn{1}{|c|}{GEOMETRY} \\
\multicolumn{1}{|c|}{PARAMETERS} \\
\hline
\hspace*{\fill} -- \hspace*{\fill} \\
\hline
\hline
\multicolumn{1}{|c|}{Optional} \\
\hline
\hspace*{\fill}$Area$\\
\hline
\end{tabular}
\hfill
\parbox{0.1in}{\ \vspace*{0.2in}\newline $\rightarrow$}
\hfill
\begin{tabular}[t]{|p{1.8in}|}
\hline
\multicolumn{1}{|c|}{DEVICE} \\
\multicolumn{1}{|c|}{PARAMETERS} \\
\hline
\{$I_{D} = f(Area, \beta, \lambda, V_{T0}, \alpha)$\}\\
\hline
\end{tabular}
\caption{I/V dependencies. \label{ji/v}}
\end{figure}

\underline{\sl\large Capacitances}\\[0.1in]
The drain-source capacitance
\begin{equation}
C_{DS} = Area\,C'_{DS}
\end{equation}
The gate-source capacitance
\begin{equation}
C_{GS} = \left\{ \begin{array}{ll}
         Area\,C'_{GS}\left(1 - {{\textstyle V_{GS}}\over{\textstyle \phi_J}}
         \right)^{-M}
         & V_{GS} \le F_C \phi_J\\
         Area\,C'_{GS}\left(1 -F_C\right)^{-(1+M)}
         \left[1-F_C(1+M)+M {{\textstyle V_{GS}}\over{\textstyle \phi_J}}
         \right]^{-M}
         & V_{GS} > F_C \phi_J
         \end{array} \right. %}
\end{equation}
models charge storage at the gate-source depletion layer.
The gate-drain capacitance
\begin{equation}
C_{GD} = \left\{ \begin{array}{ll}
         Area\,C'_{GD}\left(1 - {{\textstyle V_{GD}}\over{\textstyle \phi_J}}
         \right)^{-M}
         & V_{GD} \le F_C \phi_J\\
         Area\,C'_{GD}\left(1 -F_C\right)^{-(1+M)}
         \left[1-F_C(1+M)+M {{\textstyle V_{GD}}\over{\textstyle \phi_J}}
         \right]^{-M}
         & V_{GD} > F_C \phi_J
         \end{array} \right. %}
\end{equation}
models charge storage at the gate-drain depletion layer.
The capacitance parameter dependencies are summarized in figure
\ref{jcap}.\\[0.2in]
\begin{figure}[hbp]
\begin{tabular}[t]{|p{1in}|}
\hline
\multicolumn{1}{|c|}{PROCESS} \\
\multicolumn{1}{|c|}{PARAMETERS} \\
\hline
{\tt FC} \hfill $F_C$\\
{\tt M} \hfill $M$\\
{\tt PB} \hfill $\phi_J$\\
\hline
\end{tabular}
\hfill
\parbox{0.1in}{\ \vspace*{0.2in}\newline +}
\hfill
\begin{tabular}[t]{|p{1in}|}
\hline
\multicolumn{1}{|c|}{GEOMETRY} \\
\multicolumn{1}{|c|}{PARAMETERS} \\
\hline
\hspace*{\fill} -- \hspace*{\fill} \\
\hline
$Area$\\
\hline
\end{tabular}
\hfill
\parbox{0.1in}{\ \vspace*{0.2in}\newline $\rightarrow$}
\hfill
\begin{tabular}[t]{|p{1.8in}|}
\hline
\multicolumn{1}{|c|}{DEVICE} \\
\multicolumn{1}{|c|}{PARAMETERS} \\
\hline
{\tt CDS}\hfill$C'_{DS}$\\
{\tt CGS}\hfill$C'_{GS}$\\
{\tt CGD}\hfill$C'_{GD}$\\
\{$C_{DS} = f(Area, C'_{DS})$\}\\
\{$C_{GS} =$\newline\hspace*{\fill}$ f(Area, C'_{GS}, FC, M, PB)$\}\\
\{$C_{GD} =$\newline\hspace*{\fill}$ f(Area, C'_{GD}, FC, M, PB)$\}\\
\hline
\end{tabular}
\caption{JFET capacitance dependencies. \label{jcap}}
\end{figure}

\noindent\underline{\sl \large AC Analysis}\\[0.1in]
\index{JFET, AC Analysis}
The AC analysis uses the model of figure  \ref{m.ps} with the capacitor values
evaluated at the \dc\ operating point with
\begin{equation}
g_m = {{\textstyle\partial I_{DS}} \over {\textstyle\partial V_{GS}}}
\end{equation}
and
\begin{equation}
R_{DS} = {{\textstyle\partial I_{DS}} \over {\textstyle\partial V_{DS}}}
\end{equation}
\noindent\underline{\large Noise Analysis}\\[0.1in]
\index{JFET, Noise Model}
\index{JFET, Noise Analysis}
The JFET noise model accounts for thermal noise generated in the
parasitic resistances and shot and flicker noise generated in the
drain source current generator.  The rms (root-mean-square) values of
thermal noise current generators shunting the four parasitic resistance
$R_D$, $R_G$ and $R_S$ are
\begin{eqnarray}
I_{n,D} &=& \sqrt{4kT/R_D}~\mbox{A/}\sqrt{\mbox{Hz}}\\
I_{n,S} &=& \sqrt{4kT/R_S}~\mbox{A/}\sqrt{\mbox{Hz}}
\end{eqnarray}
Shot and flicker noise are modeled by
a noise current generator in series with the drain-source current generator.
The rms value of this noise generator is
\begin{equation}
I_{n,DS} = \left( I_{\ms{SHOT},DS}^2 + I_{\ms{FLICKER},DS}^2
                \right)
\end{equation}
\begin{eqnarray}
I_{\ms{SHOT},DS} &=& \sqrt{4kTg_m\frac{2}{3}} ~~~~\mbox{A/}\sqrt{\mbox{Hz}}
~\mbox{A/}\sqrt{\mbox{Hz}}\\
I_{\ms{FLICKER},DS} &=& \sqrt{{{\textstyle\KF I_{DS}^{\AF}} 
                         \over {\textstyle f }}}
~~~~\mbox{A/}\sqrt{\mbox{Hz}}
\end{eqnarray}
where the transconductance
\begin{equation}
g_m = {{\textstyle\partial I_{DS}} \over {\textstyle\partial V_{GS}}}
\end{equation}
is evaluated at the \dc\ operating point, and $f$ is the analysis frequency.
