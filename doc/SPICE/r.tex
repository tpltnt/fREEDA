\element{R}{Resistor}
\begin{figure}[h]
\centering
\ \pfig{r_spice.ps}
\caption{R --- resistor element.}
\end{figure}

{
\form{ {\tt R}name $N_1$ $N_2$ ResistorValue {\tt IC=}$V_R$\E }
}

\spicethreeform{ {\tt R}name $N_1$ $N_2$ \B ModelName \E ResistorValue
   \B {\tt IC=}$V_R$\E
   \B{\tt L=} Length\E \B{\tt W=} Width\E }

\pspiceform{ {\tt R}name  $N_1$ $N_2$ \B ModelName \E ResistorValue \B \E}

\begin{widelist}
\item[$N_1$] is the positive element node,
\item[$N_2$] is the negative element  node, and
\item[{\tt ModelName}]  is  the optional  model name.
\item[{\it ResistorValue}]  is  the  resistance.
               (Units: F; Required; Symbol: $ResistorValue$;
\item[{\tt IC}] is the optional  initial condition specification
      Using {\tt IC=}$V_C$ is intended for use with the {\tt UIC} option
      on  the  {\tt .TRAN}  line,  when  a transient analysis is desired
      with an initial voltage $V_C$ across the capacitor
      rather than the quiescent operating point.
      Specification of the transient initial conditions using the {\tt .IC}
      statement (see page \pageref{.ICstatement}) is preferred and is more
      convenient.
\end{widelist}
\example{ R1 1 2 12.3MEG }

\modeltype{RES}
{

\modelx{RES}{\spicethree\ Only}{Resistor Model}
\label{RESmodelsp3}

\form{ {\tt .MODEL} ModelName {\tt RES(} \B  \B keyword = value\E  ... \E
{\tt )}}

\example{ .MODEL SMALLRES RES( ) }

\modelkeyword{
{\tt DEFW}    &default device width\sym{W_{\ms{DEF}}}&meters     &1e-6  \X
{\tt NARROW}  &narrowing due to side etching\sym{X_{\ms{NARROW}}}&meters     &0.0\X
	}

The \spicethree\ resistor model is a process model for a monolithicly
fabricated resistor enabling the capacitance to be determined from geometric
information.  If the parameter {\tt W} is not specified on the element
line then $Width$
defaults to {\tt DEFW} = $W_{\ms{DEF}}$. The effective dimensions are reduced
by etching so that the effective length of the capacitor is
\begin{equation}
L_{\ms{EFF}} = Length - X_{\ms{NARROW}}
\end{equation}
and the effective width is
\begin{equation}
W_{\ms{EFF}} = Width - X_{\ms{NARROW}}
\end{equation}
The value of the capacitance
\begin{equation}
{\it CapacitorValue} = C_J L_{\ms{EFF}} W_{\ms{EFF}}
          + C_{J,\ms{SW}} L_{\ms{EFF}} +  W_{\ms{EFF}} + L_{\ms{EFF}} + W_{\ms{EFF}})
\end{equation}
}

\modelx{RES}{\pspice\ Only}{Resistor Model}
\label{RESmodelpspice}
The resistor model consists of  process-related  device
data  that  allow  the  resistance  to  be  calculated  from
geometric information and to be corrected  for  temperature.
The parameters available are:

\keyword{
{\tt R}   & resistance multiplier\sym{R_{\ms{MULTIPLIER}}}&-    &  1  \X
{\tt TC1} & first order temperature coefficient\sym{T_{C1}} &$^{\circ}\mbox{C}^{-1}$    &  0\X
{\tt TC2} & second order temperature coefficient\sym{T_{C2}}
    &$^{\circ}\mbox{C}^{-2}$ &  0\X
{\tt TCE} & exponential temperature coefficient\sym{T_{CE}}
       &$\%/^{\circ}\mbox{C}^{-2}$ &  0\X
  }


The sheet resistance is used with the narrowing parameter and L and W from the
resistor line to determine the nominal resistance by the formula

\begin{equation}
R = \left\{ \begin{array}{ll}
   ResistorValue & \mbox{no model specified}\\ \\
   ResistorValue \,R_{\ms{MULTIPLIER}} \\
   \times \left[ 1 + T_{C1} (T - T_{\ms{{NOM}}}) + T_{C2}(T-T_{\ms{NOM}})^2 \right]
		    & \mbox{model specified and}\\
		    & \mbox{{\tt TCE} not specified}\\ \\
   ResistorValue\,R_{\ms{MULTIPLIER}} 1.01^{\textstyle [T_{CE} (T - T_{\ms{{NOM}}})]}
		    & \mbox{model and {\tt TCE} specified}
    \end{array} \right. %}
\end{equation}\\[0.2in]

\clearpage
\noindent\underline{\sl \large Noise Analysis}\\[0.1in]
\index{MOSFET, Noise Model}
\index{MOSFET, Noise Analysis}
The resistor noise model accounts for thermal noise generated in the resistance.
The rms (root-mean-square) values of
thermal noise current generators shunting the resistance is
\begin{equation}
I_n = \sqrt{4kT/R}~\mbox{A/}\sqrt{\mbox{Hz}}
\end{equation}
