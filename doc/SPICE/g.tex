\element{G}{Voltage-Controlled Current Source}

\begin{figure}[h]
\centering
\ \pfig{g_spice.ps}
\caption{G --- voltage-controlled current source element.}
\end{figure}

\form{
     {\tt G}name $N_{+}$ $N_{-}$ $N_{C+}$ $N_{C-}$ Transconductance\\
     {\tt G}name $N_{+}$ $N_{-}$ {\tt POLY(} D {\tt )} $N_{C+}$ $N_{C-}$
	    PolynomialCoefficients
     }

\pspiceninetytwoform{
     {\tt G}name $N_{+}$ $N_{-}$ $N_{C+}$ $N_{C-}$ Transconductance\\
     {\tt G}name $N_{+}$ $N_{-}$ {\tt POLY(} D {\tt )}
     {\tt (}$N_{C1+}$ $N_{C1-}{\tt)}$
     ... {\tt (}$N_{CD+}$ $N_{CD-}{\tt )}$
	    PolynomialCoefficients
     {\tt G}name $N_{+}$ $N_{-}$ {\tt VALUE= \{} Expression {\tt \}}\\
     {\tt G}name $N_{+}$ $N_{-}$ {\tt TABLE \{}
            Expression {\tt \}=(} TableInput {\tt ,} TableOutput {\tt )} ... \\
     {\tt G}name $N_{+}$ $N_{-}$ {\tt LAPLACE \{}
            Expression {\tt \}=\{} TransformExpression {\tt \}} \\
     {\tt G}name $N_{+}$ $N_{-}$ {\tt FREQ \{}
          Expression {\tt \}=(} Frequency{\tt ,} Magnitude{\tt ,} Phase {\tt )} ...\\
     {\tt G}name $N_{+}$ $N_{-}$ {\tt CHEBYSHEV \{}
          Expression {\tt \}=} Type{\tt ,} CutoffFrequency ... {\tt ,}
	  Phase ...\\
     }


\begin{widelist}
\item[$N_{+}$] is the positive voltage source node.
\item[$N_{-}$] is the negative voltage source node.
\item[$N_{C+}$] is the positive controlling node.
\item[$N_{C-}$] is the negative controlling  node.
\item[{\it Transconductance}] is the transconductance.
\item[{\tt POLY}] is the identifier for the polynomial form of the element
\item[{\it D}] is the degree of the poynomial. The number of pairs of
           controlling nodes must be equal to {\it Degree}.
\item[$N_{Ci+}$] the positive node of the $i$ th controlling node pair.
\item[$N_{Ci-}$] the negative node of the $i$ th controlling node pair.
\item[{\it PolynomialCoefficients}] is the set of polynomial coefficients
which must be specified in the standard polynomial coefficient format
discussed on page \pageref{section:poly}.
\item[{\tt VALUE}]
is the identifier for the \underline{value form} of the element.
\item[{\it Expression}] This is an expression of the form discussed on
page \pageref{section:poly}.
\item[{\tt TABLE}]
is the identifier for the \underline{table form} of the element.
\item[{\it TableInput}] This is the independent input of the table.
See the {\tt TABLE} parameter above.
\item[{\it TableInput}] This is the dependent output of the table.
See the {\tt TABLE} parameter above.
\item[{\tt LAPLACE}]
is the identifier for the \underline{laplace form} of the element.
\item[{\it TransformExpression}]
\item[{\tt FREQ}]
is the identifier for the \underline{frequency form} of the element.
\item[{\it Frequency}]
\item[{\it Magnitude}]
\item[{\it Phase}]
\item[{\tt CHEBYSHEV}]
is the identifier for the \underline{chebyshev form} of the element.
\item[{\it Type}]
\item[{\it CutoffFrequency}]
\item[{\it Phase}]
\end{widelist}

\example{G1 2 3 14 1 2.0}

\note{
\item Several form of the voltage-controlled voltage source element are supported
      in addition to the Linear Transconductance
      form which is the default.
      The other forms are selected based on the the identifier
      {\tt POLY},
      {\tt VALUE},
      {\tt TABLE},
      {\tt LAPLACE},
      {\tt FREQ} or
      {\tt CHEBYSHEV}.
      }
\noindent\underline{Linear Transconductance Instance}
\\[0.1in]\hspace*{\fill}\offsetparbox{\it
     {\tt G}name $N_{+}$ $N_{-}$ $N_{C+}$ $N_{C-}$ Transconductance }\\[0.1in]
The value of the voltage generator is linearly proportional to the controlling
voltage:
\begin{equation}
v_o = Transconductance\,v_c
\end{equation}
\\[0.2in]\noindent\underline{{\tt POLY}nomial Instance}
\\[0.1in]\hspace*{\fill}\offsetparbox{\it
     {\tt G}name $N_{+}$ $N_{-}$ {\tt POLY(} D {\tt )}
     {\tt (}$N_{C1+}$ $N_{C1-}{\tt)}$
     ... {\tt (}$N_{CD+}$ $N_{CD-}{\tt )}$ PolynomialCoefficients }\\[0.1in]
The value of the voltage generator is a polynomial function of the controlling
voltages:
\begin{equation}
v_o = f(v_{c1}, ...,  v_{ci}, ...  v_{cD})
\end{equation}
where the number of controlling voltages is $D$ --- the degree of the polynomial
specified on the element line.
$v_{ci}$ is the $i$th controlling voltage and is the voltage of the
$n_{ci+}$ node with respect to the $n_{ci+}$ node.
\\[0.2in]\noindent\underline{{\tt VALUE} Instance} --- \pspice92 only
\\[0.1in]\hspace*{\fill}\offsetparbox{\it
     {\tt G}name $N_{+}$ $N_{-}$ {\tt VALUE= \{} Expression {\tt \}} }\\[0.1in]
The value of the voltage generator is the resultant of an expression evaluation.
\begin{equation}
v_o = f(v_c)
\end{equation}
\\[0.2in]\noindent\underline{{\tt TABLE} Instance} --- \pspice92 only
\\[0.1in]\hspace*{\fill}\offsetparbox{\it
     {\tt G}name $N_{+}$ $N_{-}$ {\tt TABLE \{}
          Expression {\tt \}=(} TableInput {\tt ,} TableOutput {\tt )} ...}\\[0.1in]
\begin{equation}
v_o = f(v_c)
\end{equation}
\\[0.2in]\noindent\underline{{\tt LAPLACE} Instance} --- \pspice92 only
\\[0.1in]\hspace*{\fill}\offsetparbox{\it
     {\tt G}name $N_{+}$ $N_{-}$ {\tt LAPLACE \{}
            Expression {\tt \}=\{} TransformExpression {\tt \}} }\\[0.1in]
\begin{equation}
v_o = f(v_c)
\end{equation}
\\[0.2in]\noindent\underline{{\tt FREQ}} --- \pspice92 only
\\[0.1in]\hspace*{\fill}\offsetparbox{\it
     {\tt G}name $N_{+}$ $N_{-}$ {\tt FREQ \{}
       Expression {\tt \}=(} Frequency{\tt ,} Magnitude{\tt ,} Phase {\tt )} ...}
       \\[0.1in]
\begin{equation}
v_o = f(v_c)
\end{equation}
\\[0.2in]\noindent\underline{{\tt CHEBYSHEV}} --- \pspice92 only
\\[0.1in]\hspace*{\fill}\offsetparbox{\it
     {\tt G}name $N_{+}$ $N_{-}$ {\tt CHEBYSHEV \{}
          Expression {\tt \}=} Type{\tt ,} CutoffFrequency ... {\tt ,}
	  Phase ...  }\\[0.1in]
