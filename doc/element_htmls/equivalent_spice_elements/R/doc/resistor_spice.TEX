%\documentclass{article}
%\usepackage{epsf}
%\newcommand{\fig}[1]{J:/eos.ncsu.edu/users/m/mbs/mbs_group/figures/#1}
%\newcommand{\fig}[1]{../figures/#1}
%\newcommand{\pfig}[1]{\epsfbox{\fig{#1}}}

\oddsidemargin 10mm \topmargin 0.0in \textwidth 5.5in \textheight 7.375in
\evensidemargin 1.0in \headheight 0.18in \footskip 0.16in
%%%%%%%%%%%%%%%%%%%%%%%%%%%%%%%%%%%%%%%% The title
%\begin{document}
\section[R \- Resistor]{\noindent{\LARGE \textbf{Resistor}} \hspace{107mm}\huge \textbf{R}}
%\newline
\linethickness{1mm}
\line(1,0){425}
\normalsize
%\newline
%%%%%%%%%%%%%%%%%%%%%%%%%%%%%%%%%%%%%%%% the resistor figure
\begin{figure}[h]
\centerline{\epsfxsize=0.8in\pfig{r_spice.ps}} \caption{R ---
Resistor Element.}
\end{figure}
%\newline
%%%%%%%%%%%%%%%%%%%%%%%%%%%%%%%%%%%%%%%%%%% SPICE form
%\vspace{2mm}
\linethickness{0.5mm}
\line(1,0){425}
\newline
\texttt{SPICE} \textit{Form:}
\newline
\texttt{R}\textit{name} $n_1$ $n_2$
\texttt{[}\textit{ModelName}\texttt{]} \textit{ResistorValue}
[\texttt{IC}=$V_R$]
\newline
%\vspace{2mm}
%%%%%%%%%%%%%%%%%%%%%%%%%%%%%%%%%%%%%%%%%%%%%%% explanation of terms in the SPICE form
%\newline
\begin{tabular}{r l}
$n_1$ & is the positive element node, \\
$n_2$ & is the negative element node, \\
\textit{ModelName} & is the optional model name,(\textit{ModelType} is \texttt{RES}.)\\
%\end{tabular}
\textit{ResistorValue}&(Units: ohms; Required)(Corresponds to
\texttt{r} parameter in the \FDA netlist)\\
%\begin{tabular}{r l}
\texttt{IC}  & is the optional initial condition specification. Using \texttt{IC}=$V_R$ is used with \\
             & the \textit{UIC} option on the .\textit{TRAN} line when a transient analysis is desired  \\
             & with initial voltage $V_R$ across the capacitor rather than the quiescent  \\
             & operating point. Specification of  the transient initial condition using the  \\
             & .\texttt{IC} is preferred and is more convenient.
\end{tabular}
%\newline
\textit{Model Parameters:}
\newline
%%%%%%%%%%%%%%%%%%%%%%%%%%%%%%%%%%%%%%%%%%%%%% Parameters

\begin{tabular}{|c|c|c|c|}
\hline
\textbf{Name} & \textbf{Description} & \textbf{Units} & \textbf{Default} \\
\hline
R & Resistor Value & ohms & - \\
\hline
\end{tabular}
\newline
\linethickness{0.5mm} \line(1,0){425}
\newline
%%%%%%%%%%%%%%%%%%%%%%%%%%%%%%%%%%%%%%%%%%%%%%%%%%%%%%%%%%%%%%%%%%%%% example in SPICE
\textit{Example:}
\newline
\texttt{R \ 1 \ 1 R1 \ 10.2MEG}
\newline
\linethickness{0.5mm} \line(1,0){425}
\newline
\textit{Notes:}\\
The actual element in \FDA is the \texttt{R} element. See
\texttt{R} for full documentation. \\
%%%%%%%%%%%%%%%%%%%%%%%%%%%%%%%%%%%%%%%%%%%%% Credits
\linethickness{0.5mm} \line(1,0){425}
\newline
\textit{Credits:}
\newline
\begin{tabular}{l l l l}
Name & Affiliation & Date & Links \\
Carlos E. Christofferson & NC State University & Sept 2000 & \epsfxsize=1in\pfig{logo.eps} \\
cechrist@ieee.org & & & www.ncsu.edu    \\
\end{tabular}
%\end{document}
