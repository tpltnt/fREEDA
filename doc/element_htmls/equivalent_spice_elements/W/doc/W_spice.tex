%\documentclass{article}
%\usepackage{epsf,html}
%\newcommand{\fig}[1]{J:/eos.ncsu.edu/users/m/mbs/mbs_group/figures/#1}
%\newcommand{\fig}[1]{../figures/#1}
%\newcommand{\pfig}[1]{\epsfbox{\fig{#1}}}
\oddsidemargin 10mm \topmargin 0.0in \textwidth 5.5in \textheight
7.375in \evensidemargin 1.0in \headheight 0.18in \footskip 0.16in
%%%%%%%%%%%%%%%%%%%%%%%%%%%%%%%%%%%%%%%% The title
\section[W \- Current Controlled Switch]{\noindent{\LARGE \textbf{Current Controlled Switch}}
\hspace{50mm}\huge\textbf{W}} \linethickness{1mm}\line(1,0){425}
\normalsize
%%%%%%%%%%%%%%%%%%%%%%%%%%%%%%%%%%%%%%%% the resistor figure
\begin{figure}[h]
\centerline{\epsfxsize=2.4in\pfig{w_spice.ps}} \caption{W ---
current controlled switch.\label{fig:port}}
\end{figure}
%\newline
%%%%%%%%%%%%%%%%%%%%%%%%%%%%%%%%%%%%%%%%%%% SPICE form
%\vspace{2mm}
\newline
\linethickness{0.5mm} \line(1,0){425}
\newline
\texttt{SPICE} \textit{Form:}
\newline
{\tt W}name $N_1$ $N_2$ {\it VoltageSourceName ModelName} \B ON\E
\B OFF \E
\newline
%%%%%%%%%%%%%%%%%%%%%%%%%%%%%%%%%%%%%%%%%%%%%%% explanation of terms in the SPICE form
\newline
\begin{tabular}{r l}
$N_{+}$ & is the positive node of the switch.\\
$N_{-}$ & is the negative node of the switch.\\
{\it VoltageSourceName} & is the name of the voltage source the\\
& current through which is the controlling current. The voltage\\
& source must be a {\tt V} element.\\
\notforsspice{{\tt ON} & is the optional initial condition. It
is\\
& intended for use with the {\tt UIC} option on  the  {\tt
.TRAN}\\
& line, when  a transient analysis is desired starting from
other\\
& than the quiescent operating point. It is also the initial\\
& condition on the device for \dc\ analysis.\\
{\tt OFF} & is the optional initial condition. If specified the\\
& \dc\ operating point is calculated with the terminal voltages\\
& set to zero.  Once convergence is obtained, the program\\
& continues to iterate to obtain the exact  value of the
terminal\\
& voltages. The OFF option is used to enforce the solution to\\
& correspond  to a desired  state if the circuit has more than
one\\
& stable state.}
\end{tabular}
%\newline
%%%%%%%%%%%%%%%%%%%%%%%%%%%%%%%%%%%%%%%%%%%%%%% Parameter table
%\vspace{4mm}
%\linethickness{0.5mm} \line(1,0){425}
%\newline
%\textit{Example:}
%\newline
%\texttt{PORT1 1 0 PNR=1 ZL=75}
\newline
\linethickness{0.5mm} \line(1,0){425}
\newline
\textit{Description:}\\
\modeltype{ISWITCH} \eskip{W} \model{ISWITCH}{Current-Controlled
Switch Model}
\begin{figure}[h]
\centerline{\epsfxsize=1.5in\pfig{iswitch.ps}} \caption{ISWITCH
--- current controlled switch model. \label{iswitch}}
\end{figure}

\notforsspice{The current-controlled switch model is supported by
both \spicethree\ and \pspice. However the model keywords differ slightly.\\[0.1in]

\begin{tabular}{|r|l|c|c|}
\hline
\textbf{Name} & \textbf{Description} & \textbf{Units} & \textbf{Default} \\
\hline
{\tt IT}   & threshold current\sym{I_{\ms{ON}}}& A     & 0.0 \\
\hline
{\tt IH}   & hysteresis current\sym{I_{\ms{OFF}}} & A     & 0.0\\
\hline
{\tt RON} & on resistance      \sym{R_{\ms{ON}}}&$\Omega$&1.0 \\
\hline {\tt ROFF} & off resistance
\sym{R_{\ms{OFF}}}&$\Omega$&1/GMIN\X }
\end{tabular}
\newline

Care must be exercised in using the switch.  An instantaneous
switch is highly nonlinear and will very likely lead to
convergence problems. This problem is alleviated in the {\tt
ISWITCH} model by ramping the resistance of the switch from its
off value to its on value.  For this ramping action to be
effective the difference between $I_{\ms{ON}}$ and $I_{\ms{OFF}}$
must not be too small. Also the values of $R_{\ms{ON}}$ and
$R_{\ms{OFF}}$ should not be extreme. The ration
$R_{\ms{ON}}/R_{\ms{OFF}}$ should be be as small as possible.

If $R_{\ms{ON}}/R_{\ms{OFF}}$  is large, e.g.
$R_{\ms{ON}}/R_{\ms{OFF}}$ $>$ $10^{12}$, then the default error
tolerances {\tt TRTOL} and {\tt CHGTOL}, specified in a {\tt
.OPTIONS} statement (see page \pageref{.OPTIONSstatement}) may
need to be changed.
\begin{widelist}
\item[{\tt TRTOL}] Change to 1.0 from 7.0 idf there are convergence problems
during transient analysis.
\item[{\tt CHGTOL}] If a switch is across a capacitor then {\tt CHGTOL} should be
reduced to $10^{-16}$ if there are convergence problems during
transient analysis.
\end{widelist}
\eskip{W}
\noindent\underline{\bf \large Switch Model}\\[0.1in]

The switch is modeled by a current variable resistor $R$, see
figure \ref{iswitch}.

\noindent\underline{\sl \large Standard Calculations}\\[0.1in]
\begin{eqnarray}
R_{\ms{MEAN}} & = & \sqrt{R_{\ms{ON}} + R_{\ms{OFF}}} \\
R_{\ms{RATIO}}& = &       R_{\ms{ON}}/ R_{\ms{OFF}} \\
I_{\ms{MEAN}} & = & \sqrt{I_{\ms{ON}} + I_{\ms{OFF}}} \\
I_{\Delta}& = & \left({{\textstyle i - I_{\ms{MEAN}}} \over
              {\textstyle I_{\ms{ON}}- I_{\ms{OFF}}}}\right)
\end{eqnarray}
If $I_{\ms{ON}}> I_{\ms{OFF}}$ the switch resistance
\begin{equation}
R = \left\{
\begin{array}{ll}
R_{\ms{ON}}                            & i \ge I_{\ms{ON}} \\
R_{\ms{OFF}}                            & i \le I_{\ms{OFF}}\\
R_{\ms{MEAN}}\,
  R_{\ms{RATIO}}^{\textstyle 1.5 I_{\Delta}}\,
  R_{\ms{RATIO}}^{\textstyle 1.5 I_{\Delta}^3}
  & I_{\ms{OFF}} < i < I_{\ms{ON}} \\
\end{array} \right. %}
\end{equation}
If $I_{\ms{ON}}< I_{\ms{OFF}}$ the switch resistance
\begin{equation}
R = \left\{
\begin{array}{ll}
R_{\ms{ON}}                            & i \le I_{\ms{ON}} \\
R_{\ms{OFF}}                            & i \ge I_{\ms{OFF}}\\
R_{\ms{MEAN}}\,
  R_{\ms{RATIO}}^{\textstyle 1.5 I_{\Delta}}\,
  R_{\ms{RATIO}}^{\textstyle 1.5 I_{\Delta}^3}
  & I_{\ms{OFF}} < i < I_{\ms{ON}} \\
\end{array} \right. %}
\end{equation}
\noindent\underline{\sl \large Noise Analysis}\\[0.1in]
\index{ISWITCH, Noise Model} \index{ISWITCH, Noise Analysis}
\marginid{Noise Analysis} The current controlled switch noise
model accounts for thermal noise generated in the switch
resistance. The rms (root-mean-square) values of thermal noise
current generators shunting the switch resistance is
\marginid{Noise Model}
\begin{equation}
I_{n} = \sqrt{4kT/R}~\mbox{A/}\sqrt{\mbox{Hz}}
\end{equation}
where $T$ is the analysis temperature in kelvin (K), and $k$ (=
$1.3806226\,10^{-23}$~J/K) is Boltzmanns constant.
\newline
\linethickness{0.5mm} \line(1,0){425}
\newline
\textit{Notes:}\\
There is no equivalent element in \FDA.
%%%%%%%%%%%%%%%%%%%%%%%%%%%%%%%%%%%%%%%%%%%%% Credits
\newline
\linethickness{0.5mm} \line(1,0){425}
\newline
\textit{Credits:}
\newline
\begin{tabular}{l l l l}
Name & Affiliation & Date & Links \\
Carlos E. Christofferson & NC State University & Sept 2000 & \epsfxsize=1in\pfig{logo.eps} \\
cechrist@ieee.org & & & www.ncsu.edu    \\
\end{tabular}
%\end{document}
