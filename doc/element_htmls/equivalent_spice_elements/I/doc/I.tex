%\documentclass{article}
%\usepackage{epsf}
%\newcommand{\fig}[1]{J:/eos.ncsu.edu/users/m/mbs/mbs_group/figures/#1}
%\newcommand{\fig}[1]{../figures/#1}
%\newcommand{\pfig}[1]{\epsfbox{\fig{#1}}}
%\newcommand{\ms}[1]{\mbox{\scriptsize #1}}
%\newcommand{\B}{{ \rm [}}     % begin optional parameter in \form{}
%\newcommand{\E}{{\ \rm\hspace{-0.04in}] }}   % end optional parameter in \form{}

\oddsidemargin 10mm \topmargin 0.0in \textwidth 5.5in \textheight 7.375in
\evensidemargin 1.0in \headheight 0.18in \footskip 0.16in
%%%%%%%%%%%%%%%%%%%%%%%%%%%%%%%%%%%%%%%% The title
%\begin{document}
\section[I \- Independent Current Source]{\noindent{\LARGE \textbf{Independent Current Source} \hspace{35mm}\huge\textbf{I}}}
%\newline
\linethickness{1mm}
\line(1,0){425}
\normalsize
%%%%%%%%%%%%%%%%%%%%%%%%%%%%%%%%%%%%%%%% the resistor figure
\begin{figure}[h]
\centerline{\epsfxsize=0.5in\pfig{i_spice.ps}} \caption{I ---
independent current source.}
\end{figure}
%%%%%%%%%%%%%%%%%%%%%%%%%%%%%%%%%%%%%%%% form for \FDA
\linethickness{0.5mm} \line(1,0){425}
\newline
\textit{Form:}
\newline
{\tt I}name $N_{+}$ $N_{-}$ \B \B {\tt DC}\E \B {\it DCvalue}\E
\newline
      {\tt +}\B {\tt AC} \B {\it ACmagnitude} \B {\it ACphase}\E \E \E  \newline
      {\tt +} \B {\tt DISTOF1} \B {\it F1Magnitude} \B {\it F1Phase}\E \E \E
      {\tt +} \B {\tt DISTOF2} \B {\it F2Magnitude} \B {\it F2Phase}\E \E \E
\newline
%%%%%%%%%%%%%%%%%%%%%%%%%%%%%%%%%%%%%%%%
\begin{tabular}{r l}
$N_{+}$ & is the positive current source node.\\
& (Current flow is out of the positive to the negative node.)\\
$N_{-}$ & is the negative current source node.\\
{\tt DC} & is the optional keyword for the \dc\ value of the
source.\\
{\it DCvalue} & is the \dc\ current value of the source.\\
              & (Units: A; Optional; Default: 0; Symbol:
               $I_{DC}$)\\
{\tt AC} & is the keyword for the \ac\ value of the source.\\
{\it ACmagnitude} & is the \ac\ magnitude of the source used\\
& during \ac analysis. That is, it is the peak \ac\ current so\\
& that the \ac\ signal is $\mbox{{\it
ACmagnitude}}\,\mbox{sin}(\omega t +
\mbox{ACphase})$.\\
{\it ACmagnitude} & is ignored for other types of analyses.\\
               & (Units: A; Optional; Default: 1; Symbol:
               $I_{AC}$)\\
{\it ACphase} & is the ac phase of the source. It is used only in
\ac\ analysis.\\
               & (Units: Degrees; Optional; Default: 0; Symbol:
               $\phi_{\ms{AC}}$)\\
\notforsspice{ {\tt DISTOF1} & is the distortion keyword for
distortion component 1 which has frequency {\tt F1}.\\
{\it F1magnitude} & is the magnitude of the distortion component
at {\tt F1}.\\
               & (Units: A; Optional; Default: 1; Symbol:
               $I_{F1}$)\\
{\it F1phase} & is the phase of the distortion component at
{\tt F1}. \\
& (Units: Degrees; Optional; Default: 0; Symbol:$\phi_{F1}$)\\
{\tt DISTOF2} & is the distortion keyword for distortion component
2 which has frequency {\tt F2}.\\
{\it F2magnitude} & is the magnitude of the distortion component
at {\tt F2}.\\
& (Units: A; Optional; Default: 1; Symbol:$I_{F2}$)\\
{\it F2phase} & is the phase of the distortion component at {\tt
F2}.\\
               & (Units: Degrees; Optional; Default: 0; Symbol:$\phi_{F2}$)\\
}\\
{\it TransientSpecification} & is the optional transient
specification described more fully below.
\end{tabular}
%%%%%%%%%%%%%%%%%%%%%%%%%%%%%%%%%%%%%%% Parameter list
\newline
\note{
\item
The independent current source has three different sets of
parameters to describe the source for DC analysis (see {\tt .DC}
on page \pageref{.DCstatement}), AC analysis (see {\tt .AC} on
page \pageref{.ACstatement}), and transient analysis (see {\tt
.TRAN} on page \pageref{.TRANstatement}). The \dc\ value of the
source is used during bias point evaluation and \dc\ analysis is
{\it DCValue}. It is also the constant value of the current source
if no {\it TransientSpecification} is supplied. It may also be
used in conjunction with the {\tt PWL} transient specification if
a time zero value is not provided as part of the transient
specification. The \ac\ specification, indicated by the keyword
{\tt AC} is independent of the \dc\ parameters and the {\it
Transient Specification}.

\notforsspice{\item The original documentation distributed with
\spicetwo\ and \spicethree\ incorrectly stated that if a {\it
TransientSpecification} was supplied then the time-zero transient
current was used in \dc\ analysis and in determiniong the
operating point.} }

\noindent{\large \bf  Transient Specification}

Five transient specification forms are supported: pulse ({\tt
PULSE}),  exponential ({\tt EXP}),  sinusoidal ({\tt SIN}),
piece-wise  linear ({\tt PWL}),   and single-frequency FM ({\tt
SFFM}).  The default values of some of the parameters of these
transient specifications include {\tt TSTEP} which is the printing
increment and {\tt TSTOP} which is the final time (see the {\tt
.TRAN} statement on page \pageref{.TRANstatement} for further
explanation of these quantities). In the following $t$ is the
transient analysis time.
\newline

\underline{\bf{Sinusoidal}}:\\
%%%%%%%%%%%%%%%%%%%%%%put the SIN form here%%%%%%%%%%%%%%%%%%%%
\texttt{SIN( $I_O$ $I_A$ \B $F$ \E \B $T_D$ \E \B $\theta$ \E {\tt )}}\\
\textit{Parameters:}
\begin{table}[h]
\begin{tabular}{|c|c|c|c|}
\hline
Name&Description&Units&Default\\
\hline
$I_O$ & voltage offset & V & \scriptsize{REQUIRED}\\
\hline
$I_A$ & voltage amplitude & V & \scriptsize{REQUIRED}\\
\hline
$F$ & frequency & Hz & 1/{\texttt{TSTOP}}\\
\hline
$T_D$ & time delay & s & 0\\
\hline
$\Theta$ & damping factor & 1/s & 0\\
\hline
$\phi$ & phase & degree & 0\\
\par
\hline
\end{tabular}
\end{table}
%%%%%%%%%%%%%%%%%%%%%%%%%%%%%%%%%%%%%%% example in \FDA
\newline
\linethickness{0.5mm} \line(1,0){425}
\newline
\textit{Example:}
\newline
\texttt{ISIGNAL 20 5 SIN(0.1 0.8 2 1 0.3)}
\newline
\linethickness{0.5mm} \line(1,0){425}
\newline
\textit{Description:}\\
The sinusoidal transient waveform is defined by
\begin{equation}
i = \left\{ \begin{array}{ll}
I_0                         & t \le T_D\\
I_0 + I_1 e^{-[\textstyle (t -T_D)\Theta]} \sin{2\pi[F(t-T_D) +
\phi/360]}
                            & t > T_D
\end{array} \right. %}
\end{equation}
\begin{figure}[h]
\centering
\input{isin}
\caption[Current source transient sine ({\tt SIN})
waveform]{Current source transient sine ({\tt SIN}) waveform
for\newline \hspace*{\fill} {\tt SIN(0.1 0.8 2 1 0.3 )}.
\label{fig:isin} \hspace*{\fill} }
\end{figure}
\newline
%\linethickness{0.5mm} \line(1,0){425}
%\newline
%\textit{Notes:}\\
%The actual element in \FDA is the \texttt{isource} element.
%See \texttt{isource} for full documentation.\\
%%%%%%%%%%%%%%%%%%%%%%%%%%%%%%%%%%%%%%% Parameter list
\newline

\underline{\bf{Exponential}}:\\
%%%%%%%%%%%%%%%%%%%%%%put the SIN form here%%%%%%%%%%%%%%%%%%%%
\texttt{EXP($I_1$ $I_2$ \B $T_{D1}$ \E \B $\tau_1$ \E
       \B $T_{D2}$ \E \B$\tau_2$ \E)}\\
\textit{Parameters:}
\begin{table}[h]
\begin{tabular}{|c|c|c|c|}
\hline
Name&Description&Units&Default\\
\hline
$I_1$& initial voltage & V & \scriptsize{REQUIRED}\\
\hline
$I_2$ & pulsed voltage & V & \scriptsize{REQUIRED}\\
\hline
$T_{D1}$ & rise delay time & s & 0.0\\
\hline
$\tau_1$ & rise time constant & s & {\tt TSTEP}\\
\hline
$T_{D2}$ & fall delay time & s & {\tt TSTEP}\\
\hline
$\tau_2$ & fall time constant & s &  {\tt TSTEP}\\
\par
\hline
\end{tabular}
\end{table}
%%%%%%%%%%%%%%%%%%%%%%%%%%%%%%%%%%%%%%% example in \FDA
\newline
\linethickness{0.5mm} \line(1,0){425}
\newline
\textit{Example:}
\newline
\texttt{ISIGNAL \ 2\ 0\ EXP(0.1 0.8 1 0.35 2 1)}
\newline
\linethickness{0.5mm} \line(1,0){425}
\newline
\textit{Description:}\\
The exponential transient is a single-shot event specifying two
exponentials. The current is $I_1$ for the first $T_{D1}$ seconds
at which it begins increasing exponentially towards $I_2$ with a
time constant of $\tau_1$ seconds.  At time $T_{D2}$ the current
exponentially decays towards $I_1$ with a time constant of
$\tau_2$. That is,
\begin{equation}
i = \left\{ \begin{array}{ll}
     I_1                                           & t \le T_{D1}\\
     I_1+(I_2-I_1)(1-e^{\textstyle (-(t-T_{D1})/\tau_1)})  & T_{D1} < t \le T_{D2}\\
     I_1+(I_2-I_1)(1-e^{\textstyle (-(t-T_{D1})/\tau_1)})
        +(I_1-I_2)(1-e^{\textstyle (-(t-T_{D2})/\tau_2)})  &  t > T_{D2}
     \end{array} \right. %}
\end{equation}
\vspace*{-0.2in}
\begin{figure}[h]
\centering
\documentclass{article}
\usepackage{epsf}\usepackage{here}
% SUMMARY OF USEFUL MACROS
%
% 1. \marginpar[LeftText]{RightText} Standard Latex \marginpar
%
% 2. \mymarginpar{MarginText}{BodyText} Places MarginText in margin and
%     BodyText in main text opposite margin. Places lineas above and below
%     text.  Used in element and model catalogs and elsewhere.
%
% 3. \marginlabel{text} Places large text in margin and underlines. Used
%    to put element name in margin at top of page for continued
%    descriptions.  Does not automaticly put it at the top of a page
%    and so a \clearpage is required.
%
% 4. \marginid{text} Used to place a short identifier in the margin. Just one
%    word and justified to the outside edge.
%
% 5. \offset a standard indent.
%
% 6. \offsetparbox{} Places text in an offset parbox.  Use
%    \hspace*{\fill} \offsetparbox{text}        to insert an offset text.
%
% 7. \boxed{} similar to above but starts a new line and right justifies
%    offsetparbox.
%
% 8. \form
%
% 9. \example
%
%10. \begin{widelist}  .... \end{widelist} 
%    A widelist that has 1 inch wide labels that has additional left hand
%    margin of 0.6 inch.
%
%11. \spicethreeonly{} Text is inserted only if output is for spice3.
%
%12. \pspiceonly{} 
%    Text is inserted only if output is for pspice (superspice for the most
%    part is upwards compatible to pspice).
%
%13. \sspiceonly{} 
%    Text is inserted only if output is for superspice but not for pspice.

%14. \sym{}
%    Right justifies a symbol in a keyword table.
%
%15. Use the following wherever as then we can have a standard way to
%    report them.  Note that "\dc\ " is required to get a space after dc.
%    \dc
%    \ac
%    \SPICE
%
%16. \kwnote{}  This is a convenient way to include notes in a keyword table.
%    It right justifies a note in the description column and starts it on a
%    newline.
\newcommand{\kwnote}[1]{\newline\hspace*{\fill} #1}

%
%17. \kwversion{} This is a convenient way to indicate versions in a keyword
%    table. It does a \kwnote .  It should not printed when outputing for
%     specific versions.
% typical usage is \kwversion{\sspice ; \hspice} to produce the output
%            VERSIONS: SUPERSPICE; HSPICE
% typical usage is \kwversionNote{\sspice ; \hspice} to produce the output
%                 |                    VERSIONS: SUPERSPICE; HSPICE |
%for unifomity the recommended version order is:
%    \hspice \pspice \sspice \spicetwo \spicethree
\newcommand{\version}[1]{({\sc version: #1})}
\newcommand{\kwversion}[1]{\newline({\sc version: #1})}
\newcommand{\kwversionNote}[1]{\kwnote{({\sc version: #1})}}
\newcommand{\versions}[1]{({\sc versions: #1})}
\newcommand{\kwversions}[1]{\newline({\sc versions: #1})}
\newcommand{\kwversionsNote}[1]{\kwnote{({\sc versions: #1})}}

%


\newcommand{\mymargin}{1in}   % Use to set width of margin notes
\newcommand{\mymarginparsep}{0.1in}
\newcommand{\mymarginparsepplustext}{5.6in}
\newcommand{\mymarginplus}{1.1in}
\newcommand{\mymarginplustext}{6.6in}
\newcommand{\underlinehead}{\\[-0.1in] \rule{\mymarginplustext}{0.01in}}
\newcommand{\overlinefoot}{\rule{\mymarginplustext}{0.01in}\\}
% WIDEPARBOX uses margin space as well.
\newcommand{\wideparbox}[1]{\parbox[t]{\mymarginplustext}{#1}}
%
% HEADER AND FOOT
% Standard header and foot, but includes copyright notice.a
% To omit copyrite notice do not include copyrite.sty
%
\newcommand{\mycopyrite}{}
\def\ps@headings{\let\@mkboth\markboth
\def\@oddfoot{\wideparbox{\overlinefoot\rm\today \hfill \mycopyrite}}
\def\@evenfoot{\hspace*{-\mymarginplus}\wideparbox{\overlinefoot
\mycopyrite\hfill \today}}
\def\@evenhead{\hspace*{-\mymarginplus}\wideparbox{\rm
\thepage\hfill \sl \leftmark \underlinehead }}
\def\@oddhead{\wideparbox{\hbox{}\sl \rightmark \hfill
\rm\thepage \underlinehead}}\def\chaptermark##1{\markboth {\uppercase{\ifnum
\c@secnumdepth
>\m@ne
 \@chapapp\ \thechapter. \ \fi ##1}}{}}\def\sectionmark##1{\markright
 {\uppercase{\ifnum \c@secnumdepth >\z@
  \thesection. \ \fi ##1}}}}

%\oddsidemargin 0.25in \topmargin 0.0in \textwidth 6.5in \textheight 9in
%\evensidemargin 0.25in \headheight 0.18in \footskip 0.16in

%\input{epsf}
\usepackage{epsf}
\oddsidemargin 0.25in \evensidemargin 0.25in
\topmargin 0.0in
\textwidth 6.5in \textheight 9in
\headheight 0.18in \footskip 0.16in
\leftmargin -0.5in \rightmargin -0.5in

\newcommand{\fig}[1]{figures/#1}
\newcommand{\pfig}[1]{\epsfbox{\fig{#1}}}
\newcommand{\newfig}[1]{\epsffile{\fig{#1}}}

\newcommand{\fdaelement}[1]{elements/#1}
\newcommand{\spiceelement}[1]{equivalent_spice_elements/#1}

\newcommand{\FREEDA}{{\Huge{\textsl{\textsf{f}}}${\mathsf{REEDA}}^{{\tiny{\textsf{TM}}}}$}}
\newcommand{\FDA}{{\textsl{\textsf{f}}}${\mathsf{REEDA}}^{{\tiny{\textsf{TM}}}}$}
\newcommand{\notforsspice}[1]{#1}
\newcommand{\spicetwoonly}[1]{#1}
\newcommand{\spicethreeonly}[1]{#1}
\newcommand{\pspiceonly}[1]{#1}
\newcommand{\pspiceninetytwoonly}[1]{#1}
\newcommand{\sspiceonly}[1]{}
\newcommand{\fornutmeg}[1]{}
\newcommand{\sspicetwoonly}[1]{}

%%%%%%%%%%%%%%%%%%%%%%%%%%%%%%%%%%%%%%%%%%%%%%%%%%%%%%%%%%%%%%%%%%%%%%%%%%%%%%%%
\marginparwidth=\mymargin
\marginparsep=\mymarginparsep
\newcommand{\textplusmarginparwidth}{\textwidth+\marginparsep+\mymargin}
\newcommand{\X}{\\ \hline}    % line terminataion in keyword environment
\newcommand{\B}{{ \rm [}}     % begin optional parameter in \form{}
\newcommand{\E}{{\ \rm\hspace{-0.04in}] }}   % end optional parameter in \form{}
\newcommand{\expr}{{\sc Expressions supported}}
\newcommand{\reqd}{{\scriptsize REQUIRED}}
\newcommand{\omitted}{{\scriptsize OMITTED}}
\newcommand{\inferred}{{\scriptsize INFERRED}}
\newcommand{\para}{\newline{\scriptsize (PARASITIC)}}
\newcommand{\opt}{{\tiny  OPTIONAL}}

\newcommand{\Spice}{{\sc Spice}}
\newcommand{\spice}{{\sc Spice}}
\newcommand{\justspice}{{\sc Spice}}
\newcommand{\nutmeg}{{\sc NUTMEG}}
\newcommand{\probe}{{\sc Probe}}

\newcommand{\accusim}{{\sc AccuSim}}
\newcommand{\contec}{{\sc ContecSpice}}
\newcommand{\cdsspice}{{\sc CDS Spice}}
\newcommand{\hpimpulse}{{\sc HP Impulse}}
\newcommand{\hspice}{{\sc HSpice}}
\newcommand{\igspice}{{\sc IG\_SPICE}}
\newcommand{\isspice}{{\sc IsSpice}}
\newcommand{\mspice}{{\sc Microwave Spice}}
\newcommand{\pspice}{{\sc PSpice}}
\newcommand{\justpspice}{{\sc PSpice}}
\newcommand{\spectre}{{\sc Spectre}}
\newcommand{\spicetwo}{{\sc Spice2g6}}
\newcommand{\spicethree}{{\sc Spice3}}
\newcommand{\spiceplus}{{\sc SpicePlus}}
\newcommand{\sspice}{{\sc SuperSpice}}

\newcommand{\modelversion}[1]{& #1}
%%%%%%%%%%%%%%%%%%%%%%%%%%%%%%%%%%%%%%%%%%%%%%%%%%%%%%%%%%%%%%%%%%%%%%%%%%%%%%%%
% set up a counter for all occasions
%
\newcounter{count}

% set up new commands
%
\newcommand{\vshift}{\vspace{0.2in}}
% OFFSET
\newcommand{\offset}{\hspace*{0.45in} }
% OFFSETPARBOXWIDTH argument should be \textwith - offset
\newcommand{\offsetparbox}[1]{\parbox[t]{5in}{#1}}

% note: no labeling
\newcommand{\elementx}[3]{\clearpage\rm\markright{#3:#2}
\addcontentsline{toc}{section}{#1, #2: #3}
\mymarginparx{#1}{#2}{#3}\index{#1:#2}\index{#3:#2}}

% macros for sub elements
\newcommand{\subelement}[2]{\clearpage 
   \noindent\rule{\textwidth /2}{.5mm} \\[0.1in]
   {\large \bf #1} \hspace{0.2in} {\bf #2} \\
   \noindent\rule{\textwidth /2}{.5mm} \index{#1} \index{#2}}
% macros for models
\newcommand{\model}[2]{{
   \noindent\vspace{0.2in}\parbox{\textwidth}{
   \noindent\rule{\textwidth}{.5mm} \\[0.1in]
   {\large \bf #1 Model} \label{#1model} \hfill {\large #2} \\
   \noindent\rule{\textwidth}{.5mm} \index{#1} \index{#2}}}}
\newcommand{\modelx}[3]{{
   \noindent\vspace{0.2in}\parbox{\textwidth}{
   \noindent\rule{\textwidth}{.5mm} \\[0.1in]
   {\large \bf #1 Model} \hfill {\large #2} \hfill {\large #3} \\
   \noindent\rule{\textwidth}{.5mm} \index{#1} \index{#2}}}}



% BOXED
\newcommand{\boxed}[1]{\noindent
\newline \vshift \hspace*{\fill} {\tt \offsetparbox{\tt #1}}
 \vshift}


%
% KEYWORD
%
\newcommand{\keywordtable}[1]{
        \sloppy
        \hyphenation{ca-pac-i-t-an-ce} 
        \begin{center}
    \sf
        \begin{tabular}[t]
        {|p{0.58in}|p{3.07in}|p{0.55in}|p{0.60in}|}
        \hline
        \multicolumn{1}{|c}{\bf Name} &
        \multicolumn{1}{|c}{\parbox{2.77in}{\bf Description}}  &
        \multicolumn{1}{|c}{\bf Units} &
        \multicolumn{1}{|c|}{\bf Default} \X
        #1
        \end{tabular}
        \end{center}
    }

\newcommand{\keywordtwotable}[2]{
        \sloppy
        \hyphenation{ca-pac-i-t-an-ce} 
        \begin{center}
    \sf
        \begin{tabular}[t]
        {|p{0.58in}|p{2.38in}|p{0.55in}|p{0.60in}|p{0.53in}|}
        \hline
        \multicolumn{1}{|c}{\bf Name} &
        \multicolumn{1}{|c}{\parbox{2.20in}{\bf Description}}  &
        \multicolumn{1}{|c}{\bf Units} &
        \multicolumn{1}{|c}{\bf Default} &
        \multicolumn{1}{|c|}{\bf #1} \X
        #2
        \end{tabular}
        \end{center}
    }

\newcommand{\kw}[2]{
     \samepage{
     \noindent {\sl #1} \vspace{-0.5in} \\
     \keywordtable{#2} }}

\newcommand{\kwtwo}[3]{
     \samepage{
     \noindent {\sl #1} \vspace{-0.4in} \\
     \keywordtwotable{#2}{#3} }}

\newcommand{\keyword}[1]{\kw{Keywords:}{#1}}
\newcommand{\keywordtwo}[2]{\kwtwo{Keywords:}{#1}{#2}}
\newcommand{\modelkeyword}[1]{\kw{Model Keywords}{#1}}
\newcommand{\modelkeywordtwo}[2]{\kwtwo{Model Keywords}{#1}{#2}}

\newcommand{\myline}{\\[-0.1in]
\noindent \rule{\textwidth}{0.01in} \newline}

\newcommand{\myThickLine}{\\[-0.1in]
\noindent \rule{\textwidth}{0.02in} \newline}


% SPICE 2G6 FORM
%\newcommand{\spicetwoform}[1]{
%\spicetwoonly{\samepage{\noindent{\sl\spicetwo\form{#1}}}}}

% PSPICE88 FORM
%\newcommand{\pspiceform}[1]{
%\pspiceonly{\samepage{\noindent{\sl\pspice}\form{#1}}}}

% PSPICE92 FORM
%\newcommand{\pspiceninetytwoform}[1]{
%\pspiceninetytwoonly{\samepage{\noindent{\sl\pspice92\form{#1}}}}}


% SPICE3E2 FORM
%\newcommand{\spicethreeform}[1]{
%\spicethreeonly{\samepage{\noindent{\sl\spicethree\form{#1}}}}}

% FORM
\newcommand{\form}[1]{\samepage{\noindent
 {\sl Form} \myline
% \hspace*{\fill} % For some reason \fill = 0 when \pspiceform{} is used?
\offset
\it  \offsetparbox{#1}}
\\[0.1in]}

% ELEMENT FORM
\newcommand{\elementform}[1]{\samepage{\noindent
 {\sl Element Form} \myline
% \hspace*{\fill} % For some reason \fill = 0 when \pspiceform{} is used?
\offset
\it  \offsetparbox{#1}}
\\[0.1in]}

% MODEL FORM
\newcommand{\modelform}[1]{\samepage{\noindent
 {\sl Model Form} \myline
% \hspace*{\fill} % For some reason \fill = 0 when \pspiceform{} is used?
\offset
\it  \offsetparbox{#1}}
\\[0.1in]}

% LIMITS
\newcommand{\mylimits}[1]{\samepage{\noindent
 {\sl Limits} \myline
 \hspace*{\fill} \it  \offsetparbox{#1}}
 \vshift}

% EXAMPLE
\newcommand{\example}[1]{\samepage{\noindent
{\sl Example} \myline
\offset \tt  \offsetparbox{#1}}
 \vshift}

% PSPICE88 EXAMPLE
\newcommand{\pspiceexample}[1]{\samepage{\noindent
{\sl \pspice\ Example} \myline
\offset \tt  \offsetparbox{#1}}
 \vshift}

% MODEL TYPES
\newcommand{\modeltype}[1]{\samepage{\noindent
{\sl Model Type} \myline
 \hspace*{\fill} \tt \offsetparbox{#1}}
 \\[0.1in]}

% MODEL TYPES
\newcommand{\modeltypes}[1]{\samepage{\noindent
{\sl Model Types:} \myline
 \hspace*{\fill} \tt \offsetparbox{\tt #1}}
 \vshift}

% OFFSET ENUMERATE
\newcommand{\offsetenumerate}[1]{
     \offset \hspace*{-0.1in} {\begin{enumerate} #1 \end{enumerate}}}

% NOTE
\newcommand{\note}[1]{
\vshift\samepage{\noindent {\sl Note}\myline\vspace{-0.24in}}
 \offsetenumerate{#1} }

% SPECIAL NOTE
\newcommand{\specialnote}[2]{
\vshift\samepage{\noindent {\sl #1}\myline\vspace{-0.24in}}\\#2}

\newcommand{\dc}{\mbox{\tt DC}}
\newcommand{\ac}{\mbox{\tt AC}}
\newcommand{\SPICE}{\mbox{\tt SPICE}}
\newcommand{\m}[1]{{\bf #1}}                           % matrix command  \m{}

% ////// Changing nodes to terminals///////
% print terminals in \tt and enclose in a circle use outside
\newcommand{\terminal}[1]{\: \mbox{\tt #1} \!\!\!\! \bigcirc }
%
% set up environment for example
%
\newcounter{excount}
\newcounter{dummy}
\newenvironment{eg}{\vspace{0.1in}\noindent\rule{\textwidth}{.5mm}
   \begin{list}
   {{\addtocounter{excount}{1}
   \em Example\/ \arabic{chapter}.\arabic{excount}\/}:}
   {\usecounter{dummy}
   \setlength{\rightmargin}{\leftmargin}}
   }{\end{list} \rule{\textwidth}{.5mm}\vspace{0.1in}}
%
% set up environment for block
% currently this draws a horizontal line at the start of block and another
% at the end of block.
%
\newenvironment{block}{\vspace{0.1in}\noindent\rule{\textwidth}{.5mm}
   }{\rule{\textwidth}{.5mm}\vspace{0.1in}}
%

%
% Macros for element summaries
%
% macros for element summary
%\newcommand{\summaryelement}[2]{
%   \vspace{0.4in}
%   \mymarginpar{#1}{#2} 
%   \addcontentsline{toc}{section}{#1, #2}
%   \vspace{-0.6in} \\
%   \noindent Full description on page \pageref{#1element}. \vshift\\
%   }

% Summary MODEL TYPE
%\newcommand{\summarymodeltype}[1]{\samepage{\noindent
%{\sl Model Type} \myline
% \hspace*{\fill} \tt \offsetparbox{\tt #1
% \hfill Summary on \pageref{#1summary} \index{#1}}}
% }

% Summary MODEL TYPES
%\newcommand{\summarymodeltypes}[1]{\samepage{\noindent
%{\sl Model Types} \myline
% \hspace*{\fill} \tt \offsetparbox{\tt #1
% \hfill Summary on \pageref{#1summary} \index{#1}}}
% }

%
% macros for model summary
%\newcommand{\summarymodel}[2]{\clearpage
%   \addcontentsline{toc}{section}{#1, #2}
%   \mymarginpar{#1}{#2} \label{#1summary}
%   \index{#1} \index{#2}
%   \noindent Full description on page \pageref{#1model} \\[0.1in]
%   }



%
% set up wide descriptive list
%
\newenvironment{widelist}
    {\begin{list}{}{\setlength{\rightmargin}{0in} \setlength{\itemsep}{0.1in}
    \setlength{\labelwidth}{0.95in} \setlength{\labelsep}{0.1in}
\setlength{\listparindent}{0in} \setlength{\parsep}{0in}
    \setlength{\leftmargin}{1.0in}}
    }{\end{list}}

\newcommand{\STAR}{\hspace*{\fill} * \hspace*{\fill}}

\newcommand{\sym}[1]{\hspace*{\fill} ($#1$)}

%
% The thebibliography environment is redefined so the the word References is
% not output
%\def\thebibliography#1{\list
% {[\arabic{enumi}]}{\settowidth\labelwidth{[#1]}\leftmargin\labelwidth
% \advance\leftmargin\labelsep
% \usecounter{enumi}}
% \def\newblock{\hskip .11em plus .33em minus -.07em}
% \sloppy\clubpenalty4000\widowpenalty4000
% \sfcode`\.=1000\relax}
%\let\endthebibliography=\endlist
% END thebibliography environment redefinition



%\newcommand{\eskipv}[1]{\clearpage\marginlabel{#1}}
%\newcommand{\eskip}[1]{\vspace*{\fill}\clearpage\marginlabel{#1}}
%\newcommand{\eskipnv}[1]{\newpage\marginlabel{#1}}
%\newcommand{\eskipn}[1]{\vspace*{\fill}\newpage\marginlabel{#1}}

%marginlabel is very wide
%\newcommand{\eskipfullv}[1]{\clearpage\marginlabelfull{#1}}
%\newcommand{\eskipfull}[1]{\vspace*{\fill}\clearpage\marginlabelfull{#1}}
%\newcommand{\eskipfullnv}[1]{\newpage\marginlabelfull{#1}}
%\newcommand{\eskipfulln}[1]{\vspace*{\fill}\newpage\marginlabelfull{#1}}

%\newcommand{\mycontentsline}[5]{\parbox{#1}{#2}#3
%\hspace{0.1in}\dotfill\parbox{0.3in}{\hfill\pageref{#4#5}}\\[0.1in]}
%\newcommand{\mysline}[2]{\mycontentsline{1.2in}{#1}{#2}{#1}{statement}}
%\newcommand{\mymline}[2]{\mycontentsline{0.7in}{#1}{#2}{#1}{model}}
%\newcommand{\myeline}[2]{\mycontentsline{0.7in}{#1}{#2}{#1}{element}}

%\newcommand{\myincontentsline}[5]{\vspace{0.05in}\noindent\parbox{#1}{#2}#3
%\hspace{0.1in}
%\dotfill\parbox{0.7in}{\hfill Page \pageref{#4#5}}\\[0.05in]\noindent}
%\newcommand{\myinsline}[2]{\myincontentsline{1.2in}{#1}{#2}{#1}{statement}}
%\newcommand{\myinmline}[2]{\myincontentsline{0.5in}{#1}{#2}{#1}{model}}
%\newcommand{\myineline}[2]{\myincontentsline{0.5in}{#1}{#2}{#1}{element}}

%
%
% The following a symbols that could used alot.
\newcommand{\ms}[1]{\mbox{\scriptsize #1}}
\newcommand{\AF}{A_F}
\newcommand{\CBD}{C'_{BD}}
\newcommand{\CBS}{C'_{BS}}
\newcommand{\CGBO}{C_{GBO}}
\newcommand{\CGDO}{C_{GDO}}
\newcommand{\CGSO}{C_{GSO}}
\newcommand{\CJ}{C_J}
\newcommand{\CJSW}{C_{J,\ms{SW}}}
\newcommand{\DELTA}{\delta}
\newcommand{\ETA}{\eta}
\newcommand{\FC}{F_C}
\newcommand{\GAMMA}{\gamma}
\newcommand{\IS}{I_S}
\newcommand{\JS}{J_S}
\newcommand{\KAPPA}{\kappa}
\newcommand{\KF}{K_F}
\newcommand{\KP}{K_P}
\newcommand{\LAMBDA}{\lambda}
\newcommand{\LD}{X_{JL}}
\newcommand{\LEVEL}{M_J}
\newcommand{\MJ}{M_J}
\newcommand{\MJSW}{M_{J,\ms{SW}}}
\newcommand{\NSUB}{N_B}
\newcommand{\NSS}{N_{\ms{SS}}}
\newcommand{\NFS}{N_{\ms{FS}}}
\newcommand{\NEFF}{N_{\ms{EFF}}}
\newcommand{\PB}{\phi_J}
\newcommand{\PHI}{2\phi_B}
\newcommand{\RD}{R_D}
\newcommand{\RS}{R_S}
\newcommand{\RSH}{R_{\ms{SH}}}
\newcommand{\THETA}{\theta}
\newcommand{\TOX}{T_{OX}}
\newcommand{\TPG}{T_{\ms{PG}}}
\newcommand{\UCRIT}{U_C}
\newcommand{\UEXP}{U_{\ms{EXP}}}
\newcommand{\UO}{\mu_0}
\newcommand{\UTRA}{U_{\ms{TRA}}}
\newcommand{\VMAX}{V_{\ms{MAX}}}
\newcommand{\VTZERO}{V_{T0}}
\newcommand{\VTO}{V_{T0}}
\newcommand{\XJ}{X_J}
\newcommand{\Length}{L} %  \L already used
\newcommand{\N}{N}
\newcommand{\PBSW}{\phi_{J,{\ms{SW}}}}
\newcommand{\RB}{R_B}
\newcommand{\RG}{R_B}
\newcommand{\RDS}{R_{DS}}
\newcommand{\TT}{\tau_T}
\newcommand{\W}{W}
\newcommand{\WD}{W_D}
\newcommand{\XQC}{X_{QC}}
\newcommand{\JSSW}{J_{S,{\ms{SW}}}}
\newcommand{\DL}{\Delta_L}
\newcommand{\DW}{\Delta_W}
\newcommand{\DELL}{\Delta_{L,\ms{SW}}}
\newcommand{\KONE}{K_1}
\newcommand{\KTWO}{K_2}
\newcommand{\MUS}{\mu_S}
\newcommand{\MUZ}{\mu_Z}
\newcommand{\NZERO}{N_0}
\newcommand{\NB}{N_B}
\newcommand{\ND}{N_D}
\newcommand{\TEMP}{T}
\newcommand{\VDD}{V_{DD}}
\newcommand{\WDF}{W_{\ms{DF}}}
\newcommand{\VFB}{V_{\ms{FB}}}
\newcommand{\UZERO}{U_0}
\newcommand{\UONE}{U_1}
\newcommand{\XTWOE}{X_{2E}}
\newcommand{\XTWOMS}{X_{2\ms{MS}}}
\newcommand{\XTWOMZ}{X_{2\ms{MZ}}}
\newcommand{\XTWOUZERO}{X_{2\ms{U0}}}
\newcommand{\XTWOUONE}{X_{2\ms{U1}}}
\newcommand{\XTHREEE}{X_{3E}}
\newcommand{\XTHREEMS}{X_{3\ms{MS}}}
\newcommand{\XTHREEMZ}{X_{3\ms{MZ}}}
\newcommand{\XTHREEUZERO}{X_{3\ms{U0}}}
\newcommand{\XTHREEUONE}{X_{3\ms{U1}}}
\newcommand{\XPART}{X_{\ms{PART}}}
\newcommand{\PS}{P_S}
\newcommand{\PD}{P_D}
\newcommand{\NRS}{N_{RS}}
\newcommand{\NRG}{N_{RG}}
\newcommand{\NRB}{N_{RB}}
\newcommand{\NRD}{N_{RD}}


\newcommand{\Net}{{${\cal N}$}}                          % network \N
\newcommand{\Nprime}{{${\cal N}^{\prime}$}}            % another network \Nprime
\newcommand{\Nold}{{${\cal N}^{\mbox{old}}$}}          % old network  \Nold
\newcommand{\Nnew}{{${\cal N}^{\mbox{new}}$}}          % new network  \Nnew

\newcommand{\GMIN}{{G_{\ms{MIN}}}}

\newcommand{\optionitem}[2]{
\item[{\tt #1}{#2}]\label{.OPTION#1}\index{.OPTIONS, #1}\index{#1}}

\newcommand{\error}[1]{\vspace{0.1in}\noindent{\tt #1}\\}


%For numbering an equation which is incoorporated
%with text.
\newcommand{\inlineeq}{\hspace*{\fill}\refstepcounter{equation}{\rm
(\theequation)}\\}

\begin{document}
\noindent{\LARGE \textbf{Exponential current source}
\hspace{\fill}\textbf{iexp}}
\hrulefill\linethickness{0.5mm}\line(1,0){425}
\normalsize
\newline
%\newline
% the resistor figure
\begin{figure}[h]
\centerline{\epsfxsize=0.5in\epsfbox{figures/i_spice.ps}}
\caption{Independent Current Source Element.}
\end{figure}
\newline
% form for \FDA
\linethickness{0.5mm} \line(1,0){425}
\newline
\textit{Form:}
\newline
$\tt iexp$:$\langle \tt{instance\ name}\rangle$ $n_1\ n_2\ $
$\langle \tt{parameter\ list}\rangle$
\newline
\begin{tabular}{r l}
$n_1$ & is the positive element node, \\
&  \\
$n_2$ & is the negative element node. \\
%&  \\
%parameter list & see table 1 for parameter list
\end{tabular}
% Parameter list
\newline
\textit{Parameters:}
\begin{table}[H]
\begin{tabular}{|c|c|c|c|}
\hline
Parameter&Type&Default value&Required?\\
\hline
i1: Initial value (A) & DOUBLE & 0 & no\\
\hline
i2: Final current (A) & DOUBLE & 0 & no\\
\hline
tdr: Rise Time delay (s) & DOUBLE & 0 & no\\
\hline
tdf: Fall Time delay (s) & DOUBLE & 0 & no\\
\hline
tcr: Rise Time Constant (s) & DOUBLE & 0 & no\\
\hline
tcf: Fall Time Constant (s) & DOUBLE & 0 & no\\
\par
\hline
\end{tabular}
\end{table}
% example in \FDA
%\newline
\noindent\linethickness{0.5mm}\line(1,0){425}
\newline
\textit{Example:}
\newline
\texttt{iexp:isignal\ 8\ 0\ i1=0.1 i2=0.8 tdr=1 tdf=2 tcr=0.35
tcf=1}
\newline
\linethickness{0.5mm} \line(1,0){425}
\newline
\textit{Description:}\\
The exponential transient is a single-shot event specifying two
exponentials. The current is $i_1$ for the first $t_{dr}$ seconds
at which it begins increasing exponentially towards $i_2$ with a
time constant of $t_{cr}$ seconds. At time $t_{df}$ the current
exponentially decays towards $i_1$ with a time constant of
$t_{cf}$. That is, The waveform shape of an exponential current
source is given by
\begin{eqnarray}
i_1     & 0 < t \leq t_{d1}\\
i_1 + (i_2 - i_1)[1 - e^{-(t-t_{dr})/t_{cr}}]     & t_{d1} < t \leq t_{d2}\\
i_1 + (i_2 - i_1)[1 -
e^{-(t_{df}-t_{dr})/t_{cr}}]e^{-(t-t_{df})/t_{cf}} & t_{d2} < t
\leq t_{stop}
\end{eqnarray}
\begin{figure}[h]
\centerline{\epsfxsize=3in\pfig{vexp.eps}} \caption{Current source
transient exponential waveform for \texttt{iexp:isignal\ 8\ 0\
i1=0.1 i2=0.8 tdr=1 tdf=2 tcr=0.35 tcf=1}}
\end{figure}
\newline
\linethickness{0.5mm} \line(1,0){425}
\newline
\textit{Notes:}\\
This is the \texttt{I} element in the SPICE compatible netlist.\\
\linethickness{0.5mm} \line(1,0){425}
\newline
\textit{Version:}\\
2002.05.15 \\
% Credits
\newpage
\noindent\linethickness{0.5mm}\line(1,0){425}
\newline
\textit{Credits:}\\
\begin{tabular}{l l l l}
Name & Affiliation & Date & Links \\
Satish Uppathil & NC State University & May 2002 & \epsfxsize=1in\epsfbox{figures/logo.eps} \\
svuppath@eos.ncsu.edu & & & www.ncsu.edu    \\
\end{tabular}
\end{document}

\caption[Current source exponential ({\tt EXP}) waveform] {Current
source exponential ({\tt EXP}) waveform for {\tt EXP(0.1 0.8 1
0.35 2 1)} \label{fig:iexp} }
\end{figure}
\newline
\linethickness{0.5mm} \line(1,0){425}
\newline
\textit{Notes:}\\
The actual element in \FDA is the \texttt{iexp} element.
See \texttt{iexp} for full documentation.\\
\newline

\underline{\bf{Pulse}}:\\
%%%%%%%%%%%Put the pulse form here%%%%%%%%%%%%%%%%%%%%%%%%%%%%%%%%%%%%
{\tt PULSE(} $I_1$ $I_2$ \B $T_D$ \E \B $T_R$ \E \B $T_F$\E
\B $W$ \E \B $T$ \E {\tt )}\\
\textit{Parameters:}
\begin{table}[h]
\begin{tabular}{|c|c|c|c|}
\hline
Name&Description&Units&Default\\
\hline
$I_1$ & initial voltage & V & \scriptsize{REQUIRED}\\
\hline
$I_2$ & pulsed voltage & V & \scriptsize{REQUIRED}\\
\hline
$T_D$ & delay time & s & 0.0\\
\hline
$T_R$ & rise time & s & \texttt{TSTEP}\\
\hline
$T_F$ & fall time & s & \texttt{TSTEP}\\
\hline
W & pulse width & s & \texttt{TSTOP}\\
\hline
T & period & s & \texttt{TSTOP}\\
\par
\hline
\end{tabular}
\end{table}
%%%%%%%%%%%%%%%%%%%%%%%%%%%%%%%%%%%%%%% example in \FDA
\newline
\linethickness{0.5mm} \line(1,0){425}
\newline
\textit{Example:}
\newline
\texttt{ISIGNAL 20 5 PULSE(0 5 1N 2N 1.5N 21.9N 5N 20N)}
\newline
\linethickness{0.5mm} \line(1,0){425}
\newline
\textit{Description:}\\
The pulse transient waveform is defined by
\begin{equation}
i = \left\{ \begin{array}{ll}
I_1                         & t \le T_D\\
I_1 + {{\textstyle t'} \over {\textstyle T_R}} ({I_2}-{I_1})&0<t' \le T_R\\
I_2                         &{T_R} < t'< (T_R+W)\\
I_2 - {{\textstyle t'-W} \over {\textstyle T_F}} (I_1-I_2)
                   &(T_R+W) < t' < (T_R+W+T_F)\\
I_1           &(T_R+W+T_F) < t' < T
     \end{array} \right. %}
\end{equation}
where
\begin{equation}
t' = t - T_D -(n-1)T
\end{equation}
and $t$ is the current analysis time and $n$ is the cycle index.
The effect of this is that after an initial time delay $T_D$ the
transient waveform repeats itself every cycle.
\begin{figure}[h]
\centering
\input{ipulse}
\caption[Current source transient pulse ({\tt PULSE})
waveform]{Current source transient pulse ({\tt PULSE}) waveform
for\newline \hspace*{\fill} {\tt PULSE(0.3 1.8 1 2.5 0.3 1 0.7)}
\hspace*{\fill} \label{fig:ipulse}}
\end{figure}
\newline
\linethickness{0.5mm} \line(1,0){425}
\newline
\textit{Notes:}\\
The actual element in \FDA is the \texttt{ipulse} element.
See \texttt{ipulse} for full documentation.\\
\newline

\underline{\bf{ Piece-Wise Linear}}:\\
%%%%%%%%%%%%%%%%%%%%%%put the SIN form here%%%%%%%%%%%%%%%%%%%%
\texttt{PWL($T_1$ $I_1$\B $T_2$ $I_2$ ... $T_i$ $I_i$ ... $T_N$ $I_N$ \E )}\\
%\textit{Parameters:}
%\begin{table}[h]
%\begin{tabular}{|c|c|c|c|}
%\hline
%Name&Description&Units&Default\\
%\hline
%$V_O$ & voltage offset & V & \scriptsize{REQUIRED}\\
%\hline
%$V_A$ & voltage amplitude & V & \scriptsize{REQUIRED}\\
%\hline
%$F$ & frequency & Hz & 1/{\texttt{TSTOP}}\\
%\hline
%$T_D$ & time delay & s & 0\\
%\hline
%$\Theta$ & damping factor & 1/s & 0\\
%\hline
%$\phi$ & phase & degree & 0\\
%\par
%\hline
%\end{tabular}
%\end{table}
%%%%%%%%%%%%%%%%%%%%%%%%%%%%%%%%%%%%%%% example in \FDA
\newline
\linethickness{0.5mm} \line(1,0){425}
\newline
\textit{Example:}
\newline
\texttt{ISIGNAL\ 20\ 5\ PWL(1 0.25  1 1 2 0.5 $\ldots$ 3 0.5 4 1
$\ldots$ 4.5 1.25 $\ldots$)}
\newline
\linethickness{0.5mm} \line(1,0){425}
\newline
\textit{Description:}\\
Each pair of values ($T_i$, $I_i$) specifies that  the  value of
the  source  is $I_i$ at time = $T_i$. At times between $T_i$ and
$T_{i+1}$ the values are linearly interpolated. If $T_1 >$ 0 then
the voltage is constant at {\it DCValue} (specified on the element
line) until time $T_1$.
\begin{equation}
i = \left\{ \begin{array}{ll}
    {\it DCvalue}& t < T_1\\
    I_i         & t = T_i\\
    I_{i+1}     & t = T_{i+1}\\
    I_i + \left({{t-T_i} \over {T_{i+1} - T_i}}\right)(I_{i+1}-I_i)
                & T_i < t \le T_{i+1}\\
    I_N         & t > T_N\\
     \end{array} \right. %}
\end{equation}
\begin{figure}[h]
\centering
\input{ipwl}
\caption[Current source transient piece-wise linear ({\tt PWL})
waveform]{Current source transient piece-wise linear ({\tt PWL})
waveform for\newline\hspace*{\fill} {\tt PWL(1 0.25  1 1 2 0.5
$\ldots$ 3 0.5 4 1 $\ldots$ 4.5 1.25 $\ldots$)} with {\it DCValue
= 0.25}.  \hspace*{\fill} }
\end{figure}
\newline
\linethickness{0.5mm} \line(1,0){425}
\newline
\textit{Notes:}\\
The actual element in \FDA is the \texttt{ipwl} element.
See \texttt{ipwl} for full documentation.\\
\newline

\underline{\bf{ Single-Frequency FM}}:\\
%%%%%%%%%%%%%%%%%%%%%%put the SFFM form here%%%%%%%%%%%%%%%%%%%%
\texttt{SFFM($I_O$ $I_A$ $F_C$ $\mu$ $F_S$)}\\
\textit{Parameters:}
\begin{table}[h]
\begin{tabular}{|c|c|c|c|}
\hline
Name&Description&Units&Default\\
\hline
$I_O$ & offset current & A & \\
\hline
$I_A$ & peak amplitude of \ac\ current & A & \\
\hline
$F_C$ & carrier frequency & Hz & 1/{\texttt{TSTOP}}\\
\hline
$\mu$ & modulation index & - & 0\\
\hline
$F_S$ & signal frequency & Hz & 1/{\texttt{TSTOP}}\\
\par
\hline
\end{tabular}
\end{table}
%%%%%%%%%%%%%%%%%%%%%%%%%%%%%%%%%%%%%%% example in \FDA
\newline
\linethickness{0.5mm} \line(1,0){425}
\newline
\textit{Example:}
\newline
\texttt{ISIGNAL\ 8\ 0\ SFFM(0.2 0.7 4 0.9 1)}
\newline
\linethickness{0.5mm} \line(1,0){425}
\newline
\textit{Description:}\\
The single frequency frequency modulated transient response is
described by
\begin{equation}
i = I_O + I_A\sin{(2 \pi \, F_C t +  \mu\sin{(2 \pi F_S t)})}
\end{equation}
\begin{figure}[h]
\centering
\documentclass{article}
\usepackage{epsf}\usepackage{here}
% SUMMARY OF USEFUL MACROS
%
% 1. \marginpar[LeftText]{RightText} Standard Latex \marginpar
%
% 2. \mymarginpar{MarginText}{BodyText} Places MarginText in margin and
%     BodyText in main text opposite margin. Places lineas above and below
%     text.  Used in element and model catalogs and elsewhere.
%
% 3. \marginlabel{text} Places large text in margin and underlines. Used
%    to put element name in margin at top of page for continued
%    descriptions.  Does not automaticly put it at the top of a page
%    and so a \clearpage is required.
%
% 4. \marginid{text} Used to place a short identifier in the margin. Just one
%    word and justified to the outside edge.
%
% 5. \offset a standard indent.
%
% 6. \offsetparbox{} Places text in an offset parbox.  Use
%    \hspace*{\fill} \offsetparbox{text}        to insert an offset text.
%
% 7. \boxed{} similar to above but starts a new line and right justifies
%    offsetparbox.
%
% 8. \form
%
% 9. \example
%
%10. \begin{widelist}  .... \end{widelist} 
%    A widelist that has 1 inch wide labels that has additional left hand
%    margin of 0.6 inch.
%
%11. \spicethreeonly{} Text is inserted only if output is for spice3.
%
%12. \pspiceonly{} 
%    Text is inserted only if output is for pspice (superspice for the most
%    part is upwards compatible to pspice).
%
%13. \sspiceonly{} 
%    Text is inserted only if output is for superspice but not for pspice.

%14. \sym{}
%    Right justifies a symbol in a keyword table.
%
%15. Use the following wherever as then we can have a standard way to
%    report them.  Note that "\dc\ " is required to get a space after dc.
%    \dc
%    \ac
%    \SPICE
%
%16. \kwnote{}  This is a convenient way to include notes in a keyword table.
%    It right justifies a note in the description column and starts it on a
%    newline.
\newcommand{\kwnote}[1]{\newline\hspace*{\fill} #1}

%
%17. \kwversion{} This is a convenient way to indicate versions in a keyword
%    table. It does a \kwnote .  It should not printed when outputing for
%     specific versions.
% typical usage is \kwversion{\sspice ; \hspice} to produce the output
%            VERSIONS: SUPERSPICE; HSPICE
% typical usage is \kwversionNote{\sspice ; \hspice} to produce the output
%                 |                    VERSIONS: SUPERSPICE; HSPICE |
%for unifomity the recommended version order is:
%    \hspice \pspice \sspice \spicetwo \spicethree
\newcommand{\version}[1]{({\sc version: #1})}
\newcommand{\kwversion}[1]{\newline({\sc version: #1})}
\newcommand{\kwversionNote}[1]{\kwnote{({\sc version: #1})}}
\newcommand{\versions}[1]{({\sc versions: #1})}
\newcommand{\kwversions}[1]{\newline({\sc versions: #1})}
\newcommand{\kwversionsNote}[1]{\kwnote{({\sc versions: #1})}}

%


\newcommand{\mymargin}{1in}   % Use to set width of margin notes
\newcommand{\mymarginparsep}{0.1in}
\newcommand{\mymarginparsepplustext}{5.6in}
\newcommand{\mymarginplus}{1.1in}
\newcommand{\mymarginplustext}{6.6in}
\newcommand{\underlinehead}{\\[-0.1in] \rule{\mymarginplustext}{0.01in}}
\newcommand{\overlinefoot}{\rule{\mymarginplustext}{0.01in}\\}
% WIDEPARBOX uses margin space as well.
\newcommand{\wideparbox}[1]{\parbox[t]{\mymarginplustext}{#1}}
%
% HEADER AND FOOT
% Standard header and foot, but includes copyright notice.a
% To omit copyrite notice do not include copyrite.sty
%
\newcommand{\mycopyrite}{}
\def\ps@headings{\let\@mkboth\markboth
\def\@oddfoot{\wideparbox{\overlinefoot\rm\today \hfill \mycopyrite}}
\def\@evenfoot{\hspace*{-\mymarginplus}\wideparbox{\overlinefoot
\mycopyrite\hfill \today}}
\def\@evenhead{\hspace*{-\mymarginplus}\wideparbox{\rm
\thepage\hfill \sl \leftmark \underlinehead }}
\def\@oddhead{\wideparbox{\hbox{}\sl \rightmark \hfill
\rm\thepage \underlinehead}}\def\chaptermark##1{\markboth {\uppercase{\ifnum
\c@secnumdepth
>\m@ne
 \@chapapp\ \thechapter. \ \fi ##1}}{}}\def\sectionmark##1{\markright
 {\uppercase{\ifnum \c@secnumdepth >\z@
  \thesection. \ \fi ##1}}}}

%\oddsidemargin 0.25in \topmargin 0.0in \textwidth 6.5in \textheight 9in
%\evensidemargin 0.25in \headheight 0.18in \footskip 0.16in

%\input{epsf}
\usepackage{epsf}
\oddsidemargin 0.25in \evensidemargin 0.25in
\topmargin 0.0in
\textwidth 6.5in \textheight 9in
\headheight 0.18in \footskip 0.16in
\leftmargin -0.5in \rightmargin -0.5in

\newcommand{\fig}[1]{figures/#1}
\newcommand{\pfig}[1]{\epsfbox{\fig{#1}}}
\newcommand{\newfig}[1]{\epsffile{\fig{#1}}}

\newcommand{\fdaelement}[1]{elements/#1}
\newcommand{\spiceelement}[1]{equivalent_spice_elements/#1}

\newcommand{\FREEDA}{{\Huge{\textsl{\textsf{f}}}${\mathsf{REEDA}}^{{\tiny{\textsf{TM}}}}$}}
\newcommand{\FDA}{{\textsl{\textsf{f}}}${\mathsf{REEDA}}^{{\tiny{\textsf{TM}}}}$}
\newcommand{\notforsspice}[1]{#1}
\newcommand{\spicetwoonly}[1]{#1}
\newcommand{\spicethreeonly}[1]{#1}
\newcommand{\pspiceonly}[1]{#1}
\newcommand{\pspiceninetytwoonly}[1]{#1}
\newcommand{\sspiceonly}[1]{}
\newcommand{\fornutmeg}[1]{}
\newcommand{\sspicetwoonly}[1]{}

%%%%%%%%%%%%%%%%%%%%%%%%%%%%%%%%%%%%%%%%%%%%%%%%%%%%%%%%%%%%%%%%%%%%%%%%%%%%%%%%
\marginparwidth=\mymargin
\marginparsep=\mymarginparsep
\newcommand{\textplusmarginparwidth}{\textwidth+\marginparsep+\mymargin}
\newcommand{\X}{\\ \hline}    % line terminataion in keyword environment
\newcommand{\B}{{ \rm [}}     % begin optional parameter in \form{}
\newcommand{\E}{{\ \rm\hspace{-0.04in}] }}   % end optional parameter in \form{}
\newcommand{\expr}{{\sc Expressions supported}}
\newcommand{\reqd}{{\scriptsize REQUIRED}}
\newcommand{\omitted}{{\scriptsize OMITTED}}
\newcommand{\inferred}{{\scriptsize INFERRED}}
\newcommand{\para}{\newline{\scriptsize (PARASITIC)}}
\newcommand{\opt}{{\tiny  OPTIONAL}}

\newcommand{\Spice}{{\sc Spice}}
\newcommand{\spice}{{\sc Spice}}
\newcommand{\justspice}{{\sc Spice}}
\newcommand{\nutmeg}{{\sc NUTMEG}}
\newcommand{\probe}{{\sc Probe}}

\newcommand{\accusim}{{\sc AccuSim}}
\newcommand{\contec}{{\sc ContecSpice}}
\newcommand{\cdsspice}{{\sc CDS Spice}}
\newcommand{\hpimpulse}{{\sc HP Impulse}}
\newcommand{\hspice}{{\sc HSpice}}
\newcommand{\igspice}{{\sc IG\_SPICE}}
\newcommand{\isspice}{{\sc IsSpice}}
\newcommand{\mspice}{{\sc Microwave Spice}}
\newcommand{\pspice}{{\sc PSpice}}
\newcommand{\justpspice}{{\sc PSpice}}
\newcommand{\spectre}{{\sc Spectre}}
\newcommand{\spicetwo}{{\sc Spice2g6}}
\newcommand{\spicethree}{{\sc Spice3}}
\newcommand{\spiceplus}{{\sc SpicePlus}}
\newcommand{\sspice}{{\sc SuperSpice}}

\newcommand{\modelversion}[1]{& #1}
%%%%%%%%%%%%%%%%%%%%%%%%%%%%%%%%%%%%%%%%%%%%%%%%%%%%%%%%%%%%%%%%%%%%%%%%%%%%%%%%
% set up a counter for all occasions
%
\newcounter{count}

% set up new commands
%
\newcommand{\vshift}{\vspace{0.2in}}
% OFFSET
\newcommand{\offset}{\hspace*{0.45in} }
% OFFSETPARBOXWIDTH argument should be \textwith - offset
\newcommand{\offsetparbox}[1]{\parbox[t]{5in}{#1}}

% note: no labeling
\newcommand{\elementx}[3]{\clearpage\rm\markright{#3:#2}
\addcontentsline{toc}{section}{#1, #2: #3}
\mymarginparx{#1}{#2}{#3}\index{#1:#2}\index{#3:#2}}

% macros for sub elements
\newcommand{\subelement}[2]{\clearpage 
   \noindent\rule{\textwidth /2}{.5mm} \\[0.1in]
   {\large \bf #1} \hspace{0.2in} {\bf #2} \\
   \noindent\rule{\textwidth /2}{.5mm} \index{#1} \index{#2}}
% macros for models
\newcommand{\model}[2]{{
   \noindent\vspace{0.2in}\parbox{\textwidth}{
   \noindent\rule{\textwidth}{.5mm} \\[0.1in]
   {\large \bf #1 Model} \label{#1model} \hfill {\large #2} \\
   \noindent\rule{\textwidth}{.5mm} \index{#1} \index{#2}}}}
\newcommand{\modelx}[3]{{
   \noindent\vspace{0.2in}\parbox{\textwidth}{
   \noindent\rule{\textwidth}{.5mm} \\[0.1in]
   {\large \bf #1 Model} \hfill {\large #2} \hfill {\large #3} \\
   \noindent\rule{\textwidth}{.5mm} \index{#1} \index{#2}}}}



% BOXED
\newcommand{\boxed}[1]{\noindent
\newline \vshift \hspace*{\fill} {\tt \offsetparbox{\tt #1}}
 \vshift}


%
% KEYWORD
%
\newcommand{\keywordtable}[1]{
        \sloppy
        \hyphenation{ca-pac-i-t-an-ce} 
        \begin{center}
    \sf
        \begin{tabular}[t]
        {|p{0.58in}|p{3.07in}|p{0.55in}|p{0.60in}|}
        \hline
        \multicolumn{1}{|c}{\bf Name} &
        \multicolumn{1}{|c}{\parbox{2.77in}{\bf Description}}  &
        \multicolumn{1}{|c}{\bf Units} &
        \multicolumn{1}{|c|}{\bf Default} \X
        #1
        \end{tabular}
        \end{center}
    }

\newcommand{\keywordtwotable}[2]{
        \sloppy
        \hyphenation{ca-pac-i-t-an-ce} 
        \begin{center}
    \sf
        \begin{tabular}[t]
        {|p{0.58in}|p{2.38in}|p{0.55in}|p{0.60in}|p{0.53in}|}
        \hline
        \multicolumn{1}{|c}{\bf Name} &
        \multicolumn{1}{|c}{\parbox{2.20in}{\bf Description}}  &
        \multicolumn{1}{|c}{\bf Units} &
        \multicolumn{1}{|c}{\bf Default} &
        \multicolumn{1}{|c|}{\bf #1} \X
        #2
        \end{tabular}
        \end{center}
    }

\newcommand{\kw}[2]{
     \samepage{
     \noindent {\sl #1} \vspace{-0.5in} \\
     \keywordtable{#2} }}

\newcommand{\kwtwo}[3]{
     \samepage{
     \noindent {\sl #1} \vspace{-0.4in} \\
     \keywordtwotable{#2}{#3} }}

\newcommand{\keyword}[1]{\kw{Keywords:}{#1}}
\newcommand{\keywordtwo}[2]{\kwtwo{Keywords:}{#1}{#2}}
\newcommand{\modelkeyword}[1]{\kw{Model Keywords}{#1}}
\newcommand{\modelkeywordtwo}[2]{\kwtwo{Model Keywords}{#1}{#2}}

\newcommand{\myline}{\\[-0.1in]
\noindent \rule{\textwidth}{0.01in} \newline}

\newcommand{\myThickLine}{\\[-0.1in]
\noindent \rule{\textwidth}{0.02in} \newline}


% SPICE 2G6 FORM
%\newcommand{\spicetwoform}[1]{
%\spicetwoonly{\samepage{\noindent{\sl\spicetwo\form{#1}}}}}

% PSPICE88 FORM
%\newcommand{\pspiceform}[1]{
%\pspiceonly{\samepage{\noindent{\sl\pspice}\form{#1}}}}

% PSPICE92 FORM
%\newcommand{\pspiceninetytwoform}[1]{
%\pspiceninetytwoonly{\samepage{\noindent{\sl\pspice92\form{#1}}}}}


% SPICE3E2 FORM
%\newcommand{\spicethreeform}[1]{
%\spicethreeonly{\samepage{\noindent{\sl\spicethree\form{#1}}}}}

% FORM
\newcommand{\form}[1]{\samepage{\noindent
 {\sl Form} \myline
% \hspace*{\fill} % For some reason \fill = 0 when \pspiceform{} is used?
\offset
\it  \offsetparbox{#1}}
\\[0.1in]}

% ELEMENT FORM
\newcommand{\elementform}[1]{\samepage{\noindent
 {\sl Element Form} \myline
% \hspace*{\fill} % For some reason \fill = 0 when \pspiceform{} is used?
\offset
\it  \offsetparbox{#1}}
\\[0.1in]}

% MODEL FORM
\newcommand{\modelform}[1]{\samepage{\noindent
 {\sl Model Form} \myline
% \hspace*{\fill} % For some reason \fill = 0 when \pspiceform{} is used?
\offset
\it  \offsetparbox{#1}}
\\[0.1in]}

% LIMITS
\newcommand{\mylimits}[1]{\samepage{\noindent
 {\sl Limits} \myline
 \hspace*{\fill} \it  \offsetparbox{#1}}
 \vshift}

% EXAMPLE
\newcommand{\example}[1]{\samepage{\noindent
{\sl Example} \myline
\offset \tt  \offsetparbox{#1}}
 \vshift}

% PSPICE88 EXAMPLE
\newcommand{\pspiceexample}[1]{\samepage{\noindent
{\sl \pspice\ Example} \myline
\offset \tt  \offsetparbox{#1}}
 \vshift}

% MODEL TYPES
\newcommand{\modeltype}[1]{\samepage{\noindent
{\sl Model Type} \myline
 \hspace*{\fill} \tt \offsetparbox{#1}}
 \\[0.1in]}

% MODEL TYPES
\newcommand{\modeltypes}[1]{\samepage{\noindent
{\sl Model Types:} \myline
 \hspace*{\fill} \tt \offsetparbox{\tt #1}}
 \vshift}

% OFFSET ENUMERATE
\newcommand{\offsetenumerate}[1]{
     \offset \hspace*{-0.1in} {\begin{enumerate} #1 \end{enumerate}}}

% NOTE
\newcommand{\note}[1]{
\vshift\samepage{\noindent {\sl Note}\myline\vspace{-0.24in}}
 \offsetenumerate{#1} }

% SPECIAL NOTE
\newcommand{\specialnote}[2]{
\vshift\samepage{\noindent {\sl #1}\myline\vspace{-0.24in}}\\#2}

\newcommand{\dc}{\mbox{\tt DC}}
\newcommand{\ac}{\mbox{\tt AC}}
\newcommand{\SPICE}{\mbox{\tt SPICE}}
\newcommand{\m}[1]{{\bf #1}}                           % matrix command  \m{}

% ////// Changing nodes to terminals///////
% print terminals in \tt and enclose in a circle use outside
\newcommand{\terminal}[1]{\: \mbox{\tt #1} \!\!\!\! \bigcirc }
%
% set up environment for example
%
\newcounter{excount}
\newcounter{dummy}
\newenvironment{eg}{\vspace{0.1in}\noindent\rule{\textwidth}{.5mm}
   \begin{list}
   {{\addtocounter{excount}{1}
   \em Example\/ \arabic{chapter}.\arabic{excount}\/}:}
   {\usecounter{dummy}
   \setlength{\rightmargin}{\leftmargin}}
   }{\end{list} \rule{\textwidth}{.5mm}\vspace{0.1in}}
%
% set up environment for block
% currently this draws a horizontal line at the start of block and another
% at the end of block.
%
\newenvironment{block}{\vspace{0.1in}\noindent\rule{\textwidth}{.5mm}
   }{\rule{\textwidth}{.5mm}\vspace{0.1in}}
%

%
% Macros for element summaries
%
% macros for element summary
%\newcommand{\summaryelement}[2]{
%   \vspace{0.4in}
%   \mymarginpar{#1}{#2} 
%   \addcontentsline{toc}{section}{#1, #2}
%   \vspace{-0.6in} \\
%   \noindent Full description on page \pageref{#1element}. \vshift\\
%   }

% Summary MODEL TYPE
%\newcommand{\summarymodeltype}[1]{\samepage{\noindent
%{\sl Model Type} \myline
% \hspace*{\fill} \tt \offsetparbox{\tt #1
% \hfill Summary on \pageref{#1summary} \index{#1}}}
% }

% Summary MODEL TYPES
%\newcommand{\summarymodeltypes}[1]{\samepage{\noindent
%{\sl Model Types} \myline
% \hspace*{\fill} \tt \offsetparbox{\tt #1
% \hfill Summary on \pageref{#1summary} \index{#1}}}
% }

%
% macros for model summary
%\newcommand{\summarymodel}[2]{\clearpage
%   \addcontentsline{toc}{section}{#1, #2}
%   \mymarginpar{#1}{#2} \label{#1summary}
%   \index{#1} \index{#2}
%   \noindent Full description on page \pageref{#1model} \\[0.1in]
%   }



%
% set up wide descriptive list
%
\newenvironment{widelist}
    {\begin{list}{}{\setlength{\rightmargin}{0in} \setlength{\itemsep}{0.1in}
    \setlength{\labelwidth}{0.95in} \setlength{\labelsep}{0.1in}
\setlength{\listparindent}{0in} \setlength{\parsep}{0in}
    \setlength{\leftmargin}{1.0in}}
    }{\end{list}}

\newcommand{\STAR}{\hspace*{\fill} * \hspace*{\fill}}

\newcommand{\sym}[1]{\hspace*{\fill} ($#1$)}

%
% The thebibliography environment is redefined so the the word References is
% not output
%\def\thebibliography#1{\list
% {[\arabic{enumi}]}{\settowidth\labelwidth{[#1]}\leftmargin\labelwidth
% \advance\leftmargin\labelsep
% \usecounter{enumi}}
% \def\newblock{\hskip .11em plus .33em minus -.07em}
% \sloppy\clubpenalty4000\widowpenalty4000
% \sfcode`\.=1000\relax}
%\let\endthebibliography=\endlist
% END thebibliography environment redefinition



%\newcommand{\eskipv}[1]{\clearpage\marginlabel{#1}}
%\newcommand{\eskip}[1]{\vspace*{\fill}\clearpage\marginlabel{#1}}
%\newcommand{\eskipnv}[1]{\newpage\marginlabel{#1}}
%\newcommand{\eskipn}[1]{\vspace*{\fill}\newpage\marginlabel{#1}}

%marginlabel is very wide
%\newcommand{\eskipfullv}[1]{\clearpage\marginlabelfull{#1}}
%\newcommand{\eskipfull}[1]{\vspace*{\fill}\clearpage\marginlabelfull{#1}}
%\newcommand{\eskipfullnv}[1]{\newpage\marginlabelfull{#1}}
%\newcommand{\eskipfulln}[1]{\vspace*{\fill}\newpage\marginlabelfull{#1}}

%\newcommand{\mycontentsline}[5]{\parbox{#1}{#2}#3
%\hspace{0.1in}\dotfill\parbox{0.3in}{\hfill\pageref{#4#5}}\\[0.1in]}
%\newcommand{\mysline}[2]{\mycontentsline{1.2in}{#1}{#2}{#1}{statement}}
%\newcommand{\mymline}[2]{\mycontentsline{0.7in}{#1}{#2}{#1}{model}}
%\newcommand{\myeline}[2]{\mycontentsline{0.7in}{#1}{#2}{#1}{element}}

%\newcommand{\myincontentsline}[5]{\vspace{0.05in}\noindent\parbox{#1}{#2}#3
%\hspace{0.1in}
%\dotfill\parbox{0.7in}{\hfill Page \pageref{#4#5}}\\[0.05in]\noindent}
%\newcommand{\myinsline}[2]{\myincontentsline{1.2in}{#1}{#2}{#1}{statement}}
%\newcommand{\myinmline}[2]{\myincontentsline{0.5in}{#1}{#2}{#1}{model}}
%\newcommand{\myineline}[2]{\myincontentsline{0.5in}{#1}{#2}{#1}{element}}

%
%
% The following a symbols that could used alot.
\newcommand{\ms}[1]{\mbox{\scriptsize #1}}
\newcommand{\AF}{A_F}
\newcommand{\CBD}{C'_{BD}}
\newcommand{\CBS}{C'_{BS}}
\newcommand{\CGBO}{C_{GBO}}
\newcommand{\CGDO}{C_{GDO}}
\newcommand{\CGSO}{C_{GSO}}
\newcommand{\CJ}{C_J}
\newcommand{\CJSW}{C_{J,\ms{SW}}}
\newcommand{\DELTA}{\delta}
\newcommand{\ETA}{\eta}
\newcommand{\FC}{F_C}
\newcommand{\GAMMA}{\gamma}
\newcommand{\IS}{I_S}
\newcommand{\JS}{J_S}
\newcommand{\KAPPA}{\kappa}
\newcommand{\KF}{K_F}
\newcommand{\KP}{K_P}
\newcommand{\LAMBDA}{\lambda}
\newcommand{\LD}{X_{JL}}
\newcommand{\LEVEL}{M_J}
\newcommand{\MJ}{M_J}
\newcommand{\MJSW}{M_{J,\ms{SW}}}
\newcommand{\NSUB}{N_B}
\newcommand{\NSS}{N_{\ms{SS}}}
\newcommand{\NFS}{N_{\ms{FS}}}
\newcommand{\NEFF}{N_{\ms{EFF}}}
\newcommand{\PB}{\phi_J}
\newcommand{\PHI}{2\phi_B}
\newcommand{\RD}{R_D}
\newcommand{\RS}{R_S}
\newcommand{\RSH}{R_{\ms{SH}}}
\newcommand{\THETA}{\theta}
\newcommand{\TOX}{T_{OX}}
\newcommand{\TPG}{T_{\ms{PG}}}
\newcommand{\UCRIT}{U_C}
\newcommand{\UEXP}{U_{\ms{EXP}}}
\newcommand{\UO}{\mu_0}
\newcommand{\UTRA}{U_{\ms{TRA}}}
\newcommand{\VMAX}{V_{\ms{MAX}}}
\newcommand{\VTZERO}{V_{T0}}
\newcommand{\VTO}{V_{T0}}
\newcommand{\XJ}{X_J}
\newcommand{\Length}{L} %  \L already used
\newcommand{\N}{N}
\newcommand{\PBSW}{\phi_{J,{\ms{SW}}}}
\newcommand{\RB}{R_B}
\newcommand{\RG}{R_B}
\newcommand{\RDS}{R_{DS}}
\newcommand{\TT}{\tau_T}
\newcommand{\W}{W}
\newcommand{\WD}{W_D}
\newcommand{\XQC}{X_{QC}}
\newcommand{\JSSW}{J_{S,{\ms{SW}}}}
\newcommand{\DL}{\Delta_L}
\newcommand{\DW}{\Delta_W}
\newcommand{\DELL}{\Delta_{L,\ms{SW}}}
\newcommand{\KONE}{K_1}
\newcommand{\KTWO}{K_2}
\newcommand{\MUS}{\mu_S}
\newcommand{\MUZ}{\mu_Z}
\newcommand{\NZERO}{N_0}
\newcommand{\NB}{N_B}
\newcommand{\ND}{N_D}
\newcommand{\TEMP}{T}
\newcommand{\VDD}{V_{DD}}
\newcommand{\WDF}{W_{\ms{DF}}}
\newcommand{\VFB}{V_{\ms{FB}}}
\newcommand{\UZERO}{U_0}
\newcommand{\UONE}{U_1}
\newcommand{\XTWOE}{X_{2E}}
\newcommand{\XTWOMS}{X_{2\ms{MS}}}
\newcommand{\XTWOMZ}{X_{2\ms{MZ}}}
\newcommand{\XTWOUZERO}{X_{2\ms{U0}}}
\newcommand{\XTWOUONE}{X_{2\ms{U1}}}
\newcommand{\XTHREEE}{X_{3E}}
\newcommand{\XTHREEMS}{X_{3\ms{MS}}}
\newcommand{\XTHREEMZ}{X_{3\ms{MZ}}}
\newcommand{\XTHREEUZERO}{X_{3\ms{U0}}}
\newcommand{\XTHREEUONE}{X_{3\ms{U1}}}
\newcommand{\XPART}{X_{\ms{PART}}}
\newcommand{\PS}{P_S}
\newcommand{\PD}{P_D}
\newcommand{\NRS}{N_{RS}}
\newcommand{\NRG}{N_{RG}}
\newcommand{\NRB}{N_{RB}}
\newcommand{\NRD}{N_{RD}}


\newcommand{\Net}{{${\cal N}$}}                          % network \N
\newcommand{\Nprime}{{${\cal N}^{\prime}$}}            % another network \Nprime
\newcommand{\Nold}{{${\cal N}^{\mbox{old}}$}}          % old network  \Nold
\newcommand{\Nnew}{{${\cal N}^{\mbox{new}}$}}          % new network  \Nnew

\newcommand{\GMIN}{{G_{\ms{MIN}}}}

\newcommand{\optionitem}[2]{
\item[{\tt #1}{#2}]\label{.OPTION#1}\index{.OPTIONS, #1}\index{#1}}

\newcommand{\error}[1]{\vspace{0.1in}\noindent{\tt #1}\\}


%For numbering an equation which is incoorporated
%with text.
\newcommand{\inlineeq}{\hspace*{\fill}\refstepcounter{equation}{\rm
(\theequation)}\\}

\begin{document}
\noindent{\LARGE \textbf{Single Frequency FM current source}
\hspace{\fill}\textbf{isffm}}
\hrulefill\linethickness{0.5mm}\line(1,0){425}
\normalsize
\newline
% the resistor figure
\begin{figure}[h]
\centerline{\epsfxsize=0.5in\epsfbox{figures/i_spice.ps}}
\caption{Independent Current Source Element.}
\end{figure}
\newline
% form for \FDA
\linethickness{0.5mm} \line(1,0){425}
\newline
\textit{Form:}
$\tt isffm$:$\langle \tt{instance\ name}\rangle$
$n_1\ n_2\ $ $\langle \tt{parameter\ list}\rangle$
\newline
\begin{tabular}{r l}
$n_1$ & is the positive element node, \\
&  \\
$n_2$ & is the negative element node. \\
%&  \\
%parameter list & see table 1 for parameter list
\end{tabular}
% Parameter list
\newline
\textit{Parameters:}
\begin{table}[H]
\begin{tabular}{|c|c|c|c|}
\hline
Parameter&Type&Default value&Required?\\
\hline
io: Offset current(A) & DOUBLE & 0 & no\\
\hline
ia: RMS current amplitude (A) & DOUBLE & 0 & no\\
\hline
fcarrier: AC frequency (Hz) & DOUBLE & 0 & no\\
\hline
mdi: modulation index (Dimensionless) & DOUBLE & 0 & no\\
\hline
fsignal: Signal frequency (Hz) & DOUBLE & 0 & no\\
\par
\hline
\end{tabular}
\end{table}
% example in \FDA
%\newline
\noindent\linethickness{0.5mm}\line(1,0){425}
\newline
\textit{Example:}
\newline
\texttt{isffm:isignal\ 8\ 0\ io=0.2 ia=0.7 fcarrier=4 mdi=0.9
fsignal=1}
\newline
\linethickness{0.5mm} \line(1,0){425}
\newline
\textit{Description:}\\
The waveform shape for this source is
\begin{equation}
i = i_o + i_a[\sin(2.\pi.f_{carrier}.t) +
mdi\sin(2.\pi.f_{signal}.t)]
\end{equation}
\begin{figure}[hbp]
\centerline{\epsfxsize=3in\pfig{vsffm.eps}} \caption{Current
source single frequency frequency modulation waveform for
\texttt{isffm:isignal\ 8\ 0\ io=0.2 ia=0.7 fcarrier=4 mdi=0.9
fsignal=1}}
\end{figure}
\newline
\linethickness{0.5mm} \line(1,0){425}
\newline
\textit{Notes:}\\
This is the \texttt{I} element in the SPICE compatible netlist.\\
\linethickness{0.5mm} \line(1,0){425}
\newline
\textit{Version:}\\
2002.05.15 \\
% Credits
\newpage
\noindent\linethickness{0.5mm}\line(1,0){425}
\newline
\textit{Credits:}\\
\begin{tabular}{l l l l}
Name & Affiliation & Date & Links \\
Satish Uppathil & NC State University & May 2002 & \epsfxsize=1in\epsfbox{figures/logo.eps} \\
svuppath@eos.ncsu.edu & & & www.ncsu.edu    \\
\end{tabular}
\end{document}

\caption[Current source single frequency frequency modulation
({\tt SFFM}) waveform]{Current source single frequency frequency
modulation ({\tt SFFM}) waveform for\newline \hspace*{\fill}{\tt
SFFM(0.2 0.7 4 0.9 1)}\hspace*{\fill}. \label{fig:isffm}}
\end{figure}
\newline
\linethickness{0.5mm} \line(1,0){425}
\newline
\textit{Notes:}\\
The actual element in \FDA is the \texttt{isffm} element.
See \texttt{isffm} for full documentation.\\
\newline

\underline{\bf{Amplitude Modulation}}:\\
%%%%%%%%%%%%%%%%%%%%%%put the SIN form here%%%%%%%%%%%%%%%%%%%%
\texttt{AM\ (sa\ oc\ fm\ fc\ td)}\\
\textit{Parameters:}
\begin{table}[h]
\begin{tabular}{|c|c|c|c|}
\hline
Name&Description&Units&Default\\
\hline
sa & signal amplitude & V & 0.0\\
\hline
fc & carrier frequency & Hz & 0.0\\
\hline
fm & modulation frequency & Hz & 1/{\texttt{TSTOP}}\\
\hline
oc & offset constant & dimensionless & 0.0\\
\hline
td & delay time before start of signal & s & 0.0\\
\par
\hline
\end{tabular}
\end{table}
%%%%%%%%%%%%%%%%%%%%%%%%%%%%%%%%%%%%%%% example in \FDA
\newline
\linethickness{0.5mm} \line(1,0){425}
\newline
\textit{Example:}
\newline
\texttt{ISIGNAL 20 5 AM(10 1 100 1K 1M)}
\newline
\linethickness{0.5mm} \line(1,0){425}
\newline
\textit{Description:}\\
The waveform for this source is
\begin{equation}
i = sa(oc + \sin[2.\pi.fm.(t - td)])sin[2.\pi.fc.(t - td)]
\end{equation}
%\begin{figure}[h]
%\centerline{\epsfxsize=3in\pfig{vsin.eps}} \caption{Voltage source
%transient sine waveform for \texttt{SIN(0.1 0.8 2 1 0.3)}}
%\end{figure}
\newline
\linethickness{0.5mm} \line(1,0){425}
\newline
\textit{Notes:}\\
The actual element in \FDA is the \texttt{iam} element.
See \texttt{iam} for full documentation.\\
%%%%%%%%%%%%%%%%%%%%%%%%%%%%%%%%%%%%%%%%%%%%% Credits
\linethickness{0.5mm} \line(1,0){425}
\newline
\textit{Credits:}\\
\begin{tabular}{l l l l}
Name & Affiliation & Date & Links \\
Satish Uppathil & NC State University & Sept 2000 & \epsfxsize=1in\pfig{logo.eps} \\
svuppath@eos.ncsu.edu & & & www.ncsu.edu    \\
\end{tabular}
%\end{document}
