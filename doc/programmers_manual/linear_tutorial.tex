% This document was generated automagically by lgrind.  DO NOT EDIT.
%
% Adding elements to \FDA tutorial
%
%\documentclass[12pt,titlepage]{article}
%\usepackage{epsf}
%\usepackage[nolineno]{lgrind}
%\usepackage{lgrind}
% figures in the current directory
%\newcommand{\fig}[1]{#1}
% figures in the erl space
%\newcommand{\fig}[1]{../figures/#1}
%\newcommand{\newfig}[1]{\epsffile{\fig{#1}}}
%\input{J:/eos.ncsu.edu/users/m/mbs/mbs_group/work/book/freeda_manual/freeda_elements/common/freeda.sty}
%\input{../freeda_manual/freeda_elements/common/freeda.sty}

\topmargin 0mm \oddsidemargin -2mm \evensidemargin -2mm \textwidth
173mm \textheight 210mm

\chapter{Adding Linear Elements to \FDA} \label{linear:section}
%\author{Carlos E. Christoffersen}
%\begin{document}
%\maketitle
%----------------------------------------------------------------------
\section{The \FDA Circuit Simulator}

\FDA is a netlist-based circuit simulator. The input format of the
netlist file is similar to the \texttt{SPICE} format with
extensions for new device models, variables, sweeps, and
repetitive simulation. The program provides a variety of output
data and plots. The addition of new circuit element models and
analysis types in \FDA is much simpler than in other circuit
simulators such as Spice. For example, new element models are
coded and incorporated into the program without modification to
the high-level simulator. The circuit analysis types currently
available in \FDA are DC, AC, harmonic balance (HB) \cite{svhb},
convolution transient \cite{svtr}, wavelet transient
\cite{wavelet}, and time-marching transient \cite{carlos_phd}.
Some insight into the program architecture is given in
\cite{oopaper}.

This tutorial describes the addition of linear device models to
\FDA \footnote{Further details may be found at
http://www.freeda.org}. We assume the reader is familiar with C++
syntax and basic concepts of object-oriented programming. Some
issues such as the creation of an element with multiple reference
nodes are not yet described.

In the majority of cases the code for a new element can be written
by following the code for an existing similar element. For
example, to write a new MOS transistor mode the code for an
existing MOS model can be followed.
%----------------------------------------------------------------------
\section{Example: Adding a Linear Resistor}

We illustrate the addition of linear device models using a
step-by-step example with a simple model: a linear resistor.

\subsection{Netlist syntax}
A brief description of the netlist will help in understanding the
rest of this section.

The standard \FDA netlist syntax is common to all elements. This
differs from the \texttt{SPICE} syntax but this is also supported
but is less general in that the grammar of the standard \FDA
netlist syntax is common to all elements while the grammar of the
\texttt{SPICE} syntax is not consistent and each \texttt{SPICE}
element must be handled separately in the parser.

The standard \FDA netlist syntax is\\

$\langle \tt{netlist\ Name} \rangle$:$\langle \tt{element\
ID}\rangle$ $\langle \tt{terminal1}\rangle \cdots \langle
\tt{terminaln}\rangle$ $[\langle \tt{parameter} \rangle = \langle
\tt{value} \rangle \cdots]$
\newline
\newline
The \FDA netlist syntax for a resistor (the {\bf r} element) is,
for example, {\bf r:r1\ n1\ n2\ r=100.} \\
\newline
{\bf r:1} is the {\bf elementID}, {\bf n1} and {\bf n2} are the
names of two terminals and {\bf r} is a parameter syntax name.

\texttt{SPICE} syntax is also supported for the original
\texttt{SPICE} elements. In \texttt{SPICE} syntax the equivalent
input is {\bf r1\ n1\ n2\ 100}.

\subsection{Class Name, Required Files, etc.}

A good class name for the element is \textbf{Resistor}. By
convention (in \FDA), class names should begin with capital
letters and contain no underscores. We need a netlist name for our
element. Assume we call it ``r''.

The files containing the model description are located in the
\texttt{elements/} directory. There is a header file
(\texttt{Resistor.h}) containing the declaration of the class that
defines the element and a file containing the definition of the
class with the actual model (\texttt{Resistor.cc}). Those are the
only files needed to define this element.

\subsection{The Header File}

The header file starts with comment lines describing the element
type and sometimes a simple ASCII drawing of the element
schematic. The figure can be used to describe terminal numbering.

\LGinlinefalse \lgrinde
\L{\LB{//_\V{This}_\V{may}_\V{look}_\V{like}_\V{C}_\V{code},_\V{but}_\V{it}_\V{is}_\V{really}_\-*\-_\V{C}++_\-*\-}}
\L{\LB{//}}
\L{\LB{//_\V{This}_\V{is}_\V{an}_\V{resistor}_\V{model}}}
\L{\LB{//}} \L{\LB{//}\Tab{18}{\V{res}}}
\L{\LB{//}\Tab{11}{\V{o}\-\-\-\-/\2/\2/\2\-\-\-\-\V{o}}}
\L{\LB{//}}
\L{\LB{//_\V{by}_\V{Carlos}_\V{E}._\V{Christoffersen}}}
\endlgrinde

Header files may be included more than once in C++ programs. To
avoid multiple declarations of the classes defined in the body of
the header file, the definitions in the header file are enclosed
by following preprocessor directives:

\LGinlinefalse \lgrinde \L{\LB{\K{\#ifndef}_\V{Resistor\_h}}}
\L{\LB{\K{\#define}_\V{Resistor\_h}}}
\L{\LB{\V{class}_\V{Resistor}_:_\V{public}}}
\L{\LB{\V{Element}_\{}} \Line{__\vdots} \L{\LB{\}}}
\L{\LB{\K{\#endif}}}
\endlgrinde

The \textbf{Resistor} class is derived from the base class
\textbf{Element}. This is so for all elements in \FDA. For simple
elements, the only public functions that must be declared are the
following:

\LGinlinefalse \lgrinde
\L{\LB{\V{class}_\V{Resistor}_:_\V{public}_\V{Element}}}
\L{\LB{\{}} \L{\LB{\V{public}:}} \L{\LB{}}
\L{\LB{}\Tab{2}{\V{Resistor}(\V{const}_\V{string}\&_\V{iname});}}
\L{\LB{}}
\index{Resistor}\Proc{Resistor}\L{\LB{}\Tab{2}{\~{}\V{Resistor}()_\{\}}}
\L{\LB{}}
\index{getNetlistName}\Proc{getNetlistName}\L{\LB{}\Tab{2}{\K{static}_\V{const}_\K{char}*_\V{getNetlistName}()}}
\L{\LB{}\Tab{2}{\{}}
\L{\LB{}\Tab{4}{\K{return}_\V{einfo}.\V{name};}}
\L{\LB{}\Tab{2}{\}}} \L{\LB{}}
\L{\LB{}\Tab{2}{//_\V{fill}_\V{MNAM}}}
\L{\LB{}\Tab{2}{\V{virtual}_\K{void}_\V{fillMNAM}(\V{FreqMNAM}*_\V{mnam});}}
\L{\LB{}}
\L{\LB{}\Tab{2}{\V{virtual}_\K{void}_\V{fillMNAM}(\V{TimeMNAM}*_\V{mnam});}}
\endlgrinde

The constructor always takes the same argument, \texttt{iname}. This
is the name of the particular instance of the resistor being created
(\emph{e.g.} ``r1'').

The destructor in this case is trivial. The next function,
\texttt{getNetlistName()} must be defined in every \FDA element.
It returns the netlist name of the element (``r'' in our example)
and it is used during the netlist parsing. It is declared static
because this function is called before the actual element instance
is created.

The last two functions are the declarations of the functions to
fill the entries corresponding to this element in the
\emph{modified nodal admittance matrix} (MNAM) of the circuit
being simulated.

The private members of this class are:

\LGinlinefalse \lgrinde \L{\LB{\V{private}:}} \L{\LB{}}
\L{\LB{}\Tab{2}{//_\V{Parameter}_\V{variables}}}
\L{\LB{}\Tab{2}{\K{double}_\V{r};}} \L{\LB{}}
\L{\LB{}\Tab{2}{//_\V{Element}_\V{information}}}
\L{\LB{}\Tab{2}{\K{static}_\V{ItemInfo}_\V{einfo};}} \L{\LB{}}
\L{\LB{}\Tab{2}{//_\V{Number}_\V{of}_\V{parameters}_\V{of}_\V{this}_\V{element}}}
\L{\LB{}\Tab{2}{\K{static}_\V{const}_\K{unsigned}_\V{n\_par};}}
\L{\LB{}} \L{\LB{}\Tab{2}{//_\V{Parameter}_\V{information}}}
\L{\LB{}\Tab{2}{\K{static}_\V{ParmInfo}_\V{pinfo}[\,];}} \L{\LB{}}
\L{\LB{\};}}
\endlgrinde

The netlist parameters of this element are declared as normal
member variables. Here we have only one parameter called ``r''.
There is no conflict with the netlist name of this element. The
last three variables can be left unchanged for any other element
in \FDA. We will describe their use later.


\subsection{The Class Source File}

The include preprocessor directives go first. The
\texttt{ElementManager.h} file must be always included first. For
linear elements, we also need to include the declarations for the
MNAM classes. There are currently two implementations of the MNAM
in \FDA. One in the frequency domain (\texttt{FreqMNAM}) and one
in the time domain (\texttt{TimeMNAM}). The last included file is
the declaration of the class for our element, \texttt{Resistor.h}.
\LGinlinefalse \lgrinde
\L{\LB{\K{\#include}_\S{}\3.\,./network/ElementManager.h\3\SE{}}}
\L{\LB{\K{\#include}_\S{}\3.\,./analysis/FreqMNAM.h\3\SE{}}}
\L{\LB{\K{\#include}_\S{}\3.\,./analysis/TimeMNAM.h\3\SE{}}}
\L{\LB{\K{\#include}_\S{}\3Resistor.h\3\SE{}}}
\endlgrinde

Now we must define the static member variables:

\LGinlinefalse \lgrinde \L{\LB{//_\V{Static}_\V{members}}}
\L{\LB{\V{const}_\K{unsigned}_\V{Resistor}::\V{n\_par}_=_\N{1};}}
\L{\LB{}} \L{\LB{//_\V{Element}_\V{information}}}
\L{\LB{\V{ItemInfo}_\V{Resistor}::\V{einfo}_=_\{}}
\L{\LB{}\Tab{2}{\S{}\3r\3\SE{},}}
\L{\LB{}\Tab{2}{\S{}\3Resistor\3\SE{},}}
\L{\LB{}\Tab{2}{\S{}\3Carlos_E._Christoffersen\3\SE{},}}
\L{\LB{}\Tab{2}{\V\S{}\3\3\SE{}}}
\L{\LB{\};}} \L{\LB{}} \L{\LB{//_\V{Parameter}_\V{information}}}
\L{\LB{\V{ParmInfo}_\V{Resistor}::\V{pinfo}[\,]_=_\{}}
\L{\LB{}\Tab{2}{\{\S{}\3r\3\SE{},_\S{}\3Resistance_value_(Ohms)\3\SE{},_\V{TR\_DOUBLE},_\V{true}\}}}
\L{\LB{\};}}
\endlgrinde

"{\bf r}" under Element information is the NetlistName. "{\bf r}"
under Parameter information is where the parameter name is
defined.

The \texttt{n\_par} variable must be equal to the number of
parameters defined for the element. In our example, only one
parameter is defined. Note the use of the scope (::) operator. It
is used to indicate that the \texttt{n\_par} variable is a member
of the \textbf{Resistor} class.  The \texttt{einfo} structure
contains strings describing the element. The first string is the
netlist name of the element (``r''). This name must always be
given in lowercase letters. The second string is a more verbose
description of the element. The third string should list the
authors of the element code and the last string is used to store a
web/ftp/e-mail address where more information about the element
can be found. The idea here is give full exposure to the creator
of an element and that an element is 'owned' by the person or
group that creates the element.

The \texttt{pinfo} vector contains the description of each
parameter defined in the element. The first field (always in
lowercase) is the netlist name of the parameter followed by a more
verbose description. The units of the parameter (if any) should be
included in this description. The third field is a flag denoting
the type of the parameter. The possible flags are defined in the
\texttt{NetListItem.h} file.  The corresponding types for each
flag are given in Table \ref{ltable1}.
%
\begin{table}[htpb]
\centerline {
\begin{tabular}{|l|l|}
\hline
Flag & Corresponding type \\
\hline
TR\_INT  & \texttt{int} \\
TR\_LONG & \texttt{long int} \\
TR\_FLOAT & \texttt{float} \\
TR\_DOUBLE & \texttt{double} \\
TR\_CHAR & \texttt{char} \\
TR\_STRING & \texttt{string} \\
TR\_COMPLEX & \texttt{double\_complex} \\
TR\_BOOLEAN & \texttt{bool} \\
TR\_INT\_VECTOR & \texttt{IntVector} \\
TR\_DOUBLE\_VECTOR & \texttt{DoubleVector} \\
TR\_INT\_MATRIX & \texttt{IntMatrix} \\
TR\_DOUBLE\_MATRIX & \texttt{DoubleMatrix} \\
\hline
\end{tabular}}
\caption{Possible parameter types. \label{ltable1}}
\end{table}
%
The last field in the parameter description is a flag that is
\emph{true} if the parameter is required in the netlist (\emph{i.e.}
an error occurs if the parameter is not specified) and \emph{false}
otherwise.

The constructor definition follows:

\LGinlinefalse \lgrinde
\index{Resistor}\Proc{Resistor}\L{\LB{\V{Resistor}::\V{Resistor}(\V{const}_\V{string}\&_\V{iname})_:_\V{Element}(\&\V{einfo},_\V{pinfo},_\V{n\_par},}}
\L{\LB{\V{iname})}} \L{\LB{\{}}
\L{\LB{}\Tab{2}{//_\V{Value}_\V{of}_\V{r}_\V{is}_\V{required}}}
\L{\LB{}\Tab{2}{\V{paramvalue}[\N{0}]_=_\&\V{res};}} \L{\LB{}}
\L{\LB{}\Tab{2}{//_\V{Set}_\V{the}_\V{number}_\V{of}_\V{terminals}}}
\L{\LB{}\Tab{2}{\V{setNumTerms}(\N{2});}} \L{\LB{}}
\L{\LB{}\Tab{2}{//_\V{Set}_\V{flags}}}
\L{\LB{}\Tab{2}{\V{setFlags}(\V{LINEAR}_\|_\V{ONE\_REF}_\|_\V{TR\_FREQ\_DOMAIN});}}
\L{\LB{\}}}
\endlgrinde

The constructor for the \textbf{Element} class takes the static
member variables we defined before and the instance name
(\texttt{iname}) as arguments. In the body of the \textbf{Resistor}
constructor we must point each element of the \texttt{paramvalue}
vector to the address of the member variables corresponding to the
netlist parameters. The order of the elements in the
\texttt{paramvalue} vector must be the same order that we used in the
description in \texttt{pinfo}.

The member function \texttt{setnumTerms()}, which is derived from
the \textbf{Element} class, is used to set how many terminals our
element has.

The last function used in this constructor sets some flags that are
useful to classify the different elements in a circuit. Table
\ref{ltable2} summarizes the meaning of each possible flag.
%
\begin{table}[htpb]
\centerline {
\begin{tabular}{|l|l|}
\hline
Flag & Meaning \\
\hline
LINEAR    &  Linear in HB and tran2 analyses. \\
NONLINEAR &  Nonlinear in HB and tran2 analyses. \\
\hline
ONE\_REF   &  One internal reference node (\emph{e.g.} a normal element). \\
MULTI\_REF &  Two or more internal reference nodes. \\
\hline
TR\_TIME\_DOMAIN & The element is treated in time domain in convolution \\
               & transient analysis. This includes nonlinear elements. \\
TR\_FREQ\_DOMAIN & The element is treated in frequency domain in convolution
\\
               & transient analysis. This includes linear elements. \\
\hline
SOURCE    & The element is a source in tran2 analysis. \\
\hline
\end{tabular}}
\caption{Currently defined flags. \label{ltable2}}
\end{table}
%

\subsection{Filling the Modified Nodal Admittance Matrix}

The last two methods in the source file are the routines used to
fill the MNAM of the circuit. The first function takes a pointer
to an object of the \textbf{FreqMNAM} class.

\LGinlinefalse \lgrinde
\index{fillMNAM}\Proc{fillMNAM}\L{\LB{\K{void}_\V{Resistor}::\V{fillMNAM}(\V{FreqMNAM}*_\V{mnam})}}
\L{\LB{\{}}
\L{\LB{}\Tab{2}{//_\V{Ask}_\V{my}_\V{terminals}_\V{the}_\V{row}_\V{numbers}}}
\L{\LB{}\Tab{2}{\V{mnam}\-\!\>\V{setAdmittance}(\V{getTerminal}(\N{0})\-\!\>\V{getRC}(),_\V{getTerminal}(\N{1})\-\!\>\V{getRC}(),}}
\L{\LB{}\Tab{14}{\V{one}/\V{res});}} \L{\LB{\}}}
\endlgrinde

Here {\bf setAdmittance} sets the four element stamp of an
admittance {\bf g}. Thus the stamp is
\begin{equation}
   \left[  \begin{array}{c c}
               g & -g\\
               -g & g\\
          \end{array}    \right]
\end{equation}

The first terminal of the element is obtained using {\bf
getTerminal(0)} and the second terminal by {\bf getTerminal(1)}.
{\bf getRC()} gets the row/column index and $g = 1/r$ (i.e.
one/res) is the conductance of the element.

This is a class to represent a set of modified nodal admittance
matrices in the frequency domain with their respective source
vectors. The following is the list of functions available to fill
the matrices and the source vectors:

\LGinlinefalse \lgrinde
\L{\LB{}\Tab{2}{//\-\-\-\-\-\-\-\-\-\-\-\-\-\-\-\-\-\-\-\-\-\-\-\-\-\-\-\-\-\-\-\-\-\-\-\-\-\-\-\-\-\-\-\-\-\-\-\-\-\-\-\-\-\-\-\-\-\-\-\-\-\-}}
\L{\LB{}\Tab{2}{//_\V{Methods}_\V{to}_\V{fill}_\V{the}_\V{matrices}_\V{used}_\V{by}_\V{the}_\V{elements}}}
\L{\LB{}\Tab{2}{//\-\-\-\-\-\-\-\-\-\-\-\-\-\-\-\-\-\-\-\-\-\-\-\-\-\-\-\-\-\-\-\-\-\-\-\-\-\-\-\-\-\-\-\-\-\-\-\-\-\-\-\-\-\-\-\-\-\-\-\-\-\-}}
\L{\LB{}}
\index{setElement}\Proc{setElement}\L{\LB{}\Tab{2}{\V{inline}_\K{void}_\V{setElement}(\V{const}_\K{unsigned}\&_\V{row},_\V{const}_\K{unsigned}\&_\V{col},}}
\L{\LB{}\Tab{13}{\V{const}_\V{double\_complex}\&_\V{val});}}
\endlgrinde\LGend
\LGinlinefalse\LGbegin\lgrinde[1]
\L{\LB{}\Tab{2}{\V{inline}_\K{void}_\V{setAdmittance}(\V{const}_\K{unsigned}\&_\V{term1},_\V{const}_\K{unsigned}\&_\V{term2},}}
\L{\LB{}\Tab{16}{\V{const}_\V{double\_complex}\&_\V{val});}}
\endlgrinde\LGend
\LGinlinefalse\LGbegin\lgrinde[1]
\L{\LB{}\Tab{2}{\V{inline}_\K{void}_\V{setQuad}(\V{const}_\K{unsigned}\&_\V{row1},_\V{const}_\K{unsigned}\&_\V{row2},}}
\L{\LB{}\Tab{14}{\V{const}_\K{unsigned}\&_\V{col1},_\V{const}_\K{unsigned}\&_\V{col2},}}
\L{\LB{}\Tab{14}{\V{const}_\V{double\_complex}\&_\V{val});}}
\endlgrinde\LGend
\LGinlinefalse\LGbegin\lgrinde[1]
\L{\LB{}\Tab{2}{\V{inline}_\K{void}_\V{setOnes}(\V{const}_\K{unsigned}\&_\V{pos},_\V{const}_\K{unsigned}\&_\V{neg},}}
\L{\LB{}\Tab{14}{\V{const}_\K{unsigned}\&_\V{eqn});}}
\endlgrinde\LGend
\LGinlinefalse\LGbegin\lgrinde[1]
\L{\LB{}\Tab{2}{//\-\-\-\-\-\-\-\-\-\-\-\-\-\-\-\-\-\-\-\-\-\-\-\-\-\-\-\-\-\-\-\-\-\-\-\-\-\-\-\-\-\-\-\-\-\-\-\-\-\-\-\-\-\-\-\-\-\-\-\-\-\-}}
\L{\LB{}\Tab{2}{//_\V{Methods}_\V{to}_\V{fill}_\V{the}_\V{source}_\V{vectors}}}
\L{\LB{}\Tab{2}{//\-\-\-\-\-\-\-\-\-\-\-\-\-\-\-\-\-\-\-\-\-\-\-\-\-\-\-\-\-\-\-\-\-\-\-\-\-\-\-\-\-\-\-\-\-\-\-\-\-\-\-\-\-\-\-\-\-\-\-\-\-\-}}
\L{\LB{}}
\L{\LB{}\Tab{2}{//_\V{Set}_\V{one}_\V{element}_\V{of}_\V{the}_\V{current}_\V{source}_\V{vector}}}
\L{\LB{}\Tab{2}{\K{void}_\V{setSource}(\V{const}_\K{unsigned}\&_\V{row},_\V{const}_\V{double\_complex}\&_\V{val});}}
\L{\LB{}}
\L{\LB{}\Tab{2}{//_\V{Add}_\V{value}_\V{to}_\V{a}_\V{pair}_\V{of}_\V{elements}_\V{of}_\V{the}_\V{current}_\V{source}_\V{vector}}}
\L{\LB{}\Tab{2}{\K{void}_\V{addToSource}(\V{const}_\K{unsigned}\&_\V{pos},_\V{const}_\K{unsigned}\&_\V{neg},}}
\L{\LB{}\Tab{11}{\V{const}_\V{double\_complex}\&_\V{val});}}
\endlgrinde


The \textbf{FreqMNAM} class also provides a set of methods to
retrieve information about the matrices. One of the most used in
the element routines is:

\LGinlinefalse \lgrinde
\L{\LB{}\Tab{2}{//_\V{Get}_\V{the}_\V{current}_\V{frequency}}}
\L{\LB{}\Tab{2}{\V{inline}_\V{const}_\K{double}\&_\V{getFreq}();}}
\endlgrinde

This method returns the frequency at which the the MNAM is
expected to be calculated.

The last method is used to fill the elements in the time-domain
MNAM.

\LGinlinefalse \lgrinde
\L{\LB{\K{void}_\V{Resistor}::\V{fillMNAM}(\V{TimeMNAM}*_\V{mnam})}}
\L{\LB{\{}}
\L{\LB{}\Tab{2}{//_\V{Ask}_\V{my}_\V{terminals}_\V{the}_\V{row}_\V{numbers}}}
\L{\LB{}\Tab{2}{\V{mnam}\-\!\>\V{setMAdmittance}(\V{getTerminal}(\N{0})\-\!\>\V{getRC}(),_\V{getTerminal}(\N{1})\-\!\>\V{getRC}(),}}
\L{\LB{}\Tab{15}{\V{one}/\V{res});}} \L{\LB{\}}}
\endlgrinde

For a simple element such as our resistor example, this function
looks very similar to the frequency-domain case. Nevertheless, the
interfaces of the \textbf{FreqMNAM} and \textbf{TimeMNAM} classes
are not exactly the same. The following are the methods available
to fill the MNAM in the time domain:

\LGinlinefalse \lgrinde
\L{\LB{}\Tab{2}{//\-\-\-\-\-\-\-\-\-\-\-\-\-\-\-\-\-\-\-\-\-\-\-\-\-\-\-\-\-\-\-\-\-\-\-\-\-\-\-\-\-\-\-\-\-\-\-\-\-\-\-\-\-\-\-\-\-\-\-\-\-\-}}
\L{\LB{}\Tab{2}{//_\V{Methods}_\V{to}_\V{fill}_\V{the}_\V{matrices}_\V{used}_\V{by}_\V{the}_\V{elements}}}
\L{\LB{}\Tab{2}{//\-\-\-\-\-\-\-\-\-\-\-\-\-\-\-\-\-\-\-\-\-\-\-\-\-\-\-\-\-\-\-\-\-\-\-\-\-\-\-\-\-\-\-\-\-\-\-\-\-\-\-\-\-\-\-\-\-\-\-\-\-\-}}
\L{\LB{}}
\L{\LB{}\Tab{2}{//_\V{Methods}_\K{for}_\V{filling}_\V{M}}}
\index{setMElement}\Proc{setMElement}\L{\LB{}\Tab{2}{\V{inline}_\K{void}_\V{setMElement}(\V{const}_\K{unsigned}\&_\V{row},_\V{const}_\K{unsigned}\&_\V{col},}}
\L{\LB{}\Tab{14}{\V{const}_\K{double}\&_\V{val});}}
\endlgrinde\LGend
\LGinlinefalse\LGbegin\lgrinde[8]
\L{\LB{}\Tab{2}{\V{inline}_\K{void}_\V{setMAdmittance}(\V{const}_\K{unsigned}\&_\V{term1},_\V{const}_\K{unsigned}\&_\V{term2},}}
\L{\LB{}\Tab{17}{\V{const}_\K{double}\&_\V{val});}}
\endlgrinde\LGend
\LGinlinefalse\LGbegin\lgrinde[10]
\L{\LB{}\Tab{2}{\V{inline}_\K{void}_\V{setMQuad}(\V{const}_\K{unsigned}\&_\V{row1},_\V{const}_\K{unsigned}\&_\V{row2},}}
\L{\LB{}\Tab{15}{\V{const}_\K{unsigned}\&_\V{col1},_\V{const}_\K{unsigned}\&_\V{col2},}}
\L{\LB{}\Tab{15}{\V{const}_\K{double}\&_\V{val});}}
\endlgrinde\LGend
\LGinlinefalse\LGbegin\lgrinde[13]
\L{\LB{}\Tab{2}{\V{inline}_\K{void}_\V{setMOnes}(\V{const}_\K{unsigned}\&_\V{pos},_\V{const}_\K{unsigned}\&_\V{neg},}}
\L{\LB{}\Tab{15}{\V{const}_\K{unsigned}\&_\V{eqn});}}
\endlgrinde\LGend
\LGinlinefalse\LGbegin\lgrinde[15]
\L{\LB{}\Tab{2}{//_\V{Methods}_\K{for}_\V{filling}_\V{Mp}}}
\L{\LB{}\Tab{2}{\V{inline}_\K{void}_\V{setMpElement}(\V{const}_\K{unsigned}\&_\V{row},_\V{const}_\K{unsigned}\&_\V{col},}}
\L{\LB{}\Tab{15}{\V{const}_\K{double}\&_\V{val});}}
\endlgrinde\LGend
\LGinlinefalse\LGbegin\lgrinde[18]
\L{\LB{}\Tab{2}{\V{inline}_\K{void}_\V{setMpAdmittance}(\V{const}_\K{unsigned}\&_\V{term1},_\V{const}_\K{unsigned}\&_\V{term2},}}
\L{\LB{}\Tab{16}{\V{const}_\K{double}\&_\V{val});}}
\endlgrinde\LGend
\LGinlinefalse\LGbegin\lgrinde[20]
\L{\LB{}\Tab{2}{\V{inline}_\K{void}_\V{setMpQuad}(\V{const}_\K{unsigned}\&_\V{row1},_\V{const}_\K{unsigned}\&_\V{row2},}}
\L{\LB{}\Tab{12}{\V{const}_\K{unsigned}\&_\V{col1},_\V{const}_\K{unsigned}\&_\V{col2},}}
\L{\LB{}\Tab{12}{\V{const}_\K{double}\&_\V{val});}}
\endlgrinde\LGend
\LGinlinefalse\LGbegin\lgrinde[23]
\L{\LB{}\Tab{2}{\V{inline}_\K{void}_\V{setMpOnes}(\V{const}_\K{unsigned}\&_\V{pos},_\V{const}_\K{unsigned}\&_\V{neg},}}
\L{\LB{}\Tab{12}{\V{const}_\K{unsigned}\&_\V{eqn});}}
\endlgrinde\LGend
\LGinlinefalse\LGbegin\lgrinde[25]
\L{\LB{}\Tab{2}{//\-\-\-\-\-\-\-\-\-\-\-\-\-\-\-\-\-\-\-\-\-\-\-\-\-\-\-\-\-\-\-\-\-\-\-\-\-\-\-\-\-\-\-\-\-\-\-\-\-\-\-\-\-\-\-\-\-\-\-\-\-\-}}
\L{\LB{}\Tab{2}{//_\V{Methods}_\V{to}_\V{fill}_\V{the}_\V{source}_\V{vector}}}
\L{\LB{}\Tab{2}{//\-\-\-\-\-\-\-\-\-\-\-\-\-\-\-\-\-\-\-\-\-\-\-\-\-\-\-\-\-\-\-\-\-\-\-\-\-\-\-\-\-\-\-\-\-\-\-\-\-\-\-\-\-\-\-\-\-\-\-\-\-\-}}
\L{\LB{}\Tab{2}{//_\V{Reminder}:_\V{reference}_\V{source}_\V{element}_\V{is}_\V{not}_\V{stored}}}
\L{\LB{}}
\L{\LB{}\Tab{2}{//_\V{Set}_\V{one}_\V{element}_\V{of}_\V{the}_\V{source}_\V{vector}}}
\L{\LB{}\Tab{2}{\V{inline}_\K{void}_\V{setSource}(\V{const}_\K{unsigned}\&_\V{row},_\V{const}_\K{double}\&_\V{val});}}
\endlgrinde\LGend
\LGinlinefalse\LGbegin\lgrinde[32]
\L{\LB{}\Tab{2}{//_\V{Add}_\V{value}_\V{to}_\V{a}_\V{pair}_\V{of}_\V{elements}_\V{of}_\V{the}_\V{source}_\V{vector}}}
\L{\LB{}\Tab{2}{\V{inline}_\K{void}_\V{TimeMNAM}::\V{addToSource}(\V{const}_\K{unsigned}\&_\V{pos},_\V{const}_\K{unsigned}\&}}
\L{\LB{\V{neg},_\V{const}_\K{double}\&_\V{val});}}
\endlgrinde\LGend
\LGinlinefalse\LGbegin\lgrinde[35]
\L{\LB{}\Tab{2}{//_\V{Get}_\V{the}_\V{current}_\V{time}}}
\L{\LB{}\Tab{2}{\V{inline}_\V{const}_\K{double}\&_\V{getTime}();}}
\endlgrinde\LGend

Note that we must build two matrices denoted in the source code as
\texttt{M} and \texttt{Mp}. These correspond to the $\mathbf{G}$
and $\mathbf{C}$ matrices in Chapter 3 of Reference
\cite{carlos_phd}. Another difference with the frequency domain
case is that the source vector is handled separately, as it will
demonstrated in a future example.

\subsection{Modifications to the rest of the \FDA Source Files}

%The only modification required to add or remove elements from \FDA
%\ is (apart from writing the element source code itself) including
%the element source code name in the \texttt{makefile.list} file in
%the \texttt{elements/} directory. An example is included here:
%\begin{verbatim}
%
%# Hey emacs, this is -*- makefile -*-
%#
%# Copy to makefile.list
%
%# List here all the ".cc" files to be compiled and linked into
%fREEDA
%# SRCS = Xsubckt.cc Resistor.cc Capacitor.cc Inductor.cc
%Vsource.cc \
%    Diode.cc Tlinp4.cc Ymatrix.cc GridEx.cc Isource.cc Vccs.cc NPort.cc \
%    CPW.cc Open.cc MesfetM.cc Vct.cc Vpulse.cc MesfetC.cc \
%    TMesfetC.cc ThermalMMIC.cc ThermalGrid.cc Thermal2.cc \
%    BillISource.cc ThermalNport.cc NPortT.cc ThermalTest.cc \
%    ThermalTransf.cc Jfetn1.cc MOS3.cc
%
%# List element names (class names) here to be included in the
%# automatically generated header files. Remember for this to work the
%# base names of the header files must be equal to the class names.
%#
%ELEMNAMES = Xsubckt Resistor Capacitor Inductor Vsource \
%    Diode Tlinp4 Ymatrix GridEx Isource \
%    Vccs NPort CPW Open MesfetM Vct Vpulse MesfetC \
%    TMesfetC ThermalMMIC ThermalGrid Thermal2 BillISource ThermalNport \
%    NPortT ThermalTest ThermalTransf Jfetn1 MOS3
%
%\end{verbatim}
%Normal makefile syntax is used to define the \texttt{SRCS} and
%\texttt{ELEMNAMES} variables. The comments in the file provide a clue
%of what to include in each list. We should emphasize that the class
%name of the element must be equal to the base filename of the header
%file. Otherwise, the make process will fail.
%
%Once a new element is added or removed, it is advised to run the
%command ``make dep'' from the \texttt{simulator/} directory. During
%element development, it is not necessary to build the complete program
%each time we are trying to find errors in the compilation. The make
%program can be invoked from a source subdirectory.
%
%In the above makefile the entries required to include the resistor
%element are {\tt Resistor.cc} in the line beginning {\tt \# SRCS
%=} and {\tt Resistor} in the line beginning {\tt ELEMNAMES =}. Of
%course {\tt Resistor.h} must exist as well.
%
% Begin "updated" info...
Refer to the file \texttt{readme.txt} in the \texttt{freeda-0.1}
subdirectory for instructions on compiling a new element into \FDA.

%----------------------------------------------------------------------
\section{Template Files for New Linear Elements}

The files in this section are provided to simplify the creation of
new elements. All that is required is to replace \texttt{Element1}
by the name of the actual class (or name of the element), copy the
files with appropriate names in the \texttt{elements/} directory
and write the code to fill the MNAM.

\subsection{Header File}


\LGinlinefalse \lgrinde
\L{\LB{//_\V{This}_\V{may}_\V{look}_\V{like}_\V{C}_\V{code},_\V{but}_\V{it}_\V{is}_\V{really}_\-*\-_\V{C}++_\-*\-}}
\L{\LB{//}}
\L{\LB{//_\V{This}_\V{is}_\V{a}_\V{generic}_\V{element}_\V{template}}}
\L{\LB{//}} \L{\LB{//}\Tab{16}{+\-\-\-\-\-+}}
\L{\LB{//}\Tab{11}{\V{o}\-\-\-\-+}\Tab{22}{+\-\-\-\-\V{o}}}
\L{\LB{//}\Tab{16}{+\-\-\-\-\-+}} \L{\LB{//}} \L{\LB{}}
\L{\LB{\K{\#ifndef}_\V{Element1\_h}}}
\L{\LB{\K{\#define}_\V{Element1\_h}_\N{1}}} \L{\LB{}}
\L{\LB{\V{class}_\V{Element1}_:_\V{public}_\V{Element}}}
\L{\LB{\{}} \L{\LB{\V{public}:}} \L{\LB{}}
\L{\LB{}\Tab{2}{\V{Element1}(\V{const}_\V{string}\&_\V{iname});}}
\L{\LB{}} \L{\LB{}\Tab{2}{\~{}\V{Element1}()_\{\}}} \L{\LB{}}
\L{\LB{}\Tab{2}{\K{static}_\V{const}_\K{char}*_\V{getNetlistName}()}}
\L{\LB{}\Tab{2}{\{}}
\L{\LB{}\Tab{4}{\K{return}_\V{einfo}.\V{name};}}
\L{\LB{}\Tab{2}{\}}} \L{\LB{}}
\L{\LB{}\Tab{2}{//_\V{fill}_\V{MNAM}}}
\L{\LB{}\Tab{2}{\V{virtual}_\K{void}_\V{fillMNAM}(\V{FreqMNAM}*_\V{mnam});}}
\L{\LB{}}
\L{\LB{}\Tab{2}{\V{virtual}_\K{void}_\V{fillMNAM}(\V{TimeMNAM}*_\V{mnam});}}
%\endlgrinde\LGend
\L{\LB{}}

%\LGinlinefalse\LGbegin\lgrinde[30]
\L{\LB{\V{private}:}} \L{\LB{}}
\L{\LB{}\Tab{2}{//_\V{Parameter}_\V{variables}_(\V{replace}_\V{by}_\V{your}_\V{parameter}_\V{variables}_\V{here})}}
\L{\LB{}\Tab{2}{\K{double}_\V{par1},_\V{par2};}} \L{\LB{}}
\L{\LB{}\Tab{2}{//_\V{Element}_\V{information}}}
\L{\LB{}\Tab{2}{\K{static}_\V{ItemInfo}_\V{einfo};}} \L{\LB{}}
\L{\LB{}\Tab{2}{//_\V{Number}_\V{of}_\V{parameters}_\V{of}_\V{this}_\V{element}}}
\L{\LB{}\Tab{2}{\K{static}_\V{const}_\K{unsigned}_\V{n\_par};}}
\L{\LB{}} \L{\LB{}\Tab{2}{//_\V{Parameter}_\V{information}}}
\L{\LB{}\Tab{2}{\K{static}_\V{ParmInfo}_\V{pinfo}[\,];}} \L{\LB{}}
\L{\LB{\};}} \L{\LB{}} \L{\LB{\K{\#endif}}}
\endlgrinde


\subsection{Class Source File}

\LGinlinefalse \lgrinde
\L{\LB{\K{\#include}_\S{}\3.\,./network/ElementManager.h\3\SE{}}}
\L{\LB{\K{\#include}_\S{}\3.\,./analysis/FreqMNAM.h\3\SE{}}}
\L{\LB{\K{\#include}_\S{}\3.\,./analysis/TimeMNAM.h\3\SE{}}}
\L{\LB{\K{\#include}_\S{}\3Element1.h\3\SE{}}} \L{\LB{}}
\L{\LB{//_\V{Static}_\V{members}_(\V{set}_\V{to}_\V{the}_\V{number}_\V{of}_\V{parameters}_\V{of}_\V{your}_\V{element})}}
\L{\LB{\V{const}_\K{unsigned}_\V{Element1}::\V{n\_par}_=_;}}
\L{\LB{}} \L{\LB{//_\V{Element}_\V{information}}}
\L{\LB{\V{ItemInfo}_\V{Element1}::\V{einfo}_=_\{}}
\L{\LB{}\Tab{2}{\S{}\3res\3\SE{},}}
\L{\LB{}\Tab{2}{\S{}\3Element1\3\SE{},}}
\L{\LB{}\Tab{2}{\S{}\3Your_Name\3\SE{},}}
\L{\LB{}\Tab{2}{\S{}\3Reserved for documentation use\3\SE{}}}
\L{\LB{\};}} \L{\LB{}} \L{\LB{//_\V{Parameter}_\V{information}}}
\L{\LB{\V{ParmInfo}_\V{Element1}::\V{pinfo}[\,]_=_\{}}
\L{\LB{}\Tab{2}{\{\S{}\3par1\3\SE{},_\S{}\3Parameter_1_(Ohms)\3\SE{},_\V{TR\_DOUBLE},_\V{true}\},}}
\L{\LB{}\Tab{2}{\{\S{}\3par2\3\SE{},_\S{}\3Parameter_2_(F)\3\SE{},_\V{TR\_DOUBLE},_\V{false}\}}}
\L{\LB{\};}}
\endlgrinde\LGend
\LGinlinefalse\LGbegin\lgrinde[22]
\index{Element1}\Proc{Element1}\L{\LB{\V{Element1}::\V{Element1}(\V{const}_\V{string}\&_\V{iname})_:_\V{Element}(\&\V{einfo},_\V{pinfo},_\V{n\_par},}}
\L{\LB{\V{iname})}} \L{\LB{\{}}
\L{\LB{}\Tab{2}{//_\V{Value}_\V{of}_\V{r}_\V{is}_\V{required}}}
\L{\LB{}\Tab{2}{\V{paramvalue}[\N{0}]_=_\&\V{par1};}}
\L{\LB{}\Tab{2}{\V{paramvalue}[\N{1}]_=_\&(\V{par2}_=_\N{1e}\-\N{12});_//_\V{Example}_\V{of}_\V{how}_\V{to}_\V{set}_\V{a}_\K{default}_\V{value}.}}
\L{\LB{}}
\L{\LB{}\Tab{2}{//_\V{Set}_\V{the}_\V{number}_\V{of}_\V{terminals}_(\V{Set}_\V{to}_\V{the}_\V{number}_\V{of}_\V{terminals}_\V{of}}}
\L{\LB{}\Tab{2}{//_\V{your}_\V{element})}}
\L{\LB{}\Tab{2}{\V{setNumTerms}(\N{2});}} \L{\LB{}}
\L{\LB{}\Tab{2}{//_\V{Set}_\V{flags}_(\K{do}_\V{not}_\V{change}_\V{these}_\K{for}_\V{normal}_\V{linear}_\V{elements})}}
\L{\LB{}\Tab{2}{\V{setFlags}(\V{LINEAR}_\|_\V{ONE\_REF}_\|_\V{TR\_FREQ\_DOMAIN});}}
\L{\LB{\}}}
\endlgrinde\LGend
\LGinlinefalse\LGbegin\lgrinde[36]
\index{fillMNAM}\Proc{fillMNAM}\L{\LB{\K{void}_\V{Element1}::\V{fillMNAM}(\V{FreqMNAM}*_\V{mnam})}}
\L{\LB{\{}}
\L{\LB{}\Tab{2}{//_\V{Put}_\V{your}_\V{model}_\V{here}}}
\L{\LB{\}}}
\endlgrinde\LGend
\LGinlinefalse\LGbegin\lgrinde[40]
\index{fillMNAM}\Proc{fillMNAM}\L{\LB{\K{void}_\V{Element1}::\V{fillMNAM}(\V{TimeMNAM}*_\V{mnam})}}
\L{\LB{\{}}
\L{\LB{}\Tab{2}{//_\V{Put}_\V{your}_\V{model}_\V{here}}}
\L{\LB{\}}}
\endlgrinde\LGend
\LGinlinefalse\LGbegin\lgrinde[44]
\endlgrinde\LGend
%\end{document}

\section{Contributors}
The following contributed to this chapter
\begin{itemize}
\item[] Nikhil Kriplani
\item[] Carlos Christoffersen
\item[] Michael Steer.
\end{itemize}
