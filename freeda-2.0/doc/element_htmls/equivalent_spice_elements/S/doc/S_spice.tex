%\documentclass{article}
%\usepackage{epsf,html}
%\newcommand{\fig}[1]{J:/eos.ncsu.edu/users/m/mbs/mbs_group/figures/#1}
%\newcommand{\fig}[1]{../figures/#1}
%\newcommand{\pfig}[1]{\epsfbox{\fig{#1}}}
\oddsidemargin 10mm \topmargin 0.0in \textwidth 5.5in \textheight
7.375in \evensidemargin 1.0in \headheight 0.18in \footskip 0.16in
%%%%%%%%%%%%%%%%%%%%%%%%%%%%%%%%%%%%%%%% The title
%\begin{document}
\section[S \- Voltage Controlled Switch]{\noindent{\LARGE \textbf{Voltage Controlled Switch}} \hspace{70mm}\huge\textbf{S}}
\linethickness{1mm}
\line(1,0){425}
\normalsize
%%%%%%%%%%%%%%%%%%%%%%%%%%%%%%%%%%%%%%%% the resistor figure
\begin{figure}[h]
\centerline{\epsfxsize=2in\pfig{s_spice.ps}} \caption{S ---
voltage controlled switch element.\label{fig:port}}
\end{figure}
%\newline
%%%%%%%%%%%%%%%%%%%%%%%%%%%%%%%%%%%%%%%%%%% SPICE form
%\vspace{2mm}
\newline
\linethickness{0.5mm} \line(1,0){425}
\newline
\texttt{SPICE} \textit{Form:}
\newline
{\tt S}name $N_{+}$ $N_{-}$ $N_{C+}$ $N_{C-}$ {\it ModelName}
   \B {\tt ON}\E \B {\tt OFF}\E

%\pspiceform{{\tt S}name $N_{+}$ $N_{-}$ $N_{C+}$ $N_{C-}$
%ModelName}
%%%%%%%%%%%%%%%%%%%%%%%%%%%%%%%%%%%%%%%%%%%%%%% explanation of terms in the SPICE form
%\newline
\begin{tabular}{r l}
$N_{+}$ & is the positive node of the switch.\\
$N_{-}$ & is the negative node of the switch.\\
$N_{C+}$ & is the positive controlling node of the switch.\\
$N_{C-}$ & is the negative controlling node of the switch.\\
{\it ModelName} & is the model name and is required.\\
{\tt ON} & is the optional initial condition. It is intended for\\
& use with the {\tt UIC} option on  the  {\tt .TRAN} line,  when\\
& a transient analysis is desired starting from other than the\\
& quiescent operating point. It is also the initial condition on\\
& the device for \dc\ analysis.\\
{\tt OFF} & is the optional initial condition. If specified the\\
& \dc\ operating point is calculated with the terminal voltages
set\\
& to zero. Once convergence is obtained, the program continues
to\\
& iterate to obtain the exact value of the terminal  voltages.\\
& The OFF option is used to enforce the solution to  correspond\\
& to a desired  state if the circuit has more than one stable
state.\\
\end{tabular}
\newline
%%%%%%%%%%%%%%%%%%%%%%%%%%%%%%%%%%%%%%%%%%%%%%% Parameter table
\linethickness{0.5mm} \line(1,0){425}
\newline
\textit{Example:}
\newline
\texttt{S1 1 2 3 4 SWITCH1 \\
         S2 5 6 3 0 SM2 \\
         SWITCH1 1 2 10 0 SMODEL1}
\newline
\linethickness{0.5mm} \line(1,0){425}
\newline
\textit{Description:}\\
The initial conditions are optional.  For the voltage  controlled
switch, nodes $N_{C+}$ and N{C-} are the positive and negative
controlling nodes respectively.  For the current  controlled
switch, the controlling current is that through the specified
voltage source. The direction of positive controlling current flow
is from the positive node, through the
source, to the negative node.\\[0.1in]

\modeltype{VSWITCH}

\model{VSWITCH}{Voltage-Controlled Switch Model}
\marginid{VSWITCH}
\begin{figure}[h]
\centering \ \pfig{vswitch.ps} \caption{VSWITCH --- voltage
controlled switch model. \label{vswitch}}
\end{figure}

\notforsspice{The voltage-controlled switch model is supported by
both \spicethree\ and \pspice. However the model keywords differ slightly.\\[0.1in]

\kw{\spicethree\ keywords:}{ {\tt VT}   & threshold
voltage\sym{V_{\ms{ON}}}& V     & 0.0 \X {\tt VH}   & hysteresis
voltage\sym{V_{\ms{OFF}}} & V     & 0.0 \X {\tt RON} & on
resistance      \sym{R_{\ms{ON}}}&$\Omega$&1.0 \X {\tt ROFF} & off
resistance \sym{R_{\ms{OFF}}}&$\Omega$&1/GMIN\X } }

%\kw{\pspice\ keywords:}\\
%{ {\tt VON}   & threshold voltage\sym{V_{\ms{ON}}}& V     & 0.0 \X
%{\tt VOFF}   & hysteresis voltage\sym{V_{\ms{OFF}}} & V     & 0.0
%\X {\tt RON} & on resistance \sym{R_{\ms{ON}}}&$\Omega$&1.0    \X
%{\tt ROFF} & off resistance \sym{R_{\ms{OFF}}}&$\Omega$&1/GMIN\X }
Care must be exercised in using the switch. An instantaneous
switch is highly nonlinear and will very likely lead to
convergence problems. This problem is alleviated in the {\tt
VSWITCH} model by ramping the resistance of the switch from its
off value to its on value.  For this ramping action to be
effective the difference between $V_{\ms{ON}}$ and $V_{\ms{OFF}}$
must not be too small. Also the values of $R_{\ms{ON}}$ and
$R_{\ms{OFF}}$ should not be extreme. The ration
$R_{\ms{ON}}/R_{\ms{OFF}}$ should be be as small as possible.

If $R_{\ms{ON}}/R_{\ms{OFF}}$  is large, e.g.
$R_{\ms{ON}}/R_{\ms{OFF}}$ $>$ $10^{12}$, then the default error
tolerances {\tt TRTOL} and {\tt CHGTOL}, specified in a {\tt
.OPTIONS} statement may need to be changed.\\
{\tt TRTOL} Change to 1.0 from 7.0 idf there are convergence
problems during transient analysis.\\
{\tt CHGTOL} If a switch is across a capacitor then {\tt CHGTOL}
should be reduced to $10^{-16}$ if there are convergence problems
during transient analysis.

\eskip{S}
\noindent\underline{\bf \large Switch Model}\\[0.1in]

The switch is modeled by a voltage variable resistor $R$ and an
input input resistance $R_{\ms{IN}}$, see figure \ref{vswitch}.
$R_{\ms{IN}}$ = $1/G_{\ms{MIN}}$ to ensure that the controlling
nodes are not floating and that the voltage $v$ between the
controlling nodes cannot change instantaneously.\\[0.1in]

\noindent\underline{\sl \large Standard Calculations}\\[0.1in]
\begin{eqnarray}
R_{\ms{MEAN}} & = & \sqrt{R_{\ms{ON}} + R_{\ms{OFF}}} \\
R_{\ms{RATIO}}& = &       R_{\ms{ON}}/ R_{\ms{OFF}} \\
V_{\ms{MEAN}} & = & \sqrt{V_{\ms{ON}} + V_{\ms{OFF}}} \\
V_{\Delta}& = & \left({{\textstyle v - V_{\ms{MEAN}}} \over
              {\textstyle V_{\ms{ON}}- V_{\ms{OFF}}}}\right)
\end{eqnarray}
If $V_{\ms{ON}}> V_{\ms{OFF}}$ the switch resistance
\begin{equation}
R = \left\{
\begin{array}{ll}
R_{\ms{ON}}                            & v \ge V_{\ms{ON}} \\
R_{\ms{OFF}}                            & v \le V_{\ms{OFF}}\\
R_{\ms{MEAN}}\,
  R_{\ms{RATIO}}^{\textstyle 1.5 V_{\Delta}}\,
  R_{\ms{RATIO}}^{\textstyle 1.5 V_{\Delta}^3}
  & V_{\ms{OFF}} < v < V_{\ms{ON}} \\
\end{array} \right. %}
\end{equation}
If $V_{\ms{ON}}< V_{\ms{OFF}}$ the switch resistance
\begin{equation}
R = \left\{
\begin{array}{ll}
R_{\ms{ON}}                            & v \le V_{\ms{ON}} \\
R_{\ms{OFF}}                            & v \ge V_{\ms{OFF}}\\
R_{\ms{MEAN}}\,
  R_{\ms{RATIO}}^{\textstyle 1.5 V_{\Delta}}\,
  R_{\ms{RATIO}}^{\textstyle 1.5 V_{\Delta}^3}
  & V_{\ms{OFF}} < v < V_{\ms{ON}} \\
\end{array} \right. %}
\end{equation}
\noindent\underline{\sl \large Noise Analysis}\\[0.1in]
\index{VSWITCH, Noise Model} \index{VSWITCH, Noise Analysis}
\marginid{Noise Analysis} The voltage controlled switch noise
model accounts for thermal noise generated in the switch
resistance. The rms (root-mean-square) values of thermal noise
current generators shunting the switch resistance is
\marginid{Noise Model}
\begin{equation}
I_{n} = \sqrt{4kT/R}~\mbox{A/}\sqrt{\mbox{Hz}}
\end{equation}
where $T$ is the analysis temperature in kelvin (K), and $k$ (=
$1.3806226\,10^{-23}$~J/K) is Boltzmanns constant.
\newline
\linethickness{0.5mm} \line(1,0){425}
\newline
\textit{Notes:}\\
There is no equivalent element in \FDA.
%%%%%%%%%%%%%%%%%%%%%%%%%%%%%%%%%%%%%%%%%%%%% Credits
\newline
\linethickness{0.5mm} \line(1,0){425}
\newline
\textit{Credits:}
\newline
\begin{tabular}{l l l l}
Name & Affiliation & Date & Links \\
Carlos E. Christoffersen & NC State University & Sept 2000 & \epsfxsize=1in\pfig{logo.eps} \\
cechrist@ieee.org & & & www.ncsu.edu    \\
\end{tabular}
%\end{document}
