%\documentclass{article}
%\usepackage{epsf}
%\newcommand{\fig}[1]{J:/eos.ncsu.edu/users/m/mbs/mbs_group/figures/#1}
%\newcommand{\fig}[1]{../figures/#1}
%\newcommand{\pfig}[1]{\epsfbox{\fig{#1}}}

\oddsidemargin 10mm \topmargin 0.0in \textwidth 5.5in \textheight 7.375in
\evensidemargin 1.0in \headheight 0.18in \footskip 0.16in
%%%%%%%%%%%%%%%%%%%%%%%%%%%%%%%%%%%%%%%% The title
%\begin{document}
\section[X \- Subcircuit Call]{\noindent{\LARGE \textbf{Subcircuit Call}} \hspace{70mm}\huge \textbf{X}}
%\newline
\linethickness{1mm}
\line(1,0){425}
\normalsize
%\newline
%%%%%%%%%%%%%%%%%%%%%%%%%%%%%%%%%%%%%%%% the resistor figure
\begin{figure}[h]
\centerline{\epsfxsize=0.8in\pfig{x_spice.ps}} \caption{X ---
subcircuit call element.}
\end{figure}
\newline
%%%%%%%%%%%%%%%%%%%%%%%%%%%%%%%%%%%%%%%%%%% SPICE form
%\vspace{2mm}
\linethickness{0.5mm}
\line(1,0){425}
\newline
\texttt{SPICE} \textit{Form:}
\newline
{\tt X}name $N_1$ \B $N_2$ $N_3$ ... $N_N$\E  SubcircuitName
\newline
%\vspace{2mm}
%%%%%%%%%%%%%%%%%%%%%%%%%%%%%%%%%%%%%%%%%%%%%%% explanation of terms in the SPICE form
%\newline
\begin{tabular}{r l}
$N_1$ & is the first node of the subcircuit.\\
$N_N$ & is the $N$th node of the subcircuit.\\
{\it SubcircuitName} & is the name of the subcircuit.\\
{\tt PARAMS:} & indicates that parameters are to be passed to the
subcircuit.\\
{\tt keyword:} & is keyword corresponding to the keywords defined
in the {\tt .SUBCKT} statement.\\
{\tt value:} & is numeric value.\\
{\tt Expression:} & is an algebraic expression which evaluates to
a numeric value.\\
\end{tabular}
%\newline
\newline
\linethickness{0.5mm} \line(1,0){425}
\newline
%%%%%%%%%%%%%%%%%%%%%%%%%%%%%%%%%%%%%%%%%%%%%%%%%%%%%%%%%%%%%%%%%%%%% example in SPICE
\textit{Example:}
\newline
\texttt{X1 2 4 17 3 1 MULTI}
\newline
\linethickness{0.5mm} \line(1,0){425}
\newline
\textit{Description:}\\
Subcircuits are incorporated by using the ``{\tt X}'' element. The
number of nodes of the ``{\tt X}'' element must correspond to the
number of nodes in the definition of the subcircuit (i.e. is on
the {\tt .SUBCKT} statement.
\newline
\linethickness{0.5mm} \line(1,0){425}
\newline
\textit{Notes:}\\
The actual element in \FDA is the \texttt{X} element. See
\texttt{X} for full documentation. \\
%%%%%%%%%%%%%%%%%%%%%%%%%%%%%%%%%%%%%%%%%%%%% Credits
\linethickness{0.5mm} \line(1,0){425}
\newline
\textit{Credits:}
\newline
\begin{tabular}{l l l l}
Name & Affiliation & Date & Links \\
Carlos E. Christofferson & NC State University & Sept 2000 & \epsfxsize=1in\pfig{logo.eps} \\
cechrist@ieee.org & & & www.ncsu.edu \\
\end{tabular}
%\end{document}
