\section{A Note on the Eval Routines in fREEDA}

\noindent Most nonlinear computations require the evaluation of first and higher
derivatives of vector functions with $m$ components in $n$ real or complex
variables. fREEDA uses the package ADOL-C which automatically
computes the derivatives of non-linear functions by successive implementation
of the chain rule. These calculation occur at a fraction of the time and is
relatively free of truncation errors. The interface details to the ADOL-C
library from within fREEDA can be found in
path\_to\_freeda/simulator/network/NAdolcElement.cc.

Every nonlinear element in fREEDA has an eval function that describes the
constitutive equations for the particular element at any instant. This
function is called from within NAdolcElement.cc during evaluation of the tape.
For more details on the tape,  refer to ADOL-C documentation.

\subsection{Parameterized Device Models}

A non-linear device model can be described with the following set of equations
\begin{equation}
  \bf{v} ( t ) = \bf{v} ( x ( t ), dx / dt, \ldots,
  d^m x / dt^{^m}, x_D ( t ))
\end{equation}
\begin{equation}
  \bf{i} ( t ) = \bf{i} ( x ( t ), dx / dt, \ldots,
  d^m x / dt^{^m}, x_D ( t ))
\end{equation}
where \textbf{v}(t) and \textbf{i}(t) are vectors of voltages and currents at
the ports of the non-linear device, \textbf{x}(t) is a vector of parameters or
state variables and $\bf{x_D}$(t) is a vector of time delayed state
variables. There are several models like the microwave diode, the Gummel-Poon
BJT and Berkeley's BSIM4 which for accurate modeling, require charge to be
defined as a state variable. In some cases it may not be possible to find the
voltage as a function of charge and the only alternative is to treat charge as
a state variable.

Consider a simplified model for the microwave diode shown in Figure~\ref{diode_schematic_eval}.
\begin{figure}
\centerline{\epsfbox{diode.ps}}
\caption{\label{diode_schematic_eval}Simplified schematic for a diode model}
\end{figure}
The corresponding equations for this model are
\begin{equation}
i(v) = I_s (\exp( \alpha v ) - 1 )
\end{equation}
and
\begin{eqnarray}
c_j(v) = & C_{t0}(1 - {v / \phi})^{-\gamma} + C_{d0} \exp{(\alpha^{'}v)} & if\; v \leq .8\phi\\
         & C_{t0}(.2) ^{-\gamma} + C_{d0} \exp{(\alpha^{'}v)} & if\; v \geq .8\phi
\end{eqnarray}
where $v$ is the junction voltage. The capacitor voltage $q_j$ is given by
\begin{equation}
q_j ( v ) = \int_0^v c_j ( u ) du
\end{equation}
Accurate transient analysis requires $q_j$ to be chosen as a state variable.
We need $v$ to calculate $i_1( v )$ . Since it is
not possible to solve analytically
for $v ( q )$, the alternative is to model the diode with two state
variables, namely $v$ and $q$ . The diode equations
can now be formulated in two stages. Hence for this case, we
will have two \emph{eval} functions; \emph{eval}1 and \emph{eval}2 .
In \emph{eval}1, we will calculate $i(v)$ and
$q(v)$, both are which are functions of voltage $v$. In this case, the
values of $q(v)$ are passed to \emph{eval}2, which automatically calculates $d q /
d t$. Hence we now have the current through the capacitor as
\begin{equation}
i_c = \frac{d q}{d t}
\end{equation}
All the code that performs the actual derivatives is outside the nonlinear
model and is handled automatically. Currently in fREEDA, only three levels of
eval are supported. This means that the highest order of derivatives possible
is two. Most nonlinear elements do not require more than two levels of
derivatives. The entire formulation can be generalized as follows:
\begin{eqnarray}
  \rm{stage} \, 1 & : & f_1 ( x, x_D ) \nonumber \\
  &  & g_1 ( x, x_D )
\end{eqnarray}
\begin{eqnarray}
  \rm{stage} \, 2 & : & f_2 ( f_1, d g_1 / d t ) \nonumber \\
  &  & g_2 ( f_1, d g_1 / d t )
\end{eqnarray}
\begin{eqnarray}
  \vdots &  & \nonumber
\end{eqnarray}
\begin{eqnarray}
  \rm{stage} \, n & : & v_{} ( f_{n - 1}, d g_{n - 1} / d t ) \nonumber \\
  &  & i ( f_{n - 1}, d g_{n - 1} / d t )
\end{eqnarray}
