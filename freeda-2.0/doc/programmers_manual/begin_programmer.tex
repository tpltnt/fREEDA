\pagestyle{headings}

\pagenumbering{roman}       % Number the first few pages in roman
\vspace*{1in}
\begin{center}
    {\FREEDA}\\[0.5in]
%    \newline
%    \newline
    %\\
    {\Large Programmer's Guide}\\[0.5in]
    %\newline
    {\Large Version 1.3}\\  %
%     \\[0.5in]
      \vspace{0.5in}
%     \large\sl
    {\Large June 26, 2007}\\[0.5in]
    Compiled on \today
%     \\[0.5in]
      \vspace{0.5in}
\end{center}
\newpage

\noindent Copyright \copyright\ by the respective authors
identified throughout. \vfill \noindent All rights reserved.
Printed in the United States of America. Except as permitted under
the United States Copyright Act of 1976, no part of this
publication may be reproduced, stored in a data base or retrieval
system or transmitted, in any form or by any means, electronic,
mechanical, photocopying, recording or otherwise without the
written permission without the permission of
the publisher.\\
\vfill
%\accusim\ is a trademark of Mentor Graphics.\\
%\cdsspice\ is a trademark of Cadence Design Systems.\\
%\hpimpulse\ is a trademark of Hewlett Packard.\\
%\hspice\ is a trademark of Meta-Software.\\
%\igspice\ is a trademark of A.B. Associates.\\
%\isspice\ is a trademark of IntuSoft.\\
%\mspice\ is a trademark of EEsof.\\
%\spectre\ is a trademark of Cadence Design Systems.\\
\noindent\FDA is a trademark of Michael Steer and registration has
been applied for.\\
%\noindent\FDA is a registered trademark of Michael B. Steer.\\
\noindent\spicetwo\ is a trademark of U.C. Berkeley.\\
\spicethree\ is a trademark of U.C. Berkeley.\\
%\spiceplus \ is a trademark of Valid Logic.\\
%\sspiceonly{\pspice\ is a trademark of Compact Software Inc.\\}
\notforsspice{\pspice\ probe, parts, device equations and digital
files are trademarks of Microsim Corp.\\} All other trademarks are
the properties of their respective owners.
%\vfill
%\vfill
%\noindent This document was prepared using \LaTeX.\\
%Diagrams were prepared using idraw and gnuplot.
\vfill

{\scriptsize \baselineskip=6pt \noindent Information contained in
this work is believed to be reliable and obtained from sources
that are also believed to be reliable. However, the authors do not
guarantee the completeness or accuracy of any information
contained herein and the authors shall not be responsible for any
errors, omissions, or damages arising out of use of this
information. This work is published with the understanding that
the authors are supplying information but are not attempting to
render engineering or other professional services. If such
services are required, the assistance of an appropriate
professional should be sought.} \hspace*{\fill} \tableofcontents
\normalsize
%\listoffigures \listoftables
%\input{symbo}
%\notforsspice{
%\clearpage
%\vspace*{0.5in}
%\noindent{\Large\bf Preface}\\[0.5in]
%This book began because of a need for a better \spice\ manual for our
%students who use \spicethree\ in a multi workstation environment
%at the university and on their own IBM-PC compatible computers.
%The great many low-cost \spice manuals that are available are confined to
%just one commercial version of \spice\ (\pspice by Microsim Corporation)
%and simply cause confusion when not used with \pspice.
%Many \spice\ users use several versions of \spice\ and would
%like to know what the common denominator \spice\ syntax is.
%This book is a users manual and reference for three of the most commonly used
%\spice\ programs.  \spicetwo is used as \underline{the} common denominator
%of all \spice\ programs.
%\pspice\ and \spicethree\ extensions of this ``standard'' are clearly
%identified. The book can be used as a reference by the user of any \spice -like
%program provided that the user restricts her or himself to the syntax of
%\spicetwo.
%However personal computer versions of \pspice\ and \spicethree\ are available
%from bulletin boards (or by mail at low cost) although the \spice\ version is
%severely restricted as to the size of circuit which it can handle.
%From the initial effort of developing a universal \spice\ manual,
%and with the feedback from many \spice\ users in university and
%industry, the current manual was developed. The manual has been
%written to support the needs of both novice and experienced users.
%The first part of the book helps users get started with \spice\
%perhaps for the first time. Then the algorithms and methods that
%make \spice\ work are treated. The second part of the book is a
%catalog of \spice\ statements and elements. The element catalog
%presents the electrical models of elements in a detail that has
%not been previously presented in a concise form before with a user
%perspective. An example is the clear explanation of the
%dependencies of the parameters of the MOSFET models. A subject
%that can be particularly confusing otherwise.  Most of the
%electrical models were developed by reverse engineering source
%code as some have not been adequately documented previously. The
%reference section of the manual is well indexed and cross
%referenced making it easy to find desired information with a
%minimum number of indirections.
%We continually strive to improve this manual and welcome your
%suggestions as to its improvement and your corrections.
%\\[0.5in]
%\parbox{2in}{Michael Steer\newline
%North Carolina State University\newline
%email: mbs@ncsu.edu}\hfill
%\parbox{2in}{Paul Franzon\newline
%North Carolina State University\newline
%email: paulf@ncsu.edu}
%}
