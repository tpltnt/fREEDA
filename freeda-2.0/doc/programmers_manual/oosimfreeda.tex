%--- Hey emacs, this is -*-LaTeX-*-
\chapter{\FDA Architecture} \label{ch:oo}

%--------------------------------------------------------------------------
\section{Introduction}

The architecture of \FDA is based on faithful application of object
oriented (OO) design practice\cite{eliens,dep_inv,open,liskov,int_seg,uml_tut}
\footnote{These references are available on line at http://www.objectmentor.com/}.
The OO abstraction is well suited to modeling engineering systems, for
example, in circuit simulation circuit elements are already viewed as discrete
objects and at the same time as an integral part of a (circuit) continuum.
The OO view is a unifying concept that maps extremely well onto the way
humans perceive the world around them.  Non-OO circuit simulators
always become complicated with many layers of special cases.
Referring to circuit elements again, traditional simulation
implementations have many ``if-then'' like statements and individually
identify every element in many places for special handling.

There are a few key premises that drove the architecture of \FDA. One of
these is the adoption of a very strong OO paradigm throughout to
obtain a modular design. A second key premise is the separation of the core
components embodying numerical methods from the modeling and solver
formulation process with the result that numerical techniques
developed by computer scientists and mathematicians can be formulated
using formal correctness procedures. Thus, what is adopted here, is
that the circuit abstraction is adapted so that highly reliable and
efficient pre-developed libraries can be used.

C++ is the core implementation language and was once considered slow
for scientific applications. Advances in compilers and programming
techniques, however, have made this language
attractive and in some benchmarks C++ outperforms Fortran
\cite{kai,c++fortran}. Several OO numerical libraries have been
developed \cite{oonpage}.  Of great importance to the work described
here is the incorporation of the standard template library
(STL)~\cite{STL}. The STL is a C++ library of container classes,
algorithms, and iterators; it provides many of the basic algorithms
and data structures of computer science. The STL is a generic library,
meaning that its components are heavily parameterized: almost every
component in the STL is a template.  The current ISO/ANSI C++
standard~\cite{todd} has not been fully implemented and C++ compilers
support a variable subset of the standard. The biggest areas of
noncompliance being the templates and the standard library.

In this chapter the focus is on the design of the object-oriented
structure of \FDA. The goal in the design of \FDA was to obtain speed in
development, to use `off-the-shelf' advanced numerical techniques, and
to allow easy expansion and implementation of new models and numerical
methods.

The design intent was to to combine the advantages of previous OO circuit
simulators with these new developments as well as expanding
capability. \FDA uses C++ libraries \cite{adol-c:96,mv++} and
several written in C or Fortran \cite{FFTW,Sparse,NNES}.

OO-design embodies a body of concepts that should be understood before
a full appreciation of what follows can be made.
This fits in beautifully with the idea of code resue as here we utilize concept reuse.
The first concept required is an understanding of the Unified Markup Language (UML).
This is a universal way of presenting relationships among objects and is not specific to computer coding.
UML can be viewed as a substitute for flowcharts.
A flowchart permits a very limited set of relationships to be described
and results in what is called spaghetti code: code that has many linear streams and complex connections between the threads.
Common code is multiplicated and perhaps slightly changed in each instance
as the concept of using existing code and overriding only the differences is not supported.
UML is different and fosters code reuse and much simpler coding.
So if you are not familiar with UML study Appendix \ref{app:oo}
before proceeding.

\section{The Network Package}
%
\begin{figure}
\centerline{\epsfxsize=13cm \epsfbox{network_package.eps}}
\caption{Class diagram: The network package is the core of the
simulator.\label{fig:network}}
\end{figure}
%
%
\begin{figure}[htpb]
\centerline{\epsfxsize=6cm \epsfbox{netlistitemclass.eps}}
\caption{The {\bf NetListItem} class.}
\label{fig:netlistitemclass}
\end{figure}
%
The network package is the core of the simulator. All the elements and
the analysis classes are built upon it as shown in the class diagram
of Figure~\ref{fig:network}. R(resistor), D(diode) and MesfetM shown
here are examples of particular elements. Xsubckt is the \texttt{SPICE}
X subcircuit element. The terminals of an element are specifically
referred to as terminals and not as nodes as used in \texttt{SPICE}
as the concept of a node is more general than that of a terminal. For
example elements and terminals are both graph nodes.
(See Appendix \ref{app:oo} for a description of the class diagram syntax.)
Following the suggestion made in~\cite{davis1}, there is a {\bf NetListItem}
class that is the base for all classes of objects that appear in the input
netlist. This is the base class that handles parameters.
Figure~\ref{fig:netlistitemclass} shows some of the methods provided by
this class. All the netlist items share a common syntax so that the
element model developer does not need to worry about the details of element
parsing and there is no need to modify the parser to add new elements.
For compatibility reasons, Spice-type syntax (which does not have a consistent
grammar) is supported by the parser outside the network package. Support
for the elements using \texttt{SPICE} syntax is implemented for each
element in the parser where each \texttt{SPICE} element is renamed as a \FDA element.

%
\begin{figure}[htpb]
\centerline{\epsfxsize=6cm \epsfbox{elementclass.eps}} \caption{The
{\bf Element} class.} \label{fig:elementclass}
\end{figure}
%
The {\bf Element} class contains basic methods common to all elements
as well as the interface methods for the evaluation routines.  Some of
the methods of this class (Figure~\ref{fig:elementclass}) need to be
overridden by the derived classes. For example, in class {\bf D},
{\tt svTran()} is intended to contain the code to evaluate the time
domain response of a diode. This function is used by DC and transient
analyses. The same happens with {\tt svHB()} and {\tt fillMNAM()}.
The overhead imposed by these virtual functions is small compared to
the time spent evaluating the functions themselves and so this
approach is a good compromise between flexibility and efficiency. This
idea has been used in \cite{codecs,davis1,feldmann}. \FDA also
offers a more elaborate mechanism for nonlinear element evaluation
functions which will be described in Section \ref{sec:nlelem}.

%
%\begin{figure}[htpb]
%\centerline{\epsfxsize=13cm \epsfbox{transim_manual_example.ps}}
%\caption{Documentation generated for an element.}
%\label{fig:manual}
%\end{figure}
%
%
\begin{figure}[htpb]
\centerline{\epsfxsize=6cm \epsfbox{circuitclass.eps}} \caption{The
{\bf Circuit} class.} \label{fig:circuitclass}
\end{figure}
%
The {\bf ElementManager} class is mainly responsible for keeping a
catalog of all the existing elements. Note that this class is the only
one that `knows' about each and every type of element but this
dependency is weak.  The element list is included from an
automatically generated file. {\bf ElementManager} is also used to
automatically generate documentation for the element in html format.
The {\bf Circuit} class
represents either a main circuit or a subcircuit as a collection of
elements and terminals. It provides methods to add, remove and find
elements and terminals using different criteria and it also provides
methods related to circuit topology. More details about this class are
given in Figure~\ref{fig:circuitclass}.  All the {\bf Element} and
{\bf Terminal} instances must be stored in data structure inside the
{\bf Circuit} instance and the \emph{map} container of the
STL~\cite{STL}. The map is a \emph{Sorted Associative Container} that
associates objects of type \emph{Key} with objects of type
\emph{Data}. Here \emph{Data} is either {\bf Element} or {\bf
Terminal} and \emph{Key} is \emph{int} (the ID number). This is an
example of where the features of C++ are used to reduce development
time. This is achieved at no overhead as an optimum implementation of
these concepts is embedded in the compiler.
%
\begin{figure}[htpb]
\centerline{\epsfxsize=6cm \epsfbox{xsubcktclass.eps}} \caption{The
{\bf Xsubckt} class: This class handles the \texttt{SPICE} {\bf X} element.}
\label{fig:xsubcktclass}
\end{figure}
%
Subcircuit instances are represented by the {\bf Xsubckt} class, see
Figure~\ref{fig:xsubcktclass}.  The method {\tt attachDefinition()} is
used to associate a {\bf Circuit} instance where the actual subcircuit
is stored to the particular {\bf Xsubckt} instance and {\tt
expandToCircuit()} takes a {\bf Circuit} pointer as argument and
expands the subcircuit.
Note that before expansion the complete hierarchy of a circuit is
available in memory, so this engine could eventually be used to
perform hierarchical simulation. The reason for this is that a subcircuit
can be invoked (by an {\bf X} element) before the subcircuit is defined
(in a .SUBCKT block).

% \section{Example of Element Implementation: CPW Transmission Line}
%
% %
% \begin{figure}[htpb]
% \centerline{\epsfxsize=12cm \epsfbox{CPW_line.eps}}
% \caption{A CPW transmission line} \label{fig:CPW}
% \end{figure}
% %
% %
% \begin{figure}[htpb]
% \centerline{\epsfxsize=10cm \epsfbox{transim_cpwelem.eps}}
% \caption{Implementation of an element using another element as a building
% block} \label{fig:cpwelem}
% \end{figure}
% %
% The flexibility for element creation is illustrated by showing one
% implementation of the CPW transmission line element,
% Figure~\ref{fig:CPW}. Other implementations are also possible.  Since
% the CPW model is similar in concept to the physical transmission line
% model (Tlinp4 element) shown in Figure~\ref{fig:manual}, the CPW
% element can be implemented as shown in Figure~\ref{fig:cpwelem}. The
% {\bf Element} class (Figure~\ref{fig:elementclass}) provides a default
% {\tt init()} function which is executed after all the parameters
% values and the terminal connections are set. This method is
% \emph{overridden} by the {\bf CPW} class (this means the default
% function is replaced by a custom function for the {\bf CPW} class) as
% follows. After all the CPW parameters are set, the CPW {\tt init()}
% function calculates the equivalent parameters for a Tlinp4 element
% (see Figure~\ref{fig:manual} for a description of parameters) and
% inserts it in the circuit.  The actual transmission line equations are
% then handled by the underlying Tlinp4 element and this relationship
% applies to nearly every other aspect of the model such as the
% information to check local reference nodes
% \cite{local:reference:node:christoffersen,local:reference:node:khalil}.
%
% This kind of code reuse could also be achieved using the conventional
% procedural paradigm of programming. The advantage of the OO approach is
% that all the data, as well as all the functions needed to implement
% the physical transmission line element are contained in one class,
% thus making it easier to handle the elements as independent
% `blocks'. This is known as the \emph{encapsulation} mechanism which
% provides modularity of the code (see Appendix 2).
%
% The concept of this example can also be used to create elements
% composed of a set of other elements, or `hardwired subcircuits',
% although no elements of this type are currently implemented.

\section{The Analysis Classes}

%
\begin{figure}[htpb]
\centerline{\epsfxsize=13cm \epsfbox{analysis_package.eps}}
\caption{The analysis classes: {\tt AC}, {\tt DC}, {\tt SVTr} and {\tt SVHB} are
different analysis types.} \label{fig:analysis}
\end{figure}
%
Figure \ref{fig:analysis} shows the relation between the network
package, the elements and the analysis classes.  Each of these classes
stores analysis-specific data that would traditionally be global. This
leads to the key desired attribute of flexibility in incorporating a
new type of analysis, or even different implementations of the same
analysis type. Examples are the {\bf SVTr} and {\bf SVHB} classes
which contain the state-variable convolution transient and state-variable
harmonic balance analyses described in \cite{carlos_phd}.

There are some components common to two or more analysis types. The
natural way of handling these components is by creating a class which
is shared by the different analysis types. For example, the {\bf
FreqMNAM} class handles a Modified Nodal Admittance Matrix (MNAM) in
the frequency domain.  In a microwave simulator, the frequency domain
admittance matrix is a key element for most analysis types. Since it
is used so often, special care was taken to optimize efficiency.  The
elements fill the matrix directly without the need for intermediate
storage of the element stamp.  They do that by means of in-line
functions to reduce function call overhead. Elements can fill the
source vector in a similar way.  The elements depend on the {\bf
FreqMNAM} methods, but this is not a problem since the interface is
very unlikely to change.  The current implementation of MNAM uses the
Sparse library, described in the next Section, and is completely
encapsulated inside the {\bf FreqMNAM} class.  In the same way, the
{\bf NLSInterface} class encapsulates the nonlinear solver
routines. Therefore it is possible to replace the underlying
libraries, if that is desired, without the need for any code
modification outside the wrapper classes.  A final observation is that
it is possible to add any kind of analysis type provided that the
appropriate interface is defined and the member functions are written
for each element type.

\section{Nonlinear Elements} \label{sec:nlelem}
%
\begin{figure}[htpb]
\centerline{\epsfxsize=8cm \epsfbox{depinversion.eps}}
\caption{Dependency inversion was used to make the elements
independent of the analysis classes.} \label{fig:depinversion}
\end{figure}
%
Nonlinear elements often use service routines provided by the analysis
classes. In order to maximize code reuse and to avoid the dependence
of the element code on a particular analysis routines, in \FDA
elements depend on interface classes
(Figure~\ref{fig:depinversion}). The concept is similar to the
\emph{dependency inversion} \cite{dep_inv}. This is a technique which
relies on interface classes (normally implemented using abstract
classes in C++) to make different parts of a program independent of
each other. They only depend on the interfaces. To achieve greater
efficiency, in \FDA the dependency inversion is implemented using
a concrete class with heavy in-lining and pass-by-reference.

In this way, the element routines and the analysis depend on an
interface class, {\bf TimeDomainSV} (not shown in
Figure~\ref{fig:network} for clarity).  {\bf TimeDomainSV} is a class
that is used to exchange information between an element and a state
variable based time domain analysis. It also provides some basic
algorithms such as time differentiation methods. This approach enables
the element routines to be reused by several analysis types without
the need to modify the element code (as long as the new analysis is
state variable-based). For example, the {\bf DC} analysis uses the
same interface element as the {\bf SVTr}.

\FDA offers a more refined way to implement nonlinear elements,
based on state variables (See chapter \ref{nonlinear:addition}).
By implementing those parametric equations
using a special syntax in only one function, \FDA can obtain the
analysis functions {\tt svHB()}, {\tt svTran()} and derivatives
automatically. This mechanism is termed \emph{generic evaluation}.

%
\begin{figure}[htpb]
\centerline{\epsfxsize=16cm \epsfbox{adolcelement.eps}}
\caption{Class diagram for an element using generic evaluation.
{\tt D}(diode) and MesfetM are elements.}
\label{fig:adolcelement}
\end{figure}
%
Figure \ref{fig:adolcelement} shows the class diagram for an element
using this feature. Note that the {\bf Diode} class is derived from a
class ({\bf AdolcElement}) which itself provides the analysis routines
and deals with the analysis interfaces. The {\bf Diode} class only
needs to implement the {\tt eval()} function with the parametric
equations.

{\bf AdolcElement} uses the Adol-C library (see Section
\ref{sec:adolc} for a detailed explanation) to evaluate the parametric
function and derivatives, but the concept is independent of automatic
differentiation. If automatic differentiation were not used, then the
derived class (\emph{e.g.} the {\bf Diode} class) would have to
provide the Jacobian for the parametric equations.

The idea is in a way similar to the one used in \cite{aplac2},
\emph{i.e.} the primitive equations are `wrapped' in analysis-specific
generic functions and so there is no need to write a separate routine
for each analysis type. In the current work there are two additional
features. The first is that the generic evaluation is combined with
the state variable concept of Section \ref{review:rizzoli} and
automatic differentiation. This provides unprecedented simplicity in
creating nonlinear element models. The second is that a single
mechanism is not mandatory. There are cases where it may not be
practical to use this approach, or the overhead involved (which is
completely acceptable even for simple models such as a diode) may not
be properly amortized. In those cases, the element can implement the
analysis functions directly, without using the {\bf AdolcElement}
class.

It is important to remark that generic evaluation is implemented
efficiently so there are no superfluous calculations. The current
implementation supports elements with any number of state
variables. Each element selects the input variables as a subset of the
following: the state variables, the first derivatives, the second
derivatives and a time delayed version of the variables (the delay may
be different for each). No derivation, time delaying nor
transformation is performed on the unselected inputs.

For example, an element with only an algebraic nonlinearity (such as
the VCT: voltage controlled transducer) only selects the state
variables without any derivatives or delays. As another example, the
{\bf MesfetM} class implements the Materka-Kacprzac model for a
MESFET. It requires two state variables, but only one of them needs to
be delayed.

A consequence of having a library of elements using generic evaluation
is that it is possible to add a new analysis type by just adding the
appropriate evaluation routines to the {\bf AdolcElement} class. Thus,
the maintainance and expansion of the simulator is simplified. Put
another way the code for a circuit element need only be implemented
once and the identical code can be used by many different types of
analyses including, hopefully any future analysis type implemented
in the future.

\section{Example: Use of Polymorphism}

The concept of polymorphism is briefly explained in the introduction
to object oriented programming in Appendix \ref{app:oo}. The scheme
presented in Section \ref{sec:nlelem} constitutes a good example of the
use of polymorphism to solve a complex problem.  Consider the segment
of code of Figure \ref{fig:nleval}. This code evaluates the nonlinear
element functions in state variable convolution transient analysis. {\tt
elem\_vec} is a vector of {\bf Element} pointers implemented using the
vector container of the C++ STL.
%
\begin{figure}[htpb]
\begin{verbatim}
  // Number of elements
  int n_elem = elem_vec.size();
  int i = 0;
  // Go through all the nonlinear elements
  for (int k = 0; k < n_elem; k++) {
    // Set base index in interface object
    tdsv->setIBase(i);
    // nonlinear element evaluation
    elem_vec[k]->svTran(tdsv);
    i += elem_vec[k]->getNumberOfStates();
  }
\end{verbatim}
\caption{Nonlinear element function evaluation in convolution transient.}
\label{fig:nleval}
\end{figure}
%
There is no need to keep the size of the vector in a separate variable
as {\tt elem\_vec.size()} returns the size of the vector. Also, the
memory management of the vector is dynamic and automatic; and {\tt
elem\_vec[k]} returns the Element pointer at position $k$ in the
vector.

Each pointer inside {\tt elem\_vec} points to different kinds of
elements. For the transient routine the actual type of each element
does not matter. The line containing
{\tt elem\_vec[k]->svTran(tdsv)} will call the appropriate evaluation
routine depending on the actual type of Element pointer. The element
may implement the routine directly or through generic evaluation,
but it makes no difference for the analysis routine.

The resulting code is therefore simple and there is no need for lists
of ``if-then'' statements. Those lists would be very difficult to
maintain, because each time an element is added or removed, all the
lists would have to be updated.

\section{Documentation}

\FDA\ uses LaTeX as the main documentation language. Documentation is distributed in the directory freeda-version/doc . Element (in .../doc/fr\_elements.html ) and Analysis (in .../doc/fr\_analysis.html ) as well as the entire contents of the .../doc/elements directory are automatically generated.  Elements are handled as special way, as various implementations of \FDA\ can have different model sets (other than the standard distribution to include proprietary models for example) \FDA\ which generate html documentation of individual elements. This documentation is generated from the internal data structures so it is known to be right.  The convention is to also to provide PDF documentation of individual elements which are in the directory structure for the element. Environment Variables are discussed in the User's Guide in the subsection "Setting up the Cygwin Environment."  To create a set of analysis and element documentation ready for web use you should set the environment variables as follows

\begin{center}
\begin{tabular}{ll}
\newline {\tt \$ENV\_FREEDA\_DOCUMENTATION} & .../\emph{freeda-version}/doc \\
{\tt \$ENV\_FREEDA\_WEB\_DOCUMENTATION} & .../\emph{freeda-version}/doc \\
\end{tabular}
\end{center}

\noindent where \emph{freeda-version} is either just /myhome/freeda/freeda (if soft linking has been used) or something like /myhome/freeda/freeda/freeda-1.3 and myhome is the user name.  Then `run freeda -c' and `run freeda -c'. This will put fr\_eleemnst.html and fr\_analysis.html in .../\emph{freeda-version}/doc and all of the element pdf files from the tree
{\tt \$ENV\_FREEDA\_SIMULATOR}/elements tree in .../\emph{freeda-version}/doc/elements .

\subsection{Automatically Generated Documentation}

\FDA supports automatic generation of documentation for elements and analyses. These are derived from the internal data structures and is likely to be more up-to-date than separately generated documentation. The command lines to generate documentation are
\begin{itemize}
\item freeda -c
\item freeda -c \emph{elementName}
\item freeda -a
\end{itemize}
or, if using the cygwin environment:
\begin{itemize}
\item freeda.exe -c
\item freeda.exe -c \emph{elementName}
\item freeda.exe -a
\end{itemize}

\subsubsection{Analysis Catalog, freeda -a}

This generates the analysis catalog and puts it in the file fr\_analysis.html in the directory {\tt \$ENV\_FREEDA\_DOCUMENTATION} .


\subsubsection{Element Catalog, freeda -c}

This generates a catalog of all elements and puts it in the file fr\_elements.html in the directory {\tt \$ENV\_FREEDA\_DOCUMENTATION} (this is the environment variable). Individual html files of each element are also generated and put in the directory {\tt \$ENV\_FREEDA\_DOCUMENTATION/elements}.  The individual element html files will include a link to a more detailed pdf file for the element.  This must be generated by the developer who created the element model.  There are two ways of handling the links to the pdf files.

If the environment variable {\tt \$ENV\_FREEDA\_WEB\_DOCUMENTATION} points to a local filesystem then the element pdf's are copied from the {\tt .../simulator/elements} tree and put in the directory {\tt \$ENV\_FREEDA\_WEB\_DOCUMENTATION/elements} and the individual element html's point to the local pdfs.

If the environment variable {\tt \$ENV\_FREEDA\_WEB\_DOCUMENTATION} does not point to a local filesystem (e.g. http://www.freeda.org/doc ) then the individual element html file xxxx.html internally points to  {\tt \$ENV\_FREEDA\_WEB\_DOCUMENTATION/elements/xxxx.pdf}. Typically this would be {\tt http://www.freeda.org/doc/elements/xxxx.pdf }.

Environment Variables are discussed in the User's Guide.


\subsubsection{Individual Element Documentation, freeda -c \emph{ElementName}}

This generates an individual html file for the element with the name {\emph ElementName}. This is put in the directory {\tt \$ENV\_FREEDA\_DOCUMENTATION/elements}.  The  html file will include a link to a more detailed pdf file for the element.  This must be generated by the developer who created the element model.  There are two ways of handling the links to the pdf files.

If the environment variable {\tt \$ENV\_FREEDA\_WEB\_DOCUMENTATION} points to a local filesystem then the element's pdf's points to the pdf directly in the same directory, that is in
\newline
{\tt \$ENV\_FREEDA\_DOCUMENTATION/elements}.

If the environment variable {\tt \$ENV\_FREEDA\_WEB\_DOCUMENTATION} does not point to a local filesystem  then the individual element html file xxxx.html internally points to
\newline {\tt \$ENV\_FREEDA\_WEB\_DOCUMENTATION/elements/xxxx.pdf}.



\subsection{Enhanced Element Documentation}

The static information for an element is contained in the following data structure in the file
\newline
{\tt NetListItem.h }:

\begin{verbatim}
// Structure to hold static information about the item.
struct ItemInfo
{
  char* name; // Name of netlist item
  char* description; // Expanded name of netlist item
  char* author; // Acknowledge the author
  char* documentation; // documentation string
  char* version; // include version number yyyy_mm_dd
};
\end{verbatim}

Enhanced element documentation is contained in the {\tt documentation} variable.  This is not currently used (as of \FDA\ version 1.3) but will be included at some time. New elements should set this string.  The syntax is a comma and semicolon delimited string:

\indent
{\tt type1:arg1,arg2;type2:arg1}

The string is terminated in either a semicolon or a null.

An example for a microstrip line is

\indent
{\tt category:microstrip,transmission line}

The intent is for {\tt category:} to be used to sort documentation into categories. The developer can use any argument in category.  Other fields (e.g. tyupe2 ) are for future expansion but particularly for enhancing documentation.

\section{Contributors}
The following contributed to this chapter
\begin{itemize}
\item[] Carlos Christoffersen
\item[] Michael Steer.
\end{itemize}
