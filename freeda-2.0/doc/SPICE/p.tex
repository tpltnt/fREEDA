\element{P}{Port Element}

Some versions of spice only.

\begin{figure}[h]
\centering
\ \pfig{p_spice.ps}
\caption{P --- port element.\label{fig:port}}
\end{figure}

\form{ {\tt P}name  $N_{+}$ $N_{-}$ {\tt PNR=} PortNumber
      \B{\tt ZL=} ReferenceImpedance\E}
\begin{widelist}
\item[$N_{+}$] is the positive element node,
\item[$N_{-}$] is the negative element  node, and
\item[{\it PNR}] is the integer index of the port. The port index must be
numbered sequentially beginning at 1. That is, the first occurrence of a {\tt P}
element in the in input netlist must have {\tt PNR=1}, the second occurrence
{\tt PNR=2}, etc.
               (Units: none; Required; Symbol: $PortNumber$;)
\item[{\it ZL}] is the reference impedance of port
               (Units: $\Omega$; Optional; Default: 50~$\Omega$; Symbol: $Z_L$;)\\
\end{widelist}
\example{PORT1 1 0 PNR=1 ZL=75}

\note{
\item $V_{AS}$ in Fig. \ref{fig:port} is not visible to the user and is used
by the program to test for the S parameters.  As an example of
using the port specification with a source consider the partial
circuit  in Fig. \ref{fig:port_source}}. The spice code defining
this is \example{\it {\tt P}name  $N_{+}$ $N_{-}$ {\tt PNR=}
PortNumber
      \B{\tt ZL=} ReferenceImpedance\E\\
      \B{\tt VIN $N_{-}$ 0 {\tt PULSE} (Pulse Specification)\E}}

\begin{figure}[h]
\centering
\ \pfig{pex.ps}
\caption{Example of the usage of a P element with a pulse voltage source.
\label{fig:port_source}}
\end{figure}
