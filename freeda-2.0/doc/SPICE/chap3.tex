\chapter{Carrying On\label{chapter:chap3}}

In this chapter, we introduce the different element and control
statement lines that \spice\ allows you to use, starting with the different
types of circuit elements. In particular we discuss inductors,
active elements (diodes and transistors), and transmission lines.
We then cover the different types of analyses you can do with \spice,
starting with the most common, the transient analysis, which is
introduced in Chapter 2.   The main role of the control statements
is to specify these analyses.

{
Each implementation of \spice\ differs in the range of elements and control
statements used.  In this chapter, we describe those elements and
control statements found in all versions of \spice, i.e. those found
in \spicetwo.}

\section{Elements}

Resistors and capacitors are described briefly in Chapter 2.
We now look at the other elements, both passive and active.
The most common form of the element is described only.
For example, we do not
describe how a temperature dependency could be specified.  That
sort of detail is found in the reference section
(chapters \ref{chapter:statement} and \ref{chapter:element}).

\subsection{Inductors and Mutual Inductors}

An example showing an inductor and a mutual inductor is shown
below:
\par\noindent
\centerline{\pfig{inductor.ps}}
\par\noindent
\begin{center}
{\tt\begin{verbatim}
L1 1 2 20nH
L2 3 2 40nH
L3 4 5 40nH
K  L1 L2 0.90
\end{verbatim} }
\end{center}
\par\noindent
Note how the `dot' is placed on the first node of each inductor.

\subsection{Active Devices}

Unlike passive devices, such as resistors, active, or semiconductor,
devices can not be specified by one or a few parameter values.
To save typing effort a separate {\tt .model} line is created for
every semiconductor device type that might appear in the circuit.
In this part of the tutorial, we do not describe the meaning
of the parameters in the model line.  That can be found in the Reference
section
(chapters \ref{chapter:statement} and \ref{chapter:element}).
Instead, we give examples showing how semiconductor devices
are inserted into a \spice\ circuit.
We only describe the bipolar junction diode, bipolar junction transistor,
and the MOS field effect transistor.  Spice can also be used to describe
junction field effect transistors and Gallium Arsenide but the process is the
same.

\subsubsection{Diodes and Bipolar Transistors}

The schematic and partial \spice\ net-list for a simple TTL inverter
is shown in Figure~\ref{ttl}.
%
\begin{figure}
\centerline{\epsfysize=8.7cm\pfig{ttlgate.ps}}
\par\noindent
{\tt\begin{verbatim}
Q3 7 4 0 NQ1A
D1 8 7 DPN
Q4 6 3 8 NQ1A
R4 5 6 100
*
* Model element for NPN Transistor type NQ1A (courtesy Signetics)
.MODEL  NQ1A  NPN  IS= 1.95E-17  BF= 7.03E+01  VAF= 1.80E+01  IKF= 1.80E-02
\end{verbatim} }
\par\noindent
{\em (Remainder of model deleted)}
\par\noindent
{\tt
\begin{verbatim}
* Model for diode type DPN
.MODEL DPN D(IS= 8.17E-17  RS= 2.85E+01  N= 9.99E-01  CJO= 1.65E-13
+      VJ= 8.01E-01 M= 4.61E-01  EG= 7.99E-01  XTI= 4.00E+00)
\end{verbatim} }
\caption{TTL Circuit Description.}
\label{ttl}
\end{figure}
%
Note the following features in this circuit description:
\begin{itemize}
  \item The format for bipolar junction  transistors is:\\
  \offsetparbox{
%{\it {\tt Q}name Collector-Node Base-Node Emitter-Node Model-Name}
{\it {\tt Q}name  NCollector NBase NEmitter  \B NSubstrate\E  ModelName
 \B Area\E  \B {\tt OFF}\E\\{\tt +} \B {\tt IC=}Vbe,Vce\E }}\\
specifying a three terminal device.
  \item The format for diodes is:
\offsetparbox{
%{\it {\tt D}name Positive-Node Negative-Node Model-Name}
{\it {\tt D}name $n_1$ $n_2$ ModelName \B Area\E  \B {\tt OFF}\E
\B {\tt IC}=$V_D$\E }}
specifying a two terminal device.
  \item The use of `{\tt +}'s in the model lines to `join' different
lines in the file into one line for \spice.
\end{itemize}
It is also possible to have an `area' parameter in the diode and bipolar
transistor element line.  This area parameter is a scale factor, not
an absolute measure. An area of `1.0' (which is the default) specifies
that the model parameters are used unchanged.
Specifying another area factor causes
\spice\ to change some of the model parameters to reflect a larger or
smaller transistor.  An area of `2.0' specifies a situation equivalent
to  two transistors operating in parallel.

\subsubsection{MOSFETs}

The MOSFET element line looks quite different than the element line
for a bipolar transistor.  There are two major differences.
First, the MOSFET is a four terminal device.
Three of the terminals (source, drain, gate) have analogous functions to
the three terminals
in a bipolar transistor.  The current
passes between the drain to the source (analogous to the emitter and
collector) and is controlled
by the voltage on the gate (analogous to the base).
The fourth terminal is the `substrate' referring
to the bulk silicon in which the transistor
sits.  For correct functioning, the
substrate must be connected to the ground or Vcc node, for
n-channel and p-channel transistors respectively.

The second major difference is that the physical dimensions of the
transistor are specified in the model line.
Specifically, the channel length and width, and source and drain perimeters
and areas are specified.  These are the actual
dimensions, as they appear on the chip.  An example is shown in
Figure~\ref{mos-ex}.
%
\begin{figure}
\centerline{\pfig{cmostx.ps}}
\par\noindent
\centerline{\tt M1 0 1 2 0 nenh l=0.8u w=1.6u ad=3.2p as=3.2p pd=5.2u ad=5.2u}
\caption{Example of a MOSFET element specification.}
\label{mos-ex}
\end{figure}
%
This example matches the usual general model format:
\offsetparbox{
%{\it
%{\tt M}name Drain-Node Gate-Node Source-Node Substrate-Node Model-Name \\
%+ [l=Length-Value] [w=Length-Value] [ad=Length-Value] \\
%+ [as=Length-Value] [pd=Length-Value] [sd=Length-Value]
%}
{\it {\tt M}name NDrain NGate NSsource NBulk ModelName
              \B {\tt L}=Length\E  \B {\tt W}=Width\E  \\
      {\tt +} \B {\tt AD}=DrainDiffusionArea\E
              \B {\tt AS}=SourceDiffusionArea\E  \\
      {\tt +} \B {\tt PD}=DrainPerimeter\E  \B {\tt PS}=SourcePerimeter\E \\
      {\tt +} \B {\tt OFF}\E  \B {\tt IC=}$V_{DS},V_{GS},V_{BS}$\E
}
}
\par\noindent
where {\tt []} indicates optional parameters.
{\pspice\ supports additional element parameters.
(For the full general format please see the reference catalog.)}

Note that though the {\it drain} and {\it source}
have different physical meanings
(the {\it source} is the source of the majority carrier -- electrons for
an n-channel [nmos] device and holes for a p-channel [pmos] device),
no error is produced if they are interchanged in the \spice\
circuit description.
For example, in figure~\ref{mos-ex}, using {\tt M1 2 1 0 0 },
produces the same simulation results as using {\tt M1 0 1 2 0}.

An example of a CMOS digital inverter circuit,
together with its \spice\ model is given
in Figure~\ref{cmos-ex}.
Note the use of the {\tt .option} line in this example to fix
circuit-wide default values for {\tt L}, {\tt W}, {\tt AD}, and {\tt AS}.
%
\begin{figure}[h]
\centerline{\pfig{cmosex.ps}}
\par\noindent
{\tt \begin{verbatim}
CMOS Inverter Example
*
M0 0 1 2 0 nenh l=0.8u w=1.6u ad=3.2p as=3.2p pd=5.2u ad=5.2u
M1 5 1 2 5 penh l=0.8u w=1.6u ad=3.2p as=3.2p pd=5.2u ad=5.2u
Vcc 5 0 DC 5V
*
* following option line fixes transistor length and width, and
* drain/source area defaults
*
.options defl=0.8u defw=1.6u defad=3.2p defas=3.2p
*
*
.model nenh nmos
+    Level=2               Ld=4.000e-8        Tox=1.750000e-08
+    Nsub=1.506725e+17     Vto=0.59073        Kp=6.124495e-05
\end{verbatim}
\par\noindent
{\em (Remainder of model omitted.)}
\begin{verbatim}
.model penh pmos
+    Level=2               Ld=4.000000e-08    Tox=1.750000e-08
\end{verbatim} }
\par\noindent
{\em (Remainder of model omitted.)}
%+    Gamma=0.814660             Phi=1.00                Uo=641.842
%+    Uexp=0.0998923             Ucrit=6408.27           Delta=6.35195
%+    Vmax=132350.               Xj=7.415125e-9          Lambda=0.
%+    Nfs=1.000e+11              Neff=21.4017            Nss=3.000000e+10
%+    Tpg=-1.0000                Rsh=154                 Cgso=2.42e-10
%+    Cgdo=2.42e-10              Cj=5.4e-04              Mj=.3426
%+    Cjsw=2.15e-10              Mjsw=.2315
%*
%.end
\caption{A CMOS inverter.}
\label{cmos-ex}
\end{figure}
%

\subsection{Transmission Lines}

Though a transmission line  is a four terminal device, two of the
terminals are normally set to a common reference node, an example
of which is shown in Figure~\ref{tl1}.
%
\begin{figure}
\centerline{\pfig{transl.ps}}
\par\noindent
{\tt
\begin{verbatim}
T1 1 0 2 0 Z0=50 TD=333ps
\end{verbatim}
}
\caption{Example of a transmission line.}
\label{tl1}
\end{figure}
%
This lossless transmission line model supports only a single mode of
propagation.
If the two `reference' terminals (nodes {\tt 0} in this example)
correspond to two electrically different nodes in the
physical circuit then two modes are excited and two transmission lines
are required in the corresponding \spice\ description.

If quick simulation times are important then it is necessary to limit
the use of small transmission lines.  In a transient simulation the
minimum time step does not exceed half the propagation delay of the line.
Smaller time steps result in longer simulation times.
If this is a problem, remember that a transmission line can be safely
replaced by the equivalent lumped inductor and capacitor if the length
of the line is smaller than 1/10th of the shortest signal wavelength of
interest.

\subsection{Voltage and Current Sources}

\subsubsection{Independent Sources}

Spice supplies a number of independent voltage and current source types.
As many of the source's features only make sense in the context
of the analysis to be used, only some of the source's features
are discussed here.  In particular, we present those features that
might be used in a transient analysis (see Chapter 2 and Section 3.2.1).

Voltage supplies are specified using DC independent sources, for
example:
\par\noindent
\begin{center}
{\tt VCC 5 0 DC 5}
\end{center}
\par\noindent
for a 5~V DC power supply between nodes 5 and 0.

Any repeating non-sinusoidal waveform can be specified using the
{\tt pulse} waveform specification, an example of which was
given in Chapter 2.  {\tt pulse} is often used to describe digital clocks,
for example.

Non-repeating non-sinusoidal waveforms are specified
using the piece-wise linear ({\tt pwl}) waveform function.
One period of the {\tt pulse} example presented in Chapter 2 is
shown below in the piece-wise linear format:\\
\centerline{\pfig{pwlex.ps}}
\begin{center}
{\tt vin 1 0 pwl (0ns 0V 4ns 0V 7ns 1.5V 12ns 1.5V 14ns 0V 17ns 0V)}
\end{center}


\vfil\eject
Sinusoidal and decaying sinusoidal waveforms are specified
using the {\tt SIN} function, for example:
{\tt
\begin{verbatim}
Vin  4 0 sin(2.5 1 100meg 10ns)
\end{verbatim} }
\centerline{\pfig{sinusoid.ps}}
\vspace{3mm}\par\noindent
Spice also allows you to specify exponential and single-frequency FM
signals.  Please see the reference catalog for details.

\subsubsection{Dependent Sources}

These are the most overlooked elements \spice\ provides.
Four different types of linear dependent sources can be
specified in \spice:
\begin{itemize}
\item Voltage-controlled voltage source and current-controlled current
source:
\par\noindent
\centerline{\pfig{depsceef.ps}}
\par\noindent
\begin{center}
\begin{tabular}{ll}
{\tt E1 4 3 2 1 3.3}  & {\em A voltage gain of 3.3.} \\
{\tt F1 6 5 Va 1.7} & {\em A current gain of 1.7}
\end{tabular}
\end{center}
\item Voltage-controlled current source and current-controlled voltage
source:
\end{itemize}
\par\noindent
\centerline{\pfig{depscegh.ps}}
\par\noindent
\begin{center}
\begin{tabular}{ll}
{\tt G1 4 3 2 1 15mmho}  & {\em A transconductance of 15$\times 10^{-3}$mho
($\Omega^{-1}$).} \\
{\tt H1 6 5 Va 0.5k} & {\em A transresistance of 500 Ohms}
\end{tabular}
\end{center}
The above are linear sources.  Non-linear sources can also be specified.
For example, the following voltage-controlled current source
actually specifies a non-linear resistance that could be used as
part of a non-linear Thevenin equivalent circuit:
\par\noindent
\centerline{\pfig{thevnlin.ps}}
\par\noindent
\begin{center}
{\tt Gout 2 1 2 0   0 1m -0.6m}
\end{center}
The format used in this example is:
\begin{center}
  {\it {\tt G}xxx node1 node2 ref-node1 ref-node2 C0 C1 C2}
\end{center}
to produce a dependent source that obeys the equation:
\[
I = C0 + C1 (V(ref-node2) - V(ref-node1)) + C2 (V(ref-node2) - V(ref-node1))^2.
\]
In this case, the equation specifying the current is:
\[
 I = 0 + 1 \times 10^{-3} V_{out} - 0.6\times 10^{-3} V_{out}^2
\]
\par\noindent
The above non-linear source is quadratic and dependent on only one
other variable.
The same format can be used to specify higher order polynomials.
A source dependent on the voltages/currents on/in {\tt ND} other nodes/branches
can be specified by including a {\tt poly(nd)} statement in the element line.
For example, the following linear
voltage-controlled voltage source specifies a gated sinusoidal source:
\par\noindent
\centerline{\pfig{depeply.ps}}
\par\noindent
\begin{center}
{\tt\begin{verbatim}
Vsine 1 0 sin (0 0.5 100k 5us)
Vpulse 2 0 pwl (0ns 0V 14us 0V 15us 1V 65us 1V 66us 0V)
Egate 3 0 poly(2)  1 0   2 0  0 0 0 1
\end{verbatim} }
\end{center}

i.e. This source specifies a voltage,
\[
{\tt Vgate} = 0 + 0\times {\tt Vsine}
    +0\times {\tt Vpulse} + 1 \times {\tt Vpulse} \times {\tt Vpulse}
\]
which has the following waveform:\\
\centerline{\pfig{plssine.ps}}
\vspace{3mm}\par\noindent
In its general form, a polynomial of any complexity can be specified.  e.g.
The generalized voltage controlled voltage source,
{\tt\begin{verbatim} EX <node> <node> poly(2) V1 V2  k0 k1 k2 k3 k4 k5 k6 k7
\end{verbatim}
}
\par\noindent
specifies a controlled voltage of the form
\[
  EX = k1 + k2\times V1 + k3\times V2 + k4\times V1\times V2 + k5 V1^2 + k6 V2^2
    + k7 V1^2 V2^2
\]
This could be extended to create polynomials as a function of 3, 4, etc.
variables.  However, as a practical matter, it is very difficult to
read and understand non-linear polynomials with more than two inputs.
It is easier
to create two-input polynomials separately and combine them with
another polynomial.

\section{Analyses}

\subsection{Transient Analysis}

In the transient analysis response is observed
with one or more time-varying inputs.
A simple example is given in Chapter 2.

The first step performed by \spice\ in a transient analysis is to
compute the initial DC or bias point condition.  During this
computation it is
assumed that the voltage across capacitors is zero, the current
through inductors is zero, and the value for dependent sources is zero.
\spice\ then conducts
the transient simulation by calculating all of the voltages and
currents at a set of points in time.
In the rest of this section, we discuss a number of issues related
to transient analyses, starting with a treatment of convergence.
%We are going to discuss
%four points in the rest of this section.
%First we discuss convergence.  This leads into a treatment of
%how you can influence \spice\ in determining the initial bias point,
%and then we present how you can
%influence spice to determine the time interval between the time
%steps and thus the accuracy and computer-run-time of the
%simulation.
%Finally, we show how you can obtain a Fourier transform of a transient
%signal.

\subsubsection{DC Convergence}
\label{convergence}

During both the DC analysis and the following transient analysis
iterative numerical techniques are used to obtain a solution.
The objective of these techniques is to iterate on the value
of the node voltages and branch currents until successive iterations
only bring very small changes in their values, i.e. \spice\ {\em
converges} on a solution in the DC analysis and at every time step.

Sometimes \spice\ can not converge on a solution.
If this occurs during the DC analysis it will report this problem in the
output file with a `{\tt convergence problem}' message like.
Failure to converge in the DC analysis is usually due to an error in
specifying node numbers, circuit values or model parameter values.
These should be checked carefully before proceeding further.
However, sometimes \spice\ is having a genuine problem in converging
and you might have to help it find a solution.

In many bistable circuits (e.g. flip-flops) and positive feedback circuits,
\spice\ will not converge in the DC analysis or will converge to
an undesirable value (e.g. midway between logic-0 and logic-1 in a
latch).  One way to help \spice\ converge to the correct value is to
use the {\tt off} option to turn off devices in the feedback path, e.g.,
\begin{center}
{\tt M0 0 1 2 0 pd=5.2u ad=5.2u off}
\end{center}
allowing \spice\ to find a DC solution.  Spice turns the devices back on
during the transient analysis.
Another approach is to use {\tt nodeset} to provide `hints' to \spice\
or to specify initial conditions that force a solution.

\subsubsection{Nodeset and Initial Conditions}

The basic difference between using nodeset and specifying initial conditions
is that the latter forces nodes to the specified voltage while
nodeset only provides hints.
The values
specified by the nodeset line are only used during the first
part of the DC solution
procedure and then ignored in the later parts.  Thus if they are incorrect,
or inconsistent, convergence is not prevented.
As an example of nodeset, ts use as follows in the the simple latch,
will result in the output (node {\tt 2})
converging to 5~V (assuming a CMOS latch):
\par\noindent
\centerline{\pfig{latch.ps}}
\par\noindent
\begin{center}
{\tt .nodeset V(1)=0V}
\end{center}
\vspace{3mm}

When initial conditions are set, they are used through the entire DC solution
right to the start of the transient analysis.
For example in the circuit above, the use of
{\tt .IC V(1)=1V} will result
in {\tt V(1)} starting at 1~V in the transient analysis while
the use of {\tt .nodeset V(1)=1V} would result in  {\tt V(1)} starting
at 0~V.
An error or inconsistency in specifying initial conditions with {\tt .IC} might
prevent \spice\ from converging.

A second way to specify initial conditions is to specify them in
the element lines.  For example, the statement,
\begin{center}
{\tt C1 6 0 IC=3.1}
\end{center}
initializes the voltage across capacitor {\tt C1} to 3.1~V.  For an inductor,
the following statement will set the initial current flowing through it to
4.3~mA:
\begin{center}
{\tt L3 4 5 IC=4.3m}
\end{center}
If {\tt IC=} statements are used then it is necessary to include a ``Use IC=''
({\tt UIC}) statement in the {\tt .tran} statement, e.g.
\begin{center}
.tran  200ps 34ns UIC
\end{center}
Specifying {\tt UIC} commands \spice\ to skip the
DC bias calculation, making it is necessary
for the initial conditions to be completely specified through a combination
of {\tt IC=} and {\tt .IC} statements.  Be careful.

\subsubsection{Simulation Time Step Size}

Using a smaller time step increases both the results accuracy and computer
run-time of the simulation.  One thing to be very aware of is if short
transmission lines or very fast edges are specified
then the simulation time step will be very short.
For example, trying to
obtain a `step response' with a waveform/statement such as the
following will greatly increase rise time (and also quite likely
lead to convergence problems).
\centerline{\pfig{stepsp.ps}}
\par\noindent
\begin{center}
{\tt .Vin 4 0 PWL 0ns 0V 1ps 5V 40ns 5V}
\end{center}
It is also possible to change the
time step, and other step-related parameters, in the {\tt .options}
statement.  Please see the reference catalog for details.

\subsubsection{Transient Analysis Convergence Problems}
\label{transconv}

\spice\ might report a transient analysis convergence problem with
a message like the following:
\begin{center}
{\tt
\begin{verbatim}
*ERROR*: Convergence problem in Transient Analysis at

         TIME =

\end{verbatim} }
{\em etc.}
\end{center}
\par\noindent
Sometimes \spice\ is not so hopeful and just `dumps' you part way
through the analysis, e.g. part way into a 40~ns analysis \spice\
might suddenly stop the analysis at 34~ns and end with:
\begin{center}
{\tt
\begin{verbatim}
  3.400E-09     5.452E+00   6.602E+00   6.892E+00
Y
0         ***** JOB ABORTED
\end{verbatim}
}
\end{center}
In this case, the problem was a too-short implicit time step caused
by a very short (62.3~ps delay) transmission line:
\begin{center}
{\tt Tline3 10 0 11 0 z0=60 td=62.3ps}
\end{center}
Replacing the line with its equivalent lumped circuit,
\begin{center}
{\tt
\begin{verbatim}
Lline3 10 11 7.48nH
Cline3a 10 0 1.03pF
Cline3b 11 0 1.03pF
\end{verbatim}
}
\end{center}
solved the problem.

If your transient analysis convergence problem is not being caused
by a too short a time step, then it is most likely caused by
an error in specifying
a circuit node number or parameter value.  Your circuit and {\tt .model}
lines should be checked carefully.  Often looking at the circuit
description as specified in the output listing
is more useful than looking at the file you typed in, as the output
listing is describing what \spice\ `sees'.

However, \spice\ is a numerical program and can be quirky.  For example,
one simulation driven by the pulse
\begin{center}
{\tt V2 4 0 Pulse(0V 5V 0n 1.2n 1.2n 20n 40n)}
\end{center}
would abort about half way through the simulation.
However, turning the pulse `up-side-down' (interchanging {\tt 0V} and {\tt 5V}),
\begin{center}
{\tt V2 4 0 Pulse(5V 0V 0n 1.2n 1.2n 20n 40n)}
\end{center}
allowed the simulation to complete.

\subsubsection{Spectral Analysis -- Fourier Transform}

The {\tt .Four} control statement can be used to find the spectrum
of any time-varying signal in a transient analysis.
For example, in Chapter 2, we used
the following time-domain signal as the input to an RC circuit:
\par\noindent
\centerline{\pfig{pulseex.ps}}
\par\noindent
\begin{center}
{\tt vin 1 0 pulse (0 1.5 4ns 3ns 5ns 2ns 17ns)}
\end{center}
The addition of the control statement,
\begin{center}
{\tt .Four 58.82MegHz V(1) V(2)}
\end{center}
to this file,
specifies that the spectrum of the input ({\tt V(1)}) and output ({\tt V(2)})
voltage waveforms
are also to be obtained.  The frequency specified in this statement is the
fundamental frequency of the waveform (1/17~ns = 58.82~MHz).
As a result of this statement, the output file reports the magnitude
and phase
of the first nine harmonics for each signal.  In this case, the output
for V(1) is:
\begin{center}
{\tt
\begin{verbatim}
0****     FOURIER ANALYSIS                 TEMPERATURE =   27.000 DEG C

0***********************************************************************

 FOURIER COMPONENTS OF TRANSIENT RESPONSE V(1)
0DC COMPONENT =   5.293D-01
0HARMONIC   FREQUENCY    FOURIER    NORMALIZED    PHASE     NORMALIZED
    NO         (HZ)     COMPONENT    COMPONENT    (DEG)    PHASE (DEG)

     1      5.882D+07   7.750D-01     1.000000   -91.352       0.000

     2      1.176D+08   2.561D-01     0.330483    99.249     190.601

     3      1.765D+08   7.623D-02     0.098364    16.321     107.673

     4      2.353D+08   2.980D-02     0.038451  -123.300     -31.947

     5      2.941D+08   2.502D-02     0.032282  -167.139     -75.786

     6      3.529D+08   6.971D-03     0.008994   -46.439      44.913

     7      4.118D+08   9.869D-03     0.012734    81.719     173.071

     8      4.706D+08   1.722D-02     0.022225    -9.729      81.623

     9      5.294D+08   1.346D-02     0.017363  -169.978     -78.625
\end{verbatim} }
\end{center}
A similar table is obtained for V(2).

\subsection{DC Analyses}
\label{DC}

Spice enables you to conduct the following DC analyses:
\begin{itemize}
  \item DC solution for a particular input voltage/current condition
    ({\tt .OP}).
  \item DC solutions over a range of input conditions ({\tt .DC}).
  \item Small signal DC transfer functions, including gain, input
    and output resistance ({\tt .TF}).
  \item Sensitivity of the DC value of an output to some set of
    parametric variations ({\tt .SENS}).
\end{itemize}
These are discussed in turn.

The insertion of a line with just
\begin{center}
{\tt .OP}
\end{center}
on it asks \spice\ to determine the DC bias point of the circuit with
inductors shorted and capacitors opened, just the same as the DC analysis
conducted before a transient analysis.  It might be used in situations
where you wish to know the DC bias point but the analysis you are
doing does not request it (e.g. such as when determining a frequency response).

A command line beginning with {\tt .DC} instructs \spice\ to sweep the specified
voltage source over the specified range, reporting the DC bias point
for each combination of input conditions.
If more than one source is specified
in the {\tt .DC} statement, then the first source will be swept over its
entire range for every value of the second source.
An example is given in Figure~\ref{fig:DC} in which two analyses alternatives
are presented at the bottom.
The left hand alternative instructs \spice\ to plot the transfer characteristics
of the CMOS inverter, the right hand
example instructs \spice\ to plot
the output V-I characteristics for when Vin is 5~V.
Both examples specify that the voltage sweep is to be from 0 to 5~V in
0.1~V increments.
The resulting output V-I characteristic (obtained using the statements
on the right hand side of the Figure~\ref{fig:DC}) is plotted in
Figure~\ref{VI-out}.

\begin{figure}
\begin{center}
\par\noindent
{\pfig{cmosdc.ps}}
\par\noindent
{\tt
\begin{verbatim}
MOS Inverter
*
M0 0 1 2 0 nenh l=0.8u w=1.6u ad=3.2p as=3.2p pd=5.2u ad=5.2u
M1 2 1 5 5 penh l=0.8u w=1.6u ad=3.2p as=3.2p pd=5.2u ad=5.2u
Vcc 5 0 DC 5V
\end{verbatim} }
\par\noindent
\begin{tabular}{ll}
{\tt Vin 1 0 }      & {\tt Vin 1 0 5V} \\
                    & {\tt Vout 2 0 } \\
{\tt .DC Vin 0 5 0.1}\ \ \ \ \ \ \ \ \ \ \ \ \  & {\tt .DC Vout 0 5 0.1} \\
{\tt .print DC V(2)} & {\tt .print DC I(Vout)}
\end{tabular}
\end{center}
\caption{DC response example.
Two alternative analyses are presented at the bottom and are described
in the text.}
\label{fig:DC}
\end{figure}

%
\begin{figure}
\centerline{\epsfxsize=4in\pfig{cmosinv.ps}}
\caption{Output V-I characteristics for CMOS inverter.}
\label{VI-out}
\end{figure}
Now, if you wish to find the small signal output resistance at say {\tt Vout} =
$X$~V,
\[
  r_{out} = \left. {\partial v\over \partial i}\right|_{V_{out}=X\ V}
\]
then one way to obtain this would be to measure the slope of the plot shown
in Figure~\ref{VI-out} at {\tt Vout} = $0.5$~V.  However, \spice\ provides an
easier way to get this result as a transfer function {\tt .TF}:
\begin{center}
{\tt .TF I(vout) Vout}
\end{center}
There is no need for a {\tt .print} statement with {\tt .TF}.
Running this produces the output:
\begin{center}
{\tt
\begin{verbatim}
 OUTPUT RESISTANCE AT I(VOUT)             =  2.247D+03
\end{verbatim}
}
\end{center}

A DC sweep can also be done by specifying a slow moving input and a
conducting a transient analysis.
Sometimes this is necessary, for example in circuits
with hysteresis, such as a Schmitt Trigger.

Sometimes we wish to know the sensitivities of various output parameters
with respect to variations in circuit parameters.  For example, we might
wish to know whether to specify resistors to +/-10\% or +/-1\% in order
to guarantee a certain bias point in a transistor amplifier.  This is
done with the {\tt .sens} statement.  The following example determines
the sensitivity of the bias point of an amplifier to variations in resistance
values:\\
\centerline{\pfig{ceamp.ps}}
\par\noindent
{\em Extract from input file:}
\par\noindent
\begin{center}
{\tt
\begin{verbatim}
Common-Emitter Amplifier

R1 2 5 40k
R2 2 0 40k
* Measure the Base current
Vbase 3 2
Q1 4 3 6 NQ1A
RC1 5 4 4k
RE1 6 0 100

* AC source with unity magnitude and AC buffered
Vin 1 0 AC
Cbuff 1 2 100u

Vcc 5 0 5V

* Find the bias point
.OP

*Find the sensitivity of the bias voltage at the collector
.sens V(4)
\end{verbatim}
}
\end{center}
\par\noindent
{\em Extracts from output file reporting DC bias point and the sensitivity
analysis:}
\par\noindent
\begin{center}
{\tt
\begin{verbatim}
  NODE   VOLTAGE     NODE   VOLTAGE     NODE   VOLTAGE     NODE   VOLTAGE
NODE   VOLTAGE     NODE   VOLTAGE

 (  1)    0.0000    (  2)    1.1432    (  3)    1.1432    (  4)    0.3220    (
  5)    5.0000    (  6)    0.1237

0DC SENSITIVITIES OF OUTPUT V(4)
0        ELEMENT         ELEMENT       ELEMENT       NORMALIZED
          NAME            VALUE      SENSITIVITY    SENSITIVITY
                                    (VOLTS/UNIT) (VOLTS/PERCENT)

          R1           4.000D+04      1.114D-06      4.457D-04
          R2           4.000D+04     -3.303D-07     -1.321D-04
          RC1          4.000D+03     -7.010D-05     -2.804D-03
          RE1          1.000D+02      1.192D-03      1.192D-03
\end{verbatim}
}
\end{center}

In this case, if the value for {\tt RE1} changed by 100\% the collector voltage
would only change by only 119~mV.

\subsection{Small Signal AC Analysis}

Analog circuits are often analyzed in terms of their frequency response to
steady-state, sinusoidal, small-voltage, input signals.
With small voltage swing signals, all of the circuit elements can be treated
as being linear  around some bias point.
Three types of AC analysis can be done:
\begin{enumerate}
  \item Obtain circuit response(s) as a function of frequency using
    the {\tt .AC} analysis.
  \item Conduct a noise analysis as a function of frequency using
    a {\tt .NOISE} element together with a {\tt .AC} element.
  \item Analyze the circuit for harmonic distortion using the
    {\tt .DISTO} element together with the {\tt .AC} element.
\end{enumerate}
In this section, we discuss the first two types of analysis only.
The distortion analysis capability provided in \spicetwo is somewhat limited
and so is not presented.

Consider the frequency response of the LC filter described, with
its \spice\ file, in Figure~\ref{LC}.
%
\begin{figure}
\centerline{\pfig{lcex.ps}}
\par\noindent
\begin{center}
\begin{tabular}{ll}
{\tt RLC filter} & \\
{\tt * } & \\
{\tt Vin 1 0 AC 1V } & {\em AC signal source} \\
{\tt * } & \\
{\tt R1 1 2 15Ohm } & \\
{\tt L1 2 3 50mH } & \\
{\tt C1 3 0 1.5uF } & \\
{\tt * } & \\
{\tt .AC dec 20 100Hz 10kHz } & {\em AC analysis specification}\\
{\tt * } & \\
{\tt .print AC VM(3) } & {\em print results of AC analysis} \\
\end{tabular}
\end{center}
\caption{LC filter and the \spice\ file specifying a frequency response
analysis.  (Comments in {\em emphasis} are not part of the file.)}
\label{LC}
\end{figure}
%
There are several features in this file that differentiate it from
a file specifying a transient analysis.  First the signal source {\tt  Vin}
is specified as an {\tt AC} source, not a source in the time domain.
Here it specifies a sinusoid with a magnitude of 1~Volt.  The {\tt .AC} control
statement specifies that
we wish the frequency range to be swept over a frequency range of
100~Hz to 10~kHz in decade ({\tt dec})
increments with {\tt 20} points per decade.
i.e. The output contains a total of 40 frequency points,
20 between 100~Hz and 1~kHz
and 20 between 1~KHz and 10~Hz.  The {\tt .print} statement specifies
that this is an AC analysis and specifies that the magnitude of the voltage
({\tt VM}) of node 3 with respect to node 0 be printed at each frequency
point.  Examples of other results that can also be obtained include:
\begin{center}
\begin{tabular}{ll}
{\bf Control statement Example} & {\bf Meaning} \\
{\tt .print AC VR V(2,3)} & {\em Real part of the voltage across the inductor} \\
{\tt .print AC VI V(2,3)} & {\em Imaginary part of the voltage across the inductor} \\
{\tt .print AC VP I(Vin)} & {\em Phase of current through the voltage source} \\
{\tt .print AC VDB(3)} & {\em Voltage in dB, $10\times log_{10}(magnitude)$}
\end{tabular}
\end{center}
The results obtained by running the \spice\ file specified in Figure~\ref{LC}
are shown in Figure~\ref{LC-plot}.  Note again that a DC analysis is carried
out
before the AC analysis so as to obtain the bias point (this is not shown).

\begin{figure}
\centerline{\epsfxsize=5in\pfig{lc.ps}}
\caption{Frequency response of the circuit shown in Figure~\protect{\ref{LC}}.}
\label{LC-plot}
\end{figure}

In Section~\ref{DC}, we show how to obtain the (non-linear) output impedance
as a function of the output voltage.  For small voltage swing signals, all
impedances are linear, so we are interested in input and output impedance
as a function of frequency.
For example, we could plot the output impedance of the
LCR circuit above using the following \spice\ file:
{\tt
\begin{verbatim}
RLC filter
*
* `Short' input so that it does not form
* part of the output impedance
Vin 1 0 AC 0V
*
R1 1 2 15Ohm
L1 2 3 50mH
C1 3 0 1.5uF
*
.AC dec 20 100Hz 10kHz
*
* Measure output impedance with a current source
Iout 0 3 AC 1
*
* Measure Zout:
.print AC VM(3)
.end
\end{verbatim}
}
\par\noindent
The output impedance is plotted in Figure~\ref{zout}.

\begin{figure}
\centerline{\epsfxsize=6in\pfig{lcz.ps}}
\caption{Output impedance vs. frequency for LC
circuit shown in in Figure~\protect{\ref{LC}}.}
\label{zout}
\end{figure}

Spice is also capable of conducting a noise analysis as part of
the AC analysis.  This analysis is often useful as an aid to the
design of analog circuits.   For  full details please see the {\tt .NOISE}
and {\tt .PRINT} control statement descriptions in the reference catalog
(Part III).

\subsection{Monte Carlo Analysis}

The Monte Carlo analysis is a statistical analysis of the circuit causing
the circuit to be analyzed many times with a random change of model parameters
(parameters in a {\tt .MODEL} statement).
{It is available in the \pspice\ version only.}

The form on the Monte Carlo analysis is

{\pspiceninetytwoform{{\tt .MC} NumberOfRuns AnalysisType
        OutputSpecification OutputFunction \B {\tt LIST}\E\\
                {\tt +} \B {\tt OUTPUT(} OutputSampleType {\tt )}\E
                \B {\tt RANGE(}LowValue{\tt ,} HighValue{\tt )}\E\\
        {\tt +} \B {\tt SEED=}SeedValue\E
     }}

Monte Carlo analysis repeates \dc\ analysis as specified by the {\tt .DC} statement,
\ac\ small-signal analysis as specified by the {\tt .AC} statement, or
transient analysis as specified by the {\tt .TRAN} statement.
In the {\tt .MC} statement the way in which the results of the multiple runs
are interested is controlled by the {\it OutputSpecification}] and
{\it OutputFunction} parameters.

A typical use of Monte Carlo analysis is to predict yield of a circuit
by examining the effect of process variations such as length and width
of transistors. As well the effect of temperature on circuit performance
can be investigated.

The initial run uses the nominal parameter values given in the NETLIST.
Subsequent runs statistically vary model parameters indicated as having
either lot or device tolerances. These tolerances
are specified in a {\tt .MODEL} statement.

\subsection{Transfer Function Specification}

The transfer function specifies a small-signal \dc\ analysis from which
a small-signal transfer function and input and output resistances are computed.
The transfer function computed is the ratio of the \dc\ value of the output quantity
to the input quantity.  In the above examples the following transfer functions are
computed:\\[0.1in]
\hspace*{\fill}
\begin{tabular}{|l|l|}
\hline
EXAMPLE & Transfer Function\\
\hline
&\\
.TF V(10) VINPUT   &${{\textstyle\tt V(10)}\over{\textstyle\tt VINPUT}}$\\&\\
.TF V(10,2) ISOURCE&${{\textstyle\tt V(10,2)}\over{\textstyle\tt ISOURCE}}$\\&\\
.TF I(VLOAD) ISOURCE&${{\textstyle\tt I(VLOAD)}\over{\textstyle\tt ISOURCE}}$\\&\\
\hline
\end{tabular} \hspace*{\fill}

{
\subsection{Parameteric Analysis}
\subsection{Sensistivity and Worst Case Analysis}
}

%\section{More on Spice Errors}
 %
%In Section~\ref{err1}, \ref{convergence} and \ref{transconv}
%we discussed how most
%problems in \spice\ runs can be attributed to the circuit file writer
%mis-specifying a node name or a circuit element parameter.  Sometimes
