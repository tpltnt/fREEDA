\element{S}{Voltage Controlled Switch}
\begin{figure}[h]
\centering
\ \pfig{s_spice.ps}
\caption{S --- voltage controlled switch element.}
\end{figure}

\notforsspice{
\form{{\tt S}name $N_{+}$ $N_{-}$ $N_{C+}$ $N_{C-}$ ModelName
   \B {\tt ON}\E \B {\tt OFF}\E }}

\pspiceform{{\tt S}name $N_{+}$ $N_{-}$ $N_{C+}$ $N_{C-}$
ModelName}

\example{S1 1 2 3 4 SWITCH1 \\
         S2 5 6 3 0 SM2 \\
         SWITCH1 1 2 10 0 SMODEL1}

\begin{widelist}
\item[$N_{+}$] is the positive node of the switch.
\item[$N_{-}$] is the negative node of the switch.
\item[$N_{C+}$] is the positive controlling node of the switch.
\item[$N_{C-}$] is the negative controlling node of the switch.
\item[{\it ModelName}] is the model name and is required.
\notforsspice{
\end{widelist}

\begin{widelist}
\item[{\tt ON}] is the optional initial condition.
It is intended for use with the {\tt UIC} option
on  the  {\tt .TRAN}  line,  when  a transient analysis is desired
starting from other than the quiescent operating point.
It is also the initial condition on the device for \dc\ analysis.
\item[{\tt OFF}] is the optional initial condition.
If specified the \dc\ operating point is calculated with the terminal voltages
set to zero.  Once convergence is obtained, the
program continues to iterate to obtain the exact  value of
the  terminal  voltages.  The OFF option is used to enforce the solution
to  correspond  to  a  desired  state if the circuit has more than one stable
state.
}
\end{widelist}
\notforsspice{ The initial conditions are optional.  For the
voltage  controlled switch, nodes $N_{C+}$ and N{C-} are the
positive and negative controlling nodes respectively.  For the
current  controlled switch, the controlling current is that
through the specified voltage source. The direction of positive
controlling current flow is from the positive node, through the
source, to the negative node.\\[0.1in]}
\modeltype{VSWITCH}

\model{VSWITCH}{Voltage-Controlled Switch Model}
\begin{figure}[h]
\centering
\ \pfig{vswitch.ps}
\caption{VSWITCH --- voltage controlled switch model. \label{vswitch}}
\end{figure}

\notforsspice{
The voltage-controlled switch model is supported by
both \spicethree\ and \pspice. However the model keywords differ slightly.\\[0.1in]

\kw{\spicethree\ keywords:}{
{\tt VT}   & threshold voltage\sym{V_{\ms{ON}}}& V     & 0.0    \X
{\tt VH}   & hysteresis voltage\sym{V_{\ms{OFF}}} & V     & 0.0    \X
{\tt RON}  & on resistance      \sym{R_{\ms{ON}}}&$\Omega$&1.0    \X
{\tt ROFF} & off resistance     \sym{R_{\ms{OFF}}}&$\Omega$&1/GMIN\X
}
}

\kw{\pspice\ keywords:}{
{\tt VON}   & threshold voltage\sym{V_{\ms{ON}}}& V     & 0.0    \X
{\tt VOFF}   & hysteresis voltage\sym{V_{\ms{OFF}}} & V     & 0.0    \X
{\tt RON}  & on resistance      \sym{R_{\ms{ON}}}&$\Omega$&1.0    \X
{\tt ROFF} & off resistance     \sym{R_{\ms{OFF}}}&$\Omega$&1/GMIN\X
}

Care must be exercised in using the switch.  An instantaneous switch
is highly nonlinear and will very likely lead to convergence problems.
This problem is alleviated in the {\tt VSWITCH} model by ramping the resistance
of the switch from its off value to its on value.  For this ramping action to be
effective the difference between $V_{\ms{ON}}$ and $V_{\ms{OFF}}$
must not be too small. Also the values of $R_{\ms{ON}}$ and $R_{\ms{OFF}}$
should not be extreme.
The ration $R_{\ms{ON}}/R_{\ms{OFF}}$ should be be as small as possible.

If $R_{\ms{ON}}/R_{\ms{OFF}}$  is large, e.g.
$R_{\ms{ON}}/R_{\ms{OFF}}$ $>$ $10^{12}$, then the default error tolerances
{\tt TRTOL} and {\tt CHGTOL}, specified in a {\tt .OPTIONS} statement
(see page \pageref{.OPTIONSstatement}) may need to be changed.
\begin{widelist}
\item[{\tt TRTOL}] Change to 1.0 from 7.0 idf there are convergence problems
during transient analysis.
\item[{\tt CHGTOL}] If a switch is across a capacitor then {\tt CHGTOL} should be
reduced to $10^{-16}$ if there are convergence problems
during transient analysis.
\end{widelist}

\noindent\underline{\bf \large Switch Model}\\[0.1in]

The switch is modeled by a voltage variable resistor $R$ and an input
input resistance $R_{\ms{IN}}$, see figure \ref{vswitch}.
$R_{\ms{IN}}$ = $1/G_{\ms{MIN}}$ to ensure that the
controlling nodes are not floating and that the voltage $v$ between the
controlling nodes can not change instantaneously.
\notforsspice{ $G_{\ms{MIN}}$ = {\tt GMIN} is described on page \pageref{GMIN}.}
\\[0.1in]

\noindent\underline{\sl \large Standard Calculations}\\[0.1in]
\begin{eqnarray}
R_{\ms{MEAN}} & = & \sqrt{R_{\ms{ON}} + R_{\ms{OFF}}} \\
R_{\ms{RATIO}}& = &       R_{\ms{ON}}/ R_{\ms{OFF}} \\
V_{\ms{MEAN}} & = & \sqrt{V_{\ms{ON}} + V_{\ms{OFF}}} \\
V_{\Delta}& = & \left({{\textstyle v - V_{\ms{MEAN}}} \over
              {\textstyle V_{\ms{ON}}- V_{\ms{OFF}}}}\right)
\end{eqnarray}
If $V_{\ms{ON}}> V_{\ms{OFF}}$
the switch resistance
\begin{equation}
R = \left\{
\begin{array}{ll}
R_{\ms{ON}}                            & v \ge V_{\ms{ON}} \\
R_{\ms{OFF}}                            & v \le V_{\ms{OFF}}\\
R_{\ms{MEAN}}\,
  R_{\ms{RATIO}}^{\textstyle 1.5 V_{\Delta}}\,
  R_{\ms{RATIO}}^{\textstyle 1.5 V_{\Delta}^3}
  & V_{\ms{OFF}} < v < V_{\ms{ON}} \\
\end{array} \right. %}
\end{equation}
If $V_{\ms{ON}}< V_{\ms{OFF}}$
the switch resistance
\begin{equation}
R = \left\{
\begin{array}{ll}
R_{\ms{ON}}                            & v \le V_{\ms{ON}} \\
R_{\ms{OFF}}                            & v \ge V_{\ms{OFF}}\\
R_{\ms{MEAN}}\,
  R_{\ms{RATIO}}^{\textstyle 1.5 V_{\Delta}}\,
  R_{\ms{RATIO}}^{\textstyle 1.5 V_{\Delta}^3}
  & V_{\ms{OFF}} < v < V_{\ms{ON}} \\
\end{array} \right. %}
\end{equation}
\noindent\underline{\sl \large Noise Analysis}\\[0.1in]
\index{VSWITCH, Noise Model}
\index{VSWITCH, Noise Analysis}

The voltage controlled switch noise model accounts for thermal noise generated
in the switch resistance.
The rms (root-mean-square) values of
thermal noise current generators shunting the switch resistance is
\begin{equation}
I_{n} = \sqrt{4kT/R}~\mbox{A/}\sqrt{\mbox{Hz}}
\end{equation}
where $T$ is the analysis temperature in kelvin (K), and $k$ (=
$1.3806226\,10^{-23}$~J/K) is Boltzmans constant.
