"Unable to allocate nnnn bytes"
(malloc error)


"Out of memory"
(malloc error)








Y element:

Minimum of W or S not positive

internal overflow during discretization

number of nonequidistant lines must not exeed 200

unexpected eigenvalue error

no convergence after nnnn iterations

value is of type complex

internal overflow during discretization

number of lines must not exeed 200 if LAT=5

negative width in layer

internal overflow during matrix-reduction

number of remaining lines must not exeed 200

pivot zero during matrix reduction --- check lateral distances or discretization






You do not appear to have an 8087 Co-Processor installed.
The system will crash or produce erroneous results if this is the case





"Circuit has errors ... run aborted"
See output file for details





N or O element

"time sequence error in digital interface:",

I/O error writting digital output to file xxxxxx 

I/O error reading file xxxx for digital input

I/O error writting time to digital simulator interface

err_msg = "\nERROR: SnVHI less than or equal to SnVLO for DOUTPUT model %s";

end of file or error reading DINPUT signal name data from file xxxxx

display.c:	fprintf(OUTFILE,"internal error in NetList.  genp = NULL\n");
Referes to internal representation of netlist. Error occurs when writing
netlist out.  This error shouldn't occur.

display.c:printf("NetList: internal error.  genp = NULL\n");
Referes to internal representation of netlist. Error occurs when writing
netlist out.  This error shouldn't occur.

fseek error!!!! in FourAn.c\n");
Trying to find a string in a file. Either the file does not exist or
the string can not be found.

Circuit has errors ... run aborted

ERROR -- Unable to open probe file xxxx

Expand: Parameter syntax error

ABORT parameter error.

internal error

internal error in SwpVlt

Transient Analysis iterations limit exceeded
This limit may be overridden by using the ITL5




CPU Time limit exceeded




N and O element:
signal name nnnn for device %s not found in file xxxxx

state string too long in file xxxx

"ERROR: bad column number in file xxxx"

ERROR: bad state name in file xxxx

TIMESTEP less than or equal to zero for DINPUT model

SnRLO or SnRHI less than or equal to zero for DINPUT model

SnRLO or SnRHI greater than 1/GMIN for DINPUT model

SnTSW less than or equal to zero for model

Less than two states defined for DINPUT model

DINPUT models xxxx and yyyy use the same file but do not have the same timstep

unable to create mailbox to communicate with digital simulator");

unable to open %s for DINPUT model %s",

unable to open mailbox %s for DINPUT model %s",

file format parameter not betwwen nnn1 and nnn2 for DINPUT

TIMESTEP less than or equal to zero for DOUTPUT model

TIMESCALE less than 1 for DOUTPUT model

SnVHI less than or equal to SnVLO for DOUTPUT model

No states defined for DOUTPUT model

Convergence problem in DC sweep

Convergence problem in bias point calculation

Convergence problem in transient bias point calculation

CPU Time limit exceeded

ERROR -- .DC of TEMP or MODEL PARAMETER cannot be done if .MC specified

Model appears in DC sweep more than once

P element:
port nn is undefined

voltage source xxx which controls switch xxxx is undefined

Model xx referenced by xx is undefined

|VON - VOFF| too small for VSWITCH model

RON or ROFF less than or equal to zero for VSWITCH model

RON or ROFF greater than 1/GMIN for VSWITCH model

|ION - IOFF| too small for ISWITCH model

RON or ROFF less than or equal to zero for ISWITCH model

RON or ROFF greater than 1/GMIN for ISWITCH model

I/O ERROR on file

Only one circuit allowed per file

Only one analysis temperature alowed

Monte Carlo analysis not allowed

Unable to open probe file

A FATAL ERROR HAS OCCURRED

Subcircuit xxx is undefined

Duplicate Name:
keyword in input twice

Less than 2 connections at node xxxxxxxxx

Node xxxx is floating

Voltage loop involving xxxxxx



argument out of range


Y element
fprintf(OUTFILE,"ERROR in Coupled Line MODEL

modelxxx does not match nodes of elementxxxx

unexpected Error in Element 



Z element
mismatching nodes between element and model

ERROR in xxxxx MODEL

value out of range:

unknown error code





SPICE3E errors from FTE
OUTinterface.c:static bool printinfo = false;	/* Print informational "error messages". */
OUTinterface.c:/* Print out error messages. */
OUTinterface.c:        { "Fatal error", ERR_FATAL } ,
OUTinterface.c:int OUTerror(flags,format,names) 
OUTinterface.c:/* the read OUTerror function gets written and INPerror can be converted */
OUTinterface.c:wrd_error(mess, flags)
agraf.c:    /* Reset the max X coordinate to deal with round-off error. */
aspice.c:        perror(deck);
aspice.c:            perror(deck);
aspice.c:            perror(output);
aspice.c:        perror(spicepath);
aspice.c:            perror(p->pr_outfile);
aspice.c:        perror("gethostname");
aspice.c:        perror("socket");
aspice.c:        perror("connect");
aspice.c:                perror(wl->wl_word);
aspice.c:        perror(outfile);
aspice.c:        perror(output);
breakpoint.c:bad:    fprintf(cp_err, "Syntax error.\n");
bspice.c:extern int OUTendDomain(), OUTstopnow(), OUTerror(), OUTattributes();
bspice.c:    OUTerror,
bspice.c:    /* MFB tends to jump to 0 on errors... This will catch it. */
bspice.c:        ft_sperror(err,"SIMinit");
bspice.c:                            perror(*tv);
bspice.c:                perror(tempfile);
bspice.c:                        perror(tempfile);
bspice.c:/* fprintf(stderr, "error, .save card code is broken\n"); */
clip.c:        /* This is wierd -- round-off errors I guess. */
cmath4.c:        fprintf(cp_err, "Internal error: cx_interpolate: bad scale\n");
cspice.c:extern int OUTendDomain(), OUTstopnow(), OUTerror(), OUTattributes();
cspice.c:    OUTerror,
cspice.c:    /* MFB tends to jump to 0 on errors... This will catch it. */
cspice.c:        ft_sperror(err,"SIMinit");
cspice.c:                            perror(*tv);
cspice.c:                perror(tempfile);
cspice.c:                        perror(tempfile);
display.c:    {"error", 0, 0, 0, 0, 0, 0, nop, nop,
display.c:    internalerror(ErrorMessage);
display.c:    externalerror(
display.c:    dispdev = FindDev("error");
display.c:      dispdev = FindDev("error");
display.c:        /* just ignore, since we don't want a million error messages */
display.c:        response->option = error_option;
display.c:    internalerror(ErrorMessage);
display.c:        internalerror("DevSwitch w/o changing back");
display.c:      if (!strcmp(dispdev->name, "error")) {
display.c:        internalerror("no hardcopy device");
doplot.c:        if (response.option == error_option) return;
doplot.c:     * error from doing bad things.
doplot.c:                fprintf(cp_err, "Syntax error.\n");
doplot.c:                            "Syntax error.\n");
doplot.c:                fprintf(cp_err, "Syntax error.\n");
dotcards.c:/* fprintf(stderr, "error, .save card code is broken\n"); */
dotcards.c:                "Internal error: op vector %s not real\n",
error.c: *           $Source: /ic3/quarles/shared/FTE/RCS/error.c,v $
error.c: * Print out in more detail what a floating point error was.
error.c:/* global error message buffer */
error.c:fperror(mess, code)
error.c:        fprintf(cp_err, "%s: Unknown floating point error (# %d).\n", 
error.c:fperror(mess, code)
error.c:/* Print a spice error message. */
error.c:ft_sperror(code, mess)
error.c:/* These error messages are from internal consistency checks. */
error.c:internalerror(message)
error.c:    fprintf(stderr, "internal error:  %s\n", message);
error.c:/* These errors are from external routines like fopen. */
error.c:externalerror(message)
error.c:    fprintf(stderr, "external error:  %s\n", message);
evaluate.c:/* We are careful here to catch SIGILL and recognise them as math errors.
fourier.c:#include "SPerror.h"
fourier.c:                ft_sperror(err, "fourier");
fourier.c:             * interval - should be only for very small error in
graf.c:      internalerror("gr_init:  no range specified");
graf.c:      /* note: where is the error message generated? */
graf.c:        /* note: XXX remove it from dbs, so won't get further errors */
graphdb.c:/* returns NULL on error */
graphdb.c:      internalerror("can't allocate a listgraph");
graphdb.c:    internalerror("tried to destroy non-existent graph");
grid.c:    /* Reset the max X coordinate to deal with round-off error. */
inp.c:                perror(s);
inp.c:                    working->li_error = copy(
inp.c:                if (here->li_error) {
inp.c:                                here->li_error);
inp.c:                            here->li_error, file);
inp.c:                if (here->li_error && (type == LS_PHYSICAL)) {
inp.c:                            here->li_error);
inp.c:                            here->li_error);
inp.c:                    if (there->li_error && 
inp.c:                            there->li_error);
inp.c:                            there->li_error);
inp.c:    /* First throw away any old error messages there might be and fix
inp.c:        if (dd->li_error) {
inp.c:            tfree(dd->li_error);
inp.c:            dd->li_error = NULL;
inp.c:        if (dd->li_error)
inp.c:                dd->li_linenum, dd->li_line, dd->li_error);
inp.c:            perror(wl->wl_word);
inp.c:                perror(filename);
inp.c:                perror(filename);
inp.c:            perror(filename);
inp.c:            perror(tempfile);
inp.c:                perror(wl->wl_word);
inp.c:        perror(wl->wl_word);
mfb.c:      sprintf(ErrorMessage, "MFB error: %s", MFBError(err));
mfb.c:      externalerror(ErrorMessage);
mfb.c:      externalerror("Can't restore control of tty.");
mfb.c:      externalerror("Can't initialize MFB.");
mfb.c:/* note: do error checking */
mfb.c:      externalerror("Bad linestyle");
mfb.c:      externalerror("bad color");
nutmeg.c:extern int OUTendDomain(), OUTstopnow(), OUTerror(), OUTattributes();
nutmeg.c:    OUTerror,
nutmeg.c:    /* MFB tends to jump to 0 on errors... This will catch it. */
nutmeg.c:        ft_sperror(err,"SIMinit");
nutmeg.c:                        perror(*tv);
nutmeg.c:                perror(*tv);
nutmeg.c:                perror(tempfile);
nutmeg.c:                        perror(*tv);
nutmeg.c:        fprintf(cp_err, "Warning: error executing .spiceinit.\n");
nutmeg.c:    /* Set up (void) signal handling for fatal errors. */
nutmegif.c:#include "SPerror.h"
nutmegif.c:int OUTerror() {}
nutmegif.c:/* ARGSUSED */ char * if_errstring(code) { return ("spice error"); }
parse.c:            fprintf(cp_err, "Syntax error.\n");
parse.c:    fprintf(cp_err, "Syntax error.\n");
plot5.c:      perror(graph->devdep);
plot5.c:      internalerror("bad linestyleid");
postscript.c:      perror(graph->devdep);
postscript.c:      internalerror("bad linestyleid");
rawfile.c:        perror(name);
rawfile.c:        perror(name);
runcoms.c:            perror(wl->wl_word);
sconvert.c:                perror(infile);
sconvert.c:            perror(outfile);
sconvert.c:                    fprintf(cp_err, "Write error\n"); \
sconvert.c:        perror(name);
sconvert.c:        fprintf(cp_err, "Error: alignment error in data\n");
sconvert.c:        fprintf(cp_err, "Error: alignment error in data\n");
sconvert.c:        perror(name);
sconvert.c:/* ARGSUSED */ char *if_errstring(c) int c; { return ("error"); }
shyu.c: * error in the simulation, etc). args should be the entire command line,
shyu.c:    int error;
shyu.c:    deck.li_error = NULL;
shyu.c:            ft_sperror(err,"deleteTask");
shyu.c:        ft_sperror(err,"newTask");
shyu.c:            ft_sperror(err,"createOptions");
shyu.c:            ft_sperror(err,"createSense");
shyu.c:        current->error = INPerrCat(current->error,INPmkTemp(
shyu.c:                ft_sperror(err,"createAC"); /* or similar error message */
shyu.c:            current->error = INPerrCat(current->error,INPmkTemp(
shyu.c:        error = INPapName(ckt,which,acJob,steptype,&ptemp);
shyu.c:        if(error) current->error = INPerrCat(current->error,
shyu.c:        INPerror(error));
shyu.c:        error = INPapName(ckt,which,acJob,"numsteps",parm);
shyu.c:        if(error) current->error = INPerrCat(current->error,
shyu.c:        INPerror(error));
shyu.c:        error = INPapName(ckt,which,acJob,"start",parm);
shyu.c:        if(error) current->error = INPerrCat(current->error,
shyu.c:        INPerror(error));
shyu.c:        error = INPapName(ckt,which,acJob,"stop",parm);
shyu.c:        if(error) current->error = INPerrCat(current->error,
shyu.c:        INPerror(error));
shyu.c:            current->error = INPerrCat(current->error,INPmkTemp(
shyu.c:                ft_sperror(err,"createOP"); /* or similar error message */
shyu.c:            current->error = INPerrCat(current->error,INPmkTemp(
shyu.c:            ft_sperror(err,"createOP"); /* or similar error message */
shyu.c:        error = INPapName(ckt,which,dcJob,"name1",&ptemp);
shyu.c:        if(error) current->error = INPerrCat(current->error,INPerror(error));
shyu.c:        error = INPapName(ckt,which,dcJob,"start1",parm);
shyu.c:        if(error) current->error = INPerrCat(current->error,INPerror(error));
shyu.c:        error = INPapName(ckt,which,dcJob,"stop1",parm);
shyu.c:        if(error) current->error = INPerrCat(current->error,INPerror(error));
shyu.c:        error = INPapName(ckt,which,dcJob,"step1",parm);
shyu.c:        if(error) current->error = INPerrCat(current->error,INPerror(error));
shyu.c:            error = INPapName(ckt,which,dcJob,"name2",&ptemp);
shyu.c:            if(error) current->error= INPerrCat(current->error,INPerror(error));
shyu.c:            error = INPapName(ckt,which,dcJob,"start2",parm);
shyu.c:            if(error) current->error= INPerrCat(current->error,INPerror(error));
shyu.c:            error = INPapName(ckt,which,dcJob,"stop2",parm);
shyu.c:            if(error) current->error= INPerrCat(current->error,INPerror(error));
shyu.c:            error = INPapName(ckt,which,dcJob,"step2",parm);
shyu.c:            if(error) current->error= INPerrCat(current->error,INPerror(error));
shyu.c:                ft_sperror(err,"createTRAN"); 
shyu.c:            current->error = INPerrCat(current->error,INPmkTemp(
shyu.c:        error = INPapName(ckt,which,tranJob,"tstep",parm);
shyu.c:        if(error) current->error = INPerrCat(current->error,
shyu.c:        INPerror(error));
shyu.c:        error = INPapName(ckt,which,tranJob,"tstop",parm);
shyu.c:        if(error) current->error = INPerrCat(current->error,
shyu.c:        INPerror(error));
shyu.c:            error = INPapName(ckt,which,tranJob,"tstart",parm);
shyu.c:            if(error) current->error = INPerrCat(current->error,
shyu.c:            INPerror(error));
shyu.c:            error = INPapName(ckt,which,tranJob,"tmax",parm);
shyu.c:            if(error) current->error = INPerrCat(current->error,
shyu.c:            INPerror(error));
shyu.c:                    error = INPapName(ckt,which,tranJob,"tstart",&ptemp);
shyu.c:                    if(error) current->error = INPerrCat(current->error,
shyu.c:                    INPerror(error));
shyu.c:                    error = (*(ft_sim->setAnalysisParm))(ckt,
shyu.c:                    if(error) current->error = INPerrCat(
shyu.c:                    current->error, INPerror(error));
shyu.c:                    error = (*(ft_sim->setAnalysisParm))(ckt,
shyu.c:                    if(error) current->error = INPerrCat(
shyu.c:                    current->error, INPerror(error));
shyu.c:            current->error = INPerrCat(current->error,INPmkTemp(
shyu.c:        ft_sperror(err, "doAnalyses");
signal.c:    fperror("Error", code);
signal.c:    fprintf(cp_err, "\ninternal error -- illegal instruction\n");
signal.c:    fprintf(cp_err, "\ninternal error -- bus error\n");
signal.c:    fprintf(cp_err, "\ninternal error -- segmentation violation\n");
signal.c:        "\ninternal error -- bad argument to system call\n");
spiced.c:            perror("spiced: socket");
spiced.c:            perror("spiced: bind");
spiced.c:        perror("spiced: read");
spiced.c:        perror(program);
spiced.c:        perror("wait");
spiceif.c:        ft_sperror(err, "CKTinit");
spiceif.c:        ft_sperror(err,"newUid");
spiceif.c:        ft_sperror(err,"newTask");
spiceif.c:            ft_sperror(err,"newUid");
spiceif.c:            ft_sperror(err,"createOptions");
spiceif.c: * error in the simulation, etc). args should be the entire command line,
spiceif.c:        deck.li_error = NULL;
spiceif.c:                ft_sperror(err,"deleteTask");
spiceif.c:            ft_sperror(err,"newUid");
spiceif.c:            ft_sperror(err,"newTask");
spiceif.c:                ft_sperror(err,"newUid");
spiceif.c:                ft_sperror(err,"createOptions");
spiceif.c:        if (deck.li_error) {
spiceif.c:            /* INP produdes an E_EXISTS error here... Don't
spiceif.c:            fprintf(cp_err, "Warning: %s\n", deck.li_error);
spiceif.c:            ft_sperror(err, "doAnalyses");
spiceif.c:            ft_sperror(err, "doAnalyses");
spiceif.c:        ft_sperror(err, "setAnalysisParm(options)");
spiceif.c:/* Return a string describing an error code. */
spiceif.c:    return (INPerror(code));
spiceif.c:        ft_sperror(err, "if_getparam");
subckt.c:        if (deck->li_error)
subckt.c:            d->li_error = copy(deck->li_error);
vectors.c:                /* This used to be an error... */
vectors.c:        perror(file);
x10.c:int errorhandler();
x10.c:      internalerror("Can't open X display.");
x10.c:      internalerror(ErrorMessage);
x10.c:    /* we don't want non-fatal X errors to call exit */
x10.c:    XErrorHandler(errorhandler);
x10.c:errorhandler(display, errorev)
x10.c:XErrorEvent *errorev;
x10.c:    externalerror(XErrDescrip(errorev->error_code));
x10.c:      internalerror("can't open graph window");
x10.c:      internalerror(ErrorMessage);
x10.c:        internalerror(ErrorMessage);
x10.c:        externalerror("can't get hardware color");
x10.c:        externalerror("unrecognized return value form XAppendVertex");
x10.c:    internalerror("X_DefineColor not implemented.");
x10.c:    internalerror("X_DefineLinestyle not implemented.");
x10.c:      externalerror("Can't find graph.");
x10.c:          externalerror("Can't find graph.");
x10.c:          response->option = error_option;
x10.c:        internalerror("button_option not implemented");
x10.c:        response->option = error_option;
x10.c:        internalerror("unrecognized input type");
x10.c:        response->option = error_option;
x11.c:int errorhandler();
x11.c:	  internalerror("Can't open X display.");
x11.c:	/* we don't want non-fatal X errors to call exit */
x11.c:	XSetErrorHandler(errorhandler);
x11.c:errorhandler(display, errorev)
x11.c:XErrorEvent *errorev;
x11.c:	XGetErrorText(display, errorev->error_code, ErrorMessage, 1024);
x11.c:	externalerror(ErrorMessage);
x11.c:/* note: what about error handling?  What if XtCreateWidget fails? XXX */
x11.c:	  internalerror(ErrorMessage);
x11.c:	    externalerror(ErrorMessage);
x11.c:	internalerror("X11_Arc not implemented");
x11.c:	internalerror("X11_DefineColor not implemented.");
x11.c:	internalerror("X11_DefineLinestyle not implemented.");
x11.c:	    perror("read");
x11.c:	    internalerror("button_option not implemented");
x11.c:	    response->option = error_option;
x11.c:	    internalerror("unrecognized input type");
x11.c:	    response->option = error_option;
agraf.c:            "Error: asciiplot can't handle scale with length < 2\n");
agraf.c:    "ft_agraf: Internal Error: bad limits %lg, %lg, %lg, %lg...\r\n",
agraf.c:            fprintf(cp_err, "Error: X scale (%s) not monotonic\n",
aspice.c:        fprintf(cp_err, "Error: bad deck %s\n", deck);
aspice.c:    "Error: No spice-3 binary is available for the aspice command.\n");
aspice.c:"ft_checkkids: Internal Error: should be %d jobs done but there aren't any.\n",
aspice.c:            "ft_checkkids: Internal Error: Process %d not a job!\n",
aspice.c:        "Error: there is no remote spice host for this site.\n");
aspice.c:        fprintf(cp_err, "Error: spice/tcp: unknown service\n");
aspice.c:        fprintf(cp_err, "Error: tcp: unknown protocol\n");
aspice.c:        fprintf(cp_err, "Error: unknown host %s\n", rhost);
aspice.c:        fprintf(cp_err, "Error: remote spiced says %s\n", buf);
aspice.c:            fprintf(cp_err, "Error: no circuits loaded\n");
aspice.c:        "Error: you must supply one argument, the input deck.\n");
breakpoint.c:                "com_sttus: Internal Error: bad db %d...\n",
breakpoint.c:            fprintf(cp_err, "Error: no debugs in effect\n");
breakpoint.c:                fprintf(cp_err, "Error: %s isn't a number.\n",
breakpoint.c:                "ft_bpcheck: Internal Error: bad db %d\n", 
breakpoint.c:            fprintf(cp_err, "Error: %s: no such node\n", 
breakpoint.c:            fprintf(cp_err, "Error: %s: no such node\n", 
breakpoint.c:                "satisfied: Internal Error: bad cond %d\n", 
breakpoint.c:                "printcond: Internal Error: bad cond %d", 
bspice.c:        fprintf(cp_err, "Internal Error: jump to zero\n");
bspice.c:            fprintf(cp_err, "Error: No circuit loaded!\n");
bspice.c:bad:    fprintf(cp_err, "Internal Error: ft_cktcoms: bad commands\n");
cmath2.c:        fprintf(cp_err, "Error: can't normalize a 0 vector\n");
cmath3.c:            fprintf(cp_err, "Error: divide by 0\n");
cmath3.c:            fprintf(cp_err, "Error: divide by 0\n");
cmath4.c:        fprintf(cp_err, "Error: new scale has complex data\n");
cmath4.c:                    "Error: new scale has complex data\n");
cmath4.c:        fprintf(cp_err, "Error: old scale has complex data\n");
cmath4.c:                    "Error: old scale has complex data\n");
cmath4.c:        fprintf(cp_err, "Error: lengths don't match\n");
cmath4.c:        fprintf(cp_err, "Error: argument has complex data\n");
cmath4.c:            fprintf(cp_err, "Error: old scale not monotonic\n");
cmath4.c:            fprintf(cp_err, "Error: new scale not monotonic\n");
compose.c:            fprintf(cp_err, "Error: max dimensionality is %d\n",
compose.c:        "Error: all vectors must be of the same dimensionality\n");
compose.c:                    fprintf(cp_err, "Error: bad syntax\n");
compose.c:                            "Error: bad syntax\n");
compose.c:                            "Error: bad syntax\n");
compose.c:                    fprintf(cp_err, "Error: bad syntax\n");
compose.c:                        "Error: bad parm %s = %s\n",
compose.c:                        "Error: bad parm %s = %s\n",
compose.c:                        "Error: bad parm %s = %s\n",
compose.c:                        "Error: bad parm %s = %s\n",
compose.c:                        "Error: bad parm %s = %s\n",
compose.c:                        "Error: bad parm %s = %s\n",
compose.c:                        "Error: bad parm %s = %s\n",
compose.c:                        "Error: bad parm %s = %s\n",
compose.c:                        "Error: bad parm %s = %s\n",
compose.c:                        "Error: bad parm %s = %s\n",
compose.c:                        "Error: bad parm %s = %s\n",
compose.c:                        "Error: bad parm %s = %s\n",
compose.c:            fprintf(cp_err, "Error: step cannot = 0.0\n");
compose.c:    "Error: can have at most one of (lin, log, dec, random, gauss)\n");
compose.c:"Error: either one of (lin, log, dec, random, gauss) must be given, or all\n");
cspice.c:        fprintf(cp_err, "Internal Error: jump to zero\n");
cspice.c:            fprintf(cp_err, "Error: No circuit loaded!\n");
cspiceif.c:                    "Error: %s: no matching vectors.\n",
cspiceif.c:                        "Error: %s: no such vector.\n",
cspiceif.c:                "checkvalid: Internal Error: bad node\n");
debugcoms.c:        fprintf(cp_err, "Error: no circuit loaded.\n");
debugcoms.c:        fprintf(cp_err, "Error: no circuit loaded.\n");
define.c:            fprintf(cp_err, "Error: %s is a predefined function.\n",
define.c:        fprintf(cp_err, "trcopy: Internal Error: bad parse node\n");
device.c:        fprintf(cp_err, "Error: no circuit loaded\n");
device.c:        fprintf(cp_err, "Error: no circuit loaded\n");
diff.c:                fprintf(cp_err, "Error: no such plot %s...\n",
diff.c:            fprintf(cp_err, "Error: plot names not given.\n");
diff.c:            fprintf(cp_err, "Error: no such plot %s\n", 
diff.c:            fprintf(cp_err, "Error: no such plot %s\n",
display.c:    sprintf(ErrorMessage, "Can't find device %s.", name);
display.c:    internalerror(ErrorMessage);
display.c:    sprintf(ErrorMessage,
display.c:    internalerror(ErrorMessage);
doplot.c:        fprintf(cp_err, "Error: no vectors given\n");
doplot.c:                fprintf(cp_err, "Error: misplaced vs arg\n");
doplot.c:                        "Error: missing vs arg\n");
doplot.c:            fprintf(cp_err, "Error: %s: no such vector\n",
doplot.c:"Error: plot must be either all pole-zero or contain no poles or zeros\n");
doplot.c:            "Error: X values must be >= 0 for log scale\n");
doplot.c:            "Error: Y values must be >= 0 for log scale\n");
doplot.c:                "Error: can't interpolate %s\n", v->v_name);
doplot.c:    fprintf(cp_err, "Error: bad %s parameters.\n", name);
dotcards.c:                        "Error: bad line %s\n",
dotcards.c:                    fprintf(cp_err, "Error: bad line %s\n",
dotcards.c:                        "Error: no %s plot found\n",
dotcards.c:                    fprintf(cp_err, "Error: bad line %s\n",
dotcards.c:                        "Error: no %s plot found\n",
dotcards.c:bad:    fprintf(cp_err, "Internal Error: ft_cktcoms: bad commands\n");
dotcards.c:                fprintf(cp_err, "Error bad limits %s...\n",
dotcards.c:                fprintf(cp_err, "Error bad limits %s...\n",
error.c:char ErrorMessage[1024];
evaluate.c:    fprintf(cp_err, "Error: argument out of range for math function\n");
evaluate.c:        fprintf(cp_err, "ft_evaluate: Internal Error: bad node...\n");
evaluate.c:        fprintf(cp_err, "Error: no such vector %s\n", d->v_name);
evaluate.c:        fprintf(cp_err, "Error: no scale for vector %s\n", v->v_name);
evaluate.c:        fprintf(cp_err, "Error: strange range spec...\n");
evaluate.c:        fprintf(cp_err, "Error: something funny..\n");
evaluate.c:                "op_ind: Internal Error: len %d should be %d\n",
evaluate.c:            fprintf(cp_err, "Error: no indexing on a scalar (%s)\n",
evaluate.c:        fprintf(cp_err, "Error: index %s is not of length 1\n",
evaluate.c:            fprintf(cp_err, "Error: bad v() syntax\n");
evaluate.c:            fprintf(cp_err, "Error: no such vector %s\n", buf);
fourier.c:        fprintf(cp_err, "Error: no vectors loaded.\n");
fourier.c:        fprintf(cp_err, "Error: fourier needs real time scale\n");
fourier.c:        fprintf(cp_err, "Error: bad fund freq %s\n", wl->wl_word);
fourier.c:                    "Error: lengths don't match: %d, %d\n",
fourier.c:                fprintf(cp_err, "Error: %s isn't real!\n", 
fourier.c:                "Error: wavelength longer than time span\n");
fourier.c:                        "Error: can't interpolate\n");
grid.c:        "gr_fixgrid: Internal Error: bad limits %lg, %lg, %lg, %lg...\r\n",
grid.c:        fprintf(cp_err, "Error: 0 radius in polargrid\n");
grid.c:        fprintf(cp_err, "smithgrid: Internal Error: screwed up\n");
inp.c:                "Error: .include filename missing\n");
inp.c:                    "Error: bad listing type %s\n", s);
inp.c:        fprintf(cp_err, "Error: no circuit loaded.\n");
inp.c:        fprintf(cp_err, "inp_list: Internal Error: bad type %d\n", 
inp.c:            out_printf("Error on line %d : %s\n%s\n",
interpolate.c:        fprintf(cp_err, "Error: lengths too small to interpolate.\n");
interpolate.c:        fprintf(cp_err, "Error: degree is %d, can't interpolate...\n",
interpolate.c:            fprintf(cp_err, "ft_interpolate: Internal Error...\n");
interpolate.c:                    "interpolate: Internal Error...\n");
interpolate.c:                "Error: polyfit: x = %le, y = %le, int = %le\n",
interpolate.c:                "Error: polyfit: x = %le, y = %le, int = %le\n",
interpolate.c:            "Error: can't get transient parameters from circuit\n");
interpolate.c:         "Error: bad parameters -- start = %G, stop = %G, step = %G\n",
interpolate.c:        fprintf(cp_err, "Error: no vectors available\n");
interpolate.c:        fprintf(cp_err, "Error: non-real time scale for %s\n",
interpolate.c:        fprintf(cp_err, "Error: plot must be a transient analysis\n");
interpolate.c:                fprintf(cp_err, "Error: no such vector %s\n",
interpolate.c:        fprintf(cp_err, "Error: can't interpolate %s\n", ov->v_name);
mfb.c:extern char ErrorMessage[];
mfb.c:      sprintf(ErrorMessage, "MFB error: %s", MFBError(err));
mfb.c:      externalerror(ErrorMessage);
misccoms.c:        fprintf(cp_err, "Error: can't find help dir %s\n", HELPPATH);
nutmeg.c:        fprintf(cp_err, "main: Internal Error: jump to zero\n");
nutmeg.c:                fprintf(cp_err, "Error: bad option %s\n", *tv);
nutmeg.c:                fprintf(cp_err, "Error: no circuit loaded!\n");
options.c:                    "Error: bad type for debug var\n");
options.c:            fprintf(cp_err, "Error: bad type for debug var\n");
options.c:            fprintf(cp_err, "Error: plot name not a string\n");
options.c:            fprintf(cp_err, "Error: can't set plot name\n");
options.c:            fprintf(cp_err, "Error: can't set plot title\n");
options.c:            fprintf(cp_err, "Error: can't set plot date\n");
options.c:            "cp_usrset: Internal Error: Bad var type %d\n",
parse.c:                    "Error: %s: no matching vectors.\n",
parse.c:                        "Error: %s: no such vector.\n",
parse.c:                "checkvalid: Internal Error: bad node\n");
parse.c:#define R 4 /* Error. */
parse.c:                fprintf(cp_err, "Error: stack overflow\n");
parse.c:        fprintf(cp_err, "mkbnode: Internal Error: no such op num %d\n",
parse.c:        fprintf(cp_err, "mkunode: Internal Error: no such op num %d\n",
parse.c:            fprintf(cp_err, "Error: no such function as %s.\n", 
parse.c:        fprintf(cp_err, "Error: no function as %s with that arity.\n",
plotcurve.c:        fprintf(cp_err, "Error: polydegree is %d, can't plot...\n",
plotcurve.c:        fprintf(cp_err, "Error: bad grid size %d\n", gridsize);
plotcurve.c:            fprintf(cp_err, "Error: can't put %s on gridsize %d\n",
plotcurve.c:            fprintf(cp_err, "plotcurve: Internal Error: ack...\n");
plotcurve.c:                    "plotcurve: Internal Error: ack...\n");
postcoms.c:            fprintf(cp_err, "Error: bad let syntax\n");
postcoms.c:        fprintf(cp_err, "Error: bad variable name %s\n", vname);
postcoms.c:                fprintf(cp_err, "Error: bad variable name %s\n",
postcoms.c:            fprintf(cp_err, "Error: no such vector %s\n", vname);
postcoms.c:            fprintf(cp_err, "Error: no such vector as %s.\n", 
postcoms.c:            fprintf(cp_err, "Error: no such plot.\n");
postcoms.c:        fprintf(cp_err, "Error: bad number %s\n", wl->wl_word);
postcoms.c:        fprintf(cp_err, "Error: bad index %d\n", ind);
postcoms.c:                fprintf(cp_err, "Error: no such plot %s\n",
postcoms.c:        fprintf(cp_err, "Error: can't destroy the constant plot\n");
postcoms.c:                "Internal Error: kill plot -- not in list\n");
rawfile.c:                    "Error: misplaced Command: line\n");
rawfile.c:                    "Error: misplaced Command: line\n");
rawfile.c:                fprintf(cp_err, "Error: no plot name given\n");
rawfile.c:                        "Error: bad var line %s\n",
rawfile.c:                        "Error: bad var line %s\n",
rawfile.c:                            "Error: bad arg %s\n",
rawfile.c:                            "Error: bad arg %s\n",
rawfile.c:                fprintf(cp_err, "Error: no plot name given\n");
rawfile.c:                        "Error: no such vector %s\n",
rawfile.c:                        "Error: bad rawfile\n");
rawfile.c:                        "Error: bad rawfile\n");
rawfile.c:                        "Error: bad rawfile\n");
rawfile.c:                        "Error: bad rawfile\n");
rawfile.c:                        "Error: bad rawfile\n");
rawfile.c:            "Error: strange line in rawfile -- load aborted\n");
resource.c:        fprintf(cp_err, "Error: no rusage information on %s,\n", name);
runcoms.c:        fprintf(cp_err, "Error: there aren't any circuits loaded.\n");
runcoms.c:        fprintf(cp_err, "Error: there aren't any circuits loaded.\n");
runcoms.c:        fprintf(cp_err, "Error: circuit not parsed.\n");
runcoms.c:        fprintf(cp_err, "Error: there is no circuit loaded.\n");
sconvert.c:                fprintf(cp_err, "Error: unexpected EOF\n"); \
sconvert.c:        fprintf(cp_err, "Error: alignment error in data\n");
sconvert.c:        fprintf(cp_err, "Error: alignment error in data\n");
shyu.c:            " Error: unknown parameter on .sens - ignored \n"));
signal.c:    fperror("Error", code);
spiced.c:            fprintf(stderr, "Error: spice/tcp: unknown service\n");
spiced.c:            fprintf(stderr, "Error: tcp: unknown protocol\n");
spiced.c:        fprintf(stderr, "Error: bad init line: %s\n", buf);
spiceif.c:        fprintf(cp_err, "if_run: Internal Error: bad run type %s\n",
spiceif.c:            "if_option: Internal Error: bad option type %d.\n",
spiceif.c:    fprintf(cp_err, "Error: bad type given for option %s --\n", name);
spiceif.c:                "Error: no such device or model name %s\n",
spiceif.c:            "Internal Error: no parameter '%s' on device '%s'\n",
spiceif.c:                "Error: no such device or model name %s\n",
spiceif.c:            fprintf(cp_err, "Error: no such parameter %s.\n",
spiceif.c:            "parmtovar: Internal Error: bad PARM type %d.\n",
spiceif.c:                "if_getstat: Internal Error: can't get %s\n",
spiceif.c:                "if_getstat: Internal Error: can't get %s\n",
subckt.c:                fprintf(cp_err, "Error: unknown subckt: %s\n",
subckt.c:            fprintf(cp_err, "Error: misplaced %s card: %s\n", sbend,
subckt.c:                fprintf(cp_err, "Error: no %s card.\n", sbend);
subckt.c:                fprintf(cp_err, "Error: no %s card.\n", sbend);
subckt.c:        fprintf(cp_err, "Error: infinite subckt recursion\n");
subckt.c:            "settrans: Internal Error: wrong number of params\n");
subckt.c:            fprintf(cp_err, "Error: no such subcircuit: %s\n", s);
subckt.c:        fprintf(cp_err, "Error: too few nodes for BJT: %s\n", name);
types.c:            fprintf(cp_err, "Error: too many types defined\n");
types.c:                fprintf(cp_err, "Error: too many plot abs\n");
types.c:        fprintf(cp_err, "Error: missing 'p' or 'v' argument\n");
types.c:        fprintf(cp_err, "Error: no such type as '%s'\n", type);
types.c:            fprintf(cp_err, "Error: no such vector %s.\n",
vectors.c:            "Error: plot wildcard (name %s) matches nothing\n",
vectors.c:        "Error: circuit parameters only available with spice\n");
vectors.c:        fprintf(cp_err, "vec_new: Internal Error: no cur plot\n");
vectors.c:        fprintf(cp_err, "vec_free: Internal Error: plot ptr is 0\n");
vectors.c:                "vec_free: Internal Error: %s not in plot\n",
vectors.c:        fprintf(cp_err, "Error: no such plot named %s\n", name);
x10.c:      sprintf(ErrorMessage, "Can't open %s.", displayname);
x10.c:      internalerror(ErrorMessage);
x10.c:    XErrorHandler(errorhandler);
x10.c:XErrorEvent *errorev;
x10.c:    externalerror(XErrDescrip(errorev->error_code));
x10.c:      sprintf(ErrorMessage, "can't open font %s", fontname);
x10.c:      internalerror(ErrorMessage);
x10.c:        (void) sprintf(ErrorMessage,
x10.c:        internalerror(ErrorMessage);
x11.c:	XSetErrorHandler(errorhandler);
x11.c:XErrorEvent *errorev;
x11.c:	XGetErrorText(display, errorev->error_code, ErrorMessage, 1024);
x11.c:	externalerror(ErrorMessage);
x11.c:	  sprintf(ErrorMessage, "can't open font %s", fontname);
x11.c:	  internalerror(ErrorMessage);
x11.c:	    (void) sprintf(ErrorMessage,
x11.c:	    externalerror(ErrorMessage);



SMPerror.h
definitions for error codes returned by SPICE3 routines.

#define E_INTERN E_PANIC
#define E_BADMATRIX (E_PRIVATE+1)/* ill-formed matrix can't be decomposed */
#define E_SINGULAR (E_PRIVATE+2) /* matrix is singular */
#define E_ITERLIM (E_PRIVATE+3)  /* iteration limit reached,operation aborted */
#define E_ORDER (E_PRIVATE+4)    /* integration order not supported */
#define E_METHOD (E_PRIVATE+5)   /* integration method not supported */
#define E_TIMESTEP (E_PRIVATE+6) /* timestep too small */
#define E_XMISSIONLINE (E_PRIVATE+7)    /* transmission line in pz analysis */
#define E_MAGEXCEEDED (E_PRIVATE+8) /* pole-zero magnitude too large */
#define E_SHORT (E_PRIVATE+9)   /* pole-zero input or output shorted */
#define E_INISOUT (E_PRIVATE+10)    /* pole-zero input is output */
#define E_ASKCURRENT (E_PRIVATE+11) /* ac currents cannot be ASKed */
#define E_ASKPOWER (E_PRIVATE+12)   /* ac powers cannot be ASKed */
#define E_NODUNDEF (E_PRIVATE+13) /* node not defined in noise anal */
#define E_NOACINPUT (E_PRIVATE+14) /* no ac input src specified for noise */
#define E_NOF2SRC (E_PRIVATE+15) /* no source at F2 for IM distortion analysis */
#define E_NODISTO (E_PRIVATE+16) /* no distortion analysis - NODISTO defined */
#define E_NONOISE (E_PRIVATE+17) /* no noise analysis - NONOISE defined */
char *SPerror();





IFerrmsgs.h
#define E_PANIC 1       /* vague internal error for "can't get here" cases */
#define E_EXISTS 2      /* warning/error - attempt to create duplicate */
                        /* instance or model. Old one reused instead */
#define E_NODEV 3       /* attempt to modify a non-existant instance */
#define E_NOMOD 4       /* attempt to modify a non-existant model */
#define E_NOANAL 5      /* attempt to modify a non-existant analysis */
#define E_NOTERM 6      /* attempt to bind to a non-existant terminal */
#define E_BADPARM 7     /* attempt to specify a non-existant parameter */
#define E_NOMEM 8       /* insufficient memory available - VERY FATAL */
#define E_NODECON 9     /* warning/error - node already connected, old */
                        /* connection replaced */
#define E_UNSUPP 10     /* the specified operation is unsupported by the */
                        /* simulator */
#define E_PARMVAL 11    /* the parameter value specified is illegal */
#define E_NOTEMPTY 12   /* deleted still referenced item. */
#define E_NOCHANGE 13   /* simulator can't tolerate any more topology changes */
#define E_NOTFOUND 14   /* simulator can't find something it was looking for */
#define E_BAD_DOMAIN 15 /* output interface begin/end domain calls mismatched */

#define E_PRIVATE 100   /* messages above this number are private to */
                        /* the simulator and MUST be accompanied by */
                        /* a proper setting of errMsg */
                        /* this constant should be added to all such messages */
                        /* to ensure error free operation if it must be */
                        /* changed in the future */

extern char *errMsg;    /* descriptive message about what went wrong */
                        /* MUST be malloc()'d - front end will free() */
                        /* this should be a detailed message,and is assumed */
                        /* malloc()'d so that you will feel free to add */
                        /* lots of descriptive information with sprintf*/

extern char *errRtn;    /* name of the routine declaring error */
                        /* should not be malloc()'d, will not be free()'d */
                        /* This should be a simple constant in your routine */
                        /* and thus can be set correctly even if we run out */
                        /* of memory */



static char *unknownError = "Unknown error code";
static char *Pause =        "Pause requested";
static char *intern =       "Impossible error - can't occur";
static char *exists =       "Device already exists, existing one being used";
static char *nodev =        "No such device";
static char *noterm =       "No such terminal on this device";
static char *nomod =        "No such model";
static char *badparm =      "No such parameter on this device";
static char *nomem =        "Out of Memory";
static char *badmatrix =    "Matrix can't be decomposed as is";
static char *singular =     "Matrix is singular";
static char *iterlim =      "Iteration limit reached";
static char *order =        "Unsupported integration order";
static char *method =       "Unsupported integration method";
static char *timestep =     "Timestep too small";
static char *xmission =     "transmission lines not supported by pole-zero";
static char *toobig =       "magnitude overflow";
static char *isshort =      "input or output shorted";
static char *inisout =      "transfer function is 1";
static char *nodisto =	    "No distortion analysis: compiled with NODISTO macro";
static char *nonoise =	    "No noise analysis: compiled with NONOISE macro";


Unknown error code
\marginid{code}
\index{ERROR, unknown error code}

static char *Pause =        "Pause requested\n";
static char *panic =        "Impossible error - can't occur\n";
static char *exists =       "Device already exists, existing one being used\n";
static char *nodev =        "No such device\n";
static char *nomod =        "No such model\n";
static char *noanal =       "No such analysis type\n";
static char *noterm =       "No such terminal on this device\n";
static char *badparm =      "No such parameter on this device\n";
static char *nomem =        "Out of Memory\n";
static char *nodecon =      "Warning: old connection replaced\n";
static char *unsupp =       "Action unsupported by this simulator\n";
static char *parmval =      "Parameter value is illegal\n";
static char *notempty =     "Can't delete - still referenced\n";
static char *nochange =     "Sorry, simulator can't handle that now\n";
static char *notfound =     "Not found\n";
