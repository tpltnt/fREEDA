\bibliographystyle{plain}
\begin{thebibliography}{99}
\bibitem{imtiaz} S. M. S. Imtiaz and S. M. El-Ghazaly, ``Global
modeling of millimeter-wave circuits: electromagnetic simulation of
amplifiers,'' IEEE Trans. on Microwave Theory and Tech., vol 45,
pp. 2208-2217.  Dec. 1997.

\bibitem{kuo} C.-N. Kuo, R.-B. Wu, B. Houshmand, and T. Itoh, Modeling
of microwave active devices using the FDTD analysis based on the
voltage-source approach, IEEE Microwave Guided Wave Lett., vol. 6,
pp. 199-201, May 1996.

\bibitem{larique} E. Larique, S. Mons, D. Baillargeat, S. Verdeyme,
M. Aubourg, P.  Guillon, and R. Quere, ``Electromagnetic analysis for
microwave FET modeling,'' IEEE microwave and guided wave letters Vol
8, pp. 41-43, Jan. 1998.

\bibitem{todd1} T. W. Nuteson, H. Hwang, M. B. Steer, K. Naishadham,
J.W.Mink, and J. Harvey, ``Analysis of finite grid structures with
lenses in quasi-optical systems,'' IEEE Trans. Microwave Theory
Techniques, pp. 666-672, May 1997.

\bibitem{antenna} M. B. Steer, M. N. Abdullah, C. Christoffersen,
M. Summers, S. Nakazawa, A. Khalil, and J. Harvey, ``Integrated
electro-magnetic and circuit modeling of large microwave and
millimeter-wave structures,'' Proc. 1998 IEEE Antennas and Propagation
Symp., pp. 478-481, June 1998.

\bibitem{kunisch} J. Kunisch and I. Wolff, ``Steady-state analysis of
nonlinear forced and autonomous microwave circuits using the
compression approach,'' Int. J. of Microwave and Millimeter-Wave
Computer-Aided Engineering, vol. 5, No. 4, pp. 241-225, 1995
%
\bibitem{cormen:90}
T. H. Cormen, C. E. Leiserson, R. L. Rivest
\newblock {\em Introduction to Algorithms},
\newblock The MIT Press, McGraw-Hill Book Company, 1990.
%
\bibitem{eliens} A. Eli\"ens, {Principles of object-oriented
software development}, Adison-Wesley, 1995.
%
\bibitem{dep_inv} R. C. Martin. ``The dependency inversion principle,''
{C++ Report}, May 1996.
%
\bibitem{open} R. C. Martin, ``The Open Closed Principle,'' {C++
Report}, Jan. 1996.
%
\bibitem{liskov} R. C. Martin, ``The Liskov Substitution Principle,''
{C++ Report}, March 1996.
%
\bibitem{int_seg} R. C. Martin, ``The Interface Segregation Principle,''
{C++ Report}, Aug 1996.
%
\bibitem{uml_tut} R. C. Martin, ``UML Tutorial: Part 1 --- Class
Diagrams,'' Engineering Notebook Column, {C++ Report}, Aug. 1997.
%
\bibitem{kai} A. D. Robison, ``C++ Gets Faster for Scientific
Computing,'' {Computers in Physics}, vol. 10, pp. 458-462, 1996.
%
\bibitem{c++fortran} J. R. Cary and S. G. Shasharina, ``Comparison of
C++ and Fortran 90 for Object-Oriented Scientific Programming,''
Available from Los Alamos National Laboratory as Report
No. LA-UR-96-4064.
%
\bibitem{oonpage} The Object Oriented Numerics Page,
http://oonumerics.org/.
%
\bibitem{STL} Silicon Graphics, Standard Template Library
Programmer's Guide, http://www.sgi.com/Technology/STL/.
%
\bibitem{todd} T. Veldhuizen, {Techniques for Scientific C++ - Version
0.3}, Indiana University, Computer Science Department,
1999. (http://extreme.indiana.edu/~tveldhui/papers/techniques/)
%
\bibitem{adol-c:96} A. Griewank, D. Juedes, J. Utke, ``Adol-C: A
Package for the Automatic Differenciation of Algorithms Written in
C/C++,'' ACM TOMS, vol. 22(2), pp. 131-167, June 1996.
%
\bibitem{mv++} R. Pozo, {MV++ v. 1.5a}, Reference Guide, National
Institute of Standards and Technology, 1997.
%
\bibitem{FFTW} M. Frigo and S. G. Johnson, {FFTW User's Manual},
Massachusetts Institute of Technology, September 1998.
%
\bibitem{Sparse} K. S. Kundert and A. Songiovanni-Vincentelli, {Sparse
user's guide - a sparse linear equation solver}, Dept. of
Electrical Engineering and Computer Sciences, University of
California, Berkeley, Calif. 94720, Version 1.3a, Apr 1988.
%
\bibitem{NNES} R. S. Bain, {NNES user's manual}, 1993.
%
\bibitem{gnuplot} Gnuplot. Copyright(C) 1986 - 1993, 1998 Thomas
Williams, Colin Kelley and many others.
%
\bibitem{aplac1} M. Valtonen and T. Veijola, ``A microcomputer tool
especially suited for microwave circuit design in frequency and time
domain,'' Proc. URSI/IEEE National Convention on Radio
Science, Espoo, Finland, 1986, p. 20,
%
\bibitem{aplac2} M. Valtonen, P. Heikkil\"a, A. Kankkunen, K.
Mannersalo, R. Niutanen, P. Stenius, T. Veijola and J.  Virtanen,
``APLAC - A new approach to circuit simulation by object
orientation,'' {10th European Conference on Circuit Theory and
Design Dig.}, 1991.
%
\bibitem{codecs} K. Mayaram and D. O. Pederson, ``CODECS: an
object-oriented mixed-level circuit and device simulator,'' {1987
IEEE Int. Symp. on Circuits and Systems Digest}, 1987, pp 604-607.
%
\bibitem{davis1} A. Davis, ``An object-oriented approach to circuit
simulation,'' {1996 IEEE Midwest Symp. on Circuits and
Systems Dig.}, 1996, pp 313-316.
%
\bibitem{feldmann} B. Melville, P. Feldmann and S. Moinian, ``A C++
environment for analog circuit simulation,'' {1992 IEEE
Int. Conf. on Computer Design: VLSI in Computers and Processors.}
%
\bibitem{ngoya} P. Carvalho, E. Ngoya, J. Rousset and J. Obregon,
``Object-oriented design of microwave circuit simulators,'' {1993
IEEE MTT-S Int. Microwave Symp. Digest}, June 1993, pp 1491-1494.
%
\bibitem{local:reference:node:christoffersen} C. E. Christoffersen and
M. B. Steer ``Implementation of the local reference concept for
spatially distributed circuits,'' {Int. J. of RF and Microwave
Computer-Aided Eng.}, vol. 9, No. 5, 1999.
%
\bibitem{local:reference:node:khalil} A. I. Khalil and M. B. Steer
``Circuit theory for spatially distributed microwave circuits,'' {IEEE
Trans. on Microwave Theory and Techn.}, vol. 46, Oct. 1998, pp
1500-1503.
%
\bibitem{svtr} C. E. Christoffersen, M. Ozkar, M. B. Steer, M. G. Case
and M. Rodwell, ``State variable-based transient analysis using
convolution,'' {IEEE Transactions on Microwave Theory and Techniques},
Vol. 47, June 1999, pp. 882-889.
%
\bibitem{svhb} C. E. Christoffersen, M. B. Steer and M. A. Summers,
``Harmonic balance analysis for systems with circuit-field
interactions,'' {1998 IEEE Int. Microwave Symp. Dig.}, June 1998,
pp. 1131-1134.
%
\bibitem{speelpenning} B. Speelpenning. ``Compiling Fast Partial
Derivatives of Functions Given by Algorithms,'' Ph.D. thesis (Under
the supervision of W. Gear), Department of Computer Science,
University of Illinois at Urbana-Champaign, Urbana-Champaign, Ill.,
January 1980.
%
\bibitem{coleman} T. F. Coleman y G. F. Jonsson, ``The Efficient
Computation of Structured Gradients using Automatic Differentiation,''
Cornell Theory Center Technical Report CTC97TR272, April 28, 1997
%
\bibitem{rodwell} H. S. Tsai, M. J. W. Rodwell and R. A. York,
``Planar amplifier array with improved bandwidth using folded-slots,''
{IEEE Microwave and Guided Wave Letters}, vol. 4, April 1994,
pp. 112-114.
%
\bibitem{steer:abdullah:1998} M. B. Steer, M. N. Abdullah,
C. Christoffersen, M. Summers, S. Nakazawa, A. Khalil, and J.  Harvey,
``Integrated electro-magnetic and circuit modeling of large microwave
and millimeter-wave structures,''
{Proc. 1998 IEEE Antennas and Propagation Symp.},
pp. 478--481, June 1998.
%
\bibitem{mostafa} M. N. Abdulla, U.A. Mughal, and M B. Steer,
``Network Charactarization for a Finite Array of Folded-Slot Antennas
for Spatial Power Combining Application,''
{Proc. 1999 IEEE Antennas and Propagation Symp.},
July 1999.
%
\bibitem{usman} U. A. Mughal, ``Hierarchical approach to global
modeling of active antenna arrays,'' M.S. Thesis, North
Carolina State University, 1999.
%
\bibitem{rational}
Rational Software, UML Resources,
http://www.rational.com/.
%
%
\bibitem{steer:global:1999}
M. B. Steer, J. F. Harvey, J. W. Mink, M. N. Abdulla, C. E. Christoffersen,
H. M. Gutierrez, P. L. Heron, C. W. Hicks, A. I. Khalil, U. A. Mughal,
S. Nakazawa, T. W. Nuteson, J. Patwardhan, S. G. Skaggs, M. A. Summers,
S. Wang, and A. B. Yakovlev, ``Global modeling of spatially distributed
microwave and millimeter-wave systems,'' {IEEE Trans. Microwave Theory
Techniques}, June 1999, pp. 830-839.
%
\bibitem{ims99} C. E. Christoffersen, S. Nakazawa, M. A. Summers, and
M. B. Steer, ``Transient analysis of a spatial power combining
amplifier'', {1999 IEEE MTT-S Int. Microwave Symp. Dig.}, June
1999, pp. 791-794.
%
\bibitem{mark} M. A. Summers, C. E. Christoffersen, A. I. Khalil,
S. Nakazawa, T. W. Nuteson, M. B. Steer and J. W. Mink, ``An
integrated electromagnetic and nonlinear circuit simulation
environment for spatial power combining systems,'' {1998 IEEE MTT-S
Int. Microwave Symp. Dig.}, June 1998, pp. 1473-1476.
%
\bibitem{ptplot} Ptplot. http://ptolemy.eecs.berkeley.edu/java/ptplot
%
% HB papers
%
\bibitem{rizzoli:96:1}
V. Rizzoli, F. Mastri, F. Sgallari,
G. Spaletta, \newblock {\em Harmonic-Balance Simulation of Strongly
Nonlinear very Large-Size Microwave Circuits by Inexact Newton
Methods}, \newblock IEEE MTT-S Digest, 1996.
%
\bibitem{rizzoli:95:1}
V. Rizzoli, A. Costanzo, and A. Lipparini,
\newblock {\em An Electrothermal Functional Model of the Microwave FET
Suitable for Nonlinear Simulation} \newblock International Journal of
Microwave and Millimeter-Wave Computer-Aided Engineering, Vol. 5,
No. 2, 104-121 (1995).
%
\bibitem{rizzoli:92:1}
V. Rizzoli, A. Lipparini, A. Costanzo,
F. Mastri, C. Ceccetti, A. Neri and D. Masotti, \newblock {\em
State-of-the-Art Harmonic-Balance Simulation of Forced Nonlinear
Microwave Circuits by the Piecewise Technique}, \newblock IEEE
Trans. on Microwave Theory and Techniques, Vol. 40, No. 1, Jan 1992.
%
\bibitem{gounary:1997}
 M. M. Gourary, S. G. Rusakov, S. L. Ulyanov,
M. M. Zharov, K. K. Gullapalli, and B. J. Mulvaney, \newblock {\em
Iterative Solution of Linear Systems in Harmonic Balance Analysis},
\newblock IEEE MTT-S Digest, 1997.
%
\bibitem{moret:1987}
I. Moret, \newblock {\em On the Convergence of
Inexact Quasi-Newton Methods}, \newblock International J. of Computer
Math., Vol. 28, pp. 117-137, 1987.
%
\bibitem{nakhla:vlach:76}
M. S. Nakhla, J. Vlach, \newblock {\em A
Piecewise Harmonic Balance Technique for Determination of Periodic
Response of Nonlinear Systems}, \newblock IEEE Trans. on Circuits and
Systems, Vol CAS-23, No. 2, Feb 1976.
%
\bibitem{materka:kacprzac:85}
A. Materka and T. Kacprzak, \newblock
{\em Computer Calculation of Large-Signal GaAs FET Amplifier
Characteristics}, \newblock IEEE Trans. on Microwave Theory and
Techniques, Vol MTT-33, No. 2, Feb 1985.
%
\bibitem{steer1:1996}
M. B. Steer, \newblock {\em Transient and
Steady-State Analysis of Nonlinear RF and Microwave Circuits},
\newblock ECE603 class notes, August 15, 1996.
%
\bibitem{steer2:1996}
J. F. Sevic, M. B. Steer, and A. M. Pavio,
\newblock {\em Nonlinear Analysis Methods for the Simulation of
Digital Wireless Communication Systems}, \newblock International
Journal of Microwave and Millimiter-Wave Computer-Aided Engineering,
Vol. 6, No. 3, 197-216, 1996.
%
\bibitem{kunisch:wolff:95}
J. Kunisch and I. Wolff, \newblock {\em
Steady-State Analysis of Nonlinear Forced and Autonomous Microwave
Circuits Using the Compression Approach}, \newblock International
Journal of Microwave and Millimeter-Wave Computer-Aided Engineering,
Vol. 5, No. 4, 241-255 (1995).
%
\bibitem{ngoya:suarez:quere:95}
E. Ngoya, A. S. R. Sommet and
R. Qu\'er\'e, \newblock {\em Steady State Analysis of Free or Forced
Oscillators by Harmonic Balance and Stability Investigation of
Periodic and Quasi-Periodic Regimes}, \newblock International Journal
of Microwave and Millimeter-Wave Computer-Aided Engineering, Vol. 5,
No. 3, 210-223 (1995).
%
\bibitem{mharm:94}
Compact Software, \newblock {\em Microwave
Harmonica Elements Library}, \newblock (1994).
%
\bibitem{powell:1988}
 M. J. D. Powell, \newblock {\em A hybrid method
for nonlinear equations}, \newblock Numerical Methods for Nonlinear
Algebraic Equations, P. Rabinowitz, Editor, Gordon and Breach, 1988.
%
\bibitem{kundert:vincentelli:90}
K. S. Kundert, J. K. White and
A. Sangiovanni-Vincentelli, \newblock {\em Steady-state methods for
simulating analog and microwave circuits}, \newblock Boston,
Dordrecht, Kluwer Academic Publishers, 1990.
%
\bibitem{steer:1991:1}
M. B. Steer, C. Chang and G. W. Rhyne,
\newblock {\em Computer-Aided Analysis of Nonlinear Microwave Circuits
Using Frequency-Domain Nonlinear Analysis Techniques: The State of the
Art}, \newblock International Journal of Microwave and Millimeter-Wave
Computer-Aided Engineering, Vol. 1, No. 2, 181-200, 1991.
%
\bibitem{gilmore:1991:1}
R. J. Gilmore and M. B. Steer, \newblock {\em
Nonlinear Circuit Analysis Using the Method of Harmonic Balance---A
Review of the Art. II. Advanced Concepts}, \newblock International
Journal of Microwave and Millimeter-Wave Computer-Aided Engineering,
Vol. 1, No. 2, 159-180, 1991.
%
\bibitem{chang:1990}
C. R. Chang, \newblock {\em Computer-Aided
Analysis of Nonlinear Microwave Analog Circuits Using Frequency-Domain
Spectral Balance}, \newblock Ph.D. Thesis, Department of Electrical
and Computer Engineering, North Carolina State University, Raleigh,
NC, 1990.
%
\bibitem{damore:94}
D. D'Amore, P. Maffezzoni and M. Pillan, \newblock
{\em A Newton-Powell Modification Algorithm for Harmonic Balance-Based
Circuit Analysis}, \newblock IEEE Transactions on Circuits and
Systems---I: Fundamental Theory and Applications, Vol. 41, No. 2,
February 1994.
%
\bibitem{Thodesen:Kundert:96}
Y. Thodesen, K. Kundert, \newblock {\em
Parametric harmonic balance}, IEEE MTT S. International Microwave
\newblock Symposium Digest, Vol 3, 1996, IEEE, Piscataway, NJ, USA,
pp. 1361-1364.
%
\bibitem{ushida:84}
A. Ushida and L. O. Chua.  \newblock {\em
Frequency-domain analysis of nonlinear circuits driven by multi-tone
signals}, \newblock IEEE Transactions on Circuits and Systems,
Vol. CAS-31, No. 9, September 1984, pp. 766-778.
%
\bibitem{ushida:87}
A. Ushida, L. O. Chua and T. Sugawara.  \newblock
{\em A substitution algorithm for solving nonlinear circuits with
multi-frequency components}, \newblock International Journal on
Circuit Theory and Application, Vol. 15, 1987, pp. 327-355.
%
\bibitem{gilmore:84}
R. J. Gilmore and F. J. Rosenbaum, \newblock {\em
Modelling of nonlinear distortion in GaAs MESFETs}, \newblock 1984
IEEE MTT-S International Microwave Symposium Digest, May 1984,
pp. 430-431.
%
\bibitem{bava:82}
G. P. Bava, S. Benedetto, E. Biglieri, F. Filicori,
V. A. Monaco, C. Naldi, U. Pisani and V. Pozzolo, \newblock {\em
Modelling and perfomance simulation Techniques of GaAs MESFETs for
microwave power amplifiers}, \newblock ESA-ESTEC Report, Noordwijk,
Holland, March 1982.
%
\bibitem{hiroaki:93}
H. Makino and H. Asai, \newblock {\em
Relaxation-based circuit simulation techniques in the frequency
domain}, \newblock IEICE Transactions on Fundamentals of Electronics,
Communications and Computer Sciences, Vol E76-A, No. 4 Apr 1993, p
626-630.
%
\bibitem{brambilla:93}
A. Brambilla, D. D'Amor, M. Pillan, \newblock
{\em Convergence improvements of the harmonic balance method},
\newblock Proceedings IEEE International Symposium on Circuits and
Systems, Vol. 4 1993, Publ. by IEEE, IEEE Service Center, Piscataway,
NJ, USA. p 2482-2485.
%
\bibitem{rizzoli:92:2}
V. Rizzoli, A. Costanzo, P. R. Ghigi,
F. Mastri, D. Masotti, C. Cecchetti, \newblock {\em Recent advances in
harmonic-balance techniques for nonlinear microwave circuit
simulation}, \newblock AEU Arch Elektron Uebertrag Electron Commun,
Vol. 46, No. 4 Jul 1992, p 286-297.
%
\bibitem{celik:96}
M. Celik, A. Atalar, M. A. Tan, \newblock {\em New
method for the steady-state analysis of periodically excited nonlinear
circuits}, \newblock IEEE Transactions on Circuits and Systems I:
Fundamental Theory and Applications, Vol. 43, No. 12 Dec 1996, p
964-972.
%
\bibitem{Brachtendorf:95:1}
H. G. Brachtendorf, G. Welsch, R. Laur,
\newblock {\em Fast simulation of the steady-state of circuits by the
harmonic balance technique}, \newblock Proceedings IEEE International
Symposium on Circuits and Systems, Vol. 2 1995, IEEE, Piscataway, NJ,
USA, p 1388-1391.
%
\bibitem{Brachtendorf:95:2}
H. G. Brachtendorf, G. Welsch, R. Laur,
\newblock {\em Simulation tool for the analysis and verification of
the steady state of circuit designs}, \newblock International Journal
of Circuit Theory and Applications, Vol. 23, No 4 Jul-Aug 1995, p
311-323.
%
\bibitem{barbancho:96}
I. Barbancho Perez, I. Molina Fernandez,
\newblock {\em Predictor strategies for continuation methods applied
to nonlinear circuit analysis}, \newblock Industrial Applications in
Power Systems, Computer Science and Telecommunications Proceedings of
the Mediterranean Electrotechnical Conference MELECON, Vol. 3 1996,
IEEE, Piscataway, NJ, USA, p 1419-1422.
%

\bibitem{basel:paper} M. S. Basel, M. B. Steer and P. D. Franzon,
``Simulation of high speed interconnects using a convolution-based
hierarchical packaging simulator,'' \emph{IEEE Trans. on
Components, Packaging, and Manufacturing Techn.}, Vol. 18, February
1995, pp. 74-82.

\bibitem{Brazil:New} T. J. Brazil, ``A new method for the transient
simulation of causal linear systems described in the frequency
domain,'' \emph{1992 IEEE MTT-S Int. Microwave Symp. Digest}, June
1992, pp. 1485-1488.

\bibitem{gamma} P. Perry and T. J. Brazil, ``Hilbert-transform-derived
relative group delay,'' \emph{IEEE Trans. on Microwave Theory
and Techn.}, Vol 45, Aug. 1997, pt. 1, pp. 1214-1225.

\bibitem{delta} T. J. Brazil, ``Causal convolution---a
new method for the transient analysis of linear systems at microwave
frequencies,'' \emph{IEEE Trans. on Microwave Theory and
Techn.}, Vol. 43, Feb. 1995, pp. 315-23.

\bibitem{sarkar} A. R. Djordjevic and T. K. Sarkar, ``Analysis of time
response of lossy multiconductor transmission line networks,''
\emph{IEEE Trans. on Microwave Theory and Techn.}, Vol. MTT-35,
Oct. 1987, pp. 898-908.

\bibitem{alpha} D. Winkelstein, R. Pomerleau and M. B. Steer, ``Transient
simulation of complex, lossy, multi-port transmission line networks
with nonlinear digital device termination using a circuit simulator,''
\emph{Conf. Proc. IEEE SOUTHEASTCON}, Vol. 3, pp. 1239-1244.

\bibitem{theta} J. E. Schutt-Aine and R. Mittra, ``Nonlinear transient
analysis of coupled transmission lines,'' \emph{IEEE Trans. on Circuits
and Systems}, Vol. 36, Jul. 1989, pp. 959-967.

\bibitem{chan:AWE} P. K. Chan, Comments on ``Asymptotic waveform
evaluation for timing analysis,'' \emph{IEEE Trans. on Computer Aided
Design}, Vol. 10, Aug. 1991, pp. 1078-79.

\bibitem{multi:pade} M. Celik, O. Ocali, M. A. Tan, and A. Atalar,
``Pole-zero computation in microwave circuits using multipoint Pad\'e
approximation,'' \emph{IEEE Trans. on Circuits and Systems},
Jan. 1995, pp. 6-13.

\bibitem{Nakhla:Eli} E. Chiprout and M. Nakhla, ``Fast nonlinear
waveform estimation for large distributed networks,'' \emph{1992 IEEE
MTT-S Int. Microwave Symp. Digest}, Vol.3, Jun. 1992,
pp. 1341-1344.

\bibitem{Trithy} R. J. Trihy and Ronald A. Rohrer, ``AWE macromodels
for nonlinear circuits,'' \emph{Proceedings of the 36th Midwest
Symposium on Circuits and Systems}, Vol. 1, Aug. 1993, pp. 633-636.

\bibitem{Nakhla:Grif} R. Griffith and M. S. Nakhla, ``Mixed frequency/time
domain analysis of nonlinear circuits,'' \emph{IEEE Trans. on Computer
Aided Design}, Vol.11, Aug. 1992, pp. 1032-43.

\bibitem{mete} M. Ozkar, \emph{Transient analysis of spatially distributed
microwave circuits using convolution and state variables},
M. S. Thesis, Department of Electrical and Computer Engineering,
North Carolina State University.

\bibitem{Blazeck:Mittra} C. Gordon, T. Blazeck and R. Mittra, ``Time
domain simulation of multiconductor transmission lines with
frequency-dependent losses,'' \emph{IEEE Trans. on Computer
Aided Design of Integrated Circuits and Systems}, Vol. 11 Nov. 1992
pp. 1372-87.

\bibitem{aplac} P. Stenius, P. Heikkil\"a and M. Valtonen,
``Transient analysis of circuits including frequency-dependent
components using transgyrator and convolution,''
\emph{Proc. of the 11th European Conference on Circuit Theory and
Design}, Part II, 1993, pp. 1299-1304.
\end{thebibliography}

