\documentclass{article}
\usepackage{epsf}\usepackage{here}
% SUMMARY OF USEFUL MACROS
%
% 1. \marginpar[LeftText]{RightText} Standard Latex \marginpar
%
% 2. \mymarginpar{MarginText}{BodyText} Places MarginText in margin and
%     BodyText in main text opposite margin. Places lineas above and below
%     text.  Used in element and model catalogs and elsewhere.
%
% 3. \marginlabel{text} Places large text in margin and underlines. Used
%    to put element name in margin at top of page for continued
%    descriptions.  Does not automaticly put it at the top of a page
%    and so a \clearpage is required.
%
% 4. \marginid{text} Used to place a short identifier in the margin. Just one
%    word and justified to the outside edge.
%
% 5. \offset a standard indent.
%
% 6. \offsetparbox{} Places text in an offset parbox.  Use
%    \hspace*{\fill} \offsetparbox{text}        to insert an offset text.
%
% 7. \boxed{} similar to above but starts a new line and right justifies
%    offsetparbox.
%
% 8. \form
%
% 9. \example
%
%10. \begin{widelist}  .... \end{widelist} 
%    A widelist that has 1 inch wide labels that has additional left hand
%    margin of 0.6 inch.
%
%11. \spicethreeonly{} Text is inserted only if output is for spice3.
%
%12. \pspiceonly{} 
%    Text is inserted only if output is for pspice (superspice for the most
%    part is upwards compatible to pspice).
%
%13. \sspiceonly{} 
%    Text is inserted only if output is for superspice but not for pspice.

%14. \sym{}
%    Right justifies a symbol in a keyword table.
%
%15. Use the following wherever as then we can have a standard way to
%    report them.  Note that "\dc\ " is required to get a space after dc.
%    \dc
%    \ac
%    \SPICE
%
%16. \kwnote{}  This is a convenient way to include notes in a keyword table.
%    It right justifies a note in the description column and starts it on a
%    newline.
\newcommand{\kwnote}[1]{\newline\hspace*{\fill} #1}

%
%17. \kwversion{} This is a convenient way to indicate versions in a keyword
%    table. It does a \kwnote .  It should not printed when outputing for
%     specific versions.
% typical usage is \kwversion{\sspice ; \hspice} to produce the output
%            VERSIONS: SUPERSPICE; HSPICE
% typical usage is \kwversionNote{\sspice ; \hspice} to produce the output
%                 |                    VERSIONS: SUPERSPICE; HSPICE |
%for unifomity the recommended version order is:
%    \hspice \pspice \sspice \spicetwo \spicethree
\newcommand{\version}[1]{({\sc version: #1})}
\newcommand{\kwversion}[1]{\newline({\sc version: #1})}
\newcommand{\kwversionNote}[1]{\kwnote{({\sc version: #1})}}
\newcommand{\versions}[1]{({\sc versions: #1})}
\newcommand{\kwversions}[1]{\newline({\sc versions: #1})}
\newcommand{\kwversionsNote}[1]{\kwnote{({\sc versions: #1})}}

%


\newcommand{\mymargin}{1in}   % Use to set width of margin notes
\newcommand{\mymarginparsep}{0.1in}
\newcommand{\mymarginparsepplustext}{5.6in}
\newcommand{\mymarginplus}{1.1in}
\newcommand{\mymarginplustext}{6.6in}
\newcommand{\underlinehead}{\\[-0.1in] \rule{\mymarginplustext}{0.01in}}
\newcommand{\overlinefoot}{\rule{\mymarginplustext}{0.01in}\\}
% WIDEPARBOX uses margin space as well.
\newcommand{\wideparbox}[1]{\parbox[t]{\mymarginplustext}{#1}}
%
% HEADER AND FOOT
% Standard header and foot, but includes copyright notice.a
% To omit copyrite notice do not include copyrite.sty
%
\newcommand{\mycopyrite}{}
\def\ps@headings{\let\@mkboth\markboth
\def\@oddfoot{\wideparbox{\overlinefoot\rm\today \hfill \mycopyrite}}
\def\@evenfoot{\hspace*{-\mymarginplus}\wideparbox{\overlinefoot
\mycopyrite\hfill \today}}
\def\@evenhead{\hspace*{-\mymarginplus}\wideparbox{\rm
\thepage\hfill \sl \leftmark \underlinehead }}
\def\@oddhead{\wideparbox{\hbox{}\sl \rightmark \hfill
\rm\thepage \underlinehead}}\def\chaptermark##1{\markboth {\uppercase{\ifnum
\c@secnumdepth
>\m@ne
 \@chapapp\ \thechapter. \ \fi ##1}}{}}\def\sectionmark##1{\markright
 {\uppercase{\ifnum \c@secnumdepth >\z@
  \thesection. \ \fi ##1}}}}

%\oddsidemargin 0.25in \topmargin 0.0in \textwidth 6.5in \textheight 9in
%\evensidemargin 0.25in \headheight 0.18in \footskip 0.16in

%\input{epsf}
\usepackage{epsf}
\oddsidemargin 0.25in \evensidemargin 0.25in
\topmargin 0.0in
\textwidth 6.5in \textheight 9in
\headheight 0.18in \footskip 0.16in
\leftmargin -0.5in \rightmargin -0.5in

\newcommand{\fig}[1]{figures/#1}
\newcommand{\pfig}[1]{\epsfbox{\fig{#1}}}
\newcommand{\newfig}[1]{\epsffile{\fig{#1}}}

\newcommand{\fdaelement}[1]{elements/#1}
\newcommand{\spiceelement}[1]{equivalent_spice_elements/#1}

\newcommand{\FREEDA}{{\Huge{\textsl{\textsf{f}}}${\mathsf{REEDA}}^{{\tiny{\textsf{TM}}}}$}}
\newcommand{\FDA}{{\textsl{\textsf{f}}}${\mathsf{REEDA}}^{{\tiny{\textsf{TM}}}}$}
\newcommand{\notforsspice}[1]{#1}
\newcommand{\spicetwoonly}[1]{#1}
\newcommand{\spicethreeonly}[1]{#1}
\newcommand{\pspiceonly}[1]{#1}
\newcommand{\pspiceninetytwoonly}[1]{#1}
\newcommand{\sspiceonly}[1]{}
\newcommand{\fornutmeg}[1]{}
\newcommand{\sspicetwoonly}[1]{}

%%%%%%%%%%%%%%%%%%%%%%%%%%%%%%%%%%%%%%%%%%%%%%%%%%%%%%%%%%%%%%%%%%%%%%%%%%%%%%%%
\marginparwidth=\mymargin
\marginparsep=\mymarginparsep
\newcommand{\textplusmarginparwidth}{\textwidth+\marginparsep+\mymargin}
\newcommand{\X}{\\ \hline}    % line terminataion in keyword environment
\newcommand{\B}{{ \rm [}}     % begin optional parameter in \form{}
\newcommand{\E}{{\ \rm\hspace{-0.04in}] }}   % end optional parameter in \form{}
\newcommand{\expr}{{\sc Expressions supported}}
\newcommand{\reqd}{{\scriptsize REQUIRED}}
\newcommand{\omitted}{{\scriptsize OMITTED}}
\newcommand{\inferred}{{\scriptsize INFERRED}}
\newcommand{\para}{\newline{\scriptsize (PARASITIC)}}
\newcommand{\opt}{{\tiny  OPTIONAL}}

\newcommand{\Spice}{{\sc Spice}}
\newcommand{\spice}{{\sc Spice}}
\newcommand{\justspice}{{\sc Spice}}
\newcommand{\nutmeg}{{\sc NUTMEG}}
\newcommand{\probe}{{\sc Probe}}

\newcommand{\accusim}{{\sc AccuSim}}
\newcommand{\contec}{{\sc ContecSpice}}
\newcommand{\cdsspice}{{\sc CDS Spice}}
\newcommand{\hpimpulse}{{\sc HP Impulse}}
\newcommand{\hspice}{{\sc HSpice}}
\newcommand{\igspice}{{\sc IG\_SPICE}}
\newcommand{\isspice}{{\sc IsSpice}}
\newcommand{\mspice}{{\sc Microwave Spice}}
\newcommand{\pspice}{{\sc PSpice}}
\newcommand{\justpspice}{{\sc PSpice}}
\newcommand{\spectre}{{\sc Spectre}}
\newcommand{\spicetwo}{{\sc Spice2g6}}
\newcommand{\spicethree}{{\sc Spice3}}
\newcommand{\spiceplus}{{\sc SpicePlus}}
\newcommand{\sspice}{{\sc SuperSpice}}

\newcommand{\modelversion}[1]{& #1}
%%%%%%%%%%%%%%%%%%%%%%%%%%%%%%%%%%%%%%%%%%%%%%%%%%%%%%%%%%%%%%%%%%%%%%%%%%%%%%%%
% set up a counter for all occasions
%
\newcounter{count}

% set up new commands
%
\newcommand{\vshift}{\vspace{0.2in}}
% OFFSET
\newcommand{\offset}{\hspace*{0.45in} }
% OFFSETPARBOXWIDTH argument should be \textwith - offset
\newcommand{\offsetparbox}[1]{\parbox[t]{5in}{#1}}

% note: no labeling
\newcommand{\elementx}[3]{\clearpage\rm\markright{#3:#2}
\addcontentsline{toc}{section}{#1, #2: #3}
\mymarginparx{#1}{#2}{#3}\index{#1:#2}\index{#3:#2}}

% macros for sub elements
\newcommand{\subelement}[2]{\clearpage 
   \noindent\rule{\textwidth /2}{.5mm} \\[0.1in]
   {\large \bf #1} \hspace{0.2in} {\bf #2} \\
   \noindent\rule{\textwidth /2}{.5mm} \index{#1} \index{#2}}
% macros for models
\newcommand{\model}[2]{{
   \noindent\vspace{0.2in}\parbox{\textwidth}{
   \noindent\rule{\textwidth}{.5mm} \\[0.1in]
   {\large \bf #1 Model} \label{#1model} \hfill {\large #2} \\
   \noindent\rule{\textwidth}{.5mm} \index{#1} \index{#2}}}}
\newcommand{\modelx}[3]{{
   \noindent\vspace{0.2in}\parbox{\textwidth}{
   \noindent\rule{\textwidth}{.5mm} \\[0.1in]
   {\large \bf #1 Model} \hfill {\large #2} \hfill {\large #3} \\
   \noindent\rule{\textwidth}{.5mm} \index{#1} \index{#2}}}}



% BOXED
\newcommand{\boxed}[1]{\noindent
\newline \vshift \hspace*{\fill} {\tt \offsetparbox{\tt #1}}
 \vshift}


%
% KEYWORD
%
\newcommand{\keywordtable}[1]{
        \sloppy
        \hyphenation{ca-pac-i-t-an-ce} 
        \begin{center}
    \sf
        \begin{tabular}[t]
        {|p{0.58in}|p{3.07in}|p{0.55in}|p{0.60in}|}
        \hline
        \multicolumn{1}{|c}{\bf Name} &
        \multicolumn{1}{|c}{\parbox{2.77in}{\bf Description}}  &
        \multicolumn{1}{|c}{\bf Units} &
        \multicolumn{1}{|c|}{\bf Default} \X
        #1
        \end{tabular}
        \end{center}
    }

\newcommand{\keywordtwotable}[2]{
        \sloppy
        \hyphenation{ca-pac-i-t-an-ce} 
        \begin{center}
    \sf
        \begin{tabular}[t]
        {|p{0.58in}|p{2.38in}|p{0.55in}|p{0.60in}|p{0.53in}|}
        \hline
        \multicolumn{1}{|c}{\bf Name} &
        \multicolumn{1}{|c}{\parbox{2.20in}{\bf Description}}  &
        \multicolumn{1}{|c}{\bf Units} &
        \multicolumn{1}{|c}{\bf Default} &
        \multicolumn{1}{|c|}{\bf #1} \X
        #2
        \end{tabular}
        \end{center}
    }

\newcommand{\kw}[2]{
     \samepage{
     \noindent {\sl #1} \vspace{-0.5in} \\
     \keywordtable{#2} }}

\newcommand{\kwtwo}[3]{
     \samepage{
     \noindent {\sl #1} \vspace{-0.4in} \\
     \keywordtwotable{#2}{#3} }}

\newcommand{\keyword}[1]{\kw{Keywords:}{#1}}
\newcommand{\keywordtwo}[2]{\kwtwo{Keywords:}{#1}{#2}}
\newcommand{\modelkeyword}[1]{\kw{Model Keywords}{#1}}
\newcommand{\modelkeywordtwo}[2]{\kwtwo{Model Keywords}{#1}{#2}}

\newcommand{\myline}{\\[-0.1in]
\noindent \rule{\textwidth}{0.01in} \newline}

\newcommand{\myThickLine}{\\[-0.1in]
\noindent \rule{\textwidth}{0.02in} \newline}


% SPICE 2G6 FORM
%\newcommand{\spicetwoform}[1]{
%\spicetwoonly{\samepage{\noindent{\sl\spicetwo\form{#1}}}}}

% PSPICE88 FORM
%\newcommand{\pspiceform}[1]{
%\pspiceonly{\samepage{\noindent{\sl\pspice}\form{#1}}}}

% PSPICE92 FORM
%\newcommand{\pspiceninetytwoform}[1]{
%\pspiceninetytwoonly{\samepage{\noindent{\sl\pspice92\form{#1}}}}}


% SPICE3E2 FORM
%\newcommand{\spicethreeform}[1]{
%\spicethreeonly{\samepage{\noindent{\sl\spicethree\form{#1}}}}}

% FORM
\newcommand{\form}[1]{\samepage{\noindent
 {\sl Form} \myline
% \hspace*{\fill} % For some reason \fill = 0 when \pspiceform{} is used?
\offset
\it  \offsetparbox{#1}}
\\[0.1in]}

% ELEMENT FORM
\newcommand{\elementform}[1]{\samepage{\noindent
 {\sl Element Form} \myline
% \hspace*{\fill} % For some reason \fill = 0 when \pspiceform{} is used?
\offset
\it  \offsetparbox{#1}}
\\[0.1in]}

% MODEL FORM
\newcommand{\modelform}[1]{\samepage{\noindent
 {\sl Model Form} \myline
% \hspace*{\fill} % For some reason \fill = 0 when \pspiceform{} is used?
\offset
\it  \offsetparbox{#1}}
\\[0.1in]}

% LIMITS
\newcommand{\mylimits}[1]{\samepage{\noindent
 {\sl Limits} \myline
 \hspace*{\fill} \it  \offsetparbox{#1}}
 \vshift}

% EXAMPLE
\newcommand{\example}[1]{\samepage{\noindent
{\sl Example} \myline
\offset \tt  \offsetparbox{#1}}
 \vshift}

% PSPICE88 EXAMPLE
\newcommand{\pspiceexample}[1]{\samepage{\noindent
{\sl \pspice\ Example} \myline
\offset \tt  \offsetparbox{#1}}
 \vshift}

% MODEL TYPES
\newcommand{\modeltype}[1]{\samepage{\noindent
{\sl Model Type} \myline
 \hspace*{\fill} \tt \offsetparbox{#1}}
 \\[0.1in]}

% MODEL TYPES
\newcommand{\modeltypes}[1]{\samepage{\noindent
{\sl Model Types:} \myline
 \hspace*{\fill} \tt \offsetparbox{\tt #1}}
 \vshift}

% OFFSET ENUMERATE
\newcommand{\offsetenumerate}[1]{
     \offset \hspace*{-0.1in} {\begin{enumerate} #1 \end{enumerate}}}

% NOTE
\newcommand{\note}[1]{
\vshift\samepage{\noindent {\sl Note}\myline\vspace{-0.24in}}
 \offsetenumerate{#1} }

% SPECIAL NOTE
\newcommand{\specialnote}[2]{
\vshift\samepage{\noindent {\sl #1}\myline\vspace{-0.24in}}\\#2}

\newcommand{\dc}{\mbox{\tt DC}}
\newcommand{\ac}{\mbox{\tt AC}}
\newcommand{\SPICE}{\mbox{\tt SPICE}}
\newcommand{\m}[1]{{\bf #1}}                           % matrix command  \m{}

% ////// Changing nodes to terminals///////
% print terminals in \tt and enclose in a circle use outside
\newcommand{\terminal}[1]{\: \mbox{\tt #1} \!\!\!\! \bigcirc }
%
% set up environment for example
%
\newcounter{excount}
\newcounter{dummy}
\newenvironment{eg}{\vspace{0.1in}\noindent\rule{\textwidth}{.5mm}
   \begin{list}
   {{\addtocounter{excount}{1}
   \em Example\/ \arabic{chapter}.\arabic{excount}\/}:}
   {\usecounter{dummy}
   \setlength{\rightmargin}{\leftmargin}}
   }{\end{list} \rule{\textwidth}{.5mm}\vspace{0.1in}}
%
% set up environment for block
% currently this draws a horizontal line at the start of block and another
% at the end of block.
%
\newenvironment{block}{\vspace{0.1in}\noindent\rule{\textwidth}{.5mm}
   }{\rule{\textwidth}{.5mm}\vspace{0.1in}}
%

%
% Macros for element summaries
%
% macros for element summary
%\newcommand{\summaryelement}[2]{
%   \vspace{0.4in}
%   \mymarginpar{#1}{#2} 
%   \addcontentsline{toc}{section}{#1, #2}
%   \vspace{-0.6in} \\
%   \noindent Full description on page \pageref{#1element}. \vshift\\
%   }

% Summary MODEL TYPE
%\newcommand{\summarymodeltype}[1]{\samepage{\noindent
%{\sl Model Type} \myline
% \hspace*{\fill} \tt \offsetparbox{\tt #1
% \hfill Summary on \pageref{#1summary} \index{#1}}}
% }

% Summary MODEL TYPES
%\newcommand{\summarymodeltypes}[1]{\samepage{\noindent
%{\sl Model Types} \myline
% \hspace*{\fill} \tt \offsetparbox{\tt #1
% \hfill Summary on \pageref{#1summary} \index{#1}}}
% }

%
% macros for model summary
%\newcommand{\summarymodel}[2]{\clearpage
%   \addcontentsline{toc}{section}{#1, #2}
%   \mymarginpar{#1}{#2} \label{#1summary}
%   \index{#1} \index{#2}
%   \noindent Full description on page \pageref{#1model} \\[0.1in]
%   }



%
% set up wide descriptive list
%
\newenvironment{widelist}
    {\begin{list}{}{\setlength{\rightmargin}{0in} \setlength{\itemsep}{0.1in}
    \setlength{\labelwidth}{0.95in} \setlength{\labelsep}{0.1in}
\setlength{\listparindent}{0in} \setlength{\parsep}{0in}
    \setlength{\leftmargin}{1.0in}}
    }{\end{list}}

\newcommand{\STAR}{\hspace*{\fill} * \hspace*{\fill}}

\newcommand{\sym}[1]{\hspace*{\fill} ($#1$)}

%
% The thebibliography environment is redefined so the the word References is
% not output
%\def\thebibliography#1{\list
% {[\arabic{enumi}]}{\settowidth\labelwidth{[#1]}\leftmargin\labelwidth
% \advance\leftmargin\labelsep
% \usecounter{enumi}}
% \def\newblock{\hskip .11em plus .33em minus -.07em}
% \sloppy\clubpenalty4000\widowpenalty4000
% \sfcode`\.=1000\relax}
%\let\endthebibliography=\endlist
% END thebibliography environment redefinition



%\newcommand{\eskipv}[1]{\clearpage\marginlabel{#1}}
%\newcommand{\eskip}[1]{\vspace*{\fill}\clearpage\marginlabel{#1}}
%\newcommand{\eskipnv}[1]{\newpage\marginlabel{#1}}
%\newcommand{\eskipn}[1]{\vspace*{\fill}\newpage\marginlabel{#1}}

%marginlabel is very wide
%\newcommand{\eskipfullv}[1]{\clearpage\marginlabelfull{#1}}
%\newcommand{\eskipfull}[1]{\vspace*{\fill}\clearpage\marginlabelfull{#1}}
%\newcommand{\eskipfullnv}[1]{\newpage\marginlabelfull{#1}}
%\newcommand{\eskipfulln}[1]{\vspace*{\fill}\newpage\marginlabelfull{#1}}

%\newcommand{\mycontentsline}[5]{\parbox{#1}{#2}#3
%\hspace{0.1in}\dotfill\parbox{0.3in}{\hfill\pageref{#4#5}}\\[0.1in]}
%\newcommand{\mysline}[2]{\mycontentsline{1.2in}{#1}{#2}{#1}{statement}}
%\newcommand{\mymline}[2]{\mycontentsline{0.7in}{#1}{#2}{#1}{model}}
%\newcommand{\myeline}[2]{\mycontentsline{0.7in}{#1}{#2}{#1}{element}}

%\newcommand{\myincontentsline}[5]{\vspace{0.05in}\noindent\parbox{#1}{#2}#3
%\hspace{0.1in}
%\dotfill\parbox{0.7in}{\hfill Page \pageref{#4#5}}\\[0.05in]\noindent}
%\newcommand{\myinsline}[2]{\myincontentsline{1.2in}{#1}{#2}{#1}{statement}}
%\newcommand{\myinmline}[2]{\myincontentsline{0.5in}{#1}{#2}{#1}{model}}
%\newcommand{\myineline}[2]{\myincontentsline{0.5in}{#1}{#2}{#1}{element}}

%
%
% The following a symbols that could used alot.
\newcommand{\ms}[1]{\mbox{\scriptsize #1}}
\newcommand{\AF}{A_F}
\newcommand{\CBD}{C'_{BD}}
\newcommand{\CBS}{C'_{BS}}
\newcommand{\CGBO}{C_{GBO}}
\newcommand{\CGDO}{C_{GDO}}
\newcommand{\CGSO}{C_{GSO}}
\newcommand{\CJ}{C_J}
\newcommand{\CJSW}{C_{J,\ms{SW}}}
\newcommand{\DELTA}{\delta}
\newcommand{\ETA}{\eta}
\newcommand{\FC}{F_C}
\newcommand{\GAMMA}{\gamma}
\newcommand{\IS}{I_S}
\newcommand{\JS}{J_S}
\newcommand{\KAPPA}{\kappa}
\newcommand{\KF}{K_F}
\newcommand{\KP}{K_P}
\newcommand{\LAMBDA}{\lambda}
\newcommand{\LD}{X_{JL}}
\newcommand{\LEVEL}{M_J}
\newcommand{\MJ}{M_J}
\newcommand{\MJSW}{M_{J,\ms{SW}}}
\newcommand{\NSUB}{N_B}
\newcommand{\NSS}{N_{\ms{SS}}}
\newcommand{\NFS}{N_{\ms{FS}}}
\newcommand{\NEFF}{N_{\ms{EFF}}}
\newcommand{\PB}{\phi_J}
\newcommand{\PHI}{2\phi_B}
\newcommand{\RD}{R_D}
\newcommand{\RS}{R_S}
\newcommand{\RSH}{R_{\ms{SH}}}
\newcommand{\THETA}{\theta}
\newcommand{\TOX}{T_{OX}}
\newcommand{\TPG}{T_{\ms{PG}}}
\newcommand{\UCRIT}{U_C}
\newcommand{\UEXP}{U_{\ms{EXP}}}
\newcommand{\UO}{\mu_0}
\newcommand{\UTRA}{U_{\ms{TRA}}}
\newcommand{\VMAX}{V_{\ms{MAX}}}
\newcommand{\VTZERO}{V_{T0}}
\newcommand{\VTO}{V_{T0}}
\newcommand{\XJ}{X_J}
\newcommand{\Length}{L} %  \L already used
\newcommand{\N}{N}
\newcommand{\PBSW}{\phi_{J,{\ms{SW}}}}
\newcommand{\RB}{R_B}
\newcommand{\RG}{R_B}
\newcommand{\RDS}{R_{DS}}
\newcommand{\TT}{\tau_T}
\newcommand{\W}{W}
\newcommand{\WD}{W_D}
\newcommand{\XQC}{X_{QC}}
\newcommand{\JSSW}{J_{S,{\ms{SW}}}}
\newcommand{\DL}{\Delta_L}
\newcommand{\DW}{\Delta_W}
\newcommand{\DELL}{\Delta_{L,\ms{SW}}}
\newcommand{\KONE}{K_1}
\newcommand{\KTWO}{K_2}
\newcommand{\MUS}{\mu_S}
\newcommand{\MUZ}{\mu_Z}
\newcommand{\NZERO}{N_0}
\newcommand{\NB}{N_B}
\newcommand{\ND}{N_D}
\newcommand{\TEMP}{T}
\newcommand{\VDD}{V_{DD}}
\newcommand{\WDF}{W_{\ms{DF}}}
\newcommand{\VFB}{V_{\ms{FB}}}
\newcommand{\UZERO}{U_0}
\newcommand{\UONE}{U_1}
\newcommand{\XTWOE}{X_{2E}}
\newcommand{\XTWOMS}{X_{2\ms{MS}}}
\newcommand{\XTWOMZ}{X_{2\ms{MZ}}}
\newcommand{\XTWOUZERO}{X_{2\ms{U0}}}
\newcommand{\XTWOUONE}{X_{2\ms{U1}}}
\newcommand{\XTHREEE}{X_{3E}}
\newcommand{\XTHREEMS}{X_{3\ms{MS}}}
\newcommand{\XTHREEMZ}{X_{3\ms{MZ}}}
\newcommand{\XTHREEUZERO}{X_{3\ms{U0}}}
\newcommand{\XTHREEUONE}{X_{3\ms{U1}}}
\newcommand{\XPART}{X_{\ms{PART}}}
\newcommand{\PS}{P_S}
\newcommand{\PD}{P_D}
\newcommand{\NRS}{N_{RS}}
\newcommand{\NRG}{N_{RG}}
\newcommand{\NRB}{N_{RB}}
\newcommand{\NRD}{N_{RD}}


\newcommand{\Net}{{${\cal N}$}}                          % network \N
\newcommand{\Nprime}{{${\cal N}^{\prime}$}}            % another network \Nprime
\newcommand{\Nold}{{${\cal N}^{\mbox{old}}$}}          % old network  \Nold
\newcommand{\Nnew}{{${\cal N}^{\mbox{new}}$}}          % new network  \Nnew

\newcommand{\GMIN}{{G_{\ms{MIN}}}}

\newcommand{\optionitem}[2]{
\item[{\tt #1}{#2}]\label{.OPTION#1}\index{.OPTIONS, #1}\index{#1}}

\newcommand{\error}[1]{\vspace{0.1in}\noindent{\tt #1}\\}


%For numbering an equation which is incoorporated
%with text.
\newcommand{\inlineeq}{\hspace*{\fill}\refstepcounter{equation}{\rm
(\theequation)}\\}

\begin{document}
\noindent{\LARGE \textbf{Vertical Cavity Self-Emitting Laser Diode\newline(Mena et al. Model)}
\hspace*{\fill}\textbf{VCSEL}}\\
\hrulefill\linethickness{0.5mm}\line(1,0){425} \normalsize
\newline

\begin{figure}[H]
\centerline{\epsfxsize=4.5in\epsfbox{figures/vcsel.eps}} \caption{Vertical Cavity Self-Emitting Laser Diode (Mena et al. Model). The terminals labeled open should be connected to opens or very large resistors in a circuit.  The reference terminals should be specified as local reference terminals, see the command .REF, e.g. use .REF $n_{\mathrm{nref}}$ in the netlist.}
\end{figure}

\noindent\linethickness{0.5mm}\line(1,0){425}
\newline
\textit{fREEDA Form:}
$\tt VCSEL $:$\langle \tt{instance\ name}\rangle$ $n_1\ n_2\
n_n\ n_{\textrm{nref}}\ n_t\ n_{\textrm{tref}}\ n_o\ n_{\textrm{oref}}\ n_p\ n_{\textrm{pref}}\ n_l\ n_{\textrm{lref}}$ $\langle \tt{parameter\ list}\rangle$
\newline
\begin{tabular}{r l}
$n_1$ & is the electrical anode terminal \\
&  \\
$n_2$ & is the electrical cathode terminal \\
&  \\
$n_n$ & is the carrier density terminal \\
$n_{\textrm{nref}}$ & is the carrier reference terminal \\
& \\
$n_t$ & is the thermal terminal \\
$n_{\textrm{tref}}$ & is the thermal reference terminal \\
& \\
$n_o$ & is the photon density terminal \\
$n_{\textrm{oref}}$ & is the photon density reference terminal \\
& \\
$n_p$ & is the optical power terminal \\
$n_{\textrm{pref}}$ & is the optical reference terminal \\
& \\
$n_l$ & is the wavelength terminal \\
$n_{\textrm{lref}}$ & is the wavelength reference terminal \\
& \\
parameter list & see table for parameter list
\end{tabular}\\[0.1in]

Note that  
$n_n$, $n_o$, $n_p$, $n_l$ must be connected to open circuits in the circuit as zero"current" is the solution of an internal equation.  The $n_t$ must also be connected to open as the thermal resistance, $r_{th}$, is included in the model.  If $r_{th}$ is open (or a very large value) then this terminal can be connected to a thermal circuit.

The quantities available by printing the voltage at the terminals are as follows (with $n_\textrm{nref}$, $n_\textrm{oref}$, $n_\textrm{tref}$, $n_\textrm{pref}$, and $n_\textrm{lref}$ identified as local reference terminals using  the .REF command).\\
\centerline{\begin{tabular}{r l}
``voltage'' at terminal $n_1$ & is electrical voltage \\
&  \\
``voltage'' at terminal $n_2$ & is electrical voltage \\
&  \\
``voltage'' at terminal $n_n$ & is the carrier density \\
& \\
``voltage'' at terminal $n_t$ & is the temperature in centigrade \\
& \\
``voltage'' at terminal $n_o$ & is the photon density \\
& \\
``voltage'' at terminal $n_p$ & is the optical power \\
& \\
``voltage'' at terminal $n_l$ & is the optical wavelength
\end{tabular}}\\[0.1in]


\noindent\textbf{Parameter Table}\\[0.1in]

\begin{table}[ht]
\caption{Parameters for the VCSEL Laser Diode Model}
\centering{\begin{tabular}{|l| l| l| l|}
\hline
 \textbf{Parameters} & \textbf{Description} & \textbf{Values} & \textbf{Units} \\
 \hline
 {\tt $\eta_i$} &Injection Efficiency & 1 & - \\
 {\tt $\beta$} &Spontaneous Emission Coupling Coefficient & 1e-6 & - \\
 {\tt $\tau_n$} &Carrier Recombination Lifetime & 5e-9& s \\
 {\tt $k$} &Output coupling efficiency & 2.6e-8& W \\
 {\tt $g_0$} &Gain Coefficient & 1.6e4& s$^{-1}$ \\
 {\tt $n_0$} &Carrier Transparency number & 1.94e7& - \\
 {\tt $\tau_p$} &Photon Lifetime & 2.28e-12& s \\
 {\tt $a_0$} &1st temperature coefficient of the offset current & 1.246e-3& A \\
 {\tt $a_0$} &2nd temperature coefficient of the offset current & -2.545e-5& A/K\\
 {\tt $a_1$} &3rd temperature coefficient of the offset current& 2.908e-7& A/K$^2$\\
 {\tt $a_0$} &4th temperature coefficient of the offset current & -2.531e-10& A/K$^3$\\
 {\tt $a_1$} &5th temperature coefficient of the offset current& 1.022e-12& A/K$^4$\\
 {\tt $\rho$} &Refractive index change& 2.4e-9& -\\
 {\tt $n$} &Refractive Index & 3.5& -\\
 {\tt $\lambda_0$} &Wavelength& 863e-9& m\\
 {\tt $R_{th}$} &Thermal Impedance& 2.6e3& $^o$C/W\\
 {\tt $\tau_{th}$} &Thermal time constant & 1e-6& s\\
 {\tt $T_0$} &Ambient Temperature& 20& $^o$C\\
\hline
\end{tabular}}
\label{table:p:vcsel}
\end{table}

\newpage

\noindent\textbf{Description}\\[0.1in]

\begin{figure}[h]
\centerline{\epsfxsize=5.306in \epsfbox{figures/vcsel_structure.eps}}
\caption{VCSEL Structure laser.
After~\cite{neifeld:vcselst:2003}.} \label{fig:svcsel:vcsels}
\end{figure}

\noindent Vertical-Cavity Surface-Emitting Lasers (VCSEL's)  posses a single-longitudinal-mode of operation and a
circular output beam. Also their planar structure, where the
optical cavity is formed along the device's growth direction as
shown in Fig.~\ref{fig:svcsel:vcsels}, results in many important
advantages such as compatibility with on-wafer probing, and one
and two-dimensional (1-D and 2-D) integration of VCSEL arrays. The
laser diode analyzed here is an 863 nm bottom-emitting VCSEL with
a 16 mm diameter, as described in~\cite{ohiso:flip-chip:1996}. The
laser diode model is based on the simple thermal VCSEL model
developed by Mena \textit{et.al.}~\cite{mena:asimple:1999}. It is
a semi-empirical model based on the standard laser rate equations
and a thermally dependent empirical offset current. The following
sections describes in details the governing equations of the model
and its implementation in fREEDA. The work is described in References \cite{pant:neifeld:2005}
and
\cite{kanj:thesis}.\\[0.1in]

\noindent\textbf{Analysis}\\[0.1in]

\noindent One of the most recognized limitation of a VCSEL's
performance is its thermal behavior. Due to the large electrical resistance
introduced by the Distributed Bragg Reflector
(DBR's)~\cite{mena:asimple:1999} and their poor heat dissipation
characteristics, typical VCSEL's undergo relatively severe heating
and consequently can exhibit strong thermally dependent behavior.
That is why the effects of self heating on the output
characteristics of VCSEL's are very significant. For example,
thermal lensing can yield considerable differences between cw (continuous wave) and
pulsed operation, as well as altering the emission profile of the
laser's optical modes. However, the most important effect is
exhibited in the device's static $LI$ (light versus current) characteristics. First, as
with edge-emitters, VCSEL's exhibit temperature-dependent
threshold current. Also, because the active-region temperature
increases severely with the injection current, cw operation is
limited by a sharp roll-over in the output
power~\cite{hasnain:performance:1991}.

Clearly, for any VCSEL model to be effective for the design of
optoelectronic applications, the model should capture thermal
effects, in particular the temperature-dependent threshold current
and output power roll-over. Also, the model must be able to
simulate both static and dynamic modulation of the laser. To meet
the above criteria, the model should be based on temperature
dependent rate equations. The strong thermal dependence of VCSEL's
can be attributed to a number of
mechanisms~\cite{mena:asimple:1999} such as Auger recombination
and optical losses, however, the most important effects during
static, or cw, operation are due to the temperature-dependent gain
and carrier leakage out of the active region. For simplicity, the
model in~\cite{mena:asimple:1999} ignores the
temperature-dependence of the gain and the carrier leakage is
taken into account by introducing a thermally dependent empirical
offset current into the model equations.

The above threshold static $LI$ characteristics of the VCSEL can be
modeled using $P_o=\eta(T)(I-I_{th}(N,T))$, where $P_o$ is the
optical output power, $\eta(T)$ is the temperature dependent
differential slope efficiency, $I$ is the injection current, and
$I_{th}(N,T)$ is the threshold current as a function of the
carrier number $N$ and the active region temperature $T$. Assuming
that the temperature dependence of the differential slope efficiency is minimal,
and neglecting the effect of spatial hole
burning~\cite{mena:asimple:1999}, the output power expression
becomes:
\begin{equation}
\label{eqn:svcsel:po}
P_o=\eta(I-I_{th}(T))
\end{equation}
where $I_{th}(T)$ can be expressed as a constant value $I_{tho}$
plus an empirical thermal offset current $I_{\mathrm{off}}(T)$, that is
$I_{th}(T)=I_{tho}+I_{\mathrm{off}}(T)$. The temperature-dependent offset
current could be a function of any form, but for simplicity, it is
taken as a polynomial function of temperature
\begin{equation}
\label{eqn:svcsel:ioff}
I_{\mathrm{off}}(T)=a_0 + a_1 T + a_2 T^2 + a_3 T^3 + a_4 T^4 + \cdots
\end{equation}
where the coefficient $a_0-a_4$ can be determined during parameter
extraction. It is important to note that because
Eqn.~\ref{eqn:svcsel:ioff} is not exclusively an increasing
function of temperature, it should be able to capture the general
temperature dependence of the VCSEL's $LI$ curves.

Now that we have described a method to consider the thermal effect
on the leakage current, we need an expression of the temperature
characteristics of the VCSEL. While it is possible to
adopt a numerical representation of a VCSEL's temperature profile as a
function of the heat dissipation the device, a better method
and more suitable for circuit level simulations is to describe the
active region temperature via a thermal rate equation as
follows~\cite{mena:asimple:1999}:
\begin{equation}
\label{eqn:svcsel:tre}
T=T_o + (IV-P_o) R_{th} - \tau_{th}\frac{dT}{dt}
\end{equation}
where $T_o$ is the ambient temperature, $V$ is the terminal
voltage of the laser, $R_{th}$ is the VCSEL's thermal impedance
which relates temperature change to the heat power dissipation,
and $\tau_{th}$ is the thermal time constant which accounts to the
nonzero response time of the device temperature (observed to be on
the order of 1 $\mu s$~\cite{mena:asimple:1999}). Eqn.~\ref{eqn:svcsel:tre}
also captures the thermal dynamics which is important in the transient
characteristics of a VCSEL.\\[0.1in]

\noindent\textbf{Rate Equations}\\[0.1in]
\noindent As discussed before, the model should be able to
simulate both static (DC) and dynamic (transient) modulation of the
VCSEL. To do this, the model should be based on the laser rate
equations. Fortunately, the simple above-threshold LI curves
described by $P_o=\eta(I-I_{th})$ can be described by the standard
laser rate equations~\cite{agrawal:semi:1993}. Thus, by
introducing the empirical offset current $I_{\mathrm{off}}(T)$ into these
equations, the model should be able to simulate the LI curves of
the VCSEL at different temperature as well as the dynamic behavior
such as small-signal and transient modulation.

After the addition of the offset current, the laser rate equations
become:
\begin{equation}
\label{eqn:svcsel:ren}
\frac{dN}{dt}=\frac{\eta_i(I-I_{off}(T))}{q} - \frac{N}{\tau_n} -
\frac{G_o(N-N_o)S}{1+\varepsilon S}
\end{equation}
\begin{equation}
\label{eqn:svcsel:res}
\frac{dS}{dt}=-\frac{S}{\tau_p}+\frac{\beta
N}{\tau_n}+\frac{G_o(N-N_o)S}{1+\varepsilon S}
\end{equation}
where $S$ is the photon number, $N$ is the carrier number, $N_o$
is the carrier transparency number, $\eta_i$ is the injection
efficiency, $\tau_n$ is the carrier recombination lifetime, $G_o$
is the gain coefficient, $\tau_p$ is the photon lifetime, and
$\beta$ is the spontaneous emission coupling coefficient. As we
can see, the introduction of the offset current into the rate
equations is quite simple, however, it is an extremely effective
means of describing the thermal dependence of the VCSEL's
continuous wave $LI$ characteristics. In addition, since the
model is based on the rate equations, it should be able to
simulate the non-dc behavior of the VCSEL. Finally, the optical
output power is described using $P_o=kS$, where $k$ is a scaling
factor accounting for the output coupling efficiency of the laser.\\[0.1in]


\noindent\textbf{Current/Voltage characteristics}\\[0.1in]
\label{sec:svcsel:cvc}
\noindent The current-voltage relationship of the VCSEL can be expressed
in great detail based on its diode-like characteristics, however,
the voltage across the device in this model has been selected to
be an arbitrary empirical function of current and temperature as
follows:
\begin{equation}
\label{eqn:svcsel:ivc} V=f(I,T)\cdot
\end{equation}
Then, the complete electrical characteristics of the VCSEL can be
accounted for by introducing capacitors and other parasitics
components in parallel with this voltage (in which case
Eqn.~\ref{eqn:svcsel:tre} should be modified such that it depends
on the total device current and not just $I$). The advantage of
this approach is that since different VCSELs have different $IV$
characteristics, the specific form of Eqn.~\ref{eqn:svcsel:ivc}
can be determined on a device-by-device basis. For example, the $IV$
relationship could be modeled as $V=IR_s+V_T\ln(1+I/I_s)$ where
$R_s$ is a series resistance, $I_s$ is the diode's saturation
current, and $V_T$ is the diode's thermal voltage which is usually
temperature dependent. In other cases, a polynomial function of
current and temperature~\cite{mena:asimple:1999} such as
\begin{equation}
\label{eqn:svcsel:ivp}
V=(b_0+b_1T+b_2T^2+\cdots).(c_0+c_1I+c_2I^2+\cdots)
\end{equation}
can be used, where $b_n$ and $c_n$ are constant
parameters to be extracted. It is important to note that if
experimental $IV$ data is used to determine all the other parameters
of the model first, then the exact form of
Eqn.~\ref{eqn:svcsel:ivc} can be determined at the very end of
parameter extraction of a specific device. This approach actually
has may advantages. First, it allows the voltage's current and
temperature dependence to be accurately modeled. Second, it
permits the optical and electrical characteristics to to be
decoupled from one another, therefore simplifying the extraction
of the model's parameters. For the VCSEL in consideration, the
author provides only the $IV$ characteristics at room temperature.
That is why the $IV$ data was fitted simply to a polynomial function
of current as follows:
\begin{eqnarray}
\label{eqn:svcsel:ivpext}
V&=&1.721 + 275 I - 2.439~\times~10^4I^2 + 1.338~\times~10^6I^3 \nonumber \\
&&- 4.154~\times~10^7I^4 + 6.683~\times~10^8I^5 - 4.296~\times~10^9I^6~.
\end{eqnarray}\\[0.1in]


\noindent\textbf{Implementation}\\[0.1in]
\label{sec:svcsel:impl}

\noindent The large-signal model of the VCSEL follows from
the rate equations, Eqn.~\ref{eqn:svcsel:ren} and Eqn.~\ref{eqn:svcsel:res},
from the temperature dependent offset current $I_{\mathrm{off}}(T)$, from the thermal
rate equation, Eqn.~\ref{eqn:svcsel:tre}, and from the current/voltage
characteristics of the diode described in Sec.~\ref{sec:svcsel:cvc}.
However, the implementation of the above equations as they are will
lead to convergence problems and this is why variable transformation
was used and the rate equations were normalized. First, $S$ is transformed
into a new variable $X_s$ via $S=X_s/k$, and $N$ into $X_n$
via $N=z_nX_n$, where $k$ is the output coupling efficiency and $z_n$ is an
arbitrary constants in the order of $10^7$. This will ensure that
the state variables ($X_s$ and $X_n$ as discussed in the next section)
will not take on very large values. Second, the rate equations
were normalized so that every term in those equations is well behaved. That is,
Eqn.~\ref{eqn:svcsel:tre} is divided by $R_{th}$, Eqn.~\ref{eqn:svcsel:ren} is
multiplied by $q/\eta_i$ and Eqn.~\ref{eqn:svcsel:res} is multiplied by $\tau_pk$.
The resulting equations are as follows:
\begin{equation}
\label{eqn:svcsel:tre_norm}
\frac{T}{R_{th}}=\frac{T_o}{R_{th}} + (IV-P_o) - \frac{\tau_{th}}{R_{th}}\frac{dT}{dt}
\end{equation}
\begin{equation}
\label{eqn:svcsel:ren_norm}
\frac{qzn}{\eta_i}\frac{dX_n}{dt}=(I-I_{off}(T)) - \frac{qznX_n}{\eta_i\tau_n} -
\frac{q}{\eta_ik} \frac{G_o(znX_n-N_o)X_s}{1+\frac{\varepsilon}{k} X_s}
\end{equation}
\begin{equation}
\label{eqn:svcsel:res_norm}
\tau_p\frac{dX_s}{dt}=-X_s+\frac{k \tau_p \beta
znX_n}{\tau_n}+\frac{\tau_p G_o(znX_n-N_o)X_s}{1+\frac{\varepsilon}{k}X_s}
\end{equation}

Also, the model was modified to
include the output wavelength $\lambda$ which can be calculated
from the carrier density $N$ with the following
equation~\cite{neifeld:packaging:1995}:
\begin{equation}
\label{eqn:svcsel:wle}
\lambda=\lambda_o[1-\frac{\rho}{n}(N-N_o)]
\end{equation}
where $\lambda_o$ is the intrinsic band gap wavelength, $\rho$ is
the total variation of the refractive index due to the differences
in injected current levels, and $n$ is the refractive index of the
medium.

To implement the model, we start by identifying
the state variables, then by writing the model equations as a
function of those state variables and their derivatives. Since the
terminal voltage $V$ is expressed in Eqn.~\ref{eqn:svcsel:ivpext}
as a function of the current $I$ (which could be the input
terminal current or not depending on wether a parasitic capacitor
$C_l$ is included or
not), $I$ is chosen as the first state variable. Then, if $C_l$ is
not included, $V$ can be directly written as a function of $I$. On
the other hand, if $C_l$ is included, then the total input current
is expressed as $I_{tot}=I+I_{Cl}$ where $I_{Cl}=C_ldV/dt$ and can
be expressed as:
\begin{equation}
\label{eqn:svcsel:icl}
I_{Cl}=C_l(c_1+2c_2I+3c_3I^2+4c_4I^3+5c_5I^4)\cdot
\end{equation}

Second, the three rate
equations, Eqn.~\ref{eqn:svcsel:tre_norm},~\ref{eqn:svcsel:ren_norm},
and~\ref{eqn:svcsel:res_norm}, need to be satisfied.
This is done by first augmenting the model with three ports (three terminals
with their respective local reference terminals), second by letting the above
equations be equal to the current at each of the
respective port, and finally forcing that current to zero by connecting an
open circuit to that port. It is very much in the same way
Eqn.~\ref{eqn:dhld:resn} was satisfied as described in
Sec.~\ref{sec:dhld:imp}.

Finally, the voltage at the designated optical output power is
described by $P_o=(v_m+\delta)$ where the current is
meaningless. The same should also be done for implementing
Eqn.~\ref{eqn:svcsel:wle}.\\[0.1in]


\noindent\textbf{Simulations and Results}\\[0.1in]
\noindent The purpose of the following sections is to present
some of the results of the VCSEL model implemented in
fREEDA.

In Sec.~\ref{sec:svcsel:dcanl}, a DC analysis is performed on the
implemented VCSEL model. Graphs of the $IV$ and $LI$ curves at
different ambient temperature are shown and compared to
the measurments. Also, plots of the carrier
number, output wavelength, and active region temperature versus
the input current are presented.

Sec.~\ref{sec:svcsel:tranl} presents the results of the transient
analysis. The model is driven by an input current pulse of finite
rise and fall time. All the transient simulations are carried out
at 20 $^o$C ambient temperature. Graphs of the carrier number and
active region temperature versus time are presented. Also, plots
of the optical output power and wavelength chirp are shown.

In Sec.~\ref{sec:svcsel:hbanl} a Harmonic Balance simulation
is performed on the implemented VCSEL model. First, the frequency
response of the VCSEL for an input rf power of --8 dBm is presented.
Second, plots of the large signal wavelength
chirp for different bias current is presented. Finally, the power ratio of the second
harmonic to the fundamental $P_{2f}/P_f$ and of the intermodulation
distortion to the fundamental $P_{IM3}/P_f$ as a function of bias
current and temperature are shown.

The fREEDA netlists which
was used to generate the plots in the Reference \cite{kanj:thesis}.\\[0.1in]

\begin{figure}
\centerline{\epsfxsize=4.5in \epsfbox{figures/SVCSELD_v_dc.eps}}
\caption{DC Analysis comparison of the $IV$ curve of the VCSEL model and the measurement.
Measurements from~\cite{mena:asimple:1999}.}
\label{fig:svcsel:vdc}
\end{figure}


\noindent\textbf{DC Analysis}\\[0.1in]
\label{sec:svcsel:dcanl}

\noindent The input terminal current is varied from 0 mA to 37 mA
and the simulation is run at 3 different ambient temperature, 20
$^o$C, 60 $^o$C, and 100 $^o$C. Fig.~\ref{fig:svcsel:vdc} shows
the $IV$ curve of the VCSEL (only one curve is show since the $IV$
characteristics of the diode is modeled as an independent function
of temperature).

Plots in Fig.~\ref{fig:svcsel:ndc} shows a family of curves of the
carrier number in the active region as a function of the input
current at different ambient temperature. From this graph, we can
clearly deduce two things. First, and as expected, the carrier
number will not increase anymore once threshold is reached.
Second, carrier leakage starts at a much lower input current at
high ambient temperature. Also, as seen in
Fig.~\ref{fig:svcsel:ldc}, the wavelength is constant above
threshold. This is due to the fact that the output wavelength is
mainly a function of the carrier number. Finally,
Fig.~\ref{fig:svcsel:tdc} shows the plots of the active region
temperature versus the input current for different ambient
temperature. It is clear from these plots how severely the VCSEL
can be heated.

\begin{figure}[!h]
\centerline{\epsfxsize=4.5in \epsfbox{figures/SVCSELD_n_dc.eps}}
\caption{DC Analysis plots of the carrier number at different ambient
temperature.}
\label{fig:svcsel:ndc}
\end{figure}

\begin{figure}[!h]
\centerline{\epsfxsize=4.5in \epsfbox{figures/SVCSELD_l_dc.eps}}
\caption{DC Analysis plots of the output wavelength at different ambient
temperature.}
\label{fig:svcsel:ldc}
\end{figure}

The optical output power as a function of the input current is
shown in Fig.~\ref{fig:svcsel:pdc}. From these plots, we can
deduce a lot of things. First, there is a threshold current shift
at different ambient temperatures. In addition, there is a
significant reduction in the slope efficiency and the maximum
output power. Finally, the effect of carrier leakage is obvious
and manifest itself clearly in the optical output roll over and
the complete turn-off of the laser.

As we can see in the $IV$ and $LI$ plots, there is an excellent agreement
between the simulations in fREEDA and the measurements.\\[0.1in]



\begin{figure}
\centerline{\epsfxsize=4.5in \epsfbox{figures/SVCSELD_t_dc.eps}}
\caption{DC Analysis plots of the active region temperature increase at
different ambient temperature.}
\label{fig:svcsel:tdc}
\end{figure}

\begin{figure}
\centerline{\epsfxsize=4.2in \epsfbox{figures/SVCSELD_p_dc.eps}}
\caption{DC Analysis comparison of the $LI$ curves at different ambient
temperature with the measurement. Measurements from~\cite{mena:asimple:1999}.}
\label{fig:svcsel:pdc}
\end{figure}

\begin{figure}
\centerline{\epsfxsize=4.5in \epsfbox{figures/SVCSELD_n_T0_20_pulse.eps}}
\caption{Transient analysis plot of the carrier number at 20 $^o$C.}
\label{fig:svcsel:ntran}
\end{figure}

\newpage
\noindent\textbf{Transient Analysis}\\[0.1in]
\label{sec:svcsel:tranl}

\noindent While DC simulations are very important to identify key
factors such as threshold current, maximum output power and
temperature effects in the VCSEL, transient analysis is also a
crucial part in the design of OEICs. This is the main
reason why the model was based on the laser rate equations.

The VCSEL is driven by a current pulse that has a peak value of 15
mA, a period of 5 nano-seconds, and a rise and fall time of 0.1
ns. The ambient temperature is set to 20 $^o$C during
the simulations. The plots in Fig.~\ref{fig:svcsel:ntran} show
the carrier number versus time while Fig.~\ref{fig:svcsel:ltran}
shows the plots of wavelength chirp which is a critical factor in
the design of Wavelength Division Multiplexed (WDM) Systems.
Fig.~\ref{fig:svcsel:ttran} shows how fast is the increase in the
active region temperature and plots in Fig.~\ref{fig:svcsel:ptran}
shows the optical output power. In the last figure, the optical
output power shows the well known laser turn-on delay and ringing
effects.\\[0.1in]



\begin{figure}
\centerline{\epsfxsize=4.5in \epsfbox{figures/SVCSELD_l_T0_20_pulse.eps}}
\caption{Transient analysis plot of the wavelength chirp at 20 $^o$C.}
\label{fig:svcsel:ltran}
\end{figure}

\begin{figure}
\centerline{\epsfxsize=4.5in \epsfbox{figures/SVCSELD_t_T0_20_pulse.eps}}
\caption{Transient analysis plot of the increase in the active
region temperature at 20 $^o$C.}
\label{fig:svcsel:ttran}
\end{figure}

\begin{figure}
\centerline{\epsfxsize=4.5in \epsfbox{figures/SVCSELD_p_T0_20_pulse.eps}}
\caption{Transient analysis plot of the output optical power at 20 $^o$C.}
\label{fig:svcsel:ptran}
\end{figure}

\begin{figure}
\centerline{\epsfxsize=3.5in \epsfbox{figures/SVCSEL_parasitics.eps}}
\caption{Parasitic network used in HB simulation. After~\cite{bruensteiner:extraction:1999}.}
\label{fig:svcsel:parasitic}
\end{figure}

\noindent\textbf{Harmonic Balance}\\[0.1in]
\label{sec:svcsel:hbanl}

\noindent VCSEL diodes are promising light sources for low-cost, high-performance
optical microwave links in microcellular networks and high speed phased-array radar
antenna~\cite{carlsson:analog:2002}. Lately, analog fiber-optic link based on
directly modulated VCSELs was also proposed to get rid of the digital data transmission
limitations in hazardous highly radioactive environment with large temperature variation
from 50 to 200 $^o$C such as thermonuclear reactors~\cite{fernandez:toward:2002}.
It is therefore of great importance
to characterize the VCSEL's behavior for analog applications at microwave frequencies.

In this section, Harmonic Balance was used to study the VCSEL's characteristics of
importance to microwave modulation such as the modulation response, the large signal
wavelength chirp, and most importantly the VCSEL's linearity as a function of bias
current and ambient temperature. First, the laser was connected to the parasitic network shown
in Fig.~\ref{fig:svcsel:parasitic} and the VCSEL's harmonic response was characterized.
The laser was driven by a single tone rf-input power of --8 dBm and the amplitude of the first
three harmonic peaks were monitored as the signal frequency was varied. Fig.~\ref{fig:svcsel:freqres12}
shows the modulation response of the VCSEL at a bias current of 12 mA while Fig.~\ref{fig:svcsel:freqres14}
and~\ref{fig:svcsel:freqres16} are the plots at a bias current of 14 mA and 16 mA respectively.
As we can see, the first-order relaxation resonance frequency appears to be around 4 GHz
and shifts towards 5 GHz at higher bias current. Also, the laser appears extremely nonlinear
at this high input power specifically around the resonant frequency.


\begin{figure}
\centerline{\epsfxsize=4.2in \epsfbox{figures/SVCSEL_freq_res_12mA.eps}}
\caption{Frequency response of the first three harmonics for a constant
input signal power of --8 dBm at 12 mA bias current.}
\label{fig:svcsel:freqres12}
\end{figure}

\begin{figure}
\centerline{\epsfxsize=4.2in \epsfbox{figures/SVCSEL_freq_res_14mA.eps}}
\caption{Frequency response of the first three harmonics for a constant
input signal power of --8 dBm at 14 mA bias current.}
\label{fig:svcsel:freqres14}
\end{figure}

\begin{figure}
\centerline{\epsfxsize=4.2in \epsfbox{figures/SVCSEL_freq_res_16mA.eps}}
\caption{Frequency response of the first three harmonics for a constant
input signal power of --8 dBm at 16 mA bias current.}
\label{fig:svcsel:freqres16}
\end{figure}

\begin{figure}
\centerline{\epsfxsize=4.5in \epsfbox{figures/SVCSEL_chirp.eps}}
\caption{Plots of the wavelength chirp versus frequency at different
bias current.}
\label{fig:svcsel:chirp}
\end{figure}

Fig.~\ref{fig:svcsel:chirp} shows the large signal wavelength chirp as the input frequency
was varied at different bias current. The plots shows that the wavelength chirp increases with
increasing bias current and it peaks at the resonant frequency.

Second, the VCSEL was driven by a single tone RF-input signal of --20 dBm at 1 GHz. Fig.~\ref{fig:svcsel:p2f}
shows plots of the power ratio of the second harmonic to the fundamental $P_{2f}/P_f$ as a function of bias
current at different ambient temperature. The results shows that there is an increase in linearity with
bias current up to the power rollover point where linearity starts to decrease again.
It is also interesting to see this linearity level given the nonlinear $LI$ curves of the laser
(Fig.~\ref{fig:svcsel:pdc}) and that the VCSEL is mostly linear around the maximum output power bias point.
However, the VCSEL's linearity decreases considerably with increasing ambient temperature and this could
be explained by the fact that at higher ambient temperature, the same input RF power will drive the VCSEL
more into the off region leading to more harmonic distortion.

Finally, the VCSEL was driven by two signals of equal input power of -20 dBm at 1.0 GHz and 1.01 GHz.
Fig.~\ref{fig:svcsel:im3} shows the power ratio of the third-order intermodulation product to the
carrier $P_{IM3}/P_f$ as a function of the bias current at different ambient temperature. Again, the
results shows that the VCSEL's nonlinear behavior is similar to the results described in the previous
figure.


\begin{figure}
\centerline{\epsfxsize=4.5in \epsfbox{figures/SVCSEL_P2f.eps}}
\caption{Power ratio of second harmonic to fundamental
as a function of bias current for different temperature.}
\label{fig:svcsel:p2f}
\end{figure}

\begin{figure}
\centerline{\epsfxsize=4.5in \epsfbox{figures/SVCSEL_IM3.eps}}
\caption{Power ratio of third-order intermodulation products
to carrier as a function of bias current for different temperature.}
\label{fig:svcsel:im3}
\end{figure}


\begin{thebibliography}{99}

\bibitem{pant:neifeld:2005}
R. Pant, M. A. Neifeld, M. B. Steer, H. Kanj, and A. C. Cangellaris, �Electrical package impact on VCSEL-based optical interconnects ;� Optics Communication, Vol. 245, Issue 1-6, Jan. 2005, p. 315-332.

\bibitem{kanj:thesis}
H. Kanj, Circuit-Level Modeling of Laser Diodes , M.S., North Carolina State University, 2003

\bibitem{Bewtra:modeling:1995} N. Bewtra, D. A. Suda, G. L. Tan,
F. Chatenoud, and J. M. Xu, ``Modeling of Quantum-Well Lasers with
Electro-Opto-Thermal Interaction,'' {\em IEEE J. of Selected
Topics in Quantum Electronics}, Vol. 1, No. 2, pp. 331-340, June
1995.

\bibitem{morikuni:simulation:1998} J. Morikuni, G. Dare, P. Mena,
A. Harton, K. Wyatt, ``Simulation of optical interconnect devices,
circuits, and systems using analog behavioral modeling tools,''
{\em IEEE Lasers and Electro-Optics Society Annual Meeting}, Vol.
1, pp. 235--236, Dec 1998.

\bibitem{freeda:2002} {\em http://www.freeda.org}.

\bibitem{griewank:adol-c:1996} A. Griewank, D. Juedes and J. Utke,
``Adol-C: A Package for the Automatic Differenciation of
Algorithms Written in C/C++'' ACM TOMS, Vol. 22(2), pp. 131�-167,
June 1996.

\bibitem{hecht:laser:1992} J. Hecht,
``Laser Pioneers'' {\em Boston: Academic Press}, 1992.

\bibitem{dupuis:anintro:1987} R. D. Dupuis ``An Introduction to
the Developement of the Semiconductor Laser'' {\em IEEE J. of
Quantum Electronics}, Vol. QE-23, No. 6, June 1987.

\bibitem{morikuni:optoelectronic:1998} J. Morikuni, P. Mena,
A. Harton, K. Wyatt, ``Optoelectronic Computer-Aided Design,''
{\em IEEE Lasers and Electro-Optics Society Summer Topical
Meetings}, pp. 53--54, July 1998.

\bibitem{seung-woo:optical:1999} L. Seung-Woo, C. Eun-Chang, C.
Woo-Young, ``Optical interconnection system analysis using SPICE,
'' {\em The Pacific Rim Conference on Lasers and Electro-Optics},
Vol. 2, pp. 391--392, Sep 1999.

\bibitem{christoffersen:global:2001} C. E. Christoffersen,
{\em Global Modeling of Nonlinear Microwave Circuits}, Ph.D.
Dissertation, Dept. of Electrical Engineering, North Carolina
State University, 2001.

\bibitem{joyce:electrical:1978} W. B. Joyce, R. W. Dixon,
``Electrical characterization of heterostructure lasers,'' {\em in
J. Appl. Phys.}, Vol. 49, No. 7, July 1978.

\bibitem{tucker:circuit:1981} R. S. Tucker, ``Circuit model of
double-heterojunction laser below threshold,'' {\em in IEE Proc.
on Solid-State \& Electron Devices--Part I}, Vol. 128, No. 3, pp.
101--106, 1981.

\bibitem{tucker:large:1981} R. S. Tucker, ``Large-signal circuit
model for simulation of injection-laser modulation dynamics,''
{\em in IEE Proc. on Solid-State \& Electron Devices--Part I},
Vol. 128, No. 5, pp. 180--184, 1981.

\bibitem{casey:hetero:1978} H.C. Casey, M.B. Panish,
``Heterosructure Lasers, part A and part B'' {\em Academic Press},
1978.

\bibitem{coldren:diode:1995} L.A. Coldren, S.W. Corzine,
``Diode Lasers and Photonic Integrated Circuits,''
{\em John Wiley \& Sons, Inc.}, 1995.

\bibitem{rizzoli:state:1992} V. Rizzoli, A. Lipparini, A. Costanzo,
F. Mastri, C. Cecchetti, A. Neri, D. Masotti, ``State-of-the-art
harmonic-balance simulation of forced nonlinear microwave circuits
by the piecewise technique,'' {\em IEEE Transactions on Microwave
Theory and Techniques}, Vol. 40, issue 1, pp. 12--28, Jan. 1992.

\bibitem{velu:charge:2002} S. N. Velu,
{\em Charge Based Modeling in State Variable Based Simulator},
M.S. Thesis, Dept. of Electrical Engineering, North Carolina State
University, 2002.


\bibitem{way:large:1987} W. I. Way, ``Large Signal Nonlinear
Distortion Prediction for a  Single-Mode Laser Diode Under
Microwave Intensity Modulation,'' {\em IEEE J. of Lightwave
Technology}, Vol. LT-5, No. 3, March 1987.

\bibitem{iezekiel:nonlinear:1990} S. Iezekiel, C. M. Snowden,
``Nonlinear Circuit Analysis of Laser Diodes Under Microwave
Direct Modulation,'' {\em IEEE MTT-S International}, Vol. 2, pp.
937--940, 1990.

\bibitem{towe:ahistorical:2000} E. Towe, R. F. Leheny, A. Yang,
``A Historical Perspective of the Developement of the
Vertical-Cavity Surface-Emitting Laser,'' {\em IEEE J. on Selected
Topics in Quantum Electronics}, Vol.6, No. 6, Nov/Dec 2000.

\bibitem{ohiso:flip-chip:1996} Y. Ohiso, K. Tateno, Y. Kohama,
A. Wakatsuki, H. Tsunetsugu, and T. Kurokawa, ``Flip-chip bonded
0.85-$\mu m$ bottom-emitting vertical-cavity laser array on an
AlGaAs substrate,'' {\em IEEE Photon. Technol. Lett.}, Vol. 8, pp.
1115-1117, Sep 1996.

\bibitem{mena:asimple:1999} P. V. Mena, J. J. Morikuni, S.-M.
Kang, A. V. Harton, K. W. Wyatt, ``A Simple Rate-Equation-Based
Thermal VCSEL Model,'' {\em IEEE J. of Lightwave Technology}, Vol.
17, No. 5, May 1999.

\bibitem{neifeld:vcselst:2003} M. A. Neifeld (private communication),
2003.

\bibitem{hasnain:performance:1991} G. Hasnain, K. Tai, L. Yang,
Y. H. Wang, R. J. Fisher, J. D. Wynn, B. Weir, N. K. Dutta, and A.
Y. Cho, ``Performance of Gain-guided Surface Emitting Lasers with
Semiconductor Distributed Bragg Reflectors,'' {\em IEEE J. of
Quantum Electron.}, Vol. 27, No. 6, pp. 1377-1385, 1991.

\bibitem{agrawal:semi:1993} G. P. Agrawal, and N. K. Dutta,
``Semiconductor Laser, 2nd ed.'' {\em New York: Van Nostrand
Reinhold}, 1995.

\bibitem{neifeld:packaging:1995} M. A. Neifeld and W. C. Chou,
``Electrical packaging impact on source components in optical
interconnect,'' {\em IEEE Transactions on Components, Packaging,
and Manufacturing Technology --- Part B}, Vol. 18, pp. 578--595,
Aug. 1995.

\bibitem{bruensteiner:extraction:1999} M. Bruensteiner, G. C. Papen,
``Extraction of VCSEL Rate-Equation Parameters for Low-Bias System
Simulation,'' {\em IEEE J. of Selected Topics in Quantum Electron},
Vol. 5, No. 3, 1999.

\bibitem{carlsson:analog:2002} C. Carlsson, H. Martinsson, R. Schatz, J. Halonen,
A. Larsson ``Analog Modulation Properties of Oxide Confined VCSELs at
Microwave Frequencies,'' {\em IEEE J. of Lightwave
Technology}, Vol. 20, No. 9, September 2002.

\bibitem{fernandez:toward:2002} A. F. Fernandez, F. Berghmans, B. Brichard,
M. Decreton, ``Toward The Developement of Radiation-Tolerant Instrumentation
Data Links for Thermonuclear Fusion Experiments,'' {\em IEEE Transactin on
Nuclear Science}, Vol. 49, No. 6, December 2002.

\end{thebibliography}

\noindent\linethickness{0.5mm} \line(1,0){425}
\newline\textit{Known Bugs:}\\
Several aspects of the model are hard coded for a specific case.  The model needs to be generalized. \\[0.1in]
\noindent\linethickness{0.5mm} \line(1,0){425}
\newline
\textit{Version:}\\
2003.01.01 \\
% Credits
\linethickness{0.5mm} \line(1,0){425}
\newline
\textit{Credits:}\\
\begin{tabular}{l l l l}
Name & Affiliation & Date & Logo \\
Houssam Kanj & NC State University & 2003 & \epsfxsize=1in\epsfbox{figures/logo.eps} \\
www.ncsu.edu & & & \\
Michael Steer & NC State University & 2003 & \epsfxsize=1in\epsfbox{figures/logo.eps} \\
www4.ncsu.edu/~mbs & & & \\
\end{tabular}

\end{document}
